\documentclass[../VD.tex]{subfiles}

\externaldocument{../VD}

\begin{document}

\setcounter{chapter}{4}
\chapter{La diferencial de una aplicación diferenciable}\label{chap:diferenciable}

\section{Introducción}

Si \(f\colon M \to N\) es una aplicación diferenciable, se tiene para todo
\(p\in M\) una aplicación lineal \(\dif{f_p}\colon \Tangente[p]{M}\to
\Tangente[f(p)]{N}\) que lleva el vector \(v\in \Tangente[p]{M}\) representado
por la curva \(\alpha\colon (-\varepsilon,\varepsilon)\to M\) en el vector
\(\dif{f_p(v)}\) representado por la curva \(f\circ\alpha \colon (-\varepsilon,\varepsilon)\to N\). 

\begin{figure}[h]
    \centering
    \begin{tikzcd}[column sep = large]
      \Tangente[p]{M}
      \ar[d, "\cong\tau_{\varphi}^p"']
      \ar[r, "\dif{f_p}"]
      &
      \Tangente[f(p)]{N}
      \ar[d, "\tau_{\psi}^{f(p)}\cong"] \\
      \RealSet^{m}
      \ar[r, dashed, "\widetilde{\dif{f_p}}"']
      &
      \RealSet^{n} \\
      x \ar[r, maps to]
      &
      (\tau_{\psi}^{f(p)}\circ\dif{f_p}\circ(\tau_{\varphi}^{p})^{-1})(x) 
    \end{tikzcd}
    \caption{Diagrama de la aplicación diferenciable en un punto}
    \label{fig:dif-diag-dif}
  \end{figure}
  
  Estas aplicaciones están bien definidas y son lineales, cosa que se puede ver
  a partir del anterior diagrama conmutativo para cartas \((U,\varphi)\) y

  \((W,\psi)\) de \(M\) y \(N\) con \(p\in U\) y \(f(U)\subseteq W\) de la
  \cref{fig:dif-diag-dif},   
  donde \(\tau_{\varphi}^{p}\) (análogamente \(\tau_{\psi}^{f(p)}\))
  es un isomorfismo lineal visto en la \cref{def:tang-iso} que lleva \(v\) en
  el vector velocidad \((\varphi\circ \alpha)'(0)\).

  Además, se tiene que \(\widetilde{\dif{f_{p}}}=\J[\varphi(p)]{\widetilde{f}}\), con
  \(\widetilde{f}=\psi\circ f\circ\varphi^{-1}\), y esto significa que
  localmente la diferencial corresponde con la matriz jacobiana de \(f\). 

\begin{definition}[{name=[diferencial]{diferencial de una aplicación diferenciable}},
	label={def:dif-app}]
	Sea \(f \colon M^{m} \to N^{n}\) una aplicación diferenciable. Llamaremos
	\(\dif{f_{p}}\) a la \emph{diferencial} de \(f\) en \(p\).
	
	\begin{align*}
	\Tangente[p]{M} &\overset{\dif{f_{p}}}{\to} \Tangente[f(p)]{M}\\
	\Clase{\alpha} &\mapsto \Clase{f \circ \alpha}
	\end{align*}
\end{definition}

\begin{proposition}
La aplicación \(\dif{f}\colon\Tangente{M}\to \Tangente{N}\) definida al unir todos los \(\dif{f_p}\) con \(p\in M\) cuando consideramos a \(\Tangente{M}\) y \(\Tangente{N}\) como variedades es diferenciable (ver \cref{def:tang-iso})
\end{proposition}

\begin{proof}
En efecto, sean \(\widetilde{\varphi}\colon\Tangente{U}\to
\varphi(U)\times\RealSet^m\) y \(\widetilde{\psi}\colon\Tangente{W}\to
\psi(W)\times\RealSet^n\) funciones coordenada tales que
\((\Tangente{U},\widetilde{\varphi}),(\Tangente{W},\widetilde{\psi})\) son las
cartas asociadas a \((U,\varphi)\) y \((W,\psi)\) respectivamente, y con
\(f(U)\subseteq W\), tenemos por el siguiente diagrama:

\begin{center}
    \centering
    \begin{tikzcd}[column sep = large]
      \Tangente{U}
      \ar[d, "\widetilde{\varphi}"']
      \ar[r, "\dif{f}"]
      &
      \Tangente{W}
      \ar[d, "\widetilde{\psi}"] \\
      \varphi(U)\times\RealSet^{m}
      \ar[r, dashed, "\widetilde{\dif{f}}"']
      &
      \psi(W)\times\RealSet^{n} \\
      (x,y) \ar[r, maps to]
      &
      (\psi\circ f\circ\varphi^{-1}(x),\widetilde{\dif{f_p}}(y)) 
    \end{tikzcd}
  \end{center}

luego

\[\begin{array}{c}
    \widetilde{\dif{f}}(x,y)=\widetilde{\psi}\circ \dif{f}\circ
    \widetilde{\varphi}^{-1}(x,y)= \\
    (\psi \circ f\circ  \varphi^{-1}(x),\widetilde{\dif{f_{p}}})=(\psi \circ
    f\circ  \varphi^{-1}(x),\J[\varphi(p)=x]{\psi\circ f\circ\varphi^{-1}}(y)) 
\end{array}\]
que es diferenciable pues \(f\) debe serlo y \(\dif{f_{p}}\) también.
\end{proof}

\begin{remark}
Es inmediato a partir de la definición que hemos dado que \(\dif{f}\circ \dif{g}=\dif{f\circ g}\)
\end{remark}

\begin{example}
Si \(U\subseteq \RealSet^m\) es cualquier abierto visto como variedad (y en general cualquier variedad con abiertos euclídeos) ya vimos en \cref{def:tang-iso} que hay un difeomorfismo
\[
\Tangente{U}\cong U\times \RealSet^m
\]
que lleva \([\gamma]\in \Tangente[X]{U}\) en \((x,\gamma'(0))\).
\end{example}

\begin{note}
	De acuerdo con el diagrama \cref{fig:dif-diag-dif} podemos redefinir las
  nociones de inmersión y submersión en \(p\) diciendo que \(f\colon M \to N\)
  es inmersión (submersión) en \(p\) si y solo si \(\dif{f_p}\colon
  \Tangente[p]{M}\to\Tangente[f(p)]{N}\) es inyectiva (sobreyectiva).
\end{note}

\section{Puntos regulares. Valores regulares. Teorema de Sard}

Vamos a dar unas definiciones que nombren a los puntos según el rango en ellos
de una aplicación diferenciable.

\begin{definition}[{name=[regularidad de puntos]{regularidad de puntos}},
	label={def:punto-reg}]
Sea \(f\colon M^m\to N^n\) una aplicación diferenciable. Decimos que \(p\in M\)
es un \emph{punto regular} si \(\dif{f_p}\) es sobreyectiva, es decir, el rango
de \(f\) en \(p\) es \(n\).

Si \(p\) no es regular se dice que es un \emph{punto crítico o singular}
\end{definition}

\begin{definition}
Supongamos las mismas condiciones que en la \cref{def:punto-reg}, entonces un punto \(q\in N\) se dice un \emph{valor regular} si todos los puntos \(p\in f^{-1}(q)\) son regulares.

En caso contrario se dice que es un \emph{valor crítico o singular}
\end{definition}

\begin{definition}
En las condiciones de \cref{def:punto-reg} si \(q\) es valor regular a \(f^{-1}(q)\) se le llama conjunto de \emph{nivel regular}.
\end{definition}

\begin{remark}
En particular, en toda inmersión \(f\colon M\to N\) los puntos son críticos y en
toda submersión \(f\colon M\to N\) todos los puntos son regulares. 


Además, el \cref{thm:dim-inv-sub}
nos dice que los niveles regulares son subvariedades de \(M\) de dimensión \(m-n\).
\end{remark}

El siguiente teorema se dará sin demostración, que se puede leer en
\emph{Lectures in Geometry,Vol.1} de \textbf{M.M.Postnikov}.

\begin{theorem}[name={teorema de Sard}, label={thm:sard}]
Sea \(f\colon M\to N\) una aplicación diferenciable, entonces el conjunto de
valores críticos de \(f\) tiene medida nula en \(N\).
\end{theorem}

\begin{example}
Sea \(M(m)\equiv \RealSet^{n^2}\) las matrices cuadradas de orden \(n\) vistas como \(\RealSet^{n^2}\) para el orden lexicográfico. Sea \(Sim(n)\equiv \RealSet^{\frac{n(n+1)}{2}}\) las matrices simétricas se orden \(n\) (con el orden lexicográfico por encima de la diagonal).

Sea \begin{align*}
f\colon & \RealSet^{n^2}\to  \RealSet^{\frac{n(n+1)}{2}}\\
& X  \mapsto X^t X
\end{align*}

Nótese que \((X^tX)^t=X^tX\). Esta aplicación es una submersión en las matrices \(X\in f^{-1}(I_n)=\mathcal{O}(n)\) (recordar que \(X\in \mathcal{O}(n)\iff X^tX=I_n\) ya que la inversa coincide con la traspuesta).

Veamos que \(\dif{f_X}\colon\Tangente[X]{\RealSet^{n^2}}\to \Tangente[I_n]{\RealSet^{\frac{n(n+1)}{2}}}\) es sobreyectiva para todo \(X\in \mathcal{O}(n)\).

En efecto, dada cualquier matriz \(A\) tomamos \(\gamma_A\colon(-\varepsilon,\varepsilon)\to \RealSet^{n^2}\) la curva de matrices definida como
\[
\gamma_A(s)=X+sA
\]
Tenemos que \(\gamma_A(0)=X\) y \(\gamma_A'(0)=A\). Además \(\dif{f_X}([\gamma_A])\) está representada por la curva
\begin{align*}
f\circ \gamma_A(s)&=(X+sA)^t(X+sA)\\
&=(X^t+sA^t)(X+sA)\\
&=X^tX+sA^tX+sX^tA+s^2A^tA\\
&=I_n+sA^tX+sX^tA+s^2A^tA
\end{align*}
Así pues, tras identificar \(\Tangente[X]{\RealSet^{n^2}}\cong \RealSet^{n^2}\) y \(\Tangente[I_n]{\RealSet^{\frac{n(n+1)}{2}}}\cong \RealSet^{\frac{n(n+1)}{2}}\) la diferencial \(\dif{f_X}\) lleva \(\gamma_A'(0)=A\) en
\[
(f\circ \gamma_A)'(0)=A^tX+X^tA=A^tX+X^{-1}A
\]
Ahora bien, dada \(B\) una matriz simétrica cualquiera elegimos \(A=\frac{1}{2}XB\), y tenemos para esta \(A\)
\begin{align*}
(f\circ \gamma_A)'(0)&=\frac{1}{2}B^tX^tX+\frac{1}{2}X^{-1}XB\\
&=\frac{1}{2}B^tX^{-1}X+\frac{1}{2}X^{-1}XB\\
&=\frac{1}{2}B^t+\frac{1}{2}B\\
&=B
\end{align*}
Así pues, \(\dif{f_X}\) es sobreyectiva si \(X\in \mathcal{O}(n)\).

Por \ref{dim-sub-pto} como caso particular del teorema \ref{thm:dim-inv-sub} 
\(\mathcal{O}(n)\) se convierte en una variedad de dimensión
\[
n^2 -\dfrac{n(n+1)}{2}=\dfrac{n^2-n}{2}=\dfrac{n(n-1)}{2}
\]
\end{example}

\begin{proposition}\label{prop:df-iso}
  Si \(\dif{f_{p}}\colon\Tangente[p]{M}\to\Tangente[f(p)]{N}\) es inyectiva y
  \(f\) es sobreyectiva, entonces \(\dim{M}=\dim{N}\) (i.e \(\dif{f_{p}}\) es
  isomorfismo).
\end{proposition}

\begin{proof}
  Como \(\dif{f_{p}}\) inyectiva entonces \(\dim{M}\leq\dim{N}\).
  Como \(f\) es sobreyectiva entonces \(f(M)=N\) y por el \cref{lem:denso} y la
  \cref{prop:med-ldim} se tiene que \(m(f(M))\neq 0\) y \(\dim{M}\geq\dim{N}\)
  (pues si fuese \(<\) tendría medida nula).
  Finalmente, se tiene el resultado. 
\end{proof}

\begin{proposition}\label{prop:df-iny-f-biy-difeo}
Si \( f\colon M\to N \) es una inmersión biyectiva en cada \( p\in M \),
entonces \( f \) es un difeomorfismo.
\end{proposition}

\begin{proof}
Estamos en las condiciones de la \cref{prop:df-iso} por lo que ya sabemos que \(
df_p \) es isomorfismo. Por el \nameref{thm:inversa} para cada punto \( p\in M
\) se cumple que \( f\colon U_p\to f(U_p) \) es un difeomorfismo para cierto
abierto \( p\in U\subseteq M \).

Por tanto \( f^{-1}\colon N\to M \) es diferenciable pues se tiene que, \(\ \forall q\in N \), \( f^{-1}\colon f(U_p)\to U_p \) lo es si hacemos \( p=f^{-1}(q) \).
\end{proof}

\begin{note}
De los resultados anteriores se deduce que dos variedades difeomorfas tienen
necesariamente la misma dimensión, ya que si razonásemos por reducción al
absurdo tendríamos \( f\colon M \to N \) difeomorfismo con \(\dim{M}<\dim{N}\) y
como hemos visto \( f(M)=N \) ambos de medida nula, lo que contradice el
\cref{lem:denso} que utilizábamos en la prueba de la \cref{prop:df-iso}.
\end{note}

\begin{note}
	Si \(N^n\subseteq M^m\) es una subvariedad para cada punto \(p\in N\) la
  inclusión \(i\colon N\to M\) lleva \(\Tangente[p]{N}\) en el subespacio
  generado por los vectores 
	\[
	e_i^p=[\varepsilon_i^p] \quad \text{con} \; i\leq n
	\]
	si vemos \(\RealSet^n\subseteq \RealSet^m\) como el subespacio con las últimas
  \(m-n\) coordenadas nulas.
\end{note}

Este resultado se puede refinar más aún como veremos en el siguiente lema:

\begin{lemma}
Sea \(i\colon N\to M\) la inclusión,
\(\dif{i}\colon\Tangente{N}\to\Tangente{M}\) es una incrustación de
\(\Tangente{N}\) en \(\Tangente{M}\).
\end{lemma}

\begin{proof}
Si tenemos un atlas de \(N\) formado por cartas \((U,\varphi)\) con \(U=U'\cap N\) y \(\varphi'=\Restrict{\varphi}{N}\) siendo \((U',\varphi')\) cartas de \(M\) cumpliendo \(\varphi'(U'\cap N)=\varphi'(U')\cap \RealSet^n\).

Entonces las cartas \((\dif{i}\Tangente{U},\widetilde{\varphi})\) y \((\Tangente{U'},\widetilde{\varphi'})\) están ligadas por las igualdades
\begin{align*}
\widetilde{\varphi'}(\Tangente{U'}\cap (\dif{i}\Tangente{N}))&=\widetilde{\varphi'}(\dif{i}\Tangente{(U'\cap N)})\\
&=\varphi'(U'\cap N)\times (\RealSet^n\times \{0\})\\
&=(\varphi'(U')\cap \RealSet^n)\times(\RealSet^n\times \{0\})\\
                                                             &=\widetilde{\varphi'}(\Tangente{U'})\cap(\RealSet^n\!\times\!\{0\}\;\times \;\RealSet^n\!\times \!\{0\})\\
  &\subseteq\widetilde{\varphi'}(\Tangente{U'})\cap(\RealSet^{m}\times\RealSet^{m}) 
\end{align*}
Es decir, la estructura inducida coincide con la relativa.
\end{proof}

En particular, por el teorema de \nameref{thm:Whitney} tenemos que para todo \(p\in M\), \(\Tangente[p]{M}\) se puede identificar con un subespacio de \(\Tangente[p]{\RealSet^{2m+1}}\cong\RealSet^{2m+1}\) de dimensión \(m\), permitiéndonos así ver las rectas tangentes de \(M\) como vectores de \(\RealSet^{2m+1}\).

Esta es la representación de \(\Tangente{M}\) más intuitiva habitualmente.

\end{document}

%%% Local Variables:
%%% TeX-master: "../VD_ebook"
%%% End: