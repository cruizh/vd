\documentclass[../VD_completo.tex]{subfiles}

\externaldocument{../VD}

\begin{document}

\setcounter{chapter}{10}
\chapter{Curvas integrales}\label{chap:curvas-integrales}

\begin{definition}[curva integral]
  \label{def:curva-integral}
  Dado un campo tangente \(X \colon M \to \Tangente{M}\) sobre una
  \(m\)-variedad \(M\) se dice que una curva \(\alpha \colon J \to M\) definida
  sobre un intervalo abierto \(J \subseteq \RealSet\) (posiblemente \(J =
  \RealSet\)) es una \emph{curva integral} de \(X\) en \(p \in M\) si
  \(\alpha(0) = p\) y para todo \(t \in J\), \(X(\alpha(t))\) está representado
  por la curva \(\alpha_{t} \colon (-\delta, \delta) \to M\) definida por
  \(\alpha_{t}(s) = \alpha(t+s)\) con \(\delta > 0\) (dependiente de \(t\))
  elegido para que \(t + s \in J\).
\end{definition}

Si elegimos una carta \((U, \varphi)\) de \(M\) con \(p \in  U\), para \(\delta
> 0\) con \(\alpha : (-\delta, \delta) \to U\), se tiene \(\widetilde{\varphi}
\circ X \circ \widetilde{\varphi}^{-1} (\ell(t)) = (\ell(t), g^{U}(\ell(t)))\),
donde \(\ell \colon \varphi \circ \alpha \colon J \to \varphi(U)\) y \(g^{U}
\colon \varphi(U) \to \RealSet^{m}\) es la componente del campo \(X\) en la
carta \((U, \varphi)\). Por otro lado, \(\widetilde{\varphi} \colon \Tangente{U}
\to \varphi(U) \times \RealSet^{m}\) lleva el vector \([\alpha_{t}] \in
\Tangente[\alpha(t)]{M}\) en
\[
\left( \varphi \left( \alpha(t) \right), \left( \varphi \circ \alpha_{t}
  \right)'(0) \right) = \left( \ell(t),
  \Restrict{\D{\left( \varphi(\alpha(t+s))\right)}{s}}{s=0} \right) = \left( \ell(t), \ell'(t) \right)
\]

Recordemos que
\[
\Restrict{\D{\varphi(t+s)}{t}}{s=0} = \lim_{s \to 0}
\frac{\varphi(t+s)-\varphi(t)}{s} = \varphi'(t)
\]

Así pues localmente, una curva integral dentro de una carta \((U, \varphi)\)
está determinada por la igualdad \(g^{U}(\ell(t)) = \ell'(t)\), esto es el sistema
de ecuaciones diferenciales ordinarias

\begin{equation}
  \label{eq:edo-curva-integral}
  \begin{dcases}
    \ell'_{i}(t) = g^{U}_{i}(\ell_{1}(t), \dots, \ell_{m}(t)) &1 \leq i
    \leq m\\
    \ell(0) = \varphi(p)
  \end{dcases}
\end{equation}

Esto es: al pasar a coordenadas, una curva integral es aquella que tiene por
vector tangente en un punto el valor del campo en el mismo.

Las ecuaciones en \eqref{eq:edo-curva-integral} tienen solución única en el
siguiente sentido: existen abiertos \(\Omega \subseteq \varphi(U)\) y \(J\) de
\(\varphi(p)\) y \(0\) y una aplicación diferenciable \(\xi \colon \Omega \times
J \to \varphi(U)\) tal que \(\xi(x,0) = x\) para todo \(x \in \Omega\) y
\(\Style{DDisplayFunc=outset}\pderiv{\xi(x,t)}{t} = g^{U}(\xi(x,t))\) para todo
\((x,t) \in \Omega \times J\).

Así pues, \(\gamma^{p}(t) = \xi(\varphi(p), t)\) es la curva integral del
campo \(X\) con origen en \(p\). Además si \(\xi\) y \(\widehat{\xi}\) son
funciones cumpliendo las propiedades anteriores con \(\xi(x_{0},0) =
\widehat{\xi}(x_{0},0)\) entonces \(\xi(x_{0},t) = \widehat{\xi}(x_{0},t)\)
donde estén definidas ambas.

Es decir, si \(\gamma^{p}\) y \(\widehat{\gamma}^{p}\) son curvas integrales con
origen en \(p\) entonces \(\gamma^{p} = \widehat{\gamma}^{p}\) en un intervalo
común de definición alrededor de \(0\).

El sistema de ecuaciones de \eqref{eq:edo-curva-integral} codifica localmente el
campo \(X\), de forma que \(g^{U}\) es la componente del campo y las soluciones
del sistema son las curvas integrales del mismo.

Si pensamos los campos como derivaciones \(D \colon \mathcal{F}(M) \to
\mathcal{F}(M)\), entonces una curva integral de \(D\) es aquella para la que
\(D(f)(\gamma(t)) = (f \circ \gamma)'(t)\) para todo \(f \in \mathcal{F}(M)\).

En efecto, \(D = D_{X}\) para un campo \(X\) y
\[
  \Style{DDisplayFunc=inset}
  D(f)(x) = D_{X}(f)(p) = \Sum{g^{U}_{i}(\varphi(p)) \Restrict{\pderiv{f \circ \varphi^{-1}}{x_{i}}}{\varphi(p)}}{i,1,m}
\]

Entonces si \(\gamma(t)\) es curva integral de \(X\),
\begin{align*}
  D(f)(\gamma(t))
  &= \Sum{g^{U}_{i}((\varphi \circ \gamma)(t)) \Restrict{\pderiv{f \circ \varphi^{-1}}{x_{i}}}{\varphi(\gamma(t))}}{i,1,m}\\
  &= \Sum{(\varphi \circ \gamma)'_{i}(t) \Restrict{\pderiv{f \circ \varphi^{-1}}{x_{i}}}{\varphi(\gamma(t))}}{i,1,m}\\
  &= \left( f \circ \varphi^{-1} \circ \varphi \circ \gamma \right)'(t)\\
  &= \left( f \circ \gamma \right)'(t)
\end{align*}

Una propiedad clave de las curvas integrales es la siguiente:

\begin{lemma}[traslación de curvas integrales]
  \label{lem:ci-traslacion}
  Si \(\gamma^{p} \colon J \to M\) es una \nameref{def:curva-integral} con
  origen \(p \in M\) entonces si \(t \in J\) y \(q = \gamma(t)\), entonces la
  curva integral de origen q, \(\gamma^{q}\) cumple \(\gamma^{q}(s) =
  \gamma^{p}(t+s)\) siempre que \(t + s \in J\).
\end{lemma}

\begin{proof}
  \(X(\gamma^{q}(s)) = X(\gamma^{p}(t+s)) = [\gamma^{p}_{t+s}]\). Pero
  \(\gamma^{p}_{t+s}(s') = \gamma^{p}(t+s+s') = \gamma^{q}(s+s') =
  \gamma^{q}_{s}(s')\).

  Así, \(X(\gamma^{q}(s)) = [\gamma^{p}_{t+s}] = [\gamma^{q}_{s}]\) y por tanto
  \(\gamma^{q}\) es la curva integral con origen en \(q = \gamma(t)\) en el
  intervalo \((-\varepsilon, \varepsilon)\) con \(\varepsilon\) adecuado para
  que \((t - \varepsilon, t + \varepsilon) \subseteq J\).
\end{proof}

En el caso de campos \(f\)-relacionados o
\hyperref[def:f-comp]{\(h\)-compatibles}
por una aplicación diferenciable \(h \colon M \to N\) tenemos

\begin{lemma}
  Si \(Y\) es \hyperref[def:f-comp]{\(h\)-compatible} con \(X\) y \(\gamma\) es
  la curva integral de \(X\) en \(p\), entonces \(h \circ \gamma\) es la curva
  integral de \(Y\) en \(h(p)\).
\end{lemma}

\begin{proof}
  En efecto,

  \begin{center}
    \begin{tikzcd}
      M \ar[d, "X"'] \ar[r, "h"] & N \ar[d, "Y"]\\
      \Tangente{M} \ar[r, "\mathrm{d} h"'] & \Tangente{N}
    \end{tikzcd}
  \end{center}

  Tenemos \(Y(h(p)) = \mathrm{d} h X(p)\), por lo que \(Y(h(\gamma(t))) =
  \mathrm{d} h X(\gamma(t)) = \mathrm{d} h [\gamma_{t}] = [h \circ \gamma_{t}] =
  [ (h \circ \gamma)_{t}]\).

  Por tanto si \(h \colon X \to X\) y \(X\) es un campo
  \hyperref[def:f-invariante]{\(h\)-invariante}, tenemos que \(\mathrm{d} h X =
  X h\), por lo que la curva integral de \(X\) en \(h(p)\) es \(h \circ \gamma\)
  si \(\gamma\) es la curva integral en \(p\).

  En particular si \(h\) es un difeomorfismo y \(\gamma\) es la curva integral
  de \(X\) en \(p\), entonces \(h \circ \gamma\) es la curva integral del campo
  \(h * X = \mathrm{d} h X h^{-1}\) en \(h(p)\)
\end{proof}

\begin{definition}[name=flujo]
  \label{def:flujo}
  Se llama \emph{flujo} o \emph{sistema dinámico} sobre una variedad \(M\) a una
  aplicación diferenciable \(\Phi \colon \RealSet \times M \to M\) cumpliendo
  \(\Phi(0,p) = p\) y \(\Phi(s+t, p) = \Phi(s, \Phi(t, p))\) para todo \(p \in
  M\) y \(t, s \in \RealSet\).
\end{definition}

Si fijamos \(\Phi(0,p) = p\), \(\Phi_{t} \colon M \to M\) cumple \(\Phi_{0} =
\Id{M}\) y \(\Phi_{t} \circ \Phi_{s} = \Phi_{t+s}\). En particular \(\Phi_{t}\)
es difeomorfismo con inversa \(\Phi_{-t}\).

\begin{definition}[name={línea del flujo}]
\label{def:linea-flujo}
  En las condiciones anteriores, fijado \(p \in M\), \(\Phi_{p} \colon \RealSet \to M\) se
convierte en una curva en \(M\) con origen en \(x\). A \(\Phi_{p}\) se le llama
\emph{línea del flujo} \(\Phi\) en \(p\).
\end{definition}

\begin{definition}[name=órbita]
\label{def:orbita}
A la imagen \(\Phi_{p}(\RealSet)\) se le llama \emph{órbita} de \(p\) para el flujo \(\Phi\).
\end{definition}

\begin{lemma}
  Se define la relación de equivalencia \(p \sim q\) sii existe \(t \in
  \RealSet\) con \(\Phi(t,p) = q\).

  Esta relación es dde equivalencia y la clase de \(p \in M\) es justamente su
  órbita \(\Phi_{p}(\RealSet)\). Así pues las órbitas forman una partición de \(M\).
\end{lemma}

\begin{proof}\item
  \begin{subproof}[Reflexiva]
    \(\Phi(0,p) = p\).
  \end{subproof}
  
  \begin{subproof}[Simétrica]
    Si \(q = \Phi(t,p) = \Phi_{t}(p)\), entonces \(p = \Phi_{t}^{-1}(q) =
    \Phi_{-t}(q) = \Phi(-t, q)\).
  \end{subproof}

  \begin{subproof}[Transitiva]
    Si \(q = \Phi(t,p)\), \(q' = \Phi(t', q)\), entonces
    \[q' = \Phi(t', \Phi(t, p)) = \Phi(t'+t, p).\]
  \end{subproof}
\end{proof}

\begin{note}
  Nótese que si un campo \(X\) es nulo sobre una curva integral \(\gamma(t)\)
  entonces esta es nula. En efecto, \(0 = X(\gamma(t)) = \Clase{\gamma_{t}}\) para
  todo \(t\). Entonces para toda carta \((U, \varphi)\) con \(\gamma(t) \in U\),
  si \(\gamma \colon (-\varepsilon, \varepsilon) \to U\), tenemos que
  \[
    \widehat{\varphi} X \gamma(s) = \left( \varphi \gamma(s), (\varphi \circ
    \gamma_{s})'(0) \right) = \left( \varphi \gamma(s), (\varphi \gamma)'(s)
  \right)
  = \left( \varphi \gamma(s), 0 \right).
\]

Luego \(\varphi \gamma\) es constante en \((-\varepsilon, \varepsilon)\) así que
\(\gamma\) es constante en \((-\varepsilon, \varepsilon)\) y por tanto todo su
intervalo de definición.
\end{note}

Todo flujo sobre \(M\) determina de manera natural un campo tangente sobre
\(M\).

\begin{proposition}
  Dado un flujo \(\Phi \colon \RealSet \to M\), \(X_{\Phi} \colon M \to
  \Tangente{M}\) que asigna a \(p \in M\) el vector \(X_{\Phi}(p)\) representado
  por la curva \(\Phi_{p} \colon \RealSet \to M\) es un campo tangente sobre
  \(M\) llamado el \emph{campo de velocidades} del flujo \(\Phi\).

  Además \(X_{\Phi}\) es un campo \(\Phi_{t}\)-invariante para todo \(t \in
  \RealSet\) y \(\Phi_{p}\) es la curva integral de \(X_{\Phi}\) en \(p\).
\end{proposition}

\begin{proof}
  Para ver que es \(\Phi_{t}\)-invariante hay que ver que el siguiente diagrama conmuta.
  \begin{center}
    \begin{tikzcd}
      M \ar[d, "X_{\Phi}"'] \ar[r, "\Phi_{t}"] & M \ar[d, "X_{\Phi}"]\\
      \Tangente{M} \ar[r, "\mathrm{d} \Phi_{t}"'] & \Tangente{M}
    \end{tikzcd}
  \end{center}

  En efecto, \(X_{\Phi} \left( \Phi_{t}(p) \right)\) % \(= X_{\Phi} \left(
  % \Phi_{t}(p) \right)\) % (???)
  es el vector representado por la curva
  \[
    \Phi_{\Phi_{t}(p)}(s) = \Phi \left( s, \Phi_{t}(p) \right)
    = \Phi \left( s, \Phi(t,p) \right)
    = \Phi \left( s+t, p \right).
  \]

  Análogamente,
  \[
    \dif{\Phi_{t} X_{\Phi}(p)} = \dif{\Phi_{t} \Clase{\Phi_{p}}} =
    \Clase{\Phi_{t} \circ \Phi_{p}}
  \]
  pero
  \[
    \Phi_{t} \circ \Phi_{p}(s) = \Phi_{t} \left( \Phi(s,p) \right)
    = \Phi \left( t, \Phi(s, p) \right)
    = \Phi (t+s, p)
  \]

  Observar que se ha usado la continuidad de la norma de \(\RealSet\).

  Por otro lado, la curva integral de \(X_{\Phi}\) en \(p\) será una curva
  \(\gamma\) tal que \(X_{\Phi} \left( \gamma(t) \right) = \Clase{\gamma_{t}}\)
  con \(\gamma_{t}(s) = \gamma(s+t)\). Si tomamos \(\gamma = \Phi_{p}\), tenemos
  \[
    \left( \Phi_{p} \right)_{t}(s)
    = \Phi_{p}(s+t)
    = \Phi(s+t, p)
    = \Phi \left(s, \Phi(t,p) \right)
  \]

  Por otro lado \(X_{\Phi}(\Phi_{p}(t))\) está representada por la curva
  \(\Phi_{\Phi_{p}(t)}(s) = \Phi(s, \Phi_{p}(t)) = \Phi(s, \Phi(t, p)) = \left(
    \Phi_{p} \right)_{t}(s)\).
  Es decir, \(X_{\Phi} \left( \Phi_{p}(t) \right) = \Clase{\left( \Phi_{p}
    \right)_{t}}\) y \(\Phi_{p}\) es la curva integral de \(X_{\Phi}\) en \(p\).

  Queda verificar la diferenciabilidad de \(X_{\Phi}\).
  Por ser \(\Phi\) diferenciable, encontramos cartas \((U, \varphi)\) y \((W,
  \psi)\) de \(p\) y \(\varepsilon > 0\) tales que \(f = \varphi \circ \Phi
  \circ \left( \Id{} \times \psi \right)^{-1}\) es diferenciable en el diagrama

  \begin{center}
    \begin{tikzcd}
      (-\varepsilon, \varepsilon) \times W
      \ar[r, "\Phi"]
      \ar[d, "\Id{} \times \psi"']
      & U
      \ar[d, "\varphi"]\\
      (-\varepsilon, \varepsilon) \times \psi(W)
      \ar[r, dashed]
      & \varphi(U)
    \end{tikzcd}
  \end{center}

  Obsérvese que \(f(t, x) = \varphi \circ \Phi_{\psi^{-1}(x)}(t)\).

  Es inmediato comprobar que \(X_{\Phi}(W) \subseteq \Tangente{U}\):
  ya que si \(q \in W\), \(X_{\Phi}(q) = \Clase{\Phi_{q}} =
  \Clase{\overline{\Phi_{q}}}\) para \(\overline{{\Phi_{q}}} =
  \Restrict{\Phi_{q}}{(-\varepsilon, \varepsilon)}\) y para \(\Abs{t} <
  \varepsilon\) \(\overline{\Phi_{q}}(t) = \Phi(q, t) \in U\).

  Tenemos así el diagrama

  \begin{center}
    \begin{tikzcd}
      W
      \ar[r, "X_{\Phi}"]
      \ar[d, "\psi"']
      & \Tangente{U}
      \ar[d, "\widetilde{\varphi}"]\\
      \psi(W)
      \ar[r, dashed]
      & \varphi(U) \times \RealSet^{m}
    \end{tikzcd}
  \end{center}

  Entonces para \(g = \widetilde{\varphi} \circ X_{\Phi} \circ \psi^{-1}\)
  tenemos
  \begin{align*}
    g(x)
    &= \widetilde{\varphi} \left( \Clase{\Phi_{\psi^{-1}(x)}} \right)\\
    &= \widetilde{\varphi} \left( \Clase{\overline{\Phi}_{\psi^{-1}(x)}} \right)\\
    &= \left( \varphi \circ \overline{\Phi}_{\psi^{-1}(x)}(0), \left( \varphi
        \circ \overline{\Phi}_{\psi^{-1}(x)} \right)'(0) \right)\\
    &= \left( \varphi \circ \psi^{-1}(x), \Restrict{\D{f(x,t)}{t}}{t=0} \right)
  \end{align*}

  Siendo ambas funciones diferenciables en \(x\).
\end{proof}

A continuación veremos que hay tres tipos de órbitas de cualquier flujo.

\begin{proposition}
  Sea \(\Phi \colon \RealSet \times M \to M\) un flujo en \(M\). Para \(p \in
  M\), si \(\Phi_{p} \colon \RealSet \to M\) es la línea de flujo de \(\Phi\)
  en \(p\), se cumple una de las siguientes condiciones:
  \begin{enumerate}
  \item \(\Phi_{p}\) es constante.
  \item \(\Phi_{p}\) es una inmersión inyectiva.
  \item \(\Phi_{p}\) es una inmersión periódica, i.e. existe \(t_{0} \in
    \RealSet\) (el periodo) tal que \(\Phi_{p}(s) = \Phi_{p}(t)\) si y solo si
    existe un entero \(k \in \IntegerSet\) con \(s - t = k t_{0}\).
  \end{enumerate}
\end{proposition}

\begin{proof}
  Como \(\Phi_{t}\) es difeomorfismo para todo \(t\), \(\dif{\Phi_{t}} \colon
  \Tangente[p]{M} \to \Tangente[\Phi_{t}(p)]{}\) es isomorfismo y tenemos
  \(\dif{\Phi_{t}}(X_{\Phi}(p)) = \Clase{\Phi_{t} \circ \Phi_{p}}\), donde
  \(\Phi_{t} \Phi_{p}(s) = \left( \Phi_{p}(s) \right)_{t}\) por lo que
  \(\Clase{\Phi_{t} \circ \Phi_{p}} = X_{\Phi} \left( \Phi_{p}(t) \right)\).
  Así pues si el campo \(X_{\Phi}\) es nulo para algún punto \(\Phi_{p}(t)\),
  entonces \(\Phi_{\Phi}(p) = 0\) y cambiando arriba \(t\) por \(t'\), lo es en
  todos los puntos de la curva \(\Phi_{p}\). Por ser \(\Phi_{p}\) la curva
  integral del campo \(X_{\Phi}\) en \(p\), entonces esa curva sería constante.

  En caso contrario \(X_{\Phi}\) no es nulo en ningún punto de la curva
  \(\Phi_{p}\) y, en particular, \(\Clase{\Phi_{p}} = X_{\Phi} \left(
    \Phi_{p}(0) \right) \neq 0\), y por tanto \(\dif{\Phi_{p}} \colon
  \Tangente[0]{\RealSet} \to \Tangente[p]{M}\) lleva la clase de la curva
  \(\varepsilon(t) = t\) en la clase de \(\Clase{\Phi_{p}} \neq 0\).
  Así pues, \(\dif{\Phi_{p}}\) es inyectiva y \(\Phi_{p}\) es una inmersión.

  Queda ver qué ocurre si \(\Phi_{p}\) no es inyectiva. Sean \(s < s'\) con
  \(\Phi_{p}(s) = \Phi(s, p) = \Phi(s', p) = \Phi_{p}(s')\). Entonces
  \[
    \Phi_{p}(s'-s) = \Phi(s'-s, p) = \Phi \left( -s, \Phi(s', p) \right) = \Phi
    \left( -s, \Phi(s, p) \right) = \Phi(0, p) = p
  \]

  Además, de la continuidad de \(\Phi_{p}\) se sigue que \(\Phi_{p}^{-1}(p)
  \subseteq \RealSet\) es cerrado y contiene a \(0\) y a \(s'-s > 0\).

  Como \(\Phi_{p}\) es una inmersión, es localmente inyectiva alrededor del
  \(0\) por el teorema del valor medio aplicado a las composiciones \(\pi_{i}
  \circ \varphi \circ \Phi_{p} \colon (-\delta, \delta) \to U \to \varphi(U) \to
  \RealSet\), donde \((U, \varphi)\) es una carta con \(p \in U\).

  Entonces, existe \(\varepsilon > 0\) tal que \(\Phi_{p}^{-1}(p) \cap
  (-\varepsilon, \varepsilon) = \Set{0}\).

  Sea el conjunto \(S = \Set{t > 0 : \Phi_{p}(t) = p}\). Como para \(0 < t <
  \varepsilon\), \(\Phi_{p}(t) \neq 0\), \(S \subseteq \Set{t \in \RealSet, t
    \geq \varepsilon}\) y por tanto \(S = \Phi_{p}^{-1}(p) \cap [0, +\infty)\)
  es cerrado y acotado inferiormente.
  Por tanto, existe un mínimo \(t_{0} \in S\). Tenemos \(\Phi_{p}(t_{0}) = p\) y
  para todo entero \(k \geq 0\) tenemos \(\Phi_{p}(t + k t_{0}) = \Phi_{p}(t +
  t_{0} + \dotsb + t_{0}) = \Phi \left( t, \Phi(t_{0} + \dotsb + t_{0}, p)
  \right) = \Phi(t, p) = \Phi_{p}(t)\). Igual si \(k \leq 0\).

  Ahora si \(t - t' \neq k t_{0}\) para todo entero \(k \in \IntegerSet\) y
  \(\Phi_{p}(t) = \Phi_{p}(t')\), sea \(k_{0}\) el entero con \((k_{0}-1) t_{0}
  < t - t' < k_{0} t_{0}\).

  Entonces
  \begin{align*}
    \Phi_{p} \left( k_{0}t_{0} - (t-t') \right)
    &= \Phi \left(k_{0} t_{0} - t + t', p \right)\\
    &= \Phi \left( t' - t + (k_{0}-1) t_{0}, \Phi(t_{0}, p) \right)\\
    &= \Phi \left( t' - t + (k_{0}-1) t_{0}, p \right)\\
    &= \dots =\\
    &= \Phi(t-t', p)\\
    &= \Phi \left( -t', \Phi(t, p) \right)\\
    &= \Phi \left( -t', \Phi(t', p) \right)\\
    &= \Phi_{p}(0) = p
  \end{align*}

  Pero \(0 < k_{0} t_{0} - (t-t') < t_{0}\), lo que nos da una contradicción con
  la minimalidad de \(t_{0}\).
\end{proof}

\begin{note}
  En el tercer caso de la proposición anterior, las órbitas son difeomorfas a
  \(S^{1}\) pues \(\Phi_{p} \colon \RealSet \to M\) factoriza a
  \(\widetilde{\Phi}_{p} \colon S^{1} \to M\) por la aplicación \(\RealSet \to
  S^{1}\), \(t \mapsto e^{\sfrac{2 \pi i t}{t_{0}}}\).
\end{note}

\begin{example}
  La norma en \(\RealSet\) nos da un flujo canónico \(L \colon \RealSet \times
  \RealSet \to \RealSet\), \(L(x,y) = x+y\), cuyo campo asociado es \(X_{L}
  \colon \RealSet \to \Tangente{\RealSet} = \RealSet \times \RealSet\),
  \(X_{L}(t) = \Clase{L_{t}}\), \(L_{t}(s) = t+s\).

  Además, si \(X\) es un campo y \(\gamma\) una curva integral del mismo,
  tenemos
  \[
    X \left( \gamma(t) \right) = \Clase{\gamma_{t}} = \dif{\gamma} \Clase{L_{t}}
  \]

  El campo \(L\) solo tiene una órbita.
\end{example}

\begin{example}
  Sea
  \begin{align*}
    \Phi \colon \RealSet \times \RealSet^{2}
    &\to \RealSet^{2}\\
    \left( t,
    \begin{pmatrix}
      p \\ q
    \end{pmatrix}
    \right)
    &\mapsto e^{-\sfrac{t}{2}}
      \begin{pmatrix}
        \cos t & \sin t\\
        -\sin t & \cos t
      \end{pmatrix}
      \begin{pmatrix}
        p \\ q
      \end{pmatrix}
  \end{align*}

  Es un flujo pues
  \begin{align*}
    \Phi \left( 0,
    \begin{pmatrix}
      p \\ q
    \end{pmatrix} \right)
    &=
      \begin{pmatrix}
        p \\ q
      \end{pmatrix}\\
    \Phi \left( t + s,
    \begin{pmatrix}
      p \\ q
    \end{pmatrix} \right)
    &=
      e^{\left( - \sfrac{t}{2} - \sfrac{s}{2} \right)}
      \begin{pmatrix}
        \cos (t+s) & \sin (t+s)\\
        -\sin (t+s) & \cos (t+s)
      \end{pmatrix}
      \begin{pmatrix}
        p \\ q
      \end{pmatrix} =\\
    &=
      e^{-\sfrac{t}{2}}
      \begin{pmatrix}
        \cos t & \sin t\\
        -\sin t & \cos t
      \end{pmatrix}
      e^{-\sfrac{s}{2}}
      \begin{pmatrix}
        \cos s & \sin s\\
        -\sin s & \cos s
      \end{pmatrix}
      \begin{pmatrix}
        p \\ q
      \end{pmatrix}\\
    &= \Phi \left( t, \Phi \left( s,
      \begin{pmatrix}
        p \\ q
      \end{pmatrix}
\right) \right)
  \end{align*}

  Aquí se usan las fórmulas del seno y coseno de la suma de ángulos o el hecho
  de que la matriz cuadrada representa el giro de ángulo \(t\) o \(s\) del
  plano.

  En este flujo la órbita del origen se reduce al punto y las órbitas de los
  demás puntos son de tipo espiral pues cuando \(t \to \infty\) se va acercando
  al origen y cuando \(t \to -\infty\) se va girando hacia el infinito.
\end{example}

\begin{example}
  Si en el caso anterior eliminamos el coeficiente exponencial, entonces el
  flujo resultante tiene todas sus órbitas circunferencias salvo el origen que
  se reduce a un punto.
\end{example}

Ahora se plantea el problema de determinar si todo campo tangente \(X \colon M
\to \Tangente{M}\) da lugar a un flujo.

Si todas las curvas integrales de \(X\) estuvieran definidas en todo
\(\RealSet\) entonces \(\Phi_{X} \colon \RealSet \times M \to M\) dada por
\(\Phi_{X}(t, p) = \gamma_{p}(t)\) donde \(\gamma_{p}\) es la curva integral de
\(X\) con origen en \(p\) sería una buena opción.

Desafortunadamente no todas las curvas integrales están definidas en todo
\(\RealSet\) y por ello se hace necesario definir un flujo local:

\begin{definition}[name={flujo local}, label={def:flujo-local}]
  Definimos un \emph{flujo local} como una aplicación diferenciable \(\Phi \colon U \to M\),
  donde \(U\) es un abierto de \(\RealSet \times M\) con \(\Set{0} \times M
  \subseteq U\) y cada intersección \(\RealSet \times \Set{0} \cap U\) es un
  intervalo abierto \((a_{p}, b_{p})\) conteniendo al \(0\).
\end{definition}
Ahora se exigen las propiedades
\begin{enumerate}
\item \(\Phi(0,p) = p\) para todo \(p\).
\item \(\Phi \left( t, \Phi(s,p) \right) = \Phi(t+s, p)\) siempre que tenga
  sentido, i.e. \((t+s, p) \in U\), \(\left( t, \Phi(s,p) \right) \in U\) y
  \((s,p) \in U\).
\end{enumerate}


\begin{definition}[name={flujo local maximal}]
  Un \nameref{def:flujo-local} se dice \emph{maximal} si no existe otro flujo
  \(\psi \colon V \to M\) con \(U \subseteq V\), \(U \neq V\) y
  \(\Restrict{\psi}{U} = \Phi\).
\end{definition}

El siguiente teorema da la equivalencia entre flujos y campos tangentes sobre
una variedad \(M\). Una demostración se puede encontrar en el tema 9. % TODO [REF Lee 9.12]

\begin{theorem}[name={equivalencia entre campos tangentes y flujos},label={thm:campos-tangentes-flujos}]
  Existe una biyección entre campos tangentes y flujos maximales sobre \(M\).

  Si \(\Phi\) es un flujo, el campo asociado \(X_{\Phi}\) tiene a las curvas
  \(\Phi_{p}\) por curvas integrales.

  Recíprocamente, si \(X \colon M \to \Tangente{M}\) es un campo, el flujo
  asociado \(\Phi^{X} \colon U \to M\) es tal que \(\Phi^{X}_{p}\) es la curva
  integral maximal con origen en \(p\).
\end{theorem}

\begin{definition}[name={completo}, label={def:campo-tangente-completo}]
  Un campo tangente se dice \emph{completo} si todas sus curvas integrales
  maximales están definidas en todo \(\RealSet\).
\end{definition}

Si un campo tangente es completo, el teorema anterior da el siguiente caso
particular:

\begin{theorem}
  Existe una biyección entre campos tangentes completos y flujos sobre la
  variedad \(M\).
\end{theorem}

\begin{proposition}
  \label{prop:compacta-completo}
  Si \(M\) es compacta, todo campo sobre \(M\) es completo.
\end{proposition}

La demostración es consecuencia del siguiente lema:

\begin{lemma}
  Sea \(X \colon M \to \Tangente{M}\) un campo tangente tal que existe
  \(\varepsilon > 0\) tal que para toda curva integral \(\gamma\) de \(X\),
  \((-\varepsilon, \varepsilon)\) está en el dominio de definición de
  \(\gamma\). Entonces \(X\) es un campo completo.
\end{lemma}

\begin{proof}
  Supongamos que para \(p \in M\), la curva integral con origen en \(p\),
  \(\gamma^{p} \colon J \to M\) tiene un dominio acotado superiormente, i.e.
  existe \(b = \sup J \in \RealSet\).

  Sea \(t_{0}\) con \(b - \varepsilon < t_{0} < b\). Nótese que \((-\varepsilon,
  \varepsilon) \subseteq J\). Sea \(q = \gamma^{p}(t_{0})\). Entoncess, si
  \(\gamma^{q}\) es la curva integral con origen en \(q\), \(\gamma^{q}\) está
  definida en \((-\varepsilon, \varepsilon)\) por hipótesis y tiene sentido
  \(\gamma^{q}(t-t_{0})\) si \(t \in (t_{0} - \varepsilon, t_{0} +
  \varepsilon)\).

  Además por el \hyperref[lem:ci-traslacion]{lema de traslación}
  (\cref{lem:ci-traslacion}),
  \(\gamma^{q}(t-t_{0}) = \gamma^{p}(t)\) ya que \(q = \gamma^{p}(t_{0})\).
  Por tanto \(\widehat{\gamma} \colon J \cup (t_{0} - \varepsilon, t_{0} +
  \varepsilon) \to M\) dada por \(\widehat{\gamma}(t) = \gamma^{p}(t)\) si \(t
  \in J\) y \(\widehat{\gamma}(t) = \gamma^{q}(t-t_{0})\) si \(\Abs{t-t_{0}} <
  \varepsilon\) es una curva diferenciable para la que
  \[
    X \left(\widehat{\gamma}(t) \right) =
    \begin{cases}
      X \left( \gamma^{p}(t) \right) = \Clase{\gamma^{p}_{t}} &\text{si } t \in J\\
      X \left( \gamma^{q}(t) \right) = \Clase{\gamma^{q}_{t}} &\text{si }
      \Abs{t-t_{0}} < \varepsilon
    \end{cases}
  \]
  y para \(s\) con \(\Abs{s} < \delta\) de forma que \(t+s \in J\) si \(t \in
  J\) y \(t+s \in (t_{0}-\varepsilon, t_{0}+\varepsilon)\) si \(\Abs{t-t_{0}} <
  \varepsilon\), tenemos que \(\gamma^{p}_{t}(s) = \gamma^{p}(s+t) =
  \widehat{\gamma}_{t}(s)\) y \(\gamma^{q}_{t}(s) = \gamma^{q}(s+t) =
  \widehat{\gamma}_{t}(s)\).

  Por tanto \(\widehat{\gamma}\) es curva integral de \(X\) con origen en
  \(\widehat{\gamma}(0) = \gamma^{p}(0) = p\) y definida en puntos como \(t_{0}
  + \frac{\varepsilon}{2}\) fuera de \(J\). Esto contradice la maximalidad.
\end{proof}

\begin{proof}[Prueba de la \cref{prop:compacta-completo}]
  Dado \(p \in M\) y una carta \((U_{p}, \varphi_{p})\) con \(p \in U\), sabemos
  por la ecuación diferencial que define a las curvas integrales que existe un
  abierto \(\Omega_{p}\) y otro \(J_{p}\) de \(0\) y una función \(\xi_{p}
  \colon \Omega_{p} \times J_{p} \to \varphi_{p}(U_{p})\) tal que
  \(\gamma^{q}(t) = \xi_{p} \left( \varphi(q), t \right)\) es la curva integral
  de \(X\) en el intervalo \(J_{p}\).

  Ahora por compacidad podemos encontrar una cantidad finita \(\Omega_{p_{1}},
  \dots, \Omega_{p_{k}}\) con \(M = \bigcup_{j=1}^{k} \Omega_{p_{j}}\). Sea
  \(\varepsilon = \max \Set{\varepsilon_{p_{1}}, \dots, \varepsilon_{p_{k}}}\)
  con \((-\varepsilon_{p_{j}}, \varepsilon_{p_{j}}) \subseteq J_{p_{j}}\).
  De esta manera para todo \(q \in M\), \(\gamma^{q}\) está definida en el
  intervalo \((-\varepsilon, \varepsilon)\) y se puede aplicar el lema anterior
  para obtener que todas las curvas \(\gamma^{q}\) están definidas en todo \(\RealSet\).
\end{proof}

\begin{example}
  Sea el campo sobre \(\RealSet\) \(V = x^{2} \pderiv{}{x}\). Entonces para
  saber la curva integral \(\phi\) en el punto \(x_{0}\) debemos resolver la
  ecuación diferencial \(\gamma(t)^{2} = \gamma'(t)\) con \(\gamma(0) = x_{0}\).
  Entonces llegamos a \(\frac{-1}{\gamma(t)} = t + k\), de donde \(\gamma(t) =
  \frac{-1}{t+k}\) y para \(\gamma(0) = x_{0}\), \(k = - \frac{1}{x_{0}}\).

  Así pues \(\gamma(t) = \frac{-1}{t - \frac{1}{x_{0}}} = \frac{x_{0}}{1-t}\).

  Por tanto las curvas no están definidas en todo \(\RealSet\) sino solo en
  \((-\infty, 1)\).
\end{example}

\begin{example}
  Para \(M = \GL{n}{\RealSet}\) hemos visto que los campos invariantes(?) a la
  izquierda pueden ser identificados con las matrices de
  \(\Tangente[I_{n}]{\GL{n}{\RealSet}} = \Matrices{n}{n}{\RealSet}\) el campo
  \(f_{A} \colon \GL{n}{\RealSet} \to \Tangente{\GL{n}{\RealSet}}\) con
  \(f_{A}(I_{n}) = A\) está dado por el producto \(f_{A}(X) = X A\). Por tanto
  la curva integral en \(X\) será una curva \(\gamma^{X} \colon (-\varepsilon,
  \varepsilon) \to \GL{n}{\RealSet}\) tal que \(\gamma^{X}(t) A = \left(
    \gamma^{X} \right)'(t)\) pues solo tenemos una carta.

  Si derivamos de nuevo \(\gamma = \gamma^{X}\), tenemos \(\gamma''(t) =
  \gamma'(t) A = \gamma(t) A^{2}\) (pues \(A\) está fija). Reiterando las
  derivadas llegamos a
  \begin{equation}
    \label{eq:derivada-gamma-gl}
    \gamma^{n)}(t) = \gamma(t) A^{n}
  \end{equation}

  Además, \(\gamma'(0) = I_{n}\) por lo que \(\gamma^{n)}(0) = A^{n}\). Si
  escribimos \(a_{ij}\) para las entradas de \(A\) y \(a_{nij}\) para las de
  \(A^{n}\), de \eqref{eq:derivada-gamma-gl} llegamos a la siguiente relación
  para las entradas de \(b_{ij}(t)\) de \(\gamma(t)\):
  \[
    b_{ij}^{n)}(t) = \Sum{b_{ik}(t) a_{nkj}}{k,1,n}
  \]
  con \(b_{ij}^{n)}(0) = a_{nij}\). El desarrollo de Taylor alrededor del origen
  de \(b_{ij}(t)\) nos da
  \[
    b_{ij}(t) = \Sum{\frac{b_{ij}^{n)}(0)}{n!} t^{n}}{n,0,\infty}
    = \Sum{\frac{a_{nij}}{n!} t^{n}}{n,0,\infty}
  \]

  Esto es, matricialmente,
  \[
    \gamma(t) = \Sum{\frac{A^{n} \cdot t^{n}}{n!}}{n,0,\infty} = \Sum{\frac{(At)^{n}}{n!}}{n,0,\infty}
  \]

  El término de la derecha se escribe de manera abreviada \(e^{At}\) por su
  similitud con la función exponencial clásica.

  Tenemos así que \(\gamma^{A}(t) = e^{At}\) es la curva exponencial con origen
  en \(I_{n}\) del campo invariante %(a i+giente(???))
  asociado a \(A\).

  Si \(G \subseteq \GL{n}{\RealSet}\) es un subgrupo de Lie de matrices tenemos
  la misma construcción pues sólo varía que la matriz \(A\) está en el
  subespacio \(\Tangente[I]{G} \subseteq \Tangente[I]{\GL{n}{\RealSet}} =
  \Matrices{n}{n}{\RealSet}\).

  Nótese que como \(\gamma^{A}\) es una curva en \(\GL{n}{\RealSet}\) (o más
  generalmente en \(G\)), \(\gamma^{A}(1) = e^{A}\) es una matriz de
  \(\GL{n}{\RealSet}\) (generalmente \(G\)).
  Tenemoss así una aplicación llamada \emph{exponencial}
  \(\mathcal{G} \to G\) del
  álgebra de Lie al grupo de Lie.
\end{example}

\end{document}
%%% Local Variables:
%%% TeX-master: "../VD_ebook"
%%% End:
