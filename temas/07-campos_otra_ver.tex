\documentclass[../VD.tex]{subfiles}

\externaldocument{../VD}

\begin{document}

\setcounter{chapter}{6}
\chapter{Versión alternativa de campos tangentes}\label{chap:campos2}

\section{Introducción}

Pasaremos ahora a estudiar una versión alternativa y más operativa de la definición
de campos tangentes y sus propiedades.

Consideraremos los conjuntos \(\mathcal{F}(M)=\{f\colon M\to\RealSet
\ \text{diferenciables}\}\) y \(\mathcal{F}(U)=\{f\colon U\to\RealSet
\ \text{diferenciables}\}\) con \(U\subseteq M\) abierto.

\begin{definition}[name=\(\mathcal{F}(p)\)]\label{def:F(p)}
  Al conjunto de las \emph{curvas diferenciables alrededor de \(p\in M\)} lo
  notaremos \[\mathcal{F}(p)=\{f\colon U\to\RealSet \ \text{diferenciables con}\ 
  p\in U\subseteq M \ \text{abierto}\}\]
\end{definition}

\begin{definition}
  Sean \(f\colon U\to\RealSet,g\colon V\to\RealSet\) ambas funciones en
  \(\mathcal{F}(p)\) para un cierto \(p\in M\). Diremos que \(f\) y 
  \(g\) son \emph{equivalentes en \(p\)}, y lo notarenos \(f \sim_{p} g\), si
  \(\exists\Omega\subseteq U\cap V\) abierto con \(p\in\Omega\) donde \(f,g\)
  están definidas y además \(\Restrict{f}{\Omega}=\Restrict{g}{\Omega}\).

  \vline

  A cada clase de equivalencia \(\Clase{f}\) se le llama \emph{germen en \(p\)}.
  Al conjunto cociente de los gérmenes en \(p\) lo notaremos
  \(\frac{\mathcal{F}(p)}{\sim_{p}}=\colon\mathcal{G}(p)\). 
\end{definition}

\begin{lemma}\label{lem:G(p)ev}
  \(\mathcal{G}(p)\) es un espacio vectorial, de hecho, es un álgebra (tiene un
  producto).
\end{lemma}

\begin{proof}
  Sean \(f\colon U\to\RealSet\) y \(g\colon V\to\RealSet\), \(p\in
          U,V\) abiertos, \(\lambda\in\RealSet\):
  \begin{itemize}
    \item \(\Clase{f}+\Clase{g}=\Clase{\Restrict{f}{U\cap V}+\Restrict{g}{U\cap
          V}}\)
    \item \(\lambda\Clase{f}=\Clase{\lambda f}\)
    \item Elemento neutro: \(\theta_{p}\in\mathcal{G}(p), \
      \theta_{p}\equiv cte\ 0\)
    \item Producto: \(\Clase{f}\cdot\Clase{g}=\Clase{\Restrict{f}{U\cap
          V}\cdot\Restrict{g}{U\cap V}}\)
  \end{itemize}
\end{proof}

\section{Derivaciones como aplicaciones}

\begin{definition}
  Una \emph{derivación en \(p\)} es una aplicación lineal
  \(\delta\colon\mathcal{G}(p)\to\RealSet\) que cumple la \textit{regla de
    Leibniz} de las derivadas:
  \[
    \delta(\Clase{f}\cdot\Clase{g})=\delta(\Clase{f})\cdot
    g(p)+f(p)\cdot\delta(\Clase{g}) 
  \]

  A la familia de derivaciones en \(p\) la notaremos \(\mathcal{D}(p)\).
\end{definition}

\begin{remark}
  \(\delta(\theta_{p})=0\), es más, si \(c_{t}\colon
  M\to\RealSet\ cte\ t\), entonces \(\delta(\Clase{c_{t}})=0\):
  \[\begin{array}{ccc}
      \delta(\Clase{c_{t}}\Clase{f}) & =\delta(\Clase{c_{t}f})=\delta(\Clase{tf})
      \overset{\text{\(\delta\) lineal}}{=} & t\delta(\Clase{f}) \\
      \delta(\Clase{c_{t}}\Clase{f}) & \overset{\text{Leibniz}}{=}
      \delta(\Clase{c_{t}})f(p)+t\delta(\Clase{f}) &
      \end {array}\]

    Luego, \(\forall\Clase{f}\in\mathcal{G}(p)\)
    \[
      0=\delta(\Clase{c_{t}})f(p)\Rightarrow\delta(\Clase{c_{t}})=0.
    \]
\end{remark}

\begin{remark}
  Todo germen de \(\mathcal{G}(p)\) se puede representar por una función
  \(f\colon M\to\RealSet\). Esto es por el \nameref{lem:extension} de funciones
  escalares: \(\Clase{g}\in\mathcal{G}(p)\) con \(g\colon U\to\RealSet\) con
  \(p\in U\) abierto, entonces existen \(V\subseteq U\) abierto con \(p\in V\)
  y \(\widetilde{g}\colon V\to\RealSet\) con \(\widetilde{g}=g\) en \(V\), luego
  \(\Clase{\widetilde{g}}=\Clase{g}\).
\end{remark}

\begin{example}
  Dada \(\Clase{\alpha}\in\Tangente[p]{M}\)
  (\(\alpha\colon(-\varepsilon,\varepsilon)\to M, \, \alpha(0)=p\)), entonces
  definimos:
  \[\begin{array}{ll}
    \delta_{\Clase{\alpha}}\colon & \mathcal{G}(p)\to\RealSet \\
    & \Clase{f}\mapsto\delta_{\Clase{\alpha}}(\Clase{f})=(f\circ\alpha)'(0)
    \end{array}\]
  Tomaremos \(\varepsilon\) pequeño para que
  \(\alpha((-\varepsilon,\varepsilon))\subseteq U\Rightarrow
  (-\varepsilon,\varepsilon)\overset{\alpha}{\to}U\overset{f}{\to}\RealSet\)

  \vline

  Además, \(\delta_{\Clase\alpha}\) es lineal trivialmente. Veamos la regla de
  Leibniz:
  \[\begin{array}{l}
      \delta_{\Clase\alpha}(\Clase{f}\Clase{g})=\delta_{\Clase\alpha}(\Clase{fg})=\\
      ((fg)\circ\alpha)'(0)=((f\circ\alpha)\cdot(g\circ\alpha))'(0)=\\
      ((f\circ\alpha)'(0)(g\circ\alpha)(0)+(f\circ\alpha)(0)(g\circ\alpha)'(0)
      \overset{\alpha(0)=p}{=} \\ \delta_{\Clase\alpha}(\Clase{f})g(p)+f(p)
      \delta_{\Clase\alpha}(\Clase{g})
  \end{array}\]
\end{example}

\section{Isomorfía entre el espacio tangente en \(p\) y el de derivaciones en \(p\)}

\begin{remark}
  \(\mathcal{D}(p)\) es un espacio vectorial:
  \begin{itemize}
  \item \((\delta+\delta')(\Clase{f})=\delta(\Clase{f})+\delta'(\Clase{f})\)
  \item \(\lambda\delta(\Clase{f})=\delta(\lambda\Clase{f})\)
  \end{itemize}
  Y existe un homomorfismo de espacios vectoriales 
  \(\begin{array}{ll}
      \\ \phi\colon & \Tangente[p]{M}\to\mathcal{D}(p) \\
      & \Clase{\alpha}\mapsto\delta_{\Clase{\alpha}}
    \end{array}\)
\end{remark}

\begin{lemma}\label{lem:sobre-iso-derivaciones}
  Sea \(U\subseteq\RealSet^{m}\) bola abierta de centro \(0\) (en general, un
  convexo con \(0\in U\)). Si \(f\colon U\to\RealSet\) diferenciable, entonces
  existen \(g_{1},\dots,g_{m}\colon U\to\RealSet\) cumpliendo:
  \[\begin{array}{l}
      1. \quad f(x)-f(0)=\sum\limits_{i=1}^{m}x_{i}g_{i}(x),\, \forall x\in U \\
      2. \quad g_{i}(0)=\Restrict{\pderiv{f}{x_{i}}}{0}
  \end{array}\]
\end{lemma}

\begin{proof}
  Sea \(\begin{array}{ll}\\ h\colon &
          U\times\Clase{0,1}\overset{t\cdot}{\to} U\overset{f}{\to}\RealSet \\
          & (x,t)\mapsto tx\mapsto f(tx)\end{array}\).

        Claramente, si \(U\) es convexo, entonces si
        \(x\in U,\, t\in\Clase{0,1}\) se tiene \(tx\in\Clase{0,x}\subseteq U\).

        Entonces:
        \begin{enumerate}

        \item \(\forall x\in U\):
        \[\begin{array}{c}
            h(x,1)-h(x,0)=f(x)-f(0)=\int\limits_{0}^{1}{\pderiv{h(x,t)}{t}}{dt}= \\
            \sum\limits_{i=1}^{m}x_{i}\int\limits_{0}^{1}{
            \Restrict{\pderiv{f}{x_{i}}}{tx}}{dt}=\sum\limits_{i=1}^{m}x_{i}g_{i}(x)
          \end{array}\]
        \item
        \(g_{i}(0)=\int\limits_{0}^{1}{\Restrict{\pderiv{f}{x_{i}}}{0}}{dt}=\Restrict 
        {\pderiv{f}{x_{i}}}{0}\)
      \end{enumerate}
\end{proof}

\begin{proposition}
   \(\begin{array}{ll}
      \\ \phi\colon & \Tangente[p]{M}\to\mathcal{D}(p) \\
      & \Clase{\alpha}\mapsto\delta_{\Clase{\alpha}}
    \end{array}\) es un isomorfismo
\end{proposition}

\begin{proof}
  Sea la base \(\{e_{i}^{p}\}_{i=1}^{m}=\{\Clase{\varepsilon_{i}^{p}}\}_{i=1}^{m}\)
  de \(\Tangente[p]{M^{m}}\). Si \((U,\varphi)\) es carta de \(M\) con \(p\in
  U\) entonces:
  \[\begin{array}{ll}
    (*) & \varepsilon_{i}^{p}(t)=\varphi^{-1}(\varphi(p)+te_{i}),\, \Abs{t}<\varepsilon
  \end{array},\]
  es decir, con un \(t\) adecuado para el que
  \(\varphi(p)+te_{i}\in\varphi(U)\).

  Si ahora tomamos \(f\in\mathcal{F}(U)\), entonces
  \(\Clase{f}\in\mathcal{G}(p)\) y podemos aplicarle
  \(\delta_{e_{i}^{p}}\):
  \[\begin{array}{l}
      \delta_{e_{i}^{p}}(\Clase{f})=(f\circ \varepsilon_{i}^{p})'(0)=
      (f\circ\varphi^{-1}\circ\varphi\circ\varepsilon_{i}^{p})'(0)\overset{(*)}{=} \\
      ((f\circ\varphi^{-1})\circ(\varphi(p)+te_{i}^{p}))'(0)=
      \sum\limits_{j=1}^{m}\Restrict{\pderiv{f\circ\varphi^{-1}}{x_{j}}}{\varphi(p)}
      \cdot\Restrict{\frac{dx_{j}}{dt}}{t=0}\overset{(**)}{=} \\
      \Restrict{\pderiv{(f\circ\varphi^{-1})}{x_{i}}}{\varphi(p)}=
      \Restrict{\pderiv{f}{x_{i}}}{p}
    \end{array}\]
  donde hemos utilizado:
  \[\begin{array}{ll}
      (**) & x_{j}=\left\{ \begin{array}{cc}
                             \varphi(p)_{j} & j\neq i \\
                             \varphi(p)_{i}+t & j=i
                           \end{array}\right.
    \end{array}\]

  Por tanto, \(\delta_{e_{i}^{p}}=\Restrict{\pderiv{}{x_{i}}}{p}\).

  \vline
  
  Ahora debemos probar que \(\phi\) es inyectiva y sobreyectiva:

  \begin{itemize}
    \item \underline{Inyectiva:} \(\phi\) es inyectivo si y solo
      si \(\ker{\phi}=\{0\}\). En general, la imagen por \(\phi\) de un vector
      \(\Clase{\alpha}=\sum\limits_{i=1}^{m}\lambda_{i}e_{i}^{p}\) de
      \(\Tangente[p]{M}\) es \(\phi(\Clase{\alpha})= 
      \sum\limits_{i=1}^{m}\lambda_{i}\delta_{e_{i}^{p}}\).

      Sea ahora, \((U,\varphi)\) carta con \(p\in U\) y tal que
      \(U\overset{\varphi}{\to}\varphi(U)\overset{\pi_{j}}{\to}\RealSet\),
      siendo \(\pi_{j}\) la proyección en la coordenada \(j\)
      y sea \(\varphi_{j}=\pi_{j}\circ\varphi\), entonces: 
      \[\begin{array}{c}
          \delta_{\Clase{\alpha}}(\Clase{\varphi_{j}})=0\iff
          \sum\limits_{i=1}^{m}\lambda_{i}\delta_{e_{i}^{p}}(\Clase{\varphi_{j}})=
          \\ 
          \sum\limits_{i=1}^{m}\lambda_{i}
          \Restrict{\pderiv{(\varphi_{j}\circ\varphi^{-1})}{x_{i}}}{\varphi(p)}=
          \sum\limits_{i=1}^{m}\lambda_{i}
          \Restrict{\pderiv{\pi_{j}}{x_{i}}}{\varphi(p)}=0 \ (\star)
        \end{array}\]
      y resulta que \(\Restrict{\pderiv{\pi_{j}}{x_{i}}}{\varphi(p)}\) se anula
      si \(j\neq i\) y vale uno cuando son iguales, luego por \((\star)\)
      obtenemos que \(\lambda_{j}=0\) para \(j\) fijo. Si variamos \(j\) veremos
      que \(\lambda_{i}=0\ \forall i\), luego \(\ker{\phi}=\{0\}\). 

      \item \underline{Sobreyectiva:} para probar la sobreyectividad
        utilizaremos el \cref{lem:sobre-iso-derivaciones}

        Sean \(\delta\colon\mathcal{G}(p)\to\RealSet\) y
        \(\Clase{h}\in\mathcal{G}(p)\). Podemos suponer \(h\colon
        U\to\RealSet\), con \((U,\varphi)\) carta con \(p\in U,\, \varphi(U)\)
        bola abierta de centro \(\varphi(p)=0\). Sea
        \(f=h\circ\varphi^{-1}\colon\varphi(U)\to\RealSet\), \(f\) cumple el
        \cref{lem:sobre-iso-derivaciones}:

        \[
          \exists g_{i}\colon\varphi(U)\to\RealSet\colon
          \begin{dcases}
            \forall x\in\varphi(U)
            \ f(x)-f(0)=\sum\limits_{i=1}^{m}x_{i}g_{i}(x) \\
            g_{i}(0) = \Restrict{\pderiv{f}{x_{i}}}{0}
          \end{dcases}
        \]

        Sea \(\eta_{i}=g_{i}\circ\varphi\colon U\to\RealSet\). Sea \(q\in U,\
        \varphi(q)=x\). Entonces:

        \[
          h(q)-h(p)=\sum\limits_{i=1}^{m}\varphi(q)_{i}\eta_{i}(q)
        \]

        donde \(\varphi(q)_{i}=\pi\circ\varphi(q)=\varphi_{i}(q)\).

        Luego,

        \[
          h-C_{h(p)}=\sum\limits_{i=1}^{m}\varphi_{i}\eta_{i}
        \]

        Y por tanto,
        \[\begin{array}{l}
            \delta(\Clase{h-C_{h(p)}})=
            \delta(\Clase{\sum\limits_{i=1}^{m}\varphi_{i}\eta_{i}})= \\
            \sum\limits_{i=1}^{m}\varphi_{i}(p)\delta(\Clase{\eta_{i}})
            +\delta(\Clase{\varphi_{i}})\eta_{i}=
            \sum\limits_{i=1}^{m}\delta(\Clase{\varphi_{i}})\eta_{i}(p)
          \end{array}\]

        A su vez, por las propiedades de las derivaciones
        \(\delta(\Clase{h-C_{h(p)}})=\delta(\Clase{h})\). Además, observemos que
        \(\delta(\Clase{\varphi_{i}})=\lambda_{i}\in\RealSet\) (son escalares) y
        que
        \(\eta_{i}(p)=(g_{i}\circ\varphi)=g_{i}(0)=\Restrict{\pderiv{f}{x_{i}}}{0}
        =\Restrict{\pderiv{(h\circ\varphi^{-1})}{x_{i}}}{0}=
        \delta_{e_{i}^{p}}(\Clase{h})\). Así, 
         
        \[\begin{array}{l}
            \delta(\Clase{h})=\sum\limits_{i=1}^{m}\delta(\Clase{\varphi_{i}})
            \eta_{i}(p)=\sum\limits_{i=1}^{m}\lambda_{i}\delta_{e_{i}^{p}}(\Clase{h}) \\
            \Rightarrow
            \delta=\sum\limits_{i=1}^{m}\lambda_{i}\delta_{e_{i}^{p}}=
            \phi(\sum\limits_{i=1}^{m}\lambda_{i}e_{i}^{p})
          \end{array}\]

        Y nuestra función \(\phi\) es sobreyectiva. Entonces es isomorfismo y
        \(\Tangente[p]{M}\homeo_{\phi}\mathcal{D}(p)\), teniéndose que
        \(\Tangente{M}\homeo\bigcup_{p\in M}\mathcal{D}(p)\).    
  \end{itemize}
\end{proof}

\begin{remark}
  Si optamos por la definición de Espacio Tangente como derivaciones
  \(\mathcal{D}(p)\) del espacio de gérmenes \(\mathcal{G}(p)\), entonces
  \(df_{p}\colon\mathcal{D}_{M}(p)\to\mathcal{D}_{N}(f(p))\) viene dada por
  \(df_{p}(\delta)(\Clase{g})=\delta(\Clase{g\circ f})\).

  \vline
  
  En efecto, recordemos que el isomorfismo \(\phi\) viene dado por
  \(\Clase{\alpha}\mapsto\delta_{\Clase{\alpha}}\) con
  \(\delta_{\Clase{\alpha}}(\Clase{f})=\Restrict{\frac{d(f\circ\alpha)}{dt}}{t=0}\)
  por lo que
  \(df_{p}(\delta_{\Clase{\alpha}})(\Clase{g})=\delta_{\Clase{\alpha}}(\Clase{g\circ
    f})=\Restrict{\frac{d(g\circ f\circ\alpha)}{dt}}{t=0}\).

  \vline

  A su vez,
  \(\delta_{df_{p}\Clase{\alpha}}(\Clase{g})=\delta_{\Clase{f\circ\alpha}}(\Clase{g})=
  \Restrict{\frac{d(g\circ f\circ\alpha)}{dt}}{t=0}\).

  \vline

  Por tanto,
  \(df_{p}(\delta_{\Clase{\alpha}})(\Clase{g})=\delta_{df_{p}\Clase{\alpha}}(\Clase{g})\),
  es decir, el siguiente diagrama conmuta:

   \begin{center}
      \centering
      \begin{tikzcd}[column sep = large]
        \Tangente[p]{M}
        \ar[d, "df_p"']
        \ar[r, "\homeo_\phi"]
        &
        \mathcal{D}_M(p)
        \ar[d, "df_p"] \\ 
        \Tangente[f(p)]{N}
        \ar[r, "\homeo_\phi"']
        &
        \mathcal{D}_N(f(p))
      \end{tikzcd}
    \end{center}  
\end{remark}

\section{Derivaciones como operadores}

¿Cómo se ven los campos tangentes \(X\colon M\to\Tangente{M}\)?

\begin{definition}
  \(D\colon\mathcal{F}(M)\to\mathcal{F}(M)\) operador lineal se dice
  \emph{derivación} si cumple \(D(fg)=D(f)g+fD(g)\) (regla de Leibniz). Al
  espacio de derivaciones en \(M\) lo notaremos \(\mathcal{D}(M)\) (es espacio
  vectorial).
\end{definition}

\begin{definition}\label{defderivacioper}
  Si ahora \(X\colon M\to\Tangente{M}\) es un campo tangente con funciones
  componentes locales \(g^{U}\colon\varphi(U)\to\RealSet^{m}\), entonces se
  define \(D_{X}\colon\mathcal{F}(M)\to\mathcal{F}(M)\) como
  
  \[D_{X}(f)(p)=\sum\limits_{i=1}^{m}g_{i}^{U}(\varphi(p))\Restrict
  {\pderiv{(f\circ\varphi^{-1})}{x_{i}}}{\varphi(p)}=
  \sum\limits_{i=1}^{m}g_{i}^{U}(\varphi(p))\delta_{e_{i}^{p}}(\Clase{f})\]
  siendo \((U,\varphi)\) carta con \(p\in U\).
\end{definition}

\begin{proposition}
  La anterior aplicación está bien definida (por la propiedad
  \nameref{prop:comp-local}) y es derivación.

  \vline

  Además, veremos que la correspondencia \(X\mapsto D_{X}\) es biyectiva y así
  tendremos una segunda definición alternativa de campo tangente como una
  derivación sobre \(\mathcal{F}(M)\).
\end{proposition}

\begin{proof}
  \underline{AYUDA(biyección):} Tendremos que probar que existe un único
  \(X_{D}\) para cada \(D\) tal que \(\phi(X_{D})=D\) con \(\phi\) el
  isomorfismo que vimos antes.

  Si \(D\colon\mathcal{F}(M)\to\mathcal{F}(M)\)
  derivación entonces para cada \(p\in M\)
  \(D_{p}\colon\mathcal{G}(p)\to\RealSet\) con 
  \(D_{p}(\Clase{f})=D(\widetilde{f})(p)\) donde \(\widetilde{f}\) es un
  representante de \(\Clase{f}\) definido en todo \(M\) y que siempre existe por
  el \nameref{lem:extension}.
\end{proof}

\begin{lemma}
  Si \(f,g\colon M\to\RealSet\) tienen el mismo germen en \(p\), entonces sus
  derivaciones \(D(f)\) y \(D(g)\) tienen el mismo germen en \(p\) también.
\end{lemma}

\begin{proof}
  Por el lema de extensión, consideremos la función constante \(1\):
  \(c_{1}\colon U\to\RealSet\) con \(c_{1}(u)=1\). Sean \(f=g\) en
  \(U\subseteq\RealSet\) abierto. Se tiene que \(\exists\widetilde{c_{1}}\colon
  M\to\RealSet\) tal que \(\widetilde{c_{1}}=1\) en \(V\subseteq V'\subseteq U\)
  abiertos y \(\widetilde{c_{1}}=0\) en \(U\backslash V'\).
  Sea \(h=(f-g)\widetilde{c_{1}}\). Entonces \(h=cte\ 0\):
  Si \(q\in V'\), \((f-g)(q)=0\), si \(q\in M\backslash V'\),
  \(\widetilde{c_{1}}=0\). Por ser \(D\) lineal, \(D(h)=0\) y por ser derivación:
  \[
    0=D(h)=D(f-g)\widetilde{c_{1}}+(f-g)D(\widetilde{c_{1}})
  \]

  Luego si lo aplicamos en \(q\in V\), llegaremos a que \(D(f)=D(g)\) en \(V\).
\end{proof}

\end{document}
%%% Local Variables:
%%% TeX-master: "../VD_ebook"
%%% End: