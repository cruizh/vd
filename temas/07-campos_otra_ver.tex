\documentclass[../VD.tex]{subfiles}

\externaldocument{../VD}

\begin{document}

\setcounter{chapter}{6}
\chapter{Versión alternativa de campos tangentes}\label{chap:campos}

\section{Introducción}

Pasaremos ahora a estudiar una versión alternativa y más operativa de la definición
de campos tangentes y sus propiedades.

Consideraremos los conjuntos \(\mathcal{F}(M)=\{f\colon M\to\RealSet
\ \text{diferenciables}\}\) y \(\mathcal{F}(U)=\{f\colon U\to\RealSet
\ \text{diferenciables}\}\) con \(U\subseteq M\) abierto.

\begin{definition}[name=\(\mathcal{F}(p)\)]\label{def:F(p)}
  Al conjunto de las \emph{curvas diferenciables alrededor de \(p\in M\)} lo
  notaremos \[\mathcal{F}(p)=\{f\colon U\to\RealSet \ \text{diferenciables con}\ 
  p\in U\subseteq M \ \text{abierto}\}\]
\end{definition}

\begin{definition}
  Sean \(f\colon U\to\RealSet,g\colon V\to\RealSet\) ambas funciones en
  \(\mathcal{F}(p)\) para un cierto \(p\in M\). Diremos que \(f\) y 
  \(g\) son \emph{equivalentes en \(p\)}, y lo notarenos \(f \sim_{p} g\), si
  \(\exists\Omega\subseteq U\cap V\) abierto con \(p\in\Omega\) donde \(f,g\)
  están definidas y además \(\Restrict{f}{\Omega}=\Restrict{g}{\Omega}\).

  \vline

  A cada clase de equivalencia \(\Clase{f}\) se le llama \emph{gérmen en \(p\)}.
  Al conjunto cociente de los gérmenes en \(p\) lo notaremos
  \(\frac{\mathcal{F}(p)}{\sim_{p}}=\colon\mathcal{G}(p)\). 
\end{definition}

\begin{lemma}\label{lem:G(p)ev}
  \(\mathcal{G}(p)\) es un espacio vectorial, de hecho, es un álgebra (tiene un
  producto).
\end{lemma}

\begin{proof}
  
\end{proof}


\end{document}
%%% Local Variables:
%%% TeX-master: "../VD"
%%% End: