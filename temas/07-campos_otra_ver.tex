\documentclass[../VD.tex]{subfiles}

\externaldocument{../VD}

\begin{document}

\setcounter{chapter}{6}
\chapter{Versión alternativa de campos tangentes}\label{chap:campos}

\section{Introducción}

Pasaremos ahora a estudiar una versión alternativa y más operativa de la definición
de campos tangentes y sus propiedades.

Consideraremos los conjuntos \(\mathcal{F}(M)=\{f\colon M\to\RealSet
\ \text{diferenciables}\}\) y \(\mathcal{F}(U)=\{f\colon U\to\RealSet
\ \text{diferenciables}\}\) con \(U\subseteq M\) abierto.

\begin{definition}[name=\(\mathcal{F}(p)\)]\label{def:F(p)}
  Al conjunto de las \emph{curvas diferenciables alrededor de \(p\in M\)} lo
  notaremos \[\mathcal{F}(p)=\{f\colon U\to\RealSet \ \text{diferenciables con}\ 
  p\in U\subseteq M \ \text{abierto}\}\]
\end{definition}

\begin{definition}
  Sean \(f\colon U\to\RealSet,g\colon V\to\RealSet\) ambas funciones en
  \(\mathcal{F}(p)\) para un cierto \(p\in M\). Diremos que \(f\) y 
  \(g\) son \emph{equivalentes en \(p\)}, y lo notarenos \(f \sim_{p} g\), si
  \(\exists\Omega\subseteq U\cap V\) abierto con \(p\in\Omega\) donde \(f,g\)
  están definidas y además \(\Restrict{f}{\Omega}=\Restrict{g}{\Omega}\).

  \vline

  A cada clase de equivalencia \(\Clase{f}\) se le llama \emph{gérmen en \(p\)}.
  Al conjunto cociente de los gérmenes en \(p\) lo notaremos
  \(\frac{\mathcal{F}(p)}{\sim_{p}}=\colon\mathcal{G}(p)\). 
\end{definition}

\begin{lemma}\label{lem:G(p)ev}
  \(\mathcal{G}(p)\) es un espacio vectorial, de hecho, es un álgebra (tiene un
  producto).
\end{lemma}

\begin{proof}
  Sean \(f\colon U\to\RealSet\) y \(g\colon V\to\RealSet\), \(p\in
          U,V\) abiertos, \(\lambda\in\RealSet\):
  \begin{itemize}
    \item \(\Clase{f}+\Clase{g}=\Clase{\Restrict{f}{U\cap V}+\Restrict{g}{U\cap
          V}}\)
    \item \(\lambda\Clase{f}=\Clase{\lambda f}\)
    \item Elemento neutro: \(\theta_{p}\in\mathcal{G}(p), \
      \theta_{p}\equiv cte\ 0\)
    \item Producto: \(\Clase{f}\cdot\Clase{g}=\Clase{\Restrict{f}{U\cap
          V}\cdot\Restrict{g}{U\cap V}}\)
  \end{itemize}
\end{proof}

\begin{definition}
  Una \emph{derivación en \(p\)} es una aplicación lineal
  \(\delta\colon\mathcal{G}(p)\to\RealSet\) que cumple la \textit{regla de
    Leibniz} de las derivadas:
  \[
    \delta(\Clase{f}\cdot\Clase{g})=\delta(\Clase{f})\cdot
    g(p)+f(p)\cdot\delta(\Clase{g}) 
  \]

  A la familia de derivaciones en \(p\) la notaremos \(\mathcal{D}(p)\).
\end{definition}

\begin{remark}
  \(\delta(\theta_{p})=0\), es más, si \(c_{t}\colon
  M\to\RealSet\ cte\ t\), entonces \(\delta(\Clase{c_{t}})=0\):
  \[\begin{array}{ccc}
      \delta(\Clase{c_{t}}\Clase{f}) & =\delta(\Clase{c_{t}f})=\delta(\Clase{tf})
      \overset{\text{\(\delta\) lineal}}{=} & t\delta(\Clase{f}) \\
      \delta(\Clase{c_{t}}\Clase{f}) & \overset{\text{Leibniz}}{=}
      \delta(\Clase{c_{t}})f(p)+t\delta(\Clase{f}) &
      \end {array}\]

    Luego, \(\forall\Clase{f}\in\mathcal{G}(p)\)
    \[
      0=\delta(\Clase{c_{t}})f(p)\Rightarrow\delta(\Clase{c_{t}})=0.
    \]
\end{remark}

\begin{remark}
  Todo germen de \(\mathcal{G}(p)\) se puede representar por una función
  \(f\colon M\to\RealSet\). Esto es por el \nameref{lem:extension} de funciones
  escalares: \(\Clase{g}\in\mathcal{G}(p)\) con \(g\colon U\to\RealSet\) con
  \(p\in U\) abierto, entonces existen \(V\subseteq U\) abierto con \(p\in V\)
  y \(\widetilde{g}\colon V\to\RealSet\) con \(\widetilde{g}=g\) en \(V\), luego
  \(\Clase{\widetilde{g}}=\Clase{g}\).
\end{remark}

\begin{example}
  Dada \(\Clase{\alpha}\in\Tangente[p]{M}\)
  (\(\alpha\colon(-\epsilon,\epsilon)\to M, \, \alpha(0)=p\)), entonces
  definimos:
  \[\begin{array}{ll}
    \delta_{\Clase{\alpha}}\colon & \mathcal{G}(p)\to\RealSet \\
    & \Clase{f}\mapsto\delta_{\Clase{\alpha}}(\Clase{f})=(f\circ\alpha)'(0)
    \end{array}\]
  Tomaremos \(\epsilon\) pequeño para que
  \(\alpha((-\epsilon,\epsilon))\subseteq U\Rightarrow
  (-\epsilon,\epsilon)\overset{\alpha}{\to}U\overset{f}{\to}\RealSet\)

  \vline

  Además, \(\delta_{\Clase\alpha}\) es lineal trivialmente. Veamos la regla de
  Leibniz:
  \[\begin{array}{l}
      \delta_{\Clase\alpha}(\Clase{f}\Clase{g})=\delta_{\Clase\alpha}(\Clase{fg})=\\
      ((fg)\circ\alpha)'(0)=((f\circ\alpha)\cdot(g\circ\alpha))'(0)=\\
      ((f\circ\alpha)'(0)(g\circ\alpha)(0)+(f\circ\alpha)(0)(g\circ\alpha)'(0)
      \overset{\alpha(0)=p}{=} \\ \delta_{\Clase\alpha}(\Clase{f})g(p)+f(p)
      \delta_{\Clase\alpha}(\Clase{g})
  \end{array}\]
\end{example}

\begin{remark}
  \(\mathcal{D}(p)\) es un espacio vectorial:
  \begin{itemize}
  \item \((\delta+\delta')(\Clase{f})=\delta(\Clase{f})+\delta'(\Clase{f})\)
  \item \(\lambda\delta(\Clase{f})=\delta(\lambda\Clase{f})\)
  \end{itemize}
  Y existe un homomorfismo de espacios vectoriales 
  \(\begin{array}{ll}
      \\ \phi\colon & \Tangente[p]{M}\to\mathcal{D}(p) \\
      & \Clase{\alpha}\mapsto\delta_{\Clase{\alpha}}
    \end{array}\)
\end{remark}

\begin{lemma}\label{lem:sobre-iso-derivaciones}
  Sea \(U\subseteq\RealSet^{m}\) bola abierta de centro \(0\) (en general, un
  convexo con \(0\in U\)). Si \(f\colon U\to\RealSet\) diferenciable, entonces
  existen \(g_{1},\dots,g_{m}\colon U\to\RealSet\) cumpliendo:
  \[\begin{array}{l}
      1. \quad f(x)-f(0)=\sum\limits_{i=1}^{m}x_{i}g_{i}(x),\, \forall x\in U \\
      2. \quad g_{i}(0)=\Restrict{\pderiv{f}{x_{i}}}{0}
  \end{array}\]
\end{lemma}

\begin{proof}
  Sea \(\begin{array}{ll}\\ h\colon &
          U\times\Clase{0,1}\overset{t\cdot}{\to} U\overset{f}{\to}\RealSet \\
          & (x,t)\mapsto tx\mapsto f(tx)\end{array}\).

        Claramente, si \(U\) es convexo, entonces si
        \(x\in U,\, t\in\Clase{0,1}\) se tiene \(tx\in\Clase{0,x}\subseteq U\).

        Entonces:
        \begin{enumerate}

        \item \(\forall x\in U\):
        \[\begin{array}{c}
            h(x,1)-h(x,0)=f(x)-f(0)=\int\limits_{0}^{1}{\pderiv{h(x,t)}{t}}{dt}= \\
            \sum\limits_{i=1}^{m}x_{i}\int\limits_{0}^{1}{
            \Restrict{\pderiv{f}{x_{i}}}{tx}}{dt}=\sum\limits_{i=1}^{m}x_{i}g_{i}(x)
          \end{array}\]
        \item
        \(g_{i}(0)=\int\limits_{0}^{1}{\Restrict{\pderiv{f}{x_{i}}}{0}}{dt}=\Restrict 
        {\pderiv{f}{x_{i}}}{0}\)
      \end{enumerate}
\end{proof}

\begin{proposition}
  \(\phi\) es un isomorfismo
\end{proposition}

\begin{proof}
  Sea la base \(\{e_{i}^{p}\}_{i=1}^{m}=\{\Clase{\epsilon_{i}^{p}}\}_{i=1}^{m}\)
  de \(\Tangente[p]{M^{m}}\). Si \((U,\varphi)\) es carta de \(M\) con \(p\in
  U\) entonces:
  \[\begin{array}{ll}
    (*) & \epsilon_{i}^{p}(t)=\varphi^{-1}(\varphi(p)+te_{i}),\, \Abs{t}<\epsilon
  \end{array},\]
  es decir, con un \(t\) adecuado para el que
  \(\varphi(p)+te_{i}\in\varphi(U)\).

  Si ahora tomamos \(f\in\mathcal{F}(U)\), entonces
  \(\Clase{f}\in\mathcal{G}(p)\) y podemos aplicarle
  \(\delta_{e_{i}^{p}}\):
  \[\begin{array}{l}
      \delta_{e_{i}^{p}}(\Clase{f})=(f\circ \epsilon_{i}^{p})'(0)=
      (f\circ\varphi^{-1}\circ\varphi\circ\epsilon_{i}^{p})'(0)\overset{(*)}{=} \\
      ((f\circ\varphi^{-1})\circ(\varphi(p)+te_{i}^{p}))'(0)=
      \sum\limits_{j=1}^{m}\Restrict{\pderiv{f\circ\varphi^{-1}}{x_{j}}}{\varphi(p)}
      \cdot\Restrict{\frac{dx_{j}}{dt}}{t=0}\overset{(**)}{=} \\
      \Restrict{\pderiv{(f\circ\varphi^{-1})}{x_{i}}}{\varphi(p)}=
      \Restrict{\pderiv{f}{x_{i}}}{p}
    \end{array}\]
  donde hemos utilizado:
  \[\begin{array}{ll}
      (**) & x_{j}=\left\{ \begin{array}{cc}
                             \varphi(p)_{j} & j\neq i \\
                             \varphi(p)_{i}+t & j=i
                           \end{array}\right.
    \end{array}\]

  Por tanto, \(\delta_{e_{i}^{p}}=\Restrict{\pderiv{}{x_{i}}}{p}\).

  \vline
  
  Ahora debemos probar que \(\phi\) es inyectiva y sobreyectiva:

  \begin{itemize}
    \item \underline{Inyectiva:} \(\phi\) es inyectivo si y solo
      si \(\ker{\phi}=\{0\}\). En general, la imagen por \(\phi\) de un vector
      \(\Clase{\alpha}=\sum\limits_{i=1}^{m}\lambda_{i}e_{i}^{p}\) de
      \(\Tangente[p]{M}\) es \(\phi(\Clase{\alpha})= 
      \sum\limits_{i=1}^{m}\lambda_{i}\delta_{e_{i}^{p}}\).

      Sea ahora, \((U,\varphi)\) carta con \(p\in U\) y tal que
      \(U\overset{\varphi}{\to}\varphi(U)\overset{\pi_{j}}{\to}\RealSet\),
      siendo \(\pi_{j}\) la proyección en la coordenada \(j\)
      y sea \(\varphi_{j}=\pi_{j}\circ\varphi\), entonces: 
      \[\begin{array}{c}
          \delta_{\Clase{\alpha}}(\Clase{\varphi_{j}})=0\iff
          \sum\limits_{i=1}^{m}\lambda_{i}\delta_{e_{i}^{p}}(\Clase{\varphi_{j}})=
          \\ 
          \sum\limits_{i=1}^{m}\lambda_{i}
          \Restrict{\pderiv{(\varphi_{j}\circ\varphi^{-1})}{x_{i}}}{\varphi(p)}=
          \sum\limits_{i=1}^{m}\lambda_{i}
          \Restrict{\pderiv{\pi_{j}}{x_{i}}}{\varphi(p)}=0 \ (\star)
        \end{array}\]
      y resulta que \(\Restrict{\pderiv{\pi_{j}}{x_{i}}}{\varphi(p)}\) se anula
      si \(j\neq i\) y vale uno cuando son iguales, luego por \((\star)\)
      obtenemos que \(\lambda_{j}=0\) para \(j\) fijo. Si variamos \(j\) veremos
      que \(\lambda_{i}=0\ \forall i\), luego \(\ker{\phi}=\{0\}\). 

      \item \underline{Sobreyectiva:} para probar la sobreyectividad
        necesitaremos el \cref{lem:sobre-iso-derivaciones}
  \end{itemize}
  
\end{proof}


\end{document}
%%% Local Variables:
%%% TeX-master: "../VD"
%%% End: