\documentclass[../VD.tex]{subfiles}

\externaldocument{../VD}

\begin{document}

\setcounter{chapter}{6}
\chapter{Versión alternativa de campos tangentes}\label{chap:campos}

\section{Introducción}

Pasaremos ahora a estudiar una versión alternativa y más operativa de la definición
de campos tangentes y sus propiedades.

Consideraremos los conjuntos \(\mathcal{F}(M)=\{f\colon M\to\RealSet
\ \text{diferenciables}\}\) y \(\mathcal{F}(U)=\{f\colon U\to\RealSet
\ \text{diferenciables}\}\) con \(U\subseteq M\) abierto.

\begin{definition}[name=\(\mathcal{F}(p)\)]\label{def:F(p)}
  Al conjunto de las \emph{curvas diferenciables alrededor de \(p\in M\)} lo
  notaremos \[\mathcal{F}(p)=\{f\colon U\to\RealSet \ \text{diferenciables con}\ 
  p\in U\subseteq M \ \text{abierto}\}\]
\end{definition}

\begin{definition}
  Sean \(f\colon U\to\RealSet,g\colon V\to\RealSet\) ambas funciones en
  \(\mathcal{F}(p)\) para un cierto \(p\in M\). Diremos que \(f\) y 
  \(g\) son \emph{equivalentes en \(p\)}, y lo notarenos \(f \sim_{p} g\), si
  \(\exists\Omega\subseteq U\cap V\) abierto con \(p\in\Omega\) donde \(f,g\)
  están definidas y además \(\Restrict{f}{\Omega}=\Restrict{g}{\Omega}\).

  \vline

  A cada clase de equivalencia \(\Clase{f}\) se le llama \emph{gérmen en \(p\)}.
  Al conjunto cociente de los gérmenes en \(p\) lo notaremos
  \(\frac{\mathcal{F}(p)}{\sim_{p}}=\colon\mathcal{G}(p)\). 
\end{definition}

\begin{lemma}\label{lem:G(p)ev}
  \(\mathcal{G}(p)\) es un espacio vectorial, de hecho, es un álgebra (tiene un
  producto).
\end{lemma}

\begin{proof}
  Sean \(f\colon U\to\RealSet\) y \(g\colon V\to\RealSet\), \(p\in
          U,V\) abiertos, \(\lambda\in\RealSet\):
  \begin{itemize}
    \item \(\Clase{f}+\Clase{g}=\Clase{\Restrict{f}{U\cap V}+\Restrict{g}{U\cap
          V}}\)
    \item \(\lambda\Clase{f}=\Clase{\lambda f}\)
    \item Elemento neutro: \(\theta_{p}\in\mathcal{G}(p), \
      \theta_{p}\equiv cte\ 0\)
    \item Producto: \(\Clase{f}\cdot\Clase{g}=\Clase{\Restrict{f}{U\cap
          V}\cdot\Restrict{g}{U\cap V}}\)
  \end{itemize}
\end{proof}

\begin{definition}
  Una \emph{derivación en \(p\)} es una aplicación lineal
  \(\delta\colon\mathcal{G}(p)\to\RealSet\) que cumple la \textit{regla de
    Leibniz} de las derivadas:
  \[
    \delta(\Clase{f}\cdot\Clase{g})=\delta(\Clase{f})\cdot
    g(p)+f(p)\cdot\delta(\Clase{g}) 
  \]
\end{definition}

\begin{remark}
  \(\delta(\theta_{p})=0\), es más, si \(c_{t}\colon
  M\to\RealSet\ cte\ t\), entonces \(\delta(\Clase{c_{t}})=0\):
  \[\begin{array}{l}
      \delta(\Clase{c_{t}}\Clase{f})=\delta(\Clase{c_{t}f})=\delta(\Clase{tf})
      \overset{\text{\(\delta\) lineal}}{=} t\delta(\Clase{f}) \\
      \delta(\Clase{c_{t}}\Clase{f})\overset{\text{Leibniz}}{=}
      \delta(\Clase{c_{t}})f(p)+t\delta(\Clase{f})   
      \end {array}\]

    Luego, \(\forall\Clase{f}\in\mathcal{G}(p)\)
    \[
      0=\delta(\Clase{c_{t}})f(p)\Rightarrow\delta(\Clase{c_{t}})=0.
    \]
\end{remark}

\begin{remark}
  Todo germen de \(\mathcal{G}(p)\) se puede representar por una función
  \(f\colon M\to\RealSet\). Esto es por el \nameref{lem:extension} de funciones
  escalares: \(\Clase{g}\in\mathcal{G}(p)\) con \(g\colon U\to\RealSet\) con
  \(p\in U\) abierto, entonces existen \(V\subseteq U\) abierto con \(p\in V\)
  y \(\widetilde{g}\colon V\to\RealSet\) con \(\widetilde{g}=g\) en \(V\), luego
  \(\Clase{\widetilde{g}}=\Clase{g}\).
\end{remark}

\begin{example}
  Dada \(\Clase{\alpha}\in\Tangente[p]{M}\)
  (\(\alpha\colon(-\epsilon,\epsilon)\to M, \, \alpha(0)=p\)), entonces
  definimos:
  \[\begin{array}{ll}
    \delta_{\Clase{\alpha}}\colon & \mathcal{G}(p)\to\RealSet \\
    & \Clase{f}\mapsto\delta_{\Clase{\alpha}}(\Clase{f})=(f\circ\alpha)'(0)
    \end{array}\]
  Tomaremos \(\epsilon\) pequeño para que
  \(\alpha((-\epsilon,\epsilon))\subseteq U\Rightarrow
  (-\epsilon,\epsilon)\overset{\alpha}{\to}U\overset{f}{\to}\RealSet\)

  \vline

  Además, \(\delta_{\Clase\alpha}\) es lineal trivialmente. Veamos la regla de
  Leibniz:
  \[\begin{array}{l}
      \delta_{\Clase\alpha}(\Clase{f}\Clase{g})=\delta_{\Clase\alpha}(\Clase{fg})=
      ((fg)\circ\alpha)'(0)=((f\circ\alpha)\cdot(g\circ\alpha))'(0)=\\
      ((f\circ\alpha)'(0)(g\circ\alpha)(0)+(f\circ\alpha)(0)(g\circ\alpha)'(0)
      \overset{\alpha(0)=p}{=} \\ \delta_{\Clase\alpha}(\Clase{f})g(p)+f(p)
      \delta_{\Clase\alpha}(\Clase{g})  
  \end{array}\]
\end{example}

\end{document}
%%% Local Variables:
%%% TeX-master: "../VD"
%%% End: