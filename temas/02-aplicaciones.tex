\documentclass[../VD.tex]{subfiles}

\externaldocument{../VD}

\begin{document}

\setcounter{chapter}{1}
\chapter{Aplicaciones diferenciables entre variedades}
\label{chap:app}

\begin{definition}[diferenciable]
  \label{def:diferenciable}
  Sea \(f \colon M \to N\) una aplicación entre variedades. Decimos que \(f\) es
  \emph{diferenciable} en \(p \in M\) si para toda carta \((W,\psi)\) de \(N\)
  con \(f(p) \in W\) existe una carta \((U,\varphi)\) de \(M\) tal que \(p \in
  U, f(U) \subseteq W\) y \(\psi \circ f \circ \varphi^{-1} \colon \varphi(U)
  \to \psi(W)\) es diferenciable en \(\varphi(p)\). Nótese que, en la
  \cref{fig:factorizacion-diferenciable}, \(\widetilde{f} = \psi \circ f \circ
  \varphi^{-1}\).

  Se dice que \(f\) es \emph{diferenciable} si lo es para todo \(p \in M\).
\end{definition}

\begin{figure}[h]
  \centering
  \begin{tikzcd}
    U \arrow[r, "f"] \arrow[d, "\varphi"']
    & W \arrow[d, "\psi"] \\
    \varphi(U) \arrow[r, "\widetilde{f}"]
    & \psi(W)
  \end{tikzcd}
  \caption{Factorización de una aplicación diferenciable entre variedades}
  \label{fig:factorizacion-diferenciable}
\end{figure}

\begin{lemma}
  Toda \(f\) \nameref{def:diferenciable} es continua para las topologías
  asociadas a las correspondientes estructuras diferenciables.
\end{lemma}

\begin{proof}
  Sea \(G \subset N\) abierto. Veamos que \(f^{-1}(G)\) es abierto de \(M\).

  Sea \(p \in f^{-1}(G)\). Entonces \(f(p) \in G\). Por el
  lema de \nameref{lem:topnat-caract}, existe
  \((W,\psi)\) carta de \(N\) con \(f(p) \in W \subseteq G\). Como \(f\) es
  \nameref{def:diferenciable} existe \((U,\varphi)\) carta de \(U\) con \(p \in
  U\), \(f(U) \subseteq W \subseteq G\). Entonces \(p \in U \subseteq
  f^{-1}(G)\). 
\end{proof}

\begin{lemma}
  Si \(f \colon M \to N\) y \(g \colon N \to Z\) son
  \hyperref[def:diferenciable]{diferenciables}, entonces \(g \circ f\) también
  lo es.
\end{lemma}

\begin{proof}
  % TODO
\end{proof}

\begin{definition}[difeomorfismo]
  \label{def:difeomorfismo}
  \(f \colon M \to N\) se dice \emph{difeomorfismo} entre variedades si \(f\) es
  biyectiva y \(f\) y \(f^{-1}\) son \hyperref[def:diferenciable]{diferenciables}.
\end{definition}

\begin{lemma}
  \(f\) es \nameref{def:diferenciable} en \(p\) si y solo si existen cartas
  \((U,\varphi)\) de \(M\) y \((W,\psi)\) de \(N\) con \(p \in U\), \(f(U)
  \subseteq W\) y \(\widetilde{f} = \psi \circ f \circ \varphi^{-1}\) es
  diferenciable (la misma de la \cref{fig:factorizacion-diferenciable}).
\end{lemma}

\begin{proof}
  % TODO
\end{proof}

\begin{lemma}
  Supongamos que \(f \colon M \to N\) es continua. Entonces \(f\) es
  \nameref{def:diferenciable} si y solo si para todas las cartas \((U,\varphi)\)
  de \(M\) y \((W,\psi)\) de \(N\), la aplicación \(\widetilde{f}\) de la
  \cref{fig:factorizacion-diferenciable} es diferenciable.
\end{lemma}

\begin{proof}\item 
\begin{subproof}[\(\impliedby\)]
 Sea \(W',\psi'\) una carta con \(f(p)\in W' \). Como \(W\) y \(W'\) son abiertos de \(N\), es claro que \(W\cap W'\) también es abierto de \(N\). Por ser \(f\) continua su inversa conserva los abiertos, es decir, \(f^{-1}(W\cap W')\) es abierto en \(W\), con \(p \in f^{-1}(W\cap W') \).

Como \(U\) es abierto de \(M\) podemos definir el abierto \(U'\) de \(M\) como \(U\cap f^{-1}(W\cap W')\). Entonces \((U',\varphi')\) con \(\Restrict{\varphi'}{U'}\) es una carta de \(M\) (por el
\cref{lem:compat-subcartas} de \nameref{lem:compat-subcartas} y ser \(\varphi\colon U \to \varphi(U)\) homeomorfismo).

Además \(p\in U'\) y \(f(U')\subseteq W' \), entonces considerar la aplicación \(\psi \circ f \circ \varphi^{-1} \) restringida a \(\varphi'(U') \) es lo mismo que considerar \(\psi' \circ f \circ (\varphi')^{-1} \) en dicho conjunto.

Finalmente:

\begin{figure}[h]
	\centering
	\begin{tikzcd}		
		\varphi'(U')
		\arrow[rr,"	\varphi^{-1} \circ \psi'"]
		\arrow[d, "(\varphi')^{-1}\circ \varphi"]&&
		\psi'( W')
		\arrow[d, "(\psi')^{-1} \circ \psi"]\\
		\varphi(U)
		\arrow[rr, "\varphi^{-1}\circ \psi"]&&
		\psi(W)
	\end{tikzcd}
\end{figure}

Este diagrama nos permite escribir  \(\psi' \circ f \circ (\varphi')^{-1}=\varphi^{-1} \circ \psi \circ (\varphi')^{-1}\circ \varphi \circ \psi^{-1} \circ \psi' \) que es composición de funciones diferenciales como hemos visto anteriormente.
\end{subproof}

\begin{subproof}[\(\implies\)]
Para dos cartas \((U,\varphi)\) de \(M\) y \((W,\psi)\) de \(N\) con \(p\in U\) y \(f(U)\subseteq W\), la composición \(\psi\circ f \circ \varphi^{-1}\) es diferenciable.
\end{subproof}
\end{proof}

\end{document}