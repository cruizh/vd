\documentclass[../VD.tex]{subfiles}

\externaldocument{../VD}

\begin{document}

\setcounter{chapter}{1}
\chapter{Aplicaciones diferenciables entre variedades}
\label{chap:app}

\begin{definition}[diferenciable]
  \label{def:diferenciable}
  Sea \(f \colon M \to N\) una aplicación entre variedades. Decimos que \(f\) es
  \emph{diferenciable} en \(p \in M\) si para toda carta \((W,\psi)\) de \(N\)
  con \(f(p) \in W\) existe una carta \((U,\varphi)\) de \(M\) tal que \(p \in
  U, f(U) \subseteq W\) y \(\psi \circ f \circ \varphi^{-1} \colon \varphi(U)
  \to \psi(W)\) es diferenciable en \(\varphi(p)\). Nótese que, en la
  \cref{fig:factorizacion-diferenciable}, \(\widetilde{f} = \psi \circ f \circ
  \varphi^{-1}\).

  Se dice que \(f\) es \emph{diferenciable} si lo es para todo \(p \in M\).
\end{definition}

\begin{figure}[h]
  \centering
  \begin{tikzcd}
    U \arrow[r, "f"] \arrow[d, "\varphi"']
    & W \arrow[d, "\psi"] \\
    \varphi(U) \arrow[r, "\widetilde{f}"]
    & \psi(W)
  \end{tikzcd}
  \caption{Factorización de una aplicación diferenciable entre variedades}
  \label{fig:factorizacion-diferenciable}
\end{figure}

\begin{lemma}
  Toda \(f\) \nameref{def:diferenciable} es continua para las topologías
  asociadas a las correspondientes estructuras diferenciables.
\end{lemma}

\begin{proof}
  Sea \(G \subset N\) abierto. Veamos que \(f^{-1}(G)\) es abierto de \(M\).

  Sea \(p \in f^{-1}(G)\). Entonces \(f(p) \in G\). Por el
  lema de \nameref{lem:topnat-caract}, existe
  \((W,\psi)\) carta de \(N\) con \(f(p) \in W \subseteq G\). Como \(f\) es
  \nameref{def:diferenciable} existe \((U,\varphi)\) carta de \(U\) con \(p \in
  U\), \(f(U) \subseteq W \subseteq G\). Entonces \(p \in U \subseteq
  f^{-1}(G)\). 
\end{proof}

\begin{lemma}
  Si \(f \colon M \to N\) y \(g \colon N \to Z\) son
  \hyperref[def:diferenciable]{diferenciables}, entonces \(g \circ f\) también
  lo es.
\end{lemma}

\begin{proof}
  Sean \(p \in M\) y \((\Omega,\rho)\) una carta de \(Z\) con \((g \circ f)(p)
  \in \Omega\). Como \(g\) es diferenciable, existe una carta \((W,\psi)\) de
  \(N\) con \(f(p) \in W\), \(g(W) \subseteq \Omega\) y \(\widetilde{g}
  \coloneqq \rho \circ g \circ \psi^{-1}\) diferenciable. Por otra parte, como
  \(f\) es diferenciable, existe una carta \((U,\varphi)\) de \(M\) con \(p \in
  U\), \(f(U) \subseteq W\) y \(\widetilde{f} \coloneqq \psi \circ f \circ
  \varphi^{-1}\) diferenciable. Tenemos el diagrama de la \cref{fig:comp-dif-cd}.

  \begin{figure}[h]
    \centering
    \begin{tikzcd}
      U \ar[d, "\varphi"] \ar[r, "f"] &
      W \ar[d, "\psi"] \ar[r, "g"] &
      \Omega \ar[d, "\rho"]\\
      \varphi(U) \ar[r, "\widetilde{f}"] &
      \psi(W) \ar[r, "\widetilde{g}"] &
      \rho(\Omega)
    \end{tikzcd}
    \caption{Composición de aplicaciones diferenciables}
    \label{fig:comp-dif-cd}
  \end{figure}

  Ya hemos encontrado una carta \((U,\varphi)\) de \(M\) con \(p \in U\) y \((g
  \circ f)(U) \subseteq \Omega\).
  Para ver que \(g \circ f\) es diferenciable, queda probar que
  \(h \coloneqq \rho \circ g \circ f \circ \varphi^{-1}\) lo es, y resulta que
  \(h = \widetilde{g} \circ \widetilde{f}\), i.e. es composición de aplicaciones
  diferenciables (por hipótesis), luego es diferenciable.
\end{proof}

\begin{definition}[difeomorfismo]
  \label{def:difeomorfismo}
  \(f \colon M \to N\) se dice \emph{difeomorfismo} entre variedades si \(f\) es
  biyectiva y \(f\) y \(f^{-1}\) son \hyperref[def:diferenciable]{diferenciables}.
\end{definition}

\begin{lemma}
  \(f\) es \nameref{def:diferenciable} en \(p\) si y solo si existen cartas
  \((U,\varphi)\) de \(M\) y \((W,\psi)\) de \(N\) con \(p \in U\), \(f(U)
  \subseteq W\) y \(\widetilde{f} = \psi \circ f \circ \varphi^{-1}\) es
  diferenciable (la misma de la \cref{fig:factorizacion-diferenciable}).
\end{lemma}

\begin{proof}\item
\begin{subproof}[\(\implies\)]
Es cierto pues que existan esas cartas es un caso particular de que \(f\) sea
diferenciable.
\end{subproof}
\begin{subproof}[\(\impliedby\)]
  Sea \((\Omega,\rho)\) carta cualquiera de \(N\) con \(f(p)\in\Omega\). Se
  tiene que \(\Omega\cap W\) es abierto de \(N\) y \(f(p) \in \Omega\cap
  W \subseteq W\). Como además \((W,\psi)\) y \((\Omega,\rho)\) son compatibles,
  entonces \(\psi(\Omega\cap W),\rho(\Omega\cap W)\) son abiertos y
  \(\psi\circ\rho^{-1},\rho\circ\psi^{-1}\) son diferenciables.

  Sabemos que \(f \colon U \to W\) es continua, y que \(W,\Omega\) son abiertos
  de \(N\) por lo que su intersección también, luego \(f^{-1}(\Omega \cap W)\)
  es abierto de \(M\), entonces \(f^{-1}(\Omega \cap W)\cap U \eqqcolon \Gamma
  \subseteq U\) es abierto de \(M\).

Sabemos por el \cref{lem:compat-restriccion}, que
\((\Gamma,\Restrict{\varphi}{\Gamma})\) es carta de \(M\). Entonces:

\[\Gamma\overset{f}{\rightarrow}\Omega\cap W\subseteq W,\Omega\]

\begin{figure}[h]
	\centering
	\begin{tikzcd}	
		\Gamma
		\arrow[rr,"	f "]
		\arrow[d, "\varphi"]&&
		\Omega
		\arrow[d, "\rho"]\\
		\varphi(\Gamma) \ar[d, symbol=\ni]
		\arrow[rr, "\tilde{f}=\rho\circ f\circ\varphi^{-1}"]&&
		\rho(\Omega)\\
    \varphi(p)
	\end{tikzcd}

Y \(\varphi(\Gamma)\) es abierto pues \(\varphi\) es homeomorfismo y \(\Gamma\) abierto.

Con el anterior diagrama y la definición que hemos hecho de
\(\tilde{f}=\rho\circ f\circ\varphi^{-1}=(\rho\circ\psi^{-1})\circ(\psi\circ
f\circ\varphi^{-1})\) que es diferenciable pues \((\rho\circ\psi^{-1})\) lo es
por compatbilidad y \((\psi\circ f\circ\varphi^{-1})\) lo es por hipótesis,
luego se tiene que \(f\) es diferenciable.

 
\end{figure}


\end{subproof}
\end{proof}

\begin{lemma}
  Supongamos que \(f \colon M \to N\) es continua. Entonces \(f\) es
  \nameref{def:diferenciable} en \(p\) si y solo si para todas las cartas
  \((U,\varphi)\) de \(M\) y \((W,\psi)\) de \(N\) con \(p \in U\) y \(f(U)
  \subseteq W\), la aplicación \(\widetilde{f} \coloneqq \psi \circ f \circ
  \varphi^{-1}\) (la misma de la \cref{fig:factorizacion-diferenciable}) es
  diferenciable.
\end{lemma}

\begin{proof}\item 
\begin{subproof}[\(\impliedby\)]
  Sea \((W',\psi')\) una carta con \(f(p) \in W'\).
  Como \(W\) y \(W'\) son abiertos de \(N\), es claro que \(W \cap W'\) también
  es abierto de \(N\).
  Por ser \(f\) continua, \(f^{-1}(W\cap W')\) es abierto en \(M\),
  con \(p \in f^{-1}(W \cap W')\).

  Como \(U\) es abierto de \(M\) podemos definir el abierto \(U'\) de \(M\) como
  \(U \cap f^{-1}(W \cap W')\). Entonces \((U',\varphi')\) con \(\varphi'
  \coloneqq \Restrict{\varphi}{U'}\) es una carta de \(M\) por el
  \cref{lem:compat-subcartas} de \nameref{lem:compat-subcartas} y ser \(\varphi
  \colon U \to \varphi(U)\) homeomorfismo.

  Además \(p \in U'\) y \(f(U') \subseteq W' \).
  Finalmente

  \begin{figure}[h]
    \centering
    \begin{tikzcd}
      U' \ar[r, "f"] \ar[d, "\varphi'"'] & W' \ar[d, "\psi'"]\\
      \varphi'(U') \ar[d, hook, "i_{U'}"']
      \ar[r, "\hat{f}"] & \psi'(W') \ar[d, hook, "i_{W'}"]\\
      \varphi(U) \ar[r, "\widetilde{f}"'] & \psi(W')
    \end{tikzcd}
  \end{figure}

  Por construcción de \((W',\psi')\) y definición de
  \nameref{def:diferenciable}, \(f\) es diferenciable si y solo si \(\hat{f} =
  \psi' \circ f \circ (\varphi')^{-1}\)
  lo es, y efectivamente lo es porque \(\hat{f} = \Restrict{\widetilde{f}\,}{\varphi(U')}\), que es
  restricción de una aplicación diferenciable luego es diferenciable (recordemos
  que \(\widetilde{f} = \psi \circ f \circ \varphi^{-1}\) es diferenciable
  por ser \(f\) diferenciable).
\end{subproof}

\begin{subproof}[\(\implies\)]
Para dos cartas \((U,\varphi)\) de \(M\) y \((W,\psi)\) de \(N\) con \(p\in U\)
y \(f(U)\subseteq W\), la composición \(\psi\circ f \circ \varphi^{-1}\) es
diferenciable por definición.
\end{subproof}
\end{proof}

\section{Función \(f:M\to\mathbb{R}\)}

\begin{lemma}
  \(f:M\to\mathbb{R}\) es diferenciable, donde \(M\) es \(m\)-variedad \(\leftrightarrow\) \(\forall p\in M,\
  \exists(U,\varphi),\ p\in M\) y \(f\circ\varphi^{-1}:\varphi(U)\to\mathbb{R}\)
  diferenciable, donde \(\varphi(U)\subseteq\mathbb{R}^{m}\).
\end{lemma}
\begin{proof}
(Utilizando el lema anterior)
\end{proof}

\begin{definition}[\(\mathcal{F}(U)\) y \(x_{i}\)]
  Si \(U\subseteq M\) \(m\)-variedad abierto (entonces \(U\) también lo es), se
  define \[\mathcal{F}(U)=\{f:U\to\mathbb{R}\operatorname{diferenciable}\}\]
  Sea \((U,\varphi)\) carta de \(M\):
  \[
    \varphi:U\to\varphi(U)\subseteq\mathbb{R}^{m}\overset{\pi_{i}}{\to}\mathbb{R}
  \]
  \[
    \varphi(p)=(\varphi_{i}(p),\dots,\varphi_{m}(p))\overset{\pi_{i}}{\to}\varphi_{i}(p)
  \]

A \(\varphi_{i}:U\to\mathbb{R}\) se le llama \underline{función coordenada} de
la carta \((U,\varphi)\). Se denota \(x_{i}:U\to\mathbb{R}\).
\end{definition}

\end{document}

%%% Local Variables:
%%% TeX-master: "../VD"
%%% End: