\documentclass[../VD.tex]{subfiles}

\externaldocument{../VD}

\begin{document}

\setcounter{chapter}{1}
\chapter{Aplicaciones diferenciables entre variedades}
\label{chap:app}

\begin{definition}[diferenciable]
  \label{def:diferenciable}
  Sea \(f \colon M \to N\) una aplicación entre variedades. Decimos que \(f\) es
  \emph{diferenciable} en \(p \in M\) si para toda carta \((W,\psi)\) de \(N\) con \(f(p)
  \in W\) existe una carta \((U,\varphi)\) de \(M\) tal que \(p \in U, f(U)
  \subseteq W\) y \(\psi \circ f \circ \varphi^{-1} \colon \varphi(U) \to
  \psi(W)\) es diferenciable en \(\varphi(p)\). Se dice que \(f\) es \emph{diferenciable}
  si lo es para todo \(p \in M\).
\end{definition}

\begin{lemma}
  Toda \(f\) \nameref{def:diferenciable} es continua para las topologías
  asociadas a las correspondientes estructuras diferenciables.
\end{lemma}

\begin{proof}
  Sea \(G \subset N\) abierto. Veamos que \(f^{-1}(G)\) es abierto de \(M\).

  Sea \(p \in f^{-1}(G)\). Entonces \(f(p) \in G\). Por el
  lema de \nameref{lem:topnat-caract}, existe
  \((W,\psi)\) carta de \(N\) con \(f(p) \in W \subseteq G\). Como \(f\) es
  \nameref{def:diferenciable} existe \((U,\varphi)\) carta de \(U\) con \(p \in
  U\), \(f(U) \subseteq W \subseteq G\). Entonces \(p \in U \subseteq
  f^{-1}(G)\). 
\end{proof}

\begin{definition}[difeomorfismo]
  \label{def:difeomorfismo}
  \(f \colon M \to N\) se dice \emph{difeomorfismo} entre variedades si \(f\) es
  biyectiva y \(f\) y \(f^{-1}\) son \hyperref[def:diferenciable]{diferenciables}
\end{definition}

\end{document}