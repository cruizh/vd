\documentclass[../VD.tex]{subfiles}

\externaldocument{../VD}

\begin{document}

\setcounter{chapter}{2}
\chapter{Subvariedades}\label{chap:subvd}

\begin{lemma}
  Si \(N\) es subvariedad de \(M\), entonces la topología natural de la
  estructura de variedad de \(N\) es la topología relativa de la topología
  natural (DIF-topología) de \(M\).
\end{lemma}

\begin{definition}
  Una aplicación diferenciable \(f \colon M \to N\) se dice \emph{incrustación}
  (\emph{embedding}) si es inmersión inyectiva y la DIF-topología de estructura
  diferenciable de \(f(M)\) inducida por \(f \colon M \to f(M)\) coincide con la
  relativa de \(M\)
\end{definition}

\begin{lemma}
  Si \(f \colon M \to N\) es incrustación,
  \begin{enumerate}
  \item \(f(M)\) es subvariedad.
  \item La estructura de subvariedad coincide con la inducida por \(f\).
  \item \(f\) es difeomorfismo.
  \end{enumerate}
\end{lemma}

\end{document}

%%% Local Variables:
%%% TeX-master: "../VD"
%%% End: