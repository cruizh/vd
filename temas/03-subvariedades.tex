\documentclass[../VD.tex]{subfiles}

\externaldocument{../VD}

\begin{document}

\setcounter{chapter}{2}
\chapter{Subvariedades}\label{chap:subvd}

\begin{lemma}
  Si \(N\) es subvariedad de \(M\), entonces la topología natural de la
  estructura de variedad de \(N\) es la topología relativa de la topología
  natural (DIF-topología) de \(M\).
\end{lemma}

\begin{definition}
  Una aplicación diferenciable \(f \colon M \to N\) se dice \emph{incrustación}
  (\emph{embedding}) si es inmersión inyectiva y la DIF-topología de estructura
  diferenciable de \(f(M)\) inducida por \(f \colon M \to f(M)\) coincide con la
  relativa de \(M\)
\end{definition}

\begin{lemma}
  Si \(f \colon M \to N\) es incrustación,
  \begin{enumerate}
  \item \(f(M)\) es subvariedad.
  \item La estructura de subvariedad coincide con la inducida por \(f\).
  \item \(f\) es difeomorfismo.
  \end{enumerate}
\end{lemma}

%Cosas varias

\begin{theorem}[Sin demostración]%Dijo que no lo demostraria pero si lo probo
	Sea \( f\colon M^m\to N^n \) diferenciable. Sea \( X^k\subseteq N^n \) subvariedad. Supongamos \( f \) submersión en un \( p\in f^{-1}(X) \) cualquiera. Entonces, \( f^{-1}(X) \) es subvariedad de \textit{dim} \( m-(n-k) \) .
\end{theorem}

\begin{example}
	\begin{itemize}
		\item Caso particular \( X=\{q\} \) y \( f \) es submersión en \( f^{-1}(q) \) entonces \( f^{-1}(q) \) es subvariedad de \textit{dim} \( m-n \)
		\item Sea \( f\colon \RealSet^n\to \RealSet \) definida como \( f(x_1,\ldots,x_n)= \sum x_i^2 \) es submersión en \( \RealSet^n-\{0\} \) con \( X= \{1\} \) entonces \( f^{-1}(1)=S^{n-1} \)
	\end{itemize}
\end{example}

\begin{lemma}
	Sea \( f\colon M^m\to N^n \) submersión en \( p \). Para toda carta \( (W,\psi) \) de \( N \) con \( f(p)\in W \) se pueden encontrar cartas \( (U_0,\varphi_0) \) en \( M \) y \( (W_0,\psi_{0}) \) en \( N \) con \( p\in U_0 \), \( f(U_0)\subseteq W_0\subseteq W \) y \( \psi_0=\Restrict{\psi}{W_0} \)  tales que:
	\[
	\varphi_0(U_0)=\psi_{0}(W_0)\times \Omega
	\]
	donde \( \Omega \) es un abierto de \( \RealSet^{n-m} \). Además el siguiente diagrama es conmutativo
	\begin{figure}[h]
		\centering
		\begin{tikzcd}
		U_0 \arrow[r, "f"] \arrow[d, "\varphi_{0}"']
		& W_0 \arrow[d, "\psi_{0}"] \\
		\varphi_0(U_0) \arrow[r, "\psi_0\circ f\circ \varphi^{-1}" ']
		& \psi_0(W_0)
		\end{tikzcd}
		\caption{Diagrama submersión}
		\label{fig:sub}
	\end{figure}
\end{lemma}

\end{document}

%%% Local Variables:
%%% TeX-master: "../VD"
%%% End: