\documentclass[../VD.tex]{subfiles}

\externaldocument{../VD}

\begin{document}

\setcounter{chapter}{5}
\chapter{Campos (de vectores) tangentes}\label{chap:campos}

\section{Introducción}

\begin{definition}[name=campo tangente]\label{def:campo-tangente}
  Dada \(M^{m}\) variedad, un \emph{campo tangente} de \(M\) es una aplicación
  diferenciable \(X\colon M\to\Tangente{M}\) con
  \(\pi\circ X=\Id{M}\) (\(X\) es una sección de \(\pi\colon\Tangente{M}\to M\)).
\end{definition}

\begin{remark}
  Sea \((U,\varphi)\) carta de \(M\), entonces se tiene el diagrama:
  \begin{center}
    \centering
    \begin{tikzcd}[column sep = large]
      U
      \ar[d, "\varphi"']
      \ar[r, "X"]
      &
      \Tangente{U}
      \ar[d, "\widetilde{\varphi}=\varphi\times\tau_{\varphi}^{p}"] \\
      \varphi(U)
      \ar[r, "X_{\varphi}"']
      &
      \ \ \varphi(U)\times\RealSet^{m} \\
    \end{tikzcd}
  \end{center}

  donde \(X_{\varphi}(x)=(\widetilde{\varphi}\circ X\circ
  \varphi^{-1})(x)=(x,g^{U}(x))\) y \(g^{U}:\varphi(U)\to\RealSet^{m}\) se denomina
  \emph{componente local del campo X en la carta \((U,\varphi)\)}.

  Nótese que \(g^{U}(\varphi(p))=(\varphi\circ\alpha)'(0)\) si \(X(p)\) está
  representada por la curva \(\alpha\).

  El campo nulo, \(\theta\colon M\to\Tangente{M}\) es aquel que lleva \(p\in M\)
  en el vector nulo \(\theta_{p}\in\Tangente[p]{M}\) representado por la curva
  constante \(c_{p}(t)=p\) si \(t\in(-\epsilon,\epsilon)\).

  Veamos que ocurre si tomamos otra carta \((V,\psi)\) compatible con
  \((U,\varphi)\), con su correspondiente componente local
  \(g^{V}:\psi(V)\to\RealSet^{m}\), y nos preguntamos qué ocurre en \(U\cap V\)
  (que suponemos no vacío). 

  Montando dos diagramas como el anterior pero para \(U\cap V\), vistos
  mediante las funciones coordenada \(\varphi,\psi\), que sabemos que forman un
  difeomorfismo, se llega a la siguiente relación de compatibilidad (nos
  interesa la última línea):

  \[\begin{array}{l}
    \widetilde{\psi}\widetilde{\varphi}^{-1}X_{\varphi}=X_{\psi}\psi\varphi^{-1}\iff \\
    \widetilde{\psi}\widetilde{\varphi}^{-1}(x,g^{U}(x))=X_{\psi}\psi\varphi^{-1}(x)\iff
      \\
    (\psi\varphi^{-1}(x),\J[x]{\psi\varphi^{-1}}(g^{U}(x)))=(\psi\varphi^{-1}(x),
      g^{V}(\psi\varphi^{-1}(x)))\iff \\
      g^{V}(\psi\varphi^{-1}(x))=\J[x]{\psi\varphi^{-1}}(g^{U}(x))
    \end{array}\]

  Luego dar un campo tangente sobre \(M\) con atlas
  \(\{(U,\varphi)\}=\mathcal{A}\) es dar funciones
  \(g^{U}\colon\varphi(U)\to\RealSet^{m}\) sujetas a la relación de
  compatibilidad anterior.
  
\end{remark}
  
\end{document}

%%% Local Variables:
%%% TeX-master: "../VD"
%%% End: