\documentclass[../VD.tex]{subfiles}

\externaldocument{../VD}

\begin{document}

\setcounter{chapter}{5}
\chapter{Campos (de vectores) tangentes}\label{chap:campos}

\section{Introducción}

\begin{definition}[name=campo tangente]\label{def:campo-tangente}
  Dada \(M^{m}\) variedad, un \emph{campo tangente} de \(M\) es una aplicación
  diferenciable \(X\colon M\to\Tangente{M}\) con
  \(\pi\circ X=\Id{M}\) (\(X\) es una sección de \(\pi\colon\Tangente{M}\to M\)).
\end{definition}

\begin{remark}
  Sea \((U,\varphi)\) carta de \(M\), entonces se tiene el diagrama:
  \begin{center}
    \centering
    \begin{tikzcd}[column sep = large]
      U
      \ar[d, "\varphi"']
      \ar[r, "X"]
      &
      \Tangente{U}
      \ar[d, "\widetilde{\varphi}=\varphi\times\tau_{\varphi}^{p}"] \\
      \varphi(U)
      \ar[r, "X_{\varphi}"']
      &
      \ \ \varphi(U)\times\RealSet^{m}
    \end{tikzcd}
  \end{center}

  donde \(X_{\varphi}(x)=(\widetilde{\varphi}\circ X\circ
  \varphi^{-1})(x)=(x,g^{U}(x))\) y \(g^{U}:\varphi(U)\to\RealSet^{m}\) se denomina
  \emph{componente local del campo X en la carta \((U,\varphi)\)}.

  Nótese que \(g^{U}(\varphi(p))=(\varphi\circ\alpha)'(0)\) si \(X(p)\) está
  representada por la curva \(\alpha\).

  El campo nulo, \(\theta\colon M\to\Tangente{M}\) es aquel que lleva \(p\in M\)
  en el vector nulo \(\theta_{p}\in\Tangente[p]{M}\) representado por la curva
  constante \(c_{p}(t)=p\) si \(t\in(-\epsilon,\epsilon)\).

  Veamos que ocurre si tomamos otra carta \((V,\psi)\) compatible con
  \((U,\varphi)\), con su correspondiente componente local
  \(g^{V}:\psi(V)\to\RealSet^{m}\), y nos preguntamos qué ocurre en \(U\cap V\)
  (que suponemos no vacío). 
  

\end{remark}

\begin{proposition}[name=componentes locales compatibles]\label{prop:comp-local}
  Sean \((U,\varphi),(V,\psi)\) dos cartas del atlas \(\mathcal{A}\) de la
  \(m\)-variedad \(M\). Si \(X\colon M\to\Tangente{M}\) es un campo de
  vectores tangentes, entonces se tiene la siguiente relación de compatibilidad
  entre las componentes locales de \(X\) en \(U\) y \(V\):
  
  \[
    (*) g^{V}(\psi\varphi^{-1}(x))=\J[x]{\psi\varphi^{-1}}(g^{U}(x))
  \]
  
  Recíprocamente, si tenemos un atlas \(\mathcal{A}\) de \(M^{m}\) y una
  colección de funciones diferenciables
  \(g^{U}\colon\varphi(U)\to\RealSet^{m}\ \forall(U,\varphi)\in\mathcal{A}\)
  cumpliendo la relación \((*)\), entonces existe un único campo de vectores
  tangentes con \(\{g^{U}\colon(U,\varphi)\in\mathcal{A}\}\) como componentes locales.
  \end{proposition}
  
  \begin{proof}
    La demostración se deja como ejercicio al lector.

    \underline{Ayuda:}
    Montar dos diagramas como el anterior pero para \(U\cap V\), vistos
    mediante las funciones coordenada \(\varphi\) y \(\psi\) respectivamente, que
    sabemos que forman un difeomorfismo, y llegar a la siguientes relaciones:
    \[\begin{array}{l}
        \widetilde{\psi}\widetilde{\varphi}^{-1}X_{\varphi}=X_{\psi}\psi\varphi^{-1}\iff \\
        \widetilde{\psi}\widetilde{\varphi}^{-1}(x,g^{U}(x))=X_{\psi}\psi\varphi^{-1}(x)\iff
        \\
        (\psi\varphi^{-1}(x),\J[x]{\psi\varphi^{-1}}(g^{U}(x)))=(\psi\varphi^{-1}(x),
        g^{V}(\psi\varphi^{-1}(x)))\iff \\
        g^{V}(\psi\varphi^{-1}(x))=\J[x]{\psi\varphi^{-1}}(g^{U}(x))
      \end{array}\]
    
  \end{proof}
  
  Luego dar un campo tangente sobre \(M\) con atlas
  \(\{(U,\varphi)\}=\mathcal{A}\) es dar funciones
  \(g^{U}\colon\varphi(U)\to\RealSet^{m}\) sujetas a la relación de
  compatibilidad anterior.

  \section{Espacio de campos tangentes}
  
  \begin{remark}\label{obs:espacio-campos}
    Al conjunto de todos los campos tangentes de una variedad \(M\) lo
    denotaremos: \(\mathbb{X}(M)=\{X\colon M\to\Tangente{M}\ \text{campos
      tangentes}\}\)  
  \end{remark}

  \begin{lemma}[name=espacio de campos es
    vectorial]\label{lem:espacio-campos-vectorial}
    Dada una variedad \(M\), el conjunto definido en \cref{obs:espacio-campos} es
    un espacio vectorial.  
  \end{lemma}

  \begin{proof}
    Sean \(X,Y\in\mathbb{X}(M),\lambda\in\RealSet\). Entonces:
    \begin{enumerate}
    \item \((X+Y)(p)=X(p)+Y(p)\)
    \item Neutro: El elemento neutro será el campo nulo \(\theta\colon
      M\to\Tangente{M}\) tal que \(\theta(p)=0\) el vector nulo de
      \(\Tangente[p]{M}\) definido por la curva constante \(c_{p}(t)=p\)
    \item \((\lambda X)(p)=\lambda\cdot X(p)\)
    \end{enumerate}
  \end{proof}

  \begin{lemma}
    \(\mathbb{X}(M)\) espacio vectorial es un \(\mathcal{F}(M)\)-módulo sobre
    \(\mathcal{F}(M)=\{f\colon M\to\RealSet\ \text{diferenciables}\}\), siendo
    la ``multiplicación por escalar'':
    \[
      (f\cdot X)(p)=f(p)\cdot X(p)\ \text{(multiplicación conmutativa, f(p)
        coeficiente)}
    \]
  \end{lemma}

  \section{Campos \(f\)-compatibles o \(f\)-relacionados}

  \begin{definition}[name=f-compatible]\label{def:f-comp}
  Sea  \(f\colon M^{m}\to N^{n}\) diferenciable, \(X,Y\) campos tangentes tales que:

    \begin{center}
      \centering
      \begin{tikzcd}[column sep = large, row sep = mid]
        M
        \ar[d, "X"']
        \ar[r, "f"]
        &
        N
        \ar[d, "Y"] \\
        \Tangente{M}
        \ar[r, "\dif{f}"']
        &
        \Tangente{N}
      \end{tikzcd}
    \end{center}
Entonces diremos que \(X,Y\) son \(f\)-compatibles si
    \(Y\circ f=\dif{f}\circ X \text{(conmuta)}\)
  \end{definition}


  
\end{document}
%%% Local Variables:
%%% TeX-master: "../VD"
%%% End: