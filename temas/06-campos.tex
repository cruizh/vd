\documentclass[../VD.tex]{subfiles}

\externaldocument{../VD}

\begin{document}

\setcounter{chapter}{5}
\chapter{Campos (de vectores) tangentes}\label{chap:campos}

\section{Introducción}

\begin{definition}[name=campo tangente]\label{def:campo-tangente}
  Dada \(M^{m}\) variedad, un \emph{campo tangente} de \(M\) es una aplicación
  diferenciable \(X\colon M\to\Tangente{M}\) con
  \(\pi\circ X=\Id{M}\) (\(X\) es una sección de \(\pi\colon\Tangente{M}\to M\)).
\end{definition}

\begin{remark}
  Sea \((U,\varphi)\) carta de \(M\), entonces se tiene el diagrama:
  \begin{center}
    \centering
    \begin{tikzcd}[column sep = large]
      U
      \ar[d, "\varphi"']
      \ar[r, "X"]
      &
      \Tangente{U}
      \ar[d, "\widetilde{\varphi}=\varphi\times\tau_{\varphi}^{p}"] \\
      \varphi(U)
      \ar[r, "X_{\varphi}"']
      &
      \ \ \varphi(U)\times\RealSet^{m}
    \end{tikzcd}
  \end{center}

  donde \(X_{\varphi}(x)=(\widetilde{\varphi}\circ X\circ
  \varphi^{-1})(x)=(x,g^{U}(x))\) y \(g^{U}:\varphi(U)\to\RealSet^{m}\) se denomina
  \emph{componente local del campo X en la carta \((U,\varphi)\)}.

  Nótese que \(g^{U}(\varphi(p))=(\varphi\circ\alpha)'(0)\) si \(X(p)\) está
  representada por la curva \(\alpha\).

  El campo nulo, \(\theta\colon M\to\Tangente{M}\) es aquel que lleva \(p\in M\)
  en el vector nulo \(\theta_{p}\in\Tangente[p]{M}\) representado por la curva
  constante \(c_{p}(t)=p\) si \(t\in(-\varepsilon,\varepsilon)\).

  Veamos que ocurre si tomamos otra carta \((V,\psi)\) compatible con
  \((U,\varphi)\), con su correspondiente componente local
  \(g^{V}:\psi(V)\to\RealSet^{m}\), y nos preguntamos qué ocurre en \(U\cap V\)
  (que suponemos no vacío).


\end{remark}

\begin{proposition}[name=componentes locales compatibles]\label{prop:comp-local}
  Sean \((U,\varphi),(V,\psi)\) dos cartas del atlas \(\mathcal{A}\) de la
  \(m\)-variedad \(M\). Si \(X\colon M\to\Tangente{M}\) es un campo de
  vectores tangentes, entonces se tiene la siguiente relación de compatibilidad
  entre las componentes locales de \(X\) en \(U\) y \(V\):

  \[
    (*) g^{V}(\psi\varphi^{-1}(x))=\J[x]{\psi\varphi^{-1}}(g^{U}(x))
  \]

  Recíprocamente, si tenemos un atlas \(\mathcal{A}\) de \(M^{m}\) y una
  colección de funciones diferenciables
  \(g^{U}\colon\varphi(U)\to\RealSet^{m}\ \forall(U,\varphi)\in\mathcal{A}\)
  cumpliendo la relación \((*)\), entonces existe un único campo de vectores
  tangentes con \(\{g^{U}\colon(U,\varphi)\in\mathcal{A}\}\) como componentes locales.
\end{proposition}

\begin{proof}
  La demostración se deja como ejercicio al lector.

  \underline{Ayuda:}
  Montar dos diagramas como el anterior pero para \(U\cap V\), vistos
  mediante las funciones coordenada \(\varphi\) y \(\psi\) respectivamente, que
  sabemos que forman un difeomorfismo, y llegar a la siguientes relaciones:
  \[\begin{array}{l}
      \widetilde{\psi}\widetilde{\varphi}^{-1}X_{\varphi}=X_{\psi}\psi\varphi^{-1}\iff \\
      \widetilde{\psi}\widetilde{\varphi}^{-1}(x,g^{U}(x))=X_{\psi}\psi\varphi^{-1}(x)\iff
      \\
      (\psi\varphi^{-1}(x),\J[x]{\psi\varphi^{-1}}(g^{U}(x)))=(\psi\varphi^{-1}(x),
      g^{V}(\psi\varphi^{-1}(x)))\iff \\
      g^{V}(\psi\varphi^{-1}(x))=\J[x]{\psi\varphi^{-1}}(g^{U}(x))
    \end{array}\]
\end{proof}

Luego dar un campo tangente sobre \(M\) con atlas
\(\{(U,\varphi)\}=\mathcal{A}\) es dar funciones
\(g^{U}\colon\varphi(U)\to\RealSet^{m}\) sujetas a la relación de
compatibilidad anterior.

\section{Espacio de campos tangentes}

\begin{remark}\label{obs:espacio-campos}
  Al conjunto de todos los campos tangentes de una variedad \(M\) lo
  denotaremos: \(\mathbb{X}(M)=\{X\colon M\to\Tangente{M}\ \text{campos
    tangentes}\}\)
\end{remark}

\begin{lemma}[name=espacio de campos es
  vectorial]\label{lem:espacio-campos-vectorial}
  Dada una variedad \(M\), el conjunto definido en \cref{obs:espacio-campos} es
  un espacio vectorial.
\end{lemma}

\begin{proof}
  Sean \(X,Y\in\mathbb{X}(M),\lambda\in\RealSet\). Entonces:
  \begin{enumerate}
  \item \((X+Y)(p)=X(p)+Y(p)\)
  \item Neutro: El elemento neutro será el campo nulo \(\theta\colon
    M\to\Tangente{M}\) tal que \(\theta(p)=0\) el vector nulo de
    \(\Tangente[p]{M}\) definido por la curva constante \(c_{p}(t)=p\)
  \item \((\lambda X)(p)=\lambda\cdot X(p)\)
  \end{enumerate}
\end{proof}

\begin{lemma}\label{lem:e-campos-mod}
  \(\mathbb{X}(M)\) espacio vectorial es un \(\mathcal{F}(M)\)-módulo sobre
  \(\mathcal{F}(M)=\{f\colon M\to\RealSet\ \text{diferenciables}\}\), siendo
  la ``multiplicación por escalar'':
  \[
    (f\cdot X)(p)=f(p)\cdot X(p)\ \text{(multiplicación conmutativa, f(p)
      coeficiente)}
  \]
\end{lemma}

\section{Campos \(f\)-compatibles o \(f\)-relacionados}

\begin{figure}[h]
  \centering
  \begin{tikzcd}[column sep = large]
    M
    \ar[d, "X"']
    \ar[r, "f"]
    &
    N
    \ar[d, "Y"] \\
    \Tangente{M}
    \ar[r, "\dif{f}"']
    &
    \Tangente{N}
  \end{tikzcd}
  \captionof{figure}{}
  \label{fig:campos-f-comp}
\end{figure}

\begin{definition}[name=f-compatible]\label{def:f-comp}
  Sea  \(f\colon M^{m}\to N^{n}\) diferenciable, \(X,Y\) campos tangentes en
  \(M\) y \(N\) como los de la \cref{fig:campos-f-comp}. Entonces diremos que
  \(X,Y\) son \(f\)-compatibles si \(Y\circ f=\dif{f}\circ X\) (el diagrama
  conmuta \(\forall p\in M\)).

  Esto es equivalente a decir que si \(\alpha\) es una curva
  representando \(X(p)\) entonces \(f\circ\alpha\) representa el vector
  \(Y(f(p))\).


\end{definition}

\begin{definition}[name=\(f\)-invariante]\label{def:f-invariante}
  Cuando \(f\colon X\to X\) y \(X\) es \(f\)-compatible consigo mismo (es
  decir, \(X=\dif{f}\circ X\circ f^{-1}\)) entonces \(X\) se dice
  \emph{\(f\)-invariante}.
\end{definition}

\begin{proposition}
  También es posible ver la \(f\)-compatibilidad por componentes locales:

  Sean las componentes locales de \(X\),
  \(\{f^{V}\colon\psi(V)\to\RealSet^{m}\}\), respecto a un cierto
  atlas \(\mathcal{A}\) de \(M\). Entonces las componentes locales
  \(\{g^{U}\colon\varphi(U)\to\RealSet^{m}\}\) de \(Y\) (\(f\)-compatible con
  \(X\)) respecto a otro atlas \(\mathcal{B}\) de \(N\) están relacionadas con
  las anteriores por la fórmula:
  \[
    g^{U}(\varphi(f(p)))=\J[\psi(p)]{\varphi\circ f\circ \psi^{-1}}f^{V}(\psi(p))
  \]
  si \((U,\varphi),(V,\psi)\) cartas de \(N\) y \(M\) respectivamente, con
  \(f(V)\subseteq U\).
\end{proposition}

\begin{remark}
  Cabe destacar, que si \(f\) es difeomorfismo y \(X\in\mathbb{X}(M)\) sólo
  existe un campo \(Y\in\mathbb{X}(N)\) \(f\)-compatible y es \(Y=\dif{f}\circ
  X\circ f^{-1}\). Esto induce una aplicación \(f_{*}\) que estudiaremos más
  adelante.
\end{remark}

\begin{proposition}[name=función \(f_{*}\)]
  Si \(f\colon M\to N\) es difeomorfismo, entonces \(f\) induce una aplicación
  \(f_{*}\colon\mathbb{X}(M)\to\mathbb{X}(N)\) con \(f_{*}(X)=\dif{f}\circ
  X\circ f^{-1}\), que es aplicación lineal entre espacios vectoriales (por serlo
  \(\dif{f}\)).
\end{proposition}

Como se vio en \cref{lem:e-campos-mod}, \(\mathbb{X}(M)\) es un
\(\mathcal{F}(M)\)-módulo, y se tiene el siguiente resultado:

\begin{lemma}
  Sea \(f^{\#}\colon\mathcal{F}(N)\to\mathcal{F}(M)\) el homomorfismo de
  anillos dado por la composición \(f^{\#}(g)=g\circ f\) y que cumple:
  \[
    f^{\#}(g\cdot g')=(g\cdot g')(f)=(gf)\cdot(g'f)=f^{\#}(g)\cdot f^{\#}(g')
  \]
  Entonces si \(X\) es \(f\)-compatible con \(Y\) se tiene que \(f^{\#}(g)X\)
  es \(f\)-compatible con \(gY\) para todo \(g\in\mathcal{F}(N)\)
\end{lemma}

\begin{proof}
  En efecto, el diagrama

  \begin{center}
    \centering
    \begin{tikzcd}[column sep = large]
      M
      \ar[d, "f^{\#}(g)X"']
      \ar[r, "f"]
      &
      N
      \ar[d, "gY"] \\
      \Tangente{M}
      \ar[r, "\dif{f}"']
      &
      \Tangente{N}
    \end{tikzcd}
  \end{center}

  es conmutativo pues si \(\alpha\colon(-\varepsilon,\varepsilon)\to M\) es una
  curva representante de \(X(p)\), entonces \(\beta\colon(-\delta,\delta)\to
  M\) con \(\beta(s)=\alpha(gf(p)s)\) (con \(\Abs{gf(p)\delta}<\varepsilon\))
  es representante de \(f^{\#}(g)X(p)\) y \(f\circ\beta\) lo es de
  \(\dif{f}f^{\#}(g)X(p)\).

  Por otro lado, por ser \(X\) \(f\)-compatible con
  \(Y\) tenemos que \(f\circ\alpha\) es una curva representante de \(Y(f(p))\)
  y por tanto se tiene que \(\gamma(s)=f\circ\alpha(g(f(p))s)=f\circ\beta(s)\) es
  representante de \(gY(f(p))\), esto es, \(f^{\#}(g)X\) es \(f\)-compatible
  con \(gY\).
\end{proof}

\begin{corollary}
  Como consecuencia del lema anterior, se tiene que
  \(f_{*}\colon\mathbb{X}(M)\to\mathbb{X}(N)\) es un homomorfismo de
  \(\mathcal{F}(N)\)-módulos cumpliéndose \(f_{*}(f^{\#}(g)X)=gf_{*}X\).
\end{corollary}

\begin{proof}
  Si consideramos la operación de \(\mathcal{F}(N)\) sobre \(\mathbb{X}(M)\)
  reflejada por \(f^{\#}\); esto es si \(g\in\mathcal{F}(N)\) y
  \(X\in\mathbb{X}(M)\), entonces \(g\star X=f^{\#}(g)X\) y se prueba que
  \(f_{*}\colon\mathbb{X}(M)\to\mathbb{X}(N)\) es un homomorfismo de
  \(\mathcal{F}(N)\)-modulos.
\end{proof}

\begin{example}[name=campo invariante a la izquierda o derecha de un grupo de
  Lie, G]

  Por definición son los campos \(L_{g}\)-invariantes (\(R_{g}-invariantes\)) para
  \(L_{g}(x)=gx\) (\(R_{g}(x)=xg\)) \(\forall g\in G\).

  Además, \(g=L_{g}(e)\) y se tiene:

  \begin{center}
    \centering
    \begin{tikzcd}[column sep = large]
      e
      \ar[d, "X"']
      \ar[r, "L_{g}"]
      &
      g
      \ar[d, "X"] \\
      \Tangente[e]{G}\ni X(e) \ \ \ \ \ \
      \ar[r, "\dif{L_{g}}"']
      &
      \ \ \ \ \ \ X(g)\in\Tangente[g]{G}
    \end{tikzcd}
  \end{center}
  entonces \(X(g)=\dif{L_{g}}X(e)\) y:
  \[
    \begin{array}{rcl}
      \mathbb{X}_{L}(G) & \cong & \Tangente[e]{G} \\
      X & \mapsto & X(e) \\
      X(g)=\dif{L_{g}}(v) & \leftarrow & v
    \end{array}
  \]
\end{example}

\end{document}
%%% Local Variables:
%%% TeX-master: "../VD_ebook"
%%% End: