\documentclass[../VD.tex]{subfiles}

\externaldocument{../VD}

\begin{document}

\setcounter{chapter}{7}
\chapter{Álgebra de Lie}\label{chap:algebra}

\section{Introducción}

\begin{definition}\label{def:alglie}
  Un \emph{álgebra de Lie}, \(\mathcal{L}\), es un espacio vectorial dotado de
  otra operación interna, que llamaremos \emph{producto corchete},
  \(\Clase{\cdot,\cdot}\colon\mathcal{L}\times\mathcal{L}\to\mathcal{L}\) que
  verifica:
  \begin{enumerate}
  \item Es bilineal
  \item Es antisimétrica: \(\Clase{a,b}=-\Clase{b,a}\)
  \item La identidad de Jacobi (la suma cíclica es nula):
    \(\Clase{\Clase{a,b},c}+\Clase{\Clase{b,c},a}+\Clase{\Clase{c,a},b}=0,\
    \forall a,b,c\in\mathcal{L}\)
  \end{enumerate}
\end{definition}

\begin{remark}
  De la antisimetría: \(\Clase{a,a}=-\Clase{a,a}=0,\ \forall
  a\in\mathcal{L}\)

  Si \(\mathcal{L}\) es un espacio vectorial finito, a los productos corchetes
  \(\Clase{e_{i},e_{j}}=\sum\limits_{i=1}^{m}c_{ij}^{k}e_{k}\) de cada par de
  vectores de la base de \(\mathcal{L}\) se les llama constantes de estructura
  del álgebra con respecto a la base del espacio.

  En nuestro caso, no serán espacios finitos y no se tendrá este resultado.
\end{remark}

\begin{example}
  Algunos ejemplos de álgebras de Lie:
  
  \begin{itemize}
  \item \((\RealSet^{3},\wedge)\): se tiene que \(\Clase{i,j}=i\wedge
    j=k,\Clase{j,k}=i,\Clase{k,i}=j\) por lo que se cumple la identidad de
    Jacobi, \(\Clase{\Clase{i,j},k}=\Clase{k,k}=0\) y así las otras dos,
    siendo la suma cíclica nula. 
  \item \(\mathcal{M}_{n}(\RealSet)\) con \(\Clase{A,B}=AB-BA\)
  \end{itemize}
\end{example}

\section{\(\mathbb{X}(M)\) como Álgebra de Lie}

Queremos probar que \(\mathbb{X}(M)\) es un álgebra de Lie. Consideraremos los
campos como derivaciones:

\begin{proposition}
  Sean \(X,Y\in\mathbb{X}(M)\), entonces \(D_{X}D_{Y}-D_{Y}D_{X}\) también es
  una derivación. Además, \(\Clase{D_{X},D_{Y}}=D_{X}D_{Y}-D_{Y}D_{X}\) es su
  producto corchete asociado. 
\end{proposition}

\begin{proof}
  Sean \(f,g\in\mathcal{F}(M)\). Entonces, por un lado, \(D_{X}D_{Y}(fg)=\)
  \[\begin{array}{l}
      D_{X}\Clase{D_{Y}(f)g+fD_{Y}(g)}=D_{X}(D_{Y}(f)g)+D_{X}(fD_{Y}(g))\\
      =(D_{X}D_{Y}(f))g+D_{Y}(f)D_{X}(g)+D_{X}(f)D_{Y}(g)+f(D_{X}D_{Y}(g))
    \end{array}\]

  A su vez, \(D_{Y}D_{X}(fg)=\)
  \[\begin{array}{l}
      (D_{Y}D_{X}(f))g+D_{X}(f)D_{Y}(g)+
      D_{Y}(f)D_{X}(g)+f(D_{Y}D_{X}(g))
    \end{array}\]

  Si a la primera expresión le restamos la segunda, obtenemos que 
  \(D_{X}D_{Y}(fg)-D_{Y}D_{X}(fg)=\)
  \[
    \Clase{(D_{X}D_{Y}(f))-(D_{Y}D_{X}(f))}g+
    f\Clase{D_{X}D_{Y}(g)-D_{Y}D_{X}(g)}
  \]
  Luego, es \textbf{derivación} (y por ello \textbf{bilineal}). Además, es
  \textbf{antisimétrico}:
  \[\Clase{D_{Y},D_{X}}=D_{Y}D_{X}-D_{X}D_{Y}=-(D_{X}D_{Y}-D_{Y}D_{X})=
    -\Clase{D_{X},D_{Y}}\]

  Veamos ahora la identidad de Jacobi: Sean \(D_{X},D_{Y},D_{Z}\). Entonces:
  \begin{itemize}
  \item
    \(\Clase{\Clase{D_{X},D_{Y}},D_{Z}}=\Clase{D_{X}D_{Y}-D_{Y}D_{X},D_{Z}}\).
    Como es bilineal, lo anterior se corresponde con
    \(
    \Clase{D_{X}D_{Y},D_{Z}}-\Clase{D_{Y}D_{X},D_{Z}}=D_{X}D_{Y}D_{Z}-D_{Z}D_{X}D_{Y}-
    D_{Y}D_{X}D_{Z}+D_{Z}D_{Y}D_{X}
    \)
  \item
    \(\Clase{\Clase{D_{Y},D_{Z}},D_{X}}=D_{Y}D_{Z}D_{X}-D_{X}D_{Y}D_{Z}-D_{Z}D_{Y}D_{X}
    +D_{X}D_{Z}D_{Y}\)
  \item
    \(\Clase{\Clase{D_{Z},D_{X}},D_{Y}}=D_{Z}D_{X}D_{Y}-D_{Y}D_{Z}D_{X}-D_{X}D_{Z}D_{Y}
    +D_{Y}D_{X}D_{Z}\)
  \end{itemize}
  Claramente, la suma cíclica de las expresiones anteriores se anula.

  Luego nuestra operación es un \emph{corchete de Lie} y entonces
  \(\mathbb{X}(M)\) es un \emph{álgebra de Lie}.
\end{proof}

\begin{proposition}
  Sea \(\mathcal{A}=\{(U,\varphi)\}\) atlas de \(M\). Si \(\{f^{U}\},\{g^{U}\}\)
  denotan las funciones componentes de los campos \(X,Y\) respectivamente,
  entonces el campo correspondiente a \(\Clase{D_{X},D_{Y}}(\Clase{X,Y})\) tiene
  como funciones componentes locales a
  \(\Clase{f^{U},g^{U}}\colon\varphi(U)\to\RealSet^{m}\) cuya i-ésima función
  coordenada es
  \(\Clase{f^{U},g^{U}}_{i}=\sum\limits_{j=1}^{m}(f_{j}^{U}\pderiv{g_{i}^{U}}{x_{j}}-
  \pderiv{f_{i}^{U}}{x_{j}}g_{j}^{U})\)
\end{proposition}

\begin{proof}
  En \((U,\varphi)\),
  \[\begin{array}{l}
      D_{X}(h)=\sum\limits_{i=1}^{m}\pderiv{h\circ\varphi^{-1}}{x_{i}}f_{i}^{U}
      \\
      D_{Y}(h)=\sum\limits_{i=1}^{m}\pderiv{h\circ\varphi^{-1}}{x_{i}}g_{i}^{U}
    \end{array}\]

  Entonces,
  \[\begin{array}{l}
      D_{X}D_{Y}(h)=D_{X}(\sum\limits_{i=1}^{m}\pderiv{h\circ\varphi^{-1}}{x_{i}}g_{i}^{U})
      =\sum\limits_{j=1}^{m}\pderiv{(
      \sum\limits_{i=1}^{m}\pderiv{h\circ\varphi^{-1}}{x_{i}}g_{i}^{U})}{x_{j}}f_{j}^{U}
      \\
      =\sum\limits_{i,j=1}^{m}\Clase{\pderiv{^{2}h\circ\varphi^{-1}}{x_{j}\partial
      x_{i}}g_{i}^{U}f_{j}^{U}+\pderiv{h\circ\varphi^{-1}}{x_{i}}\pderiv{g_{i}^{U}}{x_{j}}
      f_{j}^{U}}
    \end{array}\]

  A su vez,
  \[\begin{array}{l}
      D_{Y}D_{X}(h)=\sum\limits_{i,j=1}^{m}\Clase{\pderiv{^{2}h\circ\varphi^{-1}}{x_{i}
      \partial x_{j}}f_{i}^{U}g_{j}^{U}+\pderiv{h\circ\varphi^{-1}}{x_{i}}
      \pderiv{f_{i}^{U}}{x_{j}}g_{j}^{U}}
    \end{array}\]
  Por tanto, restando la segunda expresión a la primera,
  \[\begin{array}{l}
      \Clase{D_{X},D_{Y}}(h)=(D_{X}D_{Y}-D_{Y}D_{X})(h)= \\
      \sum\limits_{i=1}^{m}\pderiv{h\circ\varphi^{-1}}{x_{i}}(\sum\limits_{j=1}^{m}
      \pderiv{g_{i}^{U}}{x_{j}}f_{j}^{U}-\pderiv{f_{i}^{U}}{x_{j}}g_{j}^{U})= \\
      \sum\limits_{i=1}^{m}\pderiv{h\circ\varphi^{-1}}{x_{i}}(\text{coordenada
      i-ésima})= \\
      \sum\limits_{i=1}^{m}\pderiv{h\circ\varphi^{-1}}{x_{i}}\Clase{f^{U},g^{U}}_{i}
    \end{array}\]
\end{proof}

\section{Campos h-compatibles}

Si \(X,Y\in\mathbb{X}(M),X',Y'\in\mathbb{X}(N)\), \(X,X'\) h-compatibles,
\(Y,Y'\) también, ¿entonces \(\Clase{X,Y}\) h-compatible con \(\Clase{X',Y'}\)?

\begin{lemma}
  Si \(X\in\mathbb{X}(M),Y\in\mathbb{X}(N)\) son h-compatibles, entonces
  \(\forall p\in M,\ \forall f\in\mathcal{F}(N)\) se verifica que
  \(D_{X}(f\circ h)(p)=D_{Y}(f)(h(p))\). Es decir, \(D_{X}(f\circ
  h)=D_{Y}(f)\circ h\)
\end{lemma}

\begin{proof}
  Sean \(X,Y\) campos h-compatibles.
  Entonces por definición: \(dh_{p}X(p)=Y(h(p))\),
  donde \(dh_{p}X(p)\) se corresponde con la derivación de gérmenes para el germen
  determinado por \(f\): \(\delta_{h(p)}(\Clase{f})=D_{X}(\widetilde{f}\circ
  h)(p)=D_{X}(f\circ h)(p)\) (pues \(f\) está definida en todo \(N\) y no hace
  falta extenderla).

  También \(Y(h(p))\) se corresponde con
  \(\delta'_{h(p)}(\Clase{f})=D_{Y}(f)(h(p))\).

  Por tanto, llegamos a que
  \[\begin{array}{l}
      D_{X}(f\circ h)(p)=dh_{p}X(p)=Y(h(p))=D_{Y}(f)(h(p))
    \end{array}\]
\end{proof}

\begin{proposition}
  Si \(X,X'\) son h-compatibles (es decir, se cumple \(D_{X}(f\circ h)=D_{X'}(f)\circ
  h\)), e \(Y,Y'\) también (\(D_{Y}(f\circ h)=D_{Y'}(f)\circ h\)), entonces
  \(\Clase{X,Y},\Clase{X',Y'}\) h-compatibles.
\end{proposition}

\begin{proof}
  Aplicando el lema anterior, lo que tenemos que probar es que
  \(\Clase{D_{X},D_{Y}}(f\circ h)=\Clase{D_{X'},D_{Y'}}(f)\circ h\) para ver que
  son h-compatibles.

  Por un lado,
  \[\begin{array}{ll}
      \Clase{D_{X},D_{Y}}(f\circ h)&=D_{X}(D_{Y}(f\circ h))-D_{Y}(D_{X}(f\circ
      h))\\ &=D_{X}(D_{Y'}(f)\circ h)-D_{Y}(D_{X'}(f)\circ h)\\
      &=D_{X}(f^{Y'}\circ h)-D_{Y}(f^{X'}\circ h)\\
      &=D_{X'}(f^{Y'})\circ h -
        D_{Y'}(f^{X'})\circ h \\
      &=\Clase{D_{X'},D_{Y'}}(f)\circ h      
    \end{array}\]
  donde hemos usado que tanto \(D_{Y'}(f)\) como \(D_{X'}(f)\) son a su vez
  funciones a las que podemos aplicar el resultado anterior de h-compatibilidad.
\end{proof}
\end{document}
%%% Local Variables:
%%% TeX-master: "../VD"
%%% End: