\providecommand{\main}{..}
\documentclass[\main/VD_completo.tex]{subfiles}

\externaldocument{\main/VD}
\externaldocument{\main/temas/00-intro}
\externaldocument{\main/temas/01-cartas}
\externaldocument{\main/temas/02-aplicaciones}
\externaldocument{\main/temas/03-subvariedades}
\externaldocument{\main/temas/04-tangente}
\externaldocument{\main/temas/05-diferencial}
\externaldocument{\main/temas/06-campos}
\externaldocument{\main/temas/07-campos2}
% \externaldocument{\main/temas/08-algebralie}
\externaldocument{\main/temas/09-cotangente}
\externaldocument{\main/temas/10-tensores}


\begin{document}

\setcounter{chapter}{7}
\chapter{El álgebra de Lie de una variedad}\label{chap:algebra}

\section{Álgebras de Lie}

\begin{definition}\label{def:alglie}
  Un \emph{álgebra de Lie}, \(\mathcal{L}\), es un espacio vectorial dotado de
  otra operación interna, que llamaremos \emph{producto corchete},
  \(\ProdC{\cdot}{\cdot} \colon \mathcal{L} \times \mathcal{L} \to \mathcal{L}\)
  que verifica para todo \(a,b,c\in \mathcal{L}\):
  \begin{enumerate}
  \item Es bilineal.
  \item Es antisimétrica: \(\ProdC{a}{b}=-\ProdC{b}{a}\).
  \item Cumple la identidad de Jacobi:
    \[
      \ProdC{\ProdC{a}{b}}{c} +
      \ProdC{\ProdC{b}{c}}{a} +
      \ProdC{\ProdC{c}{a}}{b} = 0.
    \]
  \end{enumerate}
\end{definition}

\begin{remark}
  De la antisimetría se sigue la igualdad: \(\ProdC{a}{a}=-\ProdC{a}{a}=0\) 
  para todo \(a\in\mathcal{L}\).

  Si \(\mathcal{L}\) es un espacio vectorial de dimensión \(n\) con base \(\{e_i\}_{i=1}^n\), las coordenadas de los productos corchetes de cada par de
  vectores de la base respecto a esa misma base
  \(\ProdC{e_{i}}{e_{j}}=\sum_{i=1}^{m}c_{ij}^{k}e_{k}\), se les llama \emph{constantes de estructura}
  de \(\mathcal{L}\) con respecto a la base dada.
\end{remark}

\begin{example}
Tenemos los dos siguientes ejemplos de álgebras de Lie:
  
  \begin{itemize}
  \item Para el producto vectorial, \((\RealSet^{3},\wedge)\) es un álgebra de Lie. En efecto, se tiene que \(\ProdC{i}{j}=i\wedge
    j=k,\ProdC{j}{k}=i,\ProdC{k}{i}=j\) por lo que se cumple la identidad de
    Jacobi, \(\ProdC{\ProdC{i}{j}}{k}=\ProdC{k}{k}=0\). Análogamente las otras dos. 
  \item El espacio vectorial de matrices cuadradas \(\mathcal{M}_{n}(\RealSet)\) es un álgebra de Lie para el producto corchete \(\ProdC{A}{B}=AB-BA\).
  \end{itemize}
\end{example}

\section{\(\mathbb{X}(M)\) como Álgebra de Lie}

Queremos probar que para cualquier variedad \(M\), el espacio vectorial de los campos sobre \(M\), \(\mathbb{X}(M)\), es un álgebra de Lie, llamada el \emph{álgebra de Lie de \(M\)}. Consideraremos los
campos como derivaciones. Entonces se tiene el siguiente producto corchete sobre \(\mathbb{X}(M)\).

\begin{proposition}
  Sean \(X,Y\in\mathbb{X}(M)\), entonces \(D_{X}D_{Y}-D_{Y}D_{X}\) también es
  una derivación. Además, \(\ProdC{D_{X}}{D_{Y}}=D_{X}D_{Y}-D_{Y}D_{X}\) es un
  producto corchete. 
\end{proposition}

\begin{proof}
  Sean \(f,g\in\mathcal{F}(M)\). Entonces, por un lado,
  \begin{align*}
    D_{X}D_{Y}(fg) &= D_{X}(D_{Y}(f)g+fD_{Y}(g))\\
    &=D_{X}(D_{Y}(f)g)+D_{X}(fD_{Y}(g))\\
                   &=(D_{X}D_{Y}(f))g + D_{Y}(f)D_{X}(g)\\
                   &+ D_{X}(f)D_{Y}(g) + f(D_{X}D_{Y}(g)).
  \end{align*}
  A su vez,
  \begin{align*}
    D_{Y}D_{X}(fg) &= (D_{Y}D_{X}(f))g + D_{X}(f)D_{Y}(g)\\
                   &+ D_{Y}(f)D_{X}(g)+f(D_{Y}D_{X}(g)).
  \end{align*}

  Si a la primera expresión le restamos la segunda, obtenemos que 
  \begin{align*}
    D_{X}D_{Y}(fg)-D_{Y}D_{X}(fg)
    &= ((D_{X}D_{Y}(f))-(D_{Y}D_{X}(f)))g\\
    &+ f(D_{X}D_{Y}(g)-D_{Y}D_{X}(g)).
  \end{align*}
  
  Luego, es derivación (y por ello \([ \ , \ ]\) es bilineal). Además, es antisimétrica:
  
  \[
    \ProdC{D_{Y}}{D_{X}} =
    D_{Y}D_{X}-D_{X}D_{Y} =
    -(D_{X}D_{Y}-D_{Y}D_{X}) =
    -\ProdC{D_{X}}{D_{Y}}.
  \]

  Veamos ahora la identidad de Jacobi: Sean \(D_{X},D_{Y},D_{Z}\). Entonces:
  \begin{itemize}
  \item
    \(\ProdC{\ProdC{D_{X}}{D_{Y}}}{D_{Z}}=\ProdC{D_{X}D_{Y}-D_{Y}D_{X}}{D_{Z}}\).

    Desarrollando cada uno de los sumandos, tenemos: 
    \begin{align*}
      \ProdC{D_{X}D_{Y}}{D_{Z}}-\ProdC{D_{Y}D_{X}}{D_{Z}}
      &= D_{X}D_{Y}D_{Z}-D_{Z}D_{X}D_{Y}\\
      &- D_{Y}D_{X}D_{Z}+D_{Z}D_{Y}D_{X}.
    \end{align*}
  
  \item
    \begin{align*}
      \ProdC{\ProdC{D_{Y}}{D_{Z}}}{D_{X}}
      &= D_{Y}D_{Z}D_{X} - D_{X}D_{Y}D_{Z}\\
      &- D_{Z}D_{Y}D_{X} +D_{X}D_{Z}D_{Y}.
    \end{align*}
  \item
    \begin{align*}
      \ProdC{\ProdC{D_{Z}}{D_{X}}}{D_{Y}}
      &= D_{Z}D_{X}D_{Y} - D_{Y}D_{Z}D_{X}\\
      &- D_{X}D_{Z}D_{Y} + D_{Y}D_{X}D_{Z}.
    \end{align*}
  \end{itemize}
  Claramente, la suma de las tres expresiones anteriores se anula. Así pues, la operación anterior es efectivamente un \emph{corchete de Lie}.
\end{proof}

A partir de ahora, el campo correspondiente al corchete \([D_X,D_Y] = D_XD_Y-D_Y D_X\) se le denota 
\([X,Y]\) y se llama el \emph{corchete de Lie} de \(X\) e \(Y\). Esto es, \(D_{[X,Y]} = [D_X,D_Y]\). 

\begin{proposition}
  Sea \(\mathcal{A}=\{(U,\varphi)\}\) atlas de la variedad \(M\). Si \(\{f^{U}\},\{g^{U}\}\)
  denotan las funciones componentes de los campos \(X,Y\) respectivamente,
  entonces el campo \([X,Y]\) tiene
  como funciones componentes locales a
  \(\ProdC{f^{U}}{g^{U}}\colon\varphi(U)\to\RealSet^{m}\) con i-ésima función
  coordenada 
  \(\ProdC{f^{U}}{g^{U}}_{i}=\sum_{j=1}^{m}(f_{j}^{U}\pderiv{g_{i}^{U}}{x_{j}}-
  \pderiv{f_{i}^{U}}{x_{j}}g_{j}^{U})\).
\end{proposition}

\begin{proof}
  Para la carta \((U,\varphi)\) tenemos, por la Definición \ref{defderivacioper},

  \begin{align*}
    D_{X}(h)&=\sum_{i=1}^{m}\pderiv{h\circ\varphi^{-1}}{x_{i}}f_{i}^{U} \mbox{ y}\\
    D_{Y}(h)&=\sum_{i=1}^{m}\pderiv{h\circ\varphi^{-1}}{x_{i}}g_{i}^{U}.
  \end{align*}
  
  Entonces,
  \begin{align*}
    D_{X}D_{Y}(h)
    &= D_{X}(\sum_{i=1}^{m}\pderiv{h\circ\varphi^{-1}}{x_{i}}g_{i}^{U})\\
    &= \sum_{j=1}^{m}\pderiv{(
      \sum_{i=1}^{m}\pderiv{h\circ\varphi^{-1}}{x_{i}}g_{i}^{U})}{x_{j}}f_{j}^{U}
    \\
    &= \sum_{i,j=1}^{m}\left[ \pderiv{h\circ\varphi^{-1}}{x_{j},x_{i}} g_{i}^{U}f_{j}^{U}+\pderiv{h\circ\varphi^{-1}}{x_{i}}\pderiv{g_{i}^{U}}{x_{j}}
      f_{j}^{U}. \right]
  \end{align*}
  
  A su vez,
  \[
    D_{Y}D_{X}(h)
    = \sum_{i,j=1}^{m} \left[
      \pderiv{h\circ\varphi^{-1}}{x_{i},x_{j}}
      f_{i}^{U}g_{j}^{U} +
      \pderiv{h\circ\varphi^{-1}}{x_{i}}
      \pderiv{f_{i}^{U}}{x_{j}}g_{j}^{U}.
      \right]
  \]
 
  Por tanto, restando la segunda expresión a la primera,
  \begin{align*}
    \sum_{i=1}^{m}\pderiv{h\circ\varphi^{-1}}{x_{i}}\ProdC{f^{U}}{g^{U}}_{i} =\ProdC{D_{X}}{D_{Y}}(h)
    &= (D_{X}D_{Y}-D_{Y}D_{X})(h)\\
    &= \sum_{i=1}^{m}\pderiv{h\circ\varphi^{-1}}{x_{i}}\left(\sum_{j=1}^{m}
      \pderiv{g_{i}^{U}}{x_{j}}f_{j}^{U}-\pderiv{f_{i}^{U}}{x_{j}}g_{j}^{U} \right).\\
     \end{align*}
De aquí se sigue inmediatamente la igualdad requerida. 
\end{proof}

\section{Corchete de Lie de campos \(h\)-compatibles}

Sean \(h: M \to N\) una aplicación diferenciable y \(X,Y\in\mathbb{X}(M),X',Y'\in\mathbb{X}(N)\) campos tales que \(X\) y \(X'\) son \(h\)-compatibles y también lo son \(Y\) e \(Y'\). Se tiene entonces que 
\(\ProdC{X}{Y}\) es \(h\)-compatible con \(\ProdC{X'}{Y'}\). Para probarlo empezamos con el siguiente lema. 

\begin{lemma}\label{lemacomp} 
Los campos $X\in \mathbb{X}(M)\) e  \(X'\in\mathbb{X}(N)\) son \(h\)-compatibles si y sólo si 
  para todo \(p\in M\) y para toda \(f\in\mathcal{F}(N)\) se verifica que
  \(D_{X}(f\circ h)(p)=D_{X'}(f)(h(p))\). Es decir, \(D_{X}(f\circ
  h)=D_{X'}(f)\circ h\)
\end{lemma}

\begin{proof}
La condición de compatibilidad nos da \(dh_{p}X(p)=X'(h(p))\) para todo \(p\). El vector 
\(X'(h(p))\) se corresponde con la derivación \(\delta'_{h(p)}\) en \(h(p)\) que vale sobre un germen 
\([g]\) en \(h(p)\) \(D_{X'}(\widetilde{g})(h(p))\), siendo \(\widetilde{g}: N \to \RealSet\) una extensión del representante  \(g\). En particular, para \(f\) tenemos 
\(\delta'_{h(p)}([f]) = D_{X'}(f)(h(p))\). 
\par
Por otro lado, la derivación \(\delta'_{h(p)}\) es la imagen por la diferencial en \(p\) de la derivación correspondiente al vector \(X(p)\), \(\delta_{p}([\gamma]) = D_X(\widetilde{\gamma})(p)\). Entonces por REFERENCIA BUSCAR EN CAPITULO7 \((dh_p\delta_p)([g])= \delta_p(\widetilde{g}\circ h)\). En particular, para \(f\) tenemos 
\[
D_{X'}(f)(h(p)) = \delta'_{h(p)}([f]) = (dh_p\delta_p)([f])= \delta_p(f\circ h) = D_X(f\circ h)(p).
\]
El recíproco es similar. 
\end{proof}

\begin{proposition}\label{corchcomp}
  Si \(X,X'\) son \(h\)-compatibles  e \(Y,Y'\) también, entonces
  los corchetes \(\ProdC{X}{Y}\) y \(\ProdC{X'}{Y'}\) son \(h\)-compatibles.
\end{proposition}

\begin{proof}
  Aplicando el Lema \ref{lemacomp}, lo que tenemos que probar es que
  \(\ProdC{D_{X}}{D_{Y}}(f\circ h)=\ProdC{D_{X'}}{D_{Y'}}(f)\circ h\).

  Ahora bien, tenemos,
  \begin{align*}
    \ProdC{D_{X}}{D_{Y}}(f\circ h)
    &= D_{X}(D_{Y}(f\circ h)) - D_{Y}(D_{X}(f \circ h))\\
    &=D_{X}(D_{Y'}(f)\circ h)-D_{Y}(D_{X'}(f)\circ h)\\
    &=D_{X'}(D_{Y'}(f)) \circ h -D_{Y}(D_{X'}(f))\circ h\\
    &=\ProdC{D_{X'}}{D_{Y'}}(f)\circ h
  \end{align*}
 donde hemos usado que tanto \(D_{Y'}(f)\) como \(D_{X'}(f)\) son a su vez
 funciones a las que podemos aplicar el Lema \ref{lemacomp}.
\end{proof}

\begin{example}
Si \(G\) es un grupo de Lie, el subespacio \(\mathbb{X}_L(G) \subseteq \mathbb{X}(G)\) de los campos invariantes a izquierda de \(G\) es cerrado para el corchete de Lie de acuerdo con la Proposición \ref{corchcomp}. A \(\mathbb{X}_L(G)\) se le llama el \emph{álgebra de Lie del grupo  \(G\)}. 
\par
Todo homomorfismo de grupos de Lie \(f: G \to H\) induce un homomorfismo de álgebras de Lie
\(f_*: \mathbb{X}_L(G) \to \mathbb{X}_L(H)\). En efecto, si \(e\in G\) y \(e'\in H\) son los elementos neutros, sea \(df_e: T_eG \to T_{e'}H$ la diferencial en \(e\). Entonces dado \(X\in \mathbb{X}_L(G)\) se define $f_*X$ como el campo \(f_*X(y)= dl_ydf_e(X(e))\) para todo \(y\in H\). Esto es, el campo invariante a izquierda definido por el vector 
\(df_e(X(e)) \in T_{e'}H\). Además \(X\) y \(f_*X\) son \(f\)-compatibles. En efecto, para todo \(\in G\) 
\[
f_*X(f(x)) = dl_{f(x)}df_e(X(e)) = d(l_{f(x)}\circ f)_e(X(e)) \stackrel{(1)}{=} d(f\circ l_x)_e(X(e)) = df_x\circ (dl_x)_e (X(e)) = df_x\circ X(x),    
\]   
donde (1) sigue de ser \(f\) homomorfismo. Hemos probado que $X$ y $f_*X$ son $f$-compatibles.
\par
Ahora si $Y\in \mathbb{X}(G)$ es otro campo invariante a izquierda, entonces \(Y\) y  \(f_*Y\) son 
\(f\)-compatibles y, por la Proposición \ref{corchcomp},  \([X,Y]\) y \([f_*X,f_*Y]\) también son \(f\)-compatibles. Por tanto, 
\[
f_*[X,Y](e') = df_e([X,Y](e)) = [f_*X,f_*Y](e'). 
\]    
Puesto que los campos en \(\mathbb{X}_L(H)$ están determinados por su valor en $e'$, se sigue entonces $f_*[X,Y] = [f_*X,f_* Y]$. 
\end{example}
\end{document}

%%% Local Variables:
%%% TeX-master: "../VD_ebook"
%%% End: