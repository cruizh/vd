\documentclass[../VD.tex]{subfiles}

\externaldocument{../VD}

\begin{document}

\setcounter{chapter}{7}
\chapter{Álgebra de Lie}\label{chap:algebra}

\section{Introducción}

\begin{definition}\label{def:alglie}
  Un \emph{álgebra de Lie}, \(\mathcal{L}\), es un espacio vectorial dotado de
  otra operación interna, que llamaremos \emph{producto corchete},
  \(\ProdC{\cdot}{\cdot} \colon \mathcal{L} \times \mathcal{L} \to \mathcal{L}\)
  que verifica:
  \begin{enumerate}
  \item Es bilineal
  \item Es antisimétrica: \(\ProdC{a}{b}=-\ProdC{b}{a}\)
  \item La identidad de Jacobi (la suma cíclica es nula):
    \[
      \ProdC{\ProdC{a}{b}}{c} +
      \ProdC{\ProdC{b}{c}}{a} +
      \ProdC{\ProdC{c}{a}}{b}
      = 0 \ \forall a,b,c \in \mathcal{L}
    \]
  \end{enumerate}
\end{definition}

\begin{remark}
  De la antisimetría: \(\ProdC{a}{a}=-\ProdC{a}{a}=0\ \forall a\in\mathcal{L}\)

  Si \(\mathcal{L}\) es un espacio vectorial finito, a los productos corchetes
  \(\ProdC{e_{i}}{e_{j}}=\sum_{i=1}^{m}c_{ij}^{k}e_{k}\) de cada par de
  vectores de la base de \(\mathcal{L}\) se les llama constantes de estructura
  del álgebra con respecto a la base del espacio.

  En nuestro caso, no serán espacios finitos y no se tendrá este resultado.
\end{remark}

\begin{example}
  Algunos ejemplos de álgebras de Lie:
  
  \begin{itemize}
  \item \((\RealSet^{3},\wedge)\): se tiene que \(\ProdC{i}{j}=i\wedge
    j=k,\ProdC{j}{k}=i,\ProdC{k}{i}=j\) por lo que se cumple la identidad de
    Jacobi, \(\ProdC{\ProdC{i}{j}}{k}=\ProdC{k}{k}=0\) y así las otras dos,
    siendo la suma cíclica nula. 
  \item \(\mathcal{M}_{n}(\RealSet)\) con \(\ProdC{A}{B}=AB-BA\)
  \end{itemize}
\end{example}

\section{\(\mathbb{X}(M)\) como Álgebra de Lie}

Queremos probar que \(\mathbb{X}(M)\) es un álgebra de Lie. Consideraremos los
campos como derivaciones:

\begin{proposition}
  Sean \(X,Y\in\mathbb{X}(M)\), entonces \(D_{X}D_{Y}-D_{Y}D_{X}\) también es
  una derivación. Además, \(\ProdC{D_{X}}{D_{Y}}=D_{X}D_{Y}-D_{Y}D_{X}\) es su
  producto corchete asociado. 
\end{proposition}

\begin{proof}
  Sean \(f,g\in\mathcal{F}(M)\). Entonces, por un lado,
  \begin{align*}
    D_{X}D_{Y}(fg) &= D_{X}(D_{Y}(f)g+fD_{Y}(g))\\
    &=D_{X}(D_{Y}(f)g)+D_{X}(fD_{Y}(g))\\
                   &=(D_{X}D_{Y}(f))g + D_{Y}(f)D_{X}(g)\\
                   &+ D_{X}(f)D_{Y}(g) + f(D_{X}D_{Y}(g))
  \end{align*}
  A su vez,
  \begin{align*}
    D_{Y}D_{X}(fg) &= (D_{Y}D_{X}(f))g + D_{X}(f)D_{Y}(g)\\
                   &+ D_{Y}(f)D_{X}(g)+f(D_{Y}D_{X}(g))
  \end{align*}

  Si a la primera expresión le restamos la segunda, obtenemos que 
  \begin{align*}
    D_{X}D_{Y}(fg)-D_{Y}D_{X}(fg)
    &= ((D_{X}D_{Y}(f))-(D_{Y}D_{X}(f)))g\\
    &+ f(D_{X}D_{Y}(g)-D_{Y}D_{X}(g))
  \end{align*}
  
  Luego, es derivación (y por ello bilineal). Además, es antisimétrica:
  
  \[
    \ProdC{D_{Y}}{D_{X}} =
    D_{Y}D_{X}-D_{X}D_{Y} =
    -(D_{X}D_{Y}-D_{Y}D_{X}) =
    -\ProdC{D_{X}}{D_{Y}}
  \]

  Veamos ahora la identidad de Jacobi: Sean \(D_{X},D_{Y},D_{Z}\). Entonces:
  \begin{itemize}
  \item
    \(\ProdC{\ProdC{D_{X}}{D_{Y}}}{D_{Z}}=\ProdC{D_{X}D_{Y}-D_{Y}D_{X}}{D_{Z}}\).

    Como es bilineal, lo anterior se corresponde con
    \begin{align*}
      \ProdC{D_{X}D_{Y}}{D_{Z}}-\ProdC{D_{Y}D_{X}}{D_{Z}}
      &= D_{X}D_{Y}D_{Z}-D_{Z}D_{X}D_{Y}\\
      &- D_{Y}D_{X}D_{Z}+D_{Z}D_{Y}D_{X}
    \end{align*}
  
  \item
    \begin{align*}
      \ProdC{\ProdC{D_{Y}}{D_{Z}}}{D_{X}}
      &= D_{Y}D_{Z}D_{X} - D_{X}D_{Y}D_{Z}\\
      &- D_{Z}D_{Y}D_{X} +D_{X}D_{Z}D_{Y}
    \end{align*}
  \item
    \begin{align*}
      \ProdC{\ProdC{D_{Z}}{D_{X}}}{D_{Y}}
      &= D_{Z}D_{X}D_{Y} - D_{Y}D_{Z}D_{X}\\
      &- D_{X}D_{Z}D_{Y} + D_{Y}D_{X}D_{Z}
    \end{align*}
  \end{itemize}
  Claramente, la suma cíclica de las expresiones anteriores se anula.

  Luego nuestra operación es un \emph{corchete de Lie} y entonces
  \(\mathbb{X}(M)\) es un \emph{álgebra de Lie}.
\end{proof}

\begin{proposition}
  Sea \(\mathcal{A}=\{(U,\varphi)\}\) atlas de \(M\). Si \(\{f^{U}\},\{g^{U}\}\)
  denotan las funciones componentes de los campos \(X,Y\) respectivamente,
  entonces el campo correspondiente a \(\ProdC{D_{X}}{D_{Y}}(\ProdC{X}{Y})\) tiene
  como funciones componentes locales a
  \(\ProdC{f^{U}}{g^{U}}\colon\varphi(U)\to\RealSet^{m}\) cuya i-ésima función
  coordenada es
  \(\ProdC{f^{U}}{g^{U}}_{i}=\sum_{j=1}^{m}(f_{j}^{U}\pderiv{g_{i}^{U}}{x_{j}}-
  \pderiv{f_{i}^{U}}{x_{j}}g_{j}^{U})\)
\end{proposition}

\begin{proof}
  En \((U,\varphi)\),

  \begin{align*}
    D_{X}(h)&=\sum_{i=1}^{m}\pderiv{h\circ\varphi^{-1}}{x_{i}}f_{i}^{U}\\
    D_{Y}(h)&=\sum_{i=1}^{m}\pderiv{h\circ\varphi^{-1}}{x_{i}}g_{i}^{U}
  \end{align*}
  
  Entonces,
  \begin{align*}
    D_{X}D_{Y}(h)
    &= D_{X}(\sum_{i=1}^{m}\pderiv{h\circ\varphi^{-1}}{x_{i}}g_{i}^{U})\\
    &= \sum_{j=1}^{m}\pderiv{(
      \sum_{i=1}^{m}\pderiv{h\circ\varphi^{-1}}{x_{i}}g_{i}^{U})}{x_{j}}f_{j}^{U}
    \\
    &= \sum_{i,j=1}^{m}\left[ \pderiv{h\circ\varphi^{-1}}{x_{j},x_{i}} g_{i}^{U}f_{j}^{U}+\pderiv{h\circ\varphi^{-1}}{x_{i}}\pderiv{g_{i}^{U}}{x_{j}}
      f_{j}^{U} \right]
  \end{align*}
  
  A su vez,
  \[
    D_{Y}D_{X}(h)
    = \sum_{i,j=1}^{m} \left[
      \pderiv{h\circ\varphi^{-1}}{x_{i},x_{j}}
      f_{i}^{U}g_{j}^{U} +
      \pderiv{h\circ\varphi^{-1}}{x_{i}}
      \pderiv{f_{i}^{U}}{x_{j}}g_{j}^{U}
      \right]
  \]
 
  Por tanto, restando la segunda expresión a la primera,
  \begin{align*}
    \ProdC{D_{X}}{D_{Y}}(h)
    &= (D_{X}D_{Y}-D_{Y}D_{X})(h)\\
    &= \sum_{i=1}^{m}\pderiv{h\circ\varphi^{-1}}{x_{i}}\left(\sum_{j=1}^{m}
      \pderiv{g_{i}^{U}}{x_{j}}f_{j}^{U}-\pderiv{f_{i}^{U}}{x_{j}}g_{j}^{U} \right)\\
    &= \sum_{i=1}^{m} \pderiv{h\circ\varphi^{-1}}{x_{i}}(\text{coordenada
      i-ésima})\\
    &= \sum_{i=1}^{m}\pderiv{h\circ\varphi^{-1}}{x_{i}}\ProdC{f^{U}}{g^{U}}_{i}
  \end{align*}
\end{proof}

\section{Campos h-compatibles}

Si \(X,Y\in\mathbb{X}(M),X',Y'\in\mathbb{X}(N)\), \(X,X'\) h-compatibles,
\(Y,Y'\) también, ¿entonces \(\ProdC{X}{Y}\) h-compatible con \(\ProdC{X'}{Y'}\)?

\begin{lemma}
  Si \(X\in\mathbb{X}(M),Y\in\mathbb{X}(N)\) son h-compatibles, entonces
  \(\forall p\in M,\ \forall f\in\mathcal{F}(N)\) se verifica que
  \(D_{X}(f\circ h)(p)=D_{Y}(f)(h(p))\). Es decir, \(D_{X}(f\circ
  h)=D_{Y}(f)\circ h\)
\end{lemma}

\begin{proof}
  Sean \(X,Y\) campos h-compatibles.
  Entonces por definición: \(dh_{p}X(p)=Y(h(p))\),
  donde \(dh_{p}X(p)\) se corresponde con la derivación de gérmenes para el germen
  determinado por \(f\): \(\delta_{h(p)}(\Clase{f})=D_{X}(\widetilde{f}\circ
  h)(p)=D_{X}(f\circ h)(p)\) (pues \(f\) está definida en todo \(N\) y no hace
  falta extenderla).

  También \(Y(h(p))\) se corresponde con
  \(\delta'_{h(p)}(\Clase{f})=D_{Y}(f)(h(p))\).

  Por tanto, llegamos a que
  \[\begin{array}{l}
      D_{X}(f\circ h)(p)=dh_{p}X(p)=Y(h(p))=D_{Y}(f)(h(p))
    \end{array}\]
\end{proof}

\begin{proposition}
  Si \(X,X'\) son h-compatibles (es decir, se cumple \(D_{X}(f\circ h)=D_{X'}(f)\circ
  h\)), e \(Y,Y'\) también (\(D_{Y}(f\circ h)=D_{Y'}(f)\circ h\)), entonces
  \(\ProdC{X}{Y},\ProdC{X'}{Y'}\) h-compatibles.
\end{proposition}

\begin{proof}
  Aplicando el lema anterior, lo que tenemos que probar es que
  \(\ProdC{D_{X}}{D_{Y}}(f\circ h)=\ProdC{D_{X'}}{D_{Y'}}(f)\circ h\) para ver que
  son h-compatibles.

  Por un lado,
  \begin{align*}
    \ProdC{D_{X}}{D_{Y}}(f\circ h)
    &= D_{X}(D_{Y}(f\circ h)) - D_{Y}(D_{X}(f \circ h))\\
    &=D_{X}(D_{Y'}(f)\circ h)-D_{Y}(D_{X'}(f)\circ h)\\
    &=D_{X}(f^{Y'}\circ h)-D_{Y}(f^{X'}\circ h)\\
    &=D_{X'}(f^{Y'})\circ h - D_{Y'}(f^{X'})\circ h \\
    &=\ProdC{D_{X'}}{D_{Y'}}(f)\circ h
  \end{align*}
  
  donde hemos usado que tanto \(D_{Y'}(f)\) como \(D_{X'}(f)\) son a su vez
  funciones a las que podemos aplicar el resultado anterior de h-compatibilidad.
\end{proof}
\end{document}

%%% Local Variables:
%%% TeX-master: "../VD_ebook"
%%% End: