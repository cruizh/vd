\documentclass[../VD.tex]{subfiles}

\externaldocument{../VD}

\begin{document}

\setcounter{chapter}{3}
\chapter{Espacio tangente}\label{chap:tangente}

\section{Curvas diferenciables y equivalentes}

\begin{definition}[{name=[curva]{curva diferenciable}}, label={def:curva}]
  Una \emph{curva diferenciable} en una variedad \(M\) \emph{con origen en} \(p
  \in M\) es una aplicación \(\alpha \colon (-\varepsilon,\varepsilon) \to M\)
  diferenciable con \(\alpha(0) = p\).
\end{definition}

\begin{remark}
  Si \((U,\varphi)\) es carta de \(M\) con \(p \in U\), dada una
  \nameref{def:curva} con origen en \(p\), siempre
  \begin{equation}
    \label{eq:exists-curva-carta}
    \exists \varepsilon' \in (0,\varepsilon) : \alpha(-\varepsilon',\varepsilon') \subseteq U
  \end{equation}
\end{remark}

\begin{definition}[{name=[equivalentes]{curvas equivalentes}},
  label={def:curvas-equiv}]
  Dadas dos \hyperref[def:curva]{curvas diferenciables} con origen en \(p\) en
  \(M\),
  \begin{align*}
    \alpha &\colon (-\varepsilon, \varepsilon) \to M\\
    \beta &\colon (-\delta,\delta) \to M
  \end{align*}
  se dicen \emph{equivalentes} si existe una carta \((U,\varphi)\) con \(p \in
  U\) de forma que \((\varphi \circ \alpha)'(0) = (\varphi \circ \beta)'(0)\).
\end{definition}

\begin{lemma}[label={lem:curva-equiv-bien-definido}]
  La definición de \nameref{def:curvas-equiv} no depende de la elección de carta.
\end{lemma}

\begin{proof}
  Sea \((V,\psi)\) otra carta con \(p \in V\). \(p \in U \cap V\), por lo que
  podemos restringir \(\alpha\) y \(\beta\) a la intersección, teniendo
  \begin{align*}
    (\psi \circ \alpha)'(0)
    &= (\psi \circ \varphi^{-1} \circ \varphi \circ \alpha)'(0)\\
    &= \J[\varphi(p)]{(\psi \circ \varphi^{-1})}(\varphi \circ \alpha)'(0)\\
    &= \J[\varphi(p)]{(\psi \circ \varphi^{-1})}(\varphi \circ \beta)'(0)\\
    &= (\psi \circ \varphi^{-1} \circ \varphi \circ \beta)'(0)\\
    &= (\psi \circ \beta)'(0)
  \end{align*}
\end{proof}

\begin{note}
  La relación ``ser \nameref{def:curvas-equiv}'' es de equivalencia.
\end{note}

\begin{definition}[name={clase de equivalencia},label={def:curva-clase}]
  Notaremos \(\Clase{a}\) a la clase de equivalencia de la curva \(\alpha\) por
  la relación anterior.
\end{definition}

\section{Espacio tangente en un punto y espacio tangente de una variedad}

\begin{definition}[name={espacio tangente}, label={def:tangente}]
  Sea \(p \in M\). Se llama \emph{espacio tangente} de \(M\) en \(p\) a
  \begin{equation}
    \label{eq:def-tangente}
    \Tangente[p]{M} = \Set{\Clase{\alpha} : \alpha \text{ curva con origen en } p}
  \end{equation}
\end{definition}

\begin{lemma}
  \(\Tangente[p]{M}\) es un espacio vectorial de dimensión \(\dim M\).
\end{lemma}

\begin{proof}
  Sea \((U,\varphi)\) carta de \(M\) con \(p \in U\). Definimos \(\Clase{\gamma}
  \coloneqq \Clase{\alpha} + \Clase{\beta}\) como sigue:
  \[
    \gamma(t) = \varphi^{-1}(\varphi \alpha(t) + \varphi \beta(t) - \varphi(p))
  \]

  Aquí, \(t \in (-\mu, \mu)\) con \(\mu\) elegido para que
  \[
    \varphi \alpha(t) + \varphi \beta(t) - \varphi(p) \in \varphi(U)
  \]

  \(\gamma(0) = p\), y \(\gamma\) es diferenciable (se deja como ejercicio).

  Hay que ver que \(\Clase{\gamma}\) está bien definida, i.e.\ que no depende de
  las representantes de \(\Clase{\alpha}\) y \(\Clase{\beta}\):
  \[
    (\varphi \circ \gamma)'(0) = (\varphi \circ \alpha)'(0) + (\varphi \circ \beta)'(0)
  \]
  que solo depende de la clase de equivalencia.
\end{proof}

\begin{note}
  El elemento neutro es \(\theta_{p} = \Clase{c_{p}}\) con \(c_{p}(t) = p
  \ \forall t\).

  La multiplicación por escalares se realiza como \(\lambda \Clase{\alpha}
  \eqqcolon \Clase{\eta}\) donde \(\eta = \alpha(\lambda t)\), con \(t \in
  (-\xi,\xi)\) con \(\xi\) elegido para que \(\alpha(\lambda t) \in \varphi(U)\).
\end{note}

\begin{definition}[name={isomorfismo del espacio tangente}, label={def:tang-iso}]
  Dada una carta \((U,\varphi)\) de \(M\) con \(p \in U\), definimos

  \begin{align*}
    \tau_{\varphi}^{p} \colon \Tangente[p]{M} &\to \RealSet^{m}\\
    \Clase{\alpha} &\mapsto (\varphi \circ \alpha)'(0)
  \end{align*}
\end{definition}

\begin{note}
  Esta aplicación es lineal y es isomorfismo.
\end{note}

\begin{proof}
  Es lineal e inyectiva por definición. Falta ver que es sobreyectiva: sea
  \(\Set{e_{i}}_{i=1}^{m}\) base de \(\RealSet^{m}\). Definimos
  \[
    \varepsilon_{i}^{p}(t) = \varphi^{-1}(\varphi(p) + t e_{i})
  \]
  con \(t\) elegido para que \(\varphi(p) + t e_{i} \in \varphi(U)\). Estas
  curvas son diferenciables y \((\varphi \circ \varepsilon_{i}^{p})'(0) =
  e_{i}\). Por tanto \(\tau_{\varphi}^{p}\) es sobreyectiva.
\end{proof}

\begin{lemma}[name={cambio de cartas en espacio
    tangente},label={lem:tang-cambio-cartas}]
  Sean \((U,\varphi)\) y \((W,\psi)\) cartas de \(M\) con \(p \in U \cap W\). Se
  tiene el diagrama conmutativo de la \cref{fig:tang-cambio-cartas}, i.e.\ se
  tiene que
  \[
    \tau_{\psi}^{p} = \J[\varphi(p)]{(\psi \circ \varphi^{-1})} \circ
    \tau_{\varphi}^{p}
  \]

  Nótese que la relación del jacobiano se tiene por el
  \cref{lem:curva-equiv-bien-definido}.
  
  \begin{figure}[h]
    \centering
    \begin{tikzcd}
      \Clase{\alpha}
      \ar[r,  mapsto, "\tau_{\varphi}^{p}"]
      \ar[dr, mapsto, "\tau_{\psi}^{p}"]&
      (\varphi \circ \alpha)'(0)
      \ar[d, mapsto, "{\J[\varphi](p){(\psi \circ \varphi^{-1})}}"]\\
      & (\psi \circ \alpha)'(0)
    \end{tikzcd}
    \caption{Cambio de cartas en espacio tangente}
    \label{fig:tang-cambio-cartas}
  \end{figure}
\end{lemma}

\begin{definition}[{name=[espacio tangente global]{espacio tangente de una variedad}},
  label={def:tangente-global}]
  Se llama \emph{espacio tangente} de \(M^{m}\) a \(\Tangente{M} = \bigcup_{p
    \in M} \Tangente[p]{M}\).
\end{definition}

\begin{proposition}
  \(\Tangente{M}\) es \nameref{def:vd} de dimensión \(2m\).
\end{proposition}

\begin{proof}
  Sea \(\Tangente{U} = \bigcup_{p \in U} \Tangente[p]{M}\). Se tiene
  \begin{center}
    \begin{tikzcd}
      \Tangente{(U \cap V)}
      \ar[r, equals]
      \ar[d, equals] &
      \Tangente{U} \cap \T{V}
      \ar[d, equals]\\
      \bigcup_{q \in U \cap V} \Tangente[q]{M}
      \ar[r, equals] &
      (\bigcup_{p \in U} \Tangente[p]{M}) \cap (\bigcup_{p' \in V} \Tangente[p']{M})
    \end{tikzcd}
  \end{center}

  Se define la siguiente aplicación:
  \begin{align*}
    \widetilde{\varphi} \colon \Tangente{U} &\to \varphi(U) \times \RealSet^{m} \subseteq \RealSet^{m} \times \RealSet^{m} = \RealSet^{2m}\\
    \widetilde{\varphi}(v_{p}) &= (\varphi(p), \tau_{\varphi}^{p}(v_{p})) = (\varphi(p), (\varphi \circ \alpha)'(0))
  \end{align*}
  con \(v_{p} = \Clase{\alpha} \in \Tangente[p]{M}\).

  Si \((V,\psi)\) es otra carta,
  \[
    \widetilde{\varphi}(\Tangente{U} \cap \Tangente{V})
    = \widetilde{\varphi}(\Tangente{(U \cap V)}) =  \varphi(U \cap V) \times \RealSet^{m}
  \]
  \[
    \widetilde{\varphi} \circ \widetilde{\psi}^{-1}(x,y) =
    (\varphi \circ \psi^{-1}(x),
    \underbracket{\tau_{\varphi}^{p} \circ
      {(\tau_{\psi}^{p})}^{-1}(y)}_{\J[x]{(\varphi \circ \psi^{-1})}(y)})
  \]
\end{proof}

\section{Resultados sobre submersiones e inmersiones del espacio tangente de una
  variedad}

\begin{proposition}
  Existe una aplicación
  \begin{align*}
    \pi \colon \Tangente{M} &\to M\\
    \Clase{\alpha} &\overset{\pi}{\mapsto} p \quad \forall p \in M
  \end{align*}
  que es submersión.
\end{proposition}

\begin{proof}
  Sea \((U,\varphi)\) carta de \(M\). Se tiene el diagrama de la
  \cref{fig:submersion-tangente}.

    \begin{figure}[h]
    \centering
    \begin{tikzcd}[column sep = large]
      \Tangente{U}
      \ar[d, "\widetilde{\varphi}"']
      \ar[r, "\pi"] &
      U
      \ar[d, "\varphi"]\\
      \varphi(U) \times \RealSet^{m}
      \ar[r, dashed, "\widetilde{\pi}"'] &
      \varphi(U)\\
      (x,y) \ar[r, maps to] & x
    \end{tikzcd}
    \caption{Submersión del espacio tangente}
    \label{fig:submersion-tangente}
  \end{figure}
  % TODO Terminar
\end{proof}

\begin{proposition}

  La submersión \(\pi\colon TM\to M\) admite una \emph{sección} \(s\colon M\to
  TM\), es decir, \(\pi\circ s=id_{M}\), dada por
  \(p\overset{s}{\mapsto}\theta_{p}\in T_{p}M\) \textbf{neutro}. Además, \(s\) es
  \textbf{incrustación} (\(s(M)\) subvariedad de \(TM\) con la topología
  inducida).
  
\end{proposition}

\begin{proof}
  Obviamente, \(\pi\circ s=id_{M}\).
  
  \begin{figure}[h]
    \centering
    \begin{tikzcd}[column sep = large]
      \Tangente{U}
      \ar[d, "\widetilde{\varphi}"']
      &
      U
      \ar[l, "s"]
      \ar[d, "\varphi"]\\
      \varphi(U) \times \RealSet^{m}
      &
      \varphi(U)
      \ar[l, dashed, "\widetilde{s}"']\\
      (x,0) & x \ar[l, maps to]
    \end{tikzcd}
    \caption{Diagrama de s}
    \label{fig:diagrama de s}
  \end{figure}
  
  Luego se tiene \(\widetilde{s}=\widetilde{\varphi}\circ s\circ\varphi^{-1}\),
  y entonces \(\widetilde{s}(x)=(x,0)\) diferenciable y de rango \(m\), luego
  \(s\) es \underline{inmersión} (además de inyectiva). Veamos que \(s(M)\) es
  subvariedad:

  \[\begin{array}{l}
      \widetilde{\varphi}(\Tangente{U}\cap s(M))=\varphi(U)\times\{0\}= \\
      (\varphi(U)\times\RealSet^{m})\cap(\RealSet^{m}\times\{0\})=
      \widetilde{\varphi}(\Tangente{U})\cap\RealSet^{m}\times\{0\} 
    \end{array}\]

  Luego es subvariedad, inmersión inyectiva, y la topología es la inducida, por
  tanto es incrustación. 
\end{proof}

\begin{example}
  Si \(M=U\overset{\operatorname{abierto}}{\subseteq}\RealSet^{m},\, p\in M\).
  Entonces \(\Tangente[p]{M}\homeo\RealSet^{m},\, [\gamma]\mapsto\gamma'(0).\)
  Lo cual podemos extender a un difeomorfismo:
  \[\begin{array}{c}
      \Tangente[p]{M}\overset{\homeo}{\to} M\times\RealSet^{m} \\
      \Clase{\gamma}\overset{\homeo}{\mapsto} (\gamma(0),\gamma'(0))
    \end{array}
  \]

  donde podemos considerar que el vector \((\gamma(0),\gamma'(0))\) indica la
  posición y la velocidad.

  Más generalmente, si \(M\) es una variedad con una única carta
  \((M,\varphi)\), entonces:
  \[\begin{array}{c}
      \Tangente[p]{M}\homeo\RealSet^{m} \\
      \Clase{\gamma}\mapsto(\varphi\circ\gamma)'(0)
    \end{array}\]
  se extiende al difeomorfismo:
  \[\begin{array}{c}
      \Tangente{M}\to M\times\RealSet^{m} \\
      \Clase{\gamma}\mapsto(\gamma(0),(\varphi\circ\gamma)'(0))
    \end{array}\]
\end{example}

\end{document}

%%% Local Variables:
%%% TeX-master: "../VD"
%%% End: