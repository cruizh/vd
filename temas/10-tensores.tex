\documentclass[../VD.tex]{subfiles}

\externaldocument{../VD}

\begin{document}

\setcounter{chapter}{9}
\chapter{Espacios de formas y tensores sobre una variedad}\label{chap:dual} 

\section{Introducción}

En \textbf{Álgebra Lineal} conocemos la extensión natural del espacio dual de
las formas.

\begin{definition}
  Si una 1-forma del espacio vectorial \(V\) es una aplicación lineal
  \(V\to\RealSet\), entonces dados \(n\) espacios vectoriales
  \(V_{1},\dots,V_{n}\) y otro espacio vectorial \(W\), una \emph{aplicación
    multilineal} \(f\colon V_{1}\times\dots\times V_{n}\to W\), esto es, \(\forall
  \lambda,\mu\in\RealSet,\ \forall v_{i},v'_{i}\in V_{i},\ 1\leq i\leq n\):
  
  \[
    f(v_{1},\dots,\lambda v_{i}+\mu v'_{i},\dots,v_{n})=\lambda
    f(\dots,v_{i},\dots)+\mu f(\dots,v'_{i},\dots).
  \]

  Cuando \(V_{1}=\dots=V_{n},\ W=\RealSet\) decimos que \(f\) es una n-forma
  sobre \(V\).
\end{definition}

Ahora se plantea cómo extender las n-formas al espacio tangente de una variedad
\(M\). Ya vimos en el caso de las 1-formas que esa construcción se puede abordar
de varias maneras:

\begin{enumerate}
\item Crear el espacio cotangente haciendo la construcción punto a punto y
  después definiendo una estructura coherente en todo \(M\): para cada \(p\in
  M\) sea \(\DualT[p]{M}\) y se define \(\DualT{M}=\bigcup_{p\in
    M}\DualT[p]{M}\), dotándolo de una estructura de variedad diferenciable
  compatible con la de \(M\) mediante la proyección
  \(\pi^{*}\colon\DualT{M}\to M\ \omega\in\DualT[p]{M}\mapsto p\) y las 1-formas
  se definen como secciones diferenciables de esa proyección.

\item Definir el espacio cotangente globalmente sobre todo \(\Tangente{M}\) por
  medio de funciones diferenciables \(f\colon\Tangente{M}\to\RealSet\) que punto
  a punto son 1-formas, es decir, \(f\colon\Tangente[p]{M}\to\RealSet\) es
  lineal para todo \(p\in M\).

\item Usar \(\mathbb{X}(M)\) el \(\mathcal{F}(M)\)-módulo de los campos sobre
  \(M\) y reescribir \(\DualT{M}\) como el dual de \(\mathbb{X}(M)\) como
  \(\mathcal{F}(M)\)-módulo. Esto es, usar la identificación
  \(\DualT{M}\cong\text{Hom}(\mathbb{X}(M),\mathcal{F}(M))\), que lleva la forma
  \(\omega\colon M\to\Tangente{M}\) a la función \(\omega(X)(p)=X(\omega(p)),\
  \forall p\in M\).
\end{enumerate}

Veamos cómo extender estas construccione equivalentes del espacio cotangente de
\(M\) para obtener la noción de n-forma sobre la variedad \(M\).

Para ello recordemos la noción de producto tensorial que permite reinterpretar
las aplicaciones multilineales como 1-formas, pasando la complicación de las
aplicaciones multilineales al espacio vectorial.

\section{Producto Tensorial}

\begin{definition}[name=producto tensorial]
  Recordemos que el \emph{producto tensorial} de los espacios vectoriales
  \(V_{1},\dots,V_{n}\), \(V_{1}\bigotimes\dots\bigotimes V_{n}\), es el espacio
  vectorial obtenido por combinaciones lineales de los símbolos
  \(v_{1}\otimes\dots\otimes\ v_{n}\) sujetos a las relaciones:

  \begin{itemize}
  \item \(\lambda(v_{1}\otimes\dots\otimes
    v_{n})=v_{1}\otimes\dots\otimes \lambda v_{i}\otimes\dots\otimes
    v_{n},\ \forall i\)

  \item \(v_{1}\otimes\dots\otimes(v_{i}+v'_{i})\otimes\dots\otimes
    v_{n}=v_{1}\otimes\dots\otimes \lambda v_{i}\otimes\dots\otimes
    v_{n}+v_{1}\otimes\dots\otimes \lambda v'_{i}\otimes\dots\otimes
    v_{n},\ \forall i\)
  \end{itemize}

  Entonces \((V_{1}\bigotimes\dots\bigotimes
  V_{n})^{*}\cong\text{Multi}_{n}(V_{1},\dots,V_{n}),\ f\mapsto
  g_{f}(v_{1},\dots,v_{n})=f(v_{1}\otimes\dots\otimes\v_{n})\).
\end{definition}

\begin{remark}
  Además, se tiene que si \(B_{i}\) es una base de \(V_{i}\) entonces
  \(\{b_{k_{1}}\otimes\dots\otimes b_{k_{n}}\}\) es una base de
  \(V_{1}\bigotimes\dots\bigotimes V_{n}\) donde
  \((b_{k_{1}},\dots,b_{k_{n}})\in B_{1}\times\dots\times B_{n}\).
  Equivalentemente, las n-formas \(\delta_{k_{1},\dots,k_{n}}\) dadas por las
  extensiones multilineales de \(\delta_{k_{1},\dots,k_{n}}(b_{k_{1}}\dots
  b_{k_{n}})=1\) y \(\delta_{k_{1},\dots,k_{n}}(b_{k'_{1}}\dots b_{k'_{n}})=0\)
  si \((k'_{1},\dots,k'_{n})\neq(k_{1},\dots,k_{n})\) es una base del espacio de
  las aplicaciones multilineales \(V_{1}\times\dots\times V_{n}\to\RealSet\).
  
  Si \(V_{1}=\dots=V_{n}=V\) denotamos por \(\bigotimes_{n}V\) a
    \(\underbracket{V\bigotimes\dots\bigotimes V}_{n\text{ veces}}\).

  Así pues si Multi\(_{n}(V)\) es el espacio vectorial de las n-formas sobre
  \(V\) tenemos Multi\(_{n}(V)\cong(\bigotimes_{n}(V))^{*}\) y
  \(\dim{\text{Multi}_{n}(V)}=(\dim{V})^{n}\).

  Además, dadas \(\alpha\in(\bigotimes_{r}V)^{*}\) y
  \(\beta\in(\bigotimes_{s}V)^{*}\) se define
  \(\alpha\otimes\beta\in(\bigotimes_{r}V\bigotimes(\bigotimes_{s}V))^{*}\) por:
  \[
    \alpha\otimes\beta(v_{1}\otimes\dots\otimes v_{r}\otimes
    v_{r+1}\otimes\dots\otimes v_{r+s})=\alpha(v_{1}\otimes\dots\otimes
    v_{r})\cdot \beta(v_{r+1}\otimes\dots\otimes v_{r+s}). 
  \]

  Obsérvese que existe un isomorfismo
  \(\rho\colon\bigotimes_{r}V\cong(\bigotimes_{r}V)^{*}\) que lleva el producto
  tensorial de 1-formas \(\alpha_{1}\otimes\dots\otimes\alpha_{r}\) en la
  1-forma sobre \(\bigotimes_{r}V\) (o n-forma sobre \(V\)):
  \[
    \rho(\alpha_{1}\otimes\dots\otimes\alpha_{r})(v_{1}\otimes\dots\otimes
    v_{r})=\alpha_{1}(v_{1})\otimes\dots\otimes\alpha_{r}(v_{r})
  \]

  Nótese que si \(\{e_{i}\}\) es una base de \(V\), los elementos básicos de
  \(\bigotimes_{r}V^{*}\), \(e_{i_{1}}^{*}\otimes\dots\otimes e_{i_{r}}^{*}\)
  pasan por \(\rho\) a los duales \((e_{i_{1}}\otimes\dots\otimes
  e_{i_{n}})^{*}\) de la base \(\{e_{i_{1}}\otimes\dots\otimes e_{i_{r}}\}\) de
  \(\bigotimes_{r}V\).
\end{remark}

Así pues, las opciones (equivalentes) para generalizar el espacio cotangente son
las siguientes:

\begin{enumerate}
\item Tomar para cada \(p\in M\) Multi\(_{n}(\Tangente[p]{M})\) y definir
  Multi\(_{n}\Tangente{M}=\bigcup_{p\in M}\text{Multi}_{n}(\Tangente[p]{M})\).
  Si se quiere usar el producto tensorial entonces se toma
  \(\bigotimes_{n}\Tangente[p]{M}\) y \(\bigotimes_{n}\Tangente{M}=\bigcup_{p\in
    M}\bigotimes_{n}\Tangente[p]{M}\). Habrá que dotar a
  Multi\(_{n}(\Tangente{M})\cong\bigotimes_{n}\Tangente{M}\) de una estructura de
  variedad de dimensión \(n\dim{M}\), para la cual la proyección
  \(\pi^{*}\colon\bigotimes_{n}\Tangente{M}\to M\) es diferenciable y definir una
  n-forma sobre \(M\) como una sección de \(\pi\).
  
\item Definir una n-forma como una aplicación diferenciable
  \(f\colon\bigotimes_{n}\Tangente{M}\to\RealSet\) tal que para cada \(p\in M\)
  se restringe a una aplicación lineal
  \(f\colon\Tangente[p]{M}\otimes\dots\otimes\Tangente[p]{M}\to\RealSet\), o,
  equivalentemente multilineal
  \(f\colon\Tangente[p]{M}\times\dots\times\Tangente[p]{M}\to\RealSet\).

  Esta construcción es equivalente a definir primero el espacio tangente
  ''producto'' \((\Tangente{M})^{n}=\bigcup_{p\in M}(\Tangente[p]{M})^{n}\),
  donde
  \((\Tangente[p]{M})^{n}=\Tangente[p]{M}\times\dots\times\Tangente[p]{M}\) que
  resulta ser una variedad de dimensión \(n\dim{M}\). Ahora una n-forma sobre
  \(M\) será una aplicación diferenciable
  \(f\colon(\Tangente{M})^{n}\to\RealSet\) tal que para todo \(p\in M\) se
  restringe a una aplicación multilineal
  \(f\colon\Tangente[p]{M}\times\dots\times\Tangente[p]{M}\to\RealSet\).

\item Extender las nociones de aplicación multilineal y producto tensorial a
  módulos sobre un anillo \(A\). Entonces considerar las aplicaciones
  multilineales de \(\mathcal{F}(M)\)-módulos
  \(\mathbb{X}(M)\times\dots\times\mathbb{X}(M)\to\RealSet\) o bien las
  aplicaciones lineales de \(\mathcal{F}(M)\)-módulos
  \(\mathbb{X}(M)\bigotimes\dots\bigotimes\mathbb{X}(M)\to\mathcal{F}(M)\).
\end{enumerate}

\begin{lemma}
  Las 3 construcciones anteriores son equivalentes.
\end{lemma}

\begin{proof}
  Ahora vemos cómo pasar de una a otra construcción:
  \begin{itemize}
  \item \((1)\iff(2)\). Si \(s\colon M\to\text{Multi}_{n}(\Tangente{M})\) es una
    sección defino
    \(f_{s}\colon\Tangente[p]{M}\times\dots\times\Tangente[p]{M}\to\RealSet\) por
    \(f_{s}(v_{1},\dots,v_{n})=s(p)(v_{1},\dots,v_{n}),\
    v_{i}\in\Tangente[p]{M}\).
    Recíprocamente, dada \(f\colon\bigotimes_{n}\Tangente{M}\to\RealSet\) se
    define \(s_{f}(p)(v_{1},\dots,v_{n})=f(v_{1}\otimes\dots\otimes v_{n}),\
    v_{i}\in\Tangente[p]{M}\), con la notación del producto tensorial.
    
  \item \((2)\iff(3)\). Si \(f\colon\bigotimes_{n}\Tangente{M}\to\RealSet\)
    entonces
    \(\phi_{f}\colon\mathbb{X}(M)\times\dots\times\mathbb{X}(M)\to\mathcal{F}(M)\)
    está dada por \(\phi_{f}(X_{1},\dots,X_{n})(p)=f(X_{1}(p),\dots,X_{n}(p))\).
    Recíprocamente, si
    \(\phi\colon\mathbb{X}(M)\times\dots\times\mathbb{X}(M)\to\mathcal{F}(M)\)
    entonces \(f_{\phi}(v_{1}\otimes\dots\otimes
    v_{n})=\phi(X_{v_{1}},\dots,X_{v_{n}})\) donde \(X_{v_{i}}\) son campos
    cualesquiera con \(X_{v_{i}}(p)=v_{i}\). 
  \end{itemize}
\end{proof}

\section{Tensores}

Ahora podemos complicarlo más y mezclar el espacio tangente \(\Tangente{M}\) y
el cotangente \(\DualT{M}\) en las construcciones anteriores y hablar de
\textbf{tensores}.

\begin{definition}[name=tensores]
  Por un \emph{tensor r-covariante} se entenderá una r-forma sobre el espacio
  cotangente de \(M\), \(\bigotimes_{r}\DualT{M}\to\RealSet\) mientras que un
  \emph{tensor s-contravariante} es una s-forma del tangente,
  \(\bigotimes_{s}\Tangente{M}\to\RealSet\).

  \par
  
  Todavía más, si consideramos la variedad
  \((\bigotimes_{s}\Tangente{M})\otimes(\bigotimes_{r}\DualT{M})=\bigcup_{p\in
    M}(\bigotimes_{s}\Tangente[p]{M})\otimes(\bigotimes_{r}\DualT[p]{M})\), la
  aplicación diferenciable
  \((\bigotimes_{s}\Tangente{M})\otimes(\bigotimes_{r}\DualT{M})\to\RealSet\)
  que es lineal en cada \(p\in M\) será entonces un \emph{tensor mixto de tipo
    (s,r)}.
\end{definition}

\begin{remark}
  Si \(m=\dim{M}\) y tomamos una carta entorno de \(p\in M\)
  \((U,\varphi=(x_{1},\dots,x_{n}))\), con la notación
  \(\dif{x_{1}},\dots,\dif{x_{n}}\) y
  \(\pderiv{}{x_{1}},\dots,\pderiv{}{x_{n}}\) para la base canónica de
  \(\DualT[p]{M}\) y \(\Tangente[p]{M}\), tenemos que un tensor r-covariante es
  de la forma
  \[\sum_{(i_{1},\dots,i_{r})}g_{i_{1},\dots,i_{r}}\dif{x_{i_{1}}}\otimes\dots\otimes
  \dif{x_{i_{r}}},\ (i_{1},\dots,i_{r})\in\{1,\dots,m\}^{r}\] y análogamente un tensor
  s-contravariante se escribe como
  \[\sum_{(j_{1},\dots,j_{s})}f_{j_{1},\dots,j_{s}}\pderiv{}{x_{j_{1}}}\otimes\dots
  \otimes\pderiv{}{x_{j_{s}}},\ (j_{1},\dots,j_{s})\in\{1,\dots,m\}^{s}.\]

  Para un tensor mixto (s,r) tenemos
  \[\sum_{(i_{1},\dots,i_{r}),(j_{1},\dots,j_{s})}h_{i_{1},ots,i_{r}}^{j_{1},\dots,j_{s}}
  \pderiv{}{x_{j_{1}}}\otimes\dots\otimes\pderiv{}{x_{j_{s}}}\otimes
  \dif{x_{i_{1}}}\otimes\dots\otimes\dif{x_{i_{r}}}.\]

  A las funciones \(f,g,h\colon\varphi(U)\to\RealSet\) se las denomina
  \emph{componentes} del correspondiente tensor en la carta \((U,\varphi)\). 
\end{remark}

\section{Formas Simétricas y Alternadas}

Existen 2 clases de formas sobre un espacio vectorial \(V\) de especial interés:
las \textbf{formas simétricas y las antisimétricas o alternadas}.

\begin{definition}[name=forma simétrica]
  Una r-forma \(\alpha\colon V\times\dots\times V\to\RealSet\) se dice
  \emph{simétrica} si
  \[\alpha(v_{1},\dots,v_{i},\dots,v_{j},\dots,v_{r})=
  \alpha(v_{1},\dots,v_{j},\dots,v_{i},\dots,v_{r})\] para todo \(1\leq i\leq
  j\leq r\). 
\end{definition}

\begin{definition}[name=forma alternada]
  Una r-forma \(\alpha\colon V\times\dots\times V\to\RealSet\) se dice
  \emph{antisimétrica o alternada} si
  \[\alpha(v_{1},\dots,v_{i},\dots,v_{j},\dots,v_{r})= 
  -\alpha(v_{1},\dots,v_{j},\dots,v_{i},\dots,v_{r})\] para todo \(1\leq i\leq
  j\leq r\).  
\end{definition}

\begin{remark}
  Las formas alternadas también son llamadas \emph{exteriores} en algunas
  referencias. Incluso es bastante habitual que se reserve el nombre de
  ''forma'' exclusivamente para las alternadas, dejando aplicación multilineal
  para lo que aquí llamamos formas en general.
\end{remark}

\begin{example}
  El producto escalar de \(\RealSet^{n}\) es una 2-forma simétrica.
\end{example}

\begin{remark}
  \begin{enumerate}
  \item Obsérvese que para una r-forma alternada \(\alpha\) sobre \(V\) se tiene
    que si \(v_{i}=v_{j},\ i\neq j\), entonces claramente
    \(\alpha(v_{1},\dots,v_{r})=0\) en virtud de su propia definición.
  \item Más aún, como toda permutación es producto de trasposiciones entonces para
    toda r-forma alternada \(\alpha\) y toda permutación \(\sigma\) de
    \(\{1,\dots,r\}\), tenemos
    \[\alpha(v_{\sigma(1)},\dots,v_{\sigma(r)})=\text{signo}(\sigma)
    \alpha(v_{1},\dots,v_{r}),\] 
    donde signo\((\sigma)\) vale \(1\) si \(\sigma\) se descompone como un
    número par de trasposiciones y \(-1\) si el número es impar.
  \item Por otro lado, si \(\alpha\) es simétrica entonces
    \[\alpha(v_{\sigma(1)},\dots,v_{\sigma(n)})=\alpha(v_{1},\dots,v_{n})\] pues
    \(\alpha\) es invariante por trasposiciones.
  \end{enumerate}
\end{remark}

Ahora se plantea el problema de buscar un ''producto'' de \(V\) adecuado
para las formas alternadas (simétricas, respectivamente).

\section{Producto Exterior o Alternado}


\end{document}

%%% Local Variables:
%%% TeX-master: "../VD"
%%% End: