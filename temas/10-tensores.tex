\providecommand{\main}{..}
\documentclass[\main/VD_completo.tex]{subfiles}

\externaldocument{\main/VD}
\externaldocument{\main/temas/00-intro}
\externaldocument{\main/temas/01-cartas}
\externaldocument{\main/temas/02-aplicaciones}
\externaldocument{\main/temas/03-subvariedades}
\externaldocument{\main/temas/04-tangente}
\externaldocument{\main/temas/05-diferencial}
\externaldocument{\main/temas/06-campos}
\externaldocument{\main/temas/07-campos2}
\externaldocument{\main/temas/08-algebralie}
\externaldocument{\main/temas/09-cotangente}
% \externaldocument{\main/temas/10-tensores}

\begin{document}

\setcounter{chapter}{9}
\chapter{Espacios de formas y tensores sobre una variedad}\label{chap:dual}

\section{Introducción}

En \textbf{Álgebra Lineal} conocemos la extensión natural del espacio dual de
las \(1\)-formas por medio de las aplicaciones multilineales. Recordemos su definición.

\begin{definition}
  Dados \(n\) espacios vectoriales
  \(V_{1},\dots,V_{n}\) y otro espacio vectorial \(W\), una aplicación
     \(f\colon V_{1}\times\dots\times V_{n}\to W\) se dice \emph{multilineal} (de orden \(n\) con valores en \(W\)) si para cualesquiera 
     \(\lambda,\mu\in\RealSet\) y \(v_{i},v'_{i}\in V_{i},\ 1\leq i\leq n\), se cumple:

  \[
    f(v_{1},\dots,\lambda v_{i}+\mu v'_{i},\dots,v_{n})=\lambda
    f(\dots,v_{i},\dots)+\mu f(\dots,v'_{i},\dots).
  \]

  Cuando \(V_{1}=\dots=V_{n}\) y \(W=\RealSet\) decimos que \(f\) es una \emph{\(n\)-forma}
  sobre \(V\).  
\end{definition}

El conjunto de las aplicaciones multilineales de orden \(n\) sobre \(W\) se denota por \(Multi(V_1,V_2,\dots,V_n; W)\) y tiene una estructura de espacio
vectorial para las operaciones obvias. Si \(W = \RealSet\) se escribirá \(Multi(V_1,V_2,\dots,V_n)\), mientras que el espacio vectorial de las  \(n\)-formas sobre \(V\) será  denotado por \(Multi_n(V)\).

\par 

Ahora se plantea cómo extender las \(n\)-formas al espacio tangente de una variedad
\(M\). Ya vimos en el caso de las \(1\)-formas que esa construcción se puede abordar
de varias maneras:

\begin{enumerate}
\item Crear el espacio cotangente haciendo la construcción punto a punto y
  después definiendo una estructura coherente en todo \(M\): para cada \(p\in
  M\) sea \(\DualT[p]{M}\) y se define \(\DualT{M}=\bigcup_{p\in
    M}\DualT[p]{M}\), dotándolo de una estructura de variedad diferenciable
  compatible con la de \(M\) mediante la proyección
  \(\pi^{*}\colon\DualT{M}\to M\), \(\omega\in\DualT[p]{M}\mapsto p\), y las \(1\)-formas
  se definen entonces como secciones diferenciables de esa proyección.

\item Definir el espacio cotangente globalmente sobre todo \(\Tangente{M}\) por
  medio de funciones diferenciables \(f\colon\Tangente{M}\to\RealSet\) que punto
  a punto son \(1\)-formas, es decir, \(f\colon\Tangente[p]{M}\to\RealSet\) es
  lineal para todo \(p\in M\).

\item Considerar  el \(\mathcal{F}(M)\)-módulo de los campos sobre
  \(M\), \(\mathbb{X}(M)\), y definir \(\DualT{M}\) como el dual de \(\mathbb{X}(M)\) como
  \(\mathcal{F}(M)\)-módulo. Esto es, usar la identificación
  \(\DualT{M}\cong\text{Hom}(\mathbb{X}(M),\mathcal{F}(M))\), que lleva la forma
  \(\omega\colon M\to\Tangente{M}\) a la función \(\omega(X)(p)=X(\omega(p)),\
  \forall p\in M\).
\end{enumerate}

Veamos cómo extender estas construcciones equivalentes del espacio cotangente de
\(M\) para obtener la noción de \(n\)-forma sobre la variedad \(M\).

Para ello recordaremos la noción de producto tensorial que permite reinterpretar
las aplicaciones multilineales como \(1\)-formas, pasando la complicación de las
aplicaciones multilineales al espacio vectorial.

\section{Producto Tensorial}

\begin{definition}[name=producto tensorial]
  Recordemos que el \emph{producto tensorial} de los espacios vectoriales
  \(V_{1},\dots,V_{n}\), \(V_{1}\bigotimes\dots\bigotimes V_{n}\), es el espacio
  vectorial obtenido por combinaciones lineales de los símbolos
  \(v_{1}\otimes\dots\otimes\ v_{n}\) sujetos a las relaciones (\(1\leq i\leq n\)):

  \begin{itemize}
  \item \(\lambda(v_{1}\otimes\dots\otimes
    v_{n})=v_{1}\otimes\dots\otimes \lambda v_{i}\otimes\dots\otimes
    v_{n}\),

  \item \(v_{1}\otimes\dots\otimes(v_{i}+v'_{i})\otimes\dots\otimes
    v_{n}=v_{1}\otimes\dots\otimes \lambda v_{i}\otimes\dots\otimes
    v_{n}+v_{1}\otimes\dots\otimes \lambda v'_{i}\otimes\dots\otimes
    v_{n}\).
  \end{itemize}

  Se tiene entonces un isomorfismo \((V_{1}\bigotimes\dots\bigotimes
  V_{n})^{*}\cong\text{Multi}(V_{1},\dots,V_{n}),\ f\mapsto
  g_{f}(v_{1},\dots,v_{n})=f(v_{1}\otimes\dots\otimes\v_{n})\).
\end{definition}

\begin{remark}
  Además, se tiene que si \(B_{i}\) es una base de \(V_{i}\) entonces
  \(\{b_{k_{1}}\otimes\dots\otimes b_{k_{n}}\}\) es una base de
  \(V_{1}\bigotimes\dots\bigotimes V_{n}\) donde
  \((b_{k_{1}},\dots,b_{k_{n}})\in B_{1}\times\dots\times B_{n}\).
  Equivalentemente, las aplicaciones \(\delta_{k_{1},\dots,k_{n}}\) dadas por las
  extensiones multilineales de \(\delta_{k_{1},\dots,k_{n}}(b_{k_{1}}\dots
  b_{k_{n}})=1\) y \(\delta_{k_{1},\dots,k_{n}}(b_{k'_{1}}\dots b_{k'_{n}})=0\)
  si \((k'_{1},\dots,k'_{n})\neq(k_{1},\dots,k_{n})\) es una base de 
  \(Multi(V_{1}\times\dots\times V_{n})\).

  Si \(V_{1}=\dots=V_{n}=V\) denotamos por \(\bigotimes_{n}V\) a
    \(\underbracket{V\bigotimes\dots\bigotimes V}_{n\text{ veces}}\). 
    Tenemos así los isomorfismos  Multi\(_{n}(V)\cong(\bigotimes_{n}(V))^{*}\) y
  \(\dim{\text{Multi}_{n}(V)}=(\dim{V})^{n}\).

  También dadas \(\alpha\in(\bigotimes_{r}V)^{*}\) y
  \(\beta\in(\bigotimes_{s}V)^{*}\) se puede definir el producto tensorial
  \(\alpha\otimes\beta\in(\bigotimes_{r+s}V)^{*}\) por:
  \[
    \alpha\otimes\beta(v_{1}\otimes\dots\otimes v_{r}\otimes
    v_{r+1}\otimes\dots\otimes v_{r+s})=\alpha(v_{1}\otimes\dots\otimes
    v_{r})\cdot \beta(v_{r+1}\otimes\dots\otimes v_{r+s}).
  \]

  Obsérvese que existe un isomorfismo
  \(\rho\colon\bigotimes_{r}V^*\cong(\bigotimes_{r}V)^{*}\) que lleva el producto
  tensorial de \(1\)-formas \(\alpha_{1}\otimes\dots\otimes\alpha_{r}\) en la
  \(1\)-forma sobre \(\bigotimes_{r}V\) (o \(n\)-forma sobre \(V\)):
  \[
    \rho(\alpha_{1}\otimes\dots\otimes\alpha_{r})(v_{1}\otimes\dots\otimes
    v_{r})=\alpha_{1}(v_{1})\cdot \alpha_2(v_2) \cdot \dots \cdot \alpha_{r-1}(v_{r-1}) \cdot \alpha_{r}(v_{r}).
  \]

  Nótese que si \(\{e_{i}\}\) es una base de \(V\), los elementos básicos de
  \(\bigotimes_{r}V^{*}\), \(e_{i_{1}}^{*}\otimes\dots\otimes e_{i_{r}}^{*}\)
  pasan por \(\rho\) a los duales \((e_{i_{1}}\otimes\dots\otimes
  e_{i_{n}})^{*}\) de la base \(\{e_{i_{1}}\otimes\dots\otimes e_{i_{r}}\}\) de
  \(\bigotimes_{r}V\).
\end{remark}

Así pues, las opciones (equivalentes) para generalizar el espacio cotangente son
las siguientes:

\begin{enumerate}
\item Tomar para cada \(p\in M\) Multi\(_{n}(\Tangente[p]{M})\) y definir
  Multi\(_{n}\Tangente{M}=\bigcup_{p\in M}\text{Multi}_{n}(\Tangente[p]{M})\).
  Si se quiere usar el producto tensorial entonces se toma
  \(\bigotimes_{n}\Tangente[p]{M}\) y \(\bigotimes_{n}\Tangente{M}=\bigcup_{p\in
    M}\bigotimes_{n}\Tangente[p]{M}\). Habrá que dotar a
  Multi\(_{n}(\Tangente{M})\cong\bigotimes_{n}\Tangente{M}\) de una estructura de
  variedad de dimensión \(n\dim{M}\), para la cual la proyección
  \(\pi^{*}\colon\bigotimes_{n}\Tangente{M}\to M\) sea diferenciable y definir una
  \(n\)-forma sobre \(M\) como una sección de \(\pi\).

\item Definir una n-forma como una aplicación diferenciable
  \(f\colon\bigotimes_{n}\Tangente{M}\to\RealSet\) tal que para cada \(p\in M\)
  se restringe a una aplicación lineal
  \(f\colon\Tangente[p]{M}\otimes\dots\otimes\Tangente[p]{M}\to\RealSet\), o,
  equivalentemente multilineal
  \(f\colon\Tangente[p]{M}\times\dots\times\Tangente[p]{M}\to\RealSet\).

  Esta construcción es equivalente a definir primero el espacio tangente
  ''producto'' \((\Tangente{M})^{n}=\bigcup_{p\in M}(\Tangente[p]{M})^{n}\),
  donde
  \((\Tangente[p]{M})^{n}=\Tangente[p]{M}\times\dots\times\Tangente[p]{M}\) que
  resulta ser una variedad de dimensión \(n\dim{M}\). Ahora una n-forma sobre
  \(M\) será una aplicación diferenciable
  \(f\colon(\Tangente{M})^{n}\to\RealSet\) tal que para todo \(p\in M\) se
  restringe a una aplicación multilineal
  \(f\colon\Tangente[p]{M}\times\dots\times\Tangente[p]{M}\to\RealSet\).

\item Extender las nociones de aplicación multilineal y producto tensorial a
  módulos sobre un anillo \(A\). Entonces considerar las aplicaciones
  multilineales de \(\mathcal{F}(M)\)-módulos
  \(\mathbb{X}(M)\times\dots\times\mathbb{X}(M)\to\RealSet\) o bien las
  aplicaciones lineales de \(\mathcal{F}(M)\)-módulos
  \(\mathbb{X}(M)\bigotimes\dots\bigotimes\mathbb{X}(M)\to\mathcal{F}(M)\).
\end{enumerate}

\begin{lemma}
  Las tres construcciones anteriores son equivalentes.
\end{lemma}

\begin{proof}
  Veamos cómo pasar de una a otra construcción:
  \begin{itemize}
  \item \((1)\iff(2)\). Si \(s\colon M\to\text{Multi}_{n}(\Tangente{M})\) es una
    sección se define
    \(f_{s}\colon\Tangente[p]{M}\times\dots\times\Tangente[p]{M}\to\RealSet\) por
    \(f_{s}(v_{1},\dots,v_{n})=s(p)(v_{1},\dots,v_{n}),\
    v_{i}\in\Tangente[p]{M}\).
    Recíprocamente, dada \(f\colon\bigotimes_{n}\Tangente{M}\to\RealSet\) se
    define \(s_{f}(p)(v_{1},\dots,v_{n})=f(v_{1}\otimes\dots\otimes v_{n}),\
    v_{i}\in\Tangente[p]{M}\), con la notación del producto tensorial.

  \item \((2)\iff(3)\). Si \(f\colon\bigotimes_{n}\Tangente{M}\to\RealSet\)
    entonces
    \(\phi_{f}\colon\mathbb{X}(M)\times\dots\times\mathbb{X}(M)\to\mathcal{F}(M)\)
    está dada por \(\phi_{f}(X_{1},\dots,X_{n})(p)=f(X_{1}(p),\dots,X_{n}(p))\).
    Recíprocamente, si
    \(\phi\colon\mathbb{X}(M)\times\dots\times\mathbb{X}(M)\to\mathcal{F}(M)\)
    entonces \(f_{\phi}(v_{1}\otimes\dots\otimes
    v_{n})=\phi(X_{v_{1}},\dots,X_{v_{n}})\) donde \(X_{v_{i}}\) son campos
    cualesquiera con \(X_{v_{i}}(p)=v_{i}\).
  \end{itemize}
\end{proof}

\section{Tensores}

Ahora podemos complicarlo más y mezclar el espacio tangente \(\Tangente{M}\) y
el cotangente \(\DualT{M}\) en las construcciones anteriores y hablar de
\textbf{tensores}.

\begin{definition}[name=tensores]
  Por un \emph{tensor \(r\)-covariante} se entenderá una \(r\)-forma sobre el espacio
  cotangente de \(M\), \(\bigotimes_{r}\DualT{M}\to\RealSet\) mientras que un
  \emph{tensor \(s\)-contravariante} es una \(s\)-forma del tangente,
  \(\bigotimes_{s}\Tangente{M}\to\RealSet\).

  \par

  Todavía más, si consideramos la variedad
  \[(\bigotimes_{s}\Tangente{M})\otimes(\bigotimes_{r}\DualT{M})=\bigcup_{p\in
    M}(\bigotimes_{s}\Tangente[p]{M})\otimes(\bigotimes_{r}\DualT[p]{M}),
    \] 
    una aplicación diferenciable
  \((\bigotimes_{s}\Tangente{M})\otimes(\bigotimes_{r}\DualT{M})\to\RealSet\)
  que es lineal en cada \(p\in M\) será entonces un \emph{tensor mixto de tipo
    \((s,r)\)}.
\end{definition}

\begin{remark}
  Si \(m=\dim{M}\) y tomamos una carta entorno de \(p\in M\)
  \((U,\varphi=(x_{1},\dots,x_{n}))\), con la notación
  \(\dif{x_{1}},\dots,\dif{x_{n}}\) y
  \(\pderiv{}{x_{1}},\dots,\pderiv{}{x_{n}}\) para la base canónica de
  \(\DualT[p]{M}\) y \(\Tangente[p]{M}\), tenemos que un tensor \(r\)-covariante se escribe 
  localmente 
  \[\sum_{(i_{1},\dots,i_{r}) \in \{1,\dots, m\}^r}g_{i_{1},\dots,i_{r}}\dif{x_{i_{1}}}\otimes\dots\otimes
  \dif{x_{i_{r}}}
  \] 
  y análogamente un tensor \(s\)-contravariante se escribe como
  \[\sum_{(j_{1},\dots,j_{s})\in\{1,\dots,m\}^{s}}f_{j_{1},\dots,j_{s}}\pderiv{}{x_{j_{1}}}\otimes\dots
  \otimes\pderiv{}{x_{j_{s}}}.\]

  Para un tensor mixto \((s,r)\) tenemos
  \[\sum_{(i_{1},\dots,i_{r},j_{1},\dots,j_{s}) \in \{1,\dots, m\}^r \times\{1,\dots, m\}^s} h_{i_{1},\dots,i_{r}}^{j_{1},\dots,j_{s}}
  \pderiv{}{x_{j_{1}}}\otimes\dots\otimes\pderiv{}{x_{j_{s}}}\otimes
  \dif{x_{i_{1}}}\otimes\dots\otimes\dif{x_{i_{r}}}.\]

 En las fórmulas anteriores las funciones 
 \[f_{j_{1},\dots,j_{s}},g_{i_{1},\dots,i_{r}},h_{i_{1},\dots,i_{r}}^{j_{1},\dots,j_{s}}\colon\varphi(U)\to\RealSet
 \] 
 se denominan \emph{componentes} del correspondiente tensor en la carta \((U,\varphi)\).
\end{remark}

\section{Formas Simétricas y Alternadas}

Existen dos clases de formas sobre un espacio vectorial \(V\) de especial interés:
las \textbf{formas simétricas y las antisimétricas o alternadas}.

\begin{definition}[name=forma simétrica]
  Una \(r\)-forma \(\alpha\colon V\times\dots\times V\to\RealSet\) se dice
  \emph{simétrica} si
  \[\alpha(v_{1},\dots,v_{i},\dots,v_{j},\dots,v_{r})=
  \alpha(v_{1},\dots,v_{j},\dots,v_{i},\dots,v_{r})\] para todo \(1\leq i\leq
  j\leq r\).
\end{definition}

\begin{definition}[name=forma alternada]
  Una \(r\)-forma \(\alpha\colon V\times\dots\times V\to\RealSet\) se dice
  \emph{antisimétrica o alternada} si
  \[\alpha(v_{1},\dots,v_{i},\dots,v_{j},\dots,v_{r})=
  -\alpha(v_{1},\dots,v_{j},\dots,v_{i},\dots,v_{r})\] para todo \(1\leq i\leq
  j\leq r\).
\end{definition}

\begin{remark}
  Las formas alternadas también son llamadas \emph{exteriores} en algunas
  referencias. Incluso es bastante habitual que se reserve el nombre de
  ''forma'' exclusivamente para las alternadas, dejando aplicación multilineal
  para lo que aquí llamamos formas en general.
\end{remark}

\begin{example}
  El producto escalar de \(\RealSet^{n}\) es una \(2\)-forma simétrica.
\end{example}

\begin{remark}\label{rem:inv-trasp}
  \begin{enumerate}
  \item Para toda \(r\)-forma alternada \(\alpha\) sobre \(V\) se tiene
    que si \(v_{i}=v_{j},\ i\neq j\), y entonces 
    \(\alpha(v_{1},\dots,v_{r})=0\) en virtud de su propia definición.
  \item Más aún, como toda permutación es producto de trasposiciones entonces para
    toda \(r\)-forma alternada \(\alpha\) y toda permutación \(\sigma\) de
    \(\{1,\dots,r\}\), tenemos
    \[\alpha(v_{\sigma(1)},\dots,v_{\sigma(r)})=s(\sigma)
    \alpha(v_{1},\dots,v_{r}),\]
    donde \(s(\sigma)\) vale \(1\) si \(\sigma\) se descompone como un
    número par de transposiciones y \(-1\) si el número es impar.
  \item Por otro lado, si \(\alpha\) es simétrica entonces
    \[\alpha(v_{\sigma(1)},\dots,v_{\sigma(n)})=\alpha(v_{1},\dots,v_{n})\] pues
    \(\alpha\) es invariante por trasposiciones.
  \end{enumerate}
\end{remark}

Ahora se plantea el problema de buscar un ''producto'' de \(V\) 
que permita describir las formas alternadas (simétricas, respectivamente) como formas duales para ese producto.

\section{Producto Exterior y simétrico}
\begin{definition}[name= producto exterior]
El \emph{\(r\)-producto  exterior (o alternado)} de \( V \), denotado \( \Lambda^rV \), es el espacio vectorial obtenido al hacer cociente del producto tensorial \( \otimes_rV=\underbracket{V\otimes \ldots \otimes V}_{r} \) por el subespacio generado por los productos \( v_1\otimes\ldots\otimes v_r\) con \( v_i= v_j \) para algún \(  i\neq j\).
\end{definition}

La clase de \( v_1\otimes\ldots\otimes v_r \) se denota \( v_1\wedge\ldots\wedge v_r \) y se llama \emph{producto exterior (o alternado)} de los vectores \( v_1,\ldots,v_r \).

De manera análoga se define el producto para las formas simétricas como sigue:

\begin{definition}[name=producto simétrico]
El \emph{\(r\)-producto simétrico} de \( V \), denotado \( S^rV \), es el espacio vectorial cociente de \( \otimes_rV \) por el subespacio generado por las diferencias
\[
(v_1\otimes \ldots\otimes v_i\otimes \ldots \otimes v_j\otimes \ldots \otimes
v_r)-
(v_1\otimes \ldots\otimes v_j\otimes \ldots \otimes v_i\otimes \ldots \otimes v_r),
\]
para todo \( i\neq j \).
\end{definition}

A la clase de \( v_{1}\otimes \ldots \otimes v_{r} \) se le denota \( v_{1}\ldots v_{r}\) y se llama \emph{producto simétrico} de los vectores \(v_{1},\ldots ,v_{r}  \).

Se tienen así aplicaciones cociente
\[
\pi_a\colon \otimes_r V\to \Lambda^rV \mbox{ y } \pi_s\colon \otimes_rV\to S^rV.
\]

Sean \( Alt_r(V) \) y \( Sim_r(V) \) los subespacios de \( Multi_r(V) \) formados por las \(r\)-formas alternas y simétricas, respectivamente. Vamos a construir aplicaciones lineales
\[
Alt\colon Multi_r(V)\to Alt_r(V), \quad Sim\colon Multi_r(V)\to Sim_r(V)
\]
que se correspondan con las proyecciones $\pi_a$ y $\pi_s$.
Para ello observamos que si \( \beta\colon V\times V\to \RealSet \) es cualquier \(r\)-forma y \( \sigma \) es una permutación de \( \{1,\ldots,r\} \), tenemos entonces una nueva \(r\)-forma \[ ^\sigma\beta\colon V\times\ldots V\to \RealSet 
\] 
definida como
\[
^\sigma\beta(v_1,\ldots,v_r)=\beta(v_{\sigma(1)},\ldots,v_{\sigma(r)})
\]

\begin{note}
Si \( \tau \) es otra permutación entonces para la composición \( \tau\sigma \) se tiene \( ^{\tau\sigma}\beta=^\tau(^\sigma\beta) \)
\end{note}

\section{Funciones de simetrización  y alternancia}

\begin{definition}
Dada la \(r \)-forma \( \beta \) se define
\[
Sim(\beta)=\frac{1}{r!}\sum_{\sigma\in \mathfrak{G}_r} \ ^\sigma\beta,
\]
donde \( \mathfrak{G}_r \) es el grupo de las permutaciones de \( r \) elementos.
\end{definition}

A \(Sim(\beta)\) se le llama la \emph{simetrización} de (o la \emph{forma simétrica asociada} a) la forma \(\beta\). 

\begin{lemma}
\( Sim(\beta) \)es efectivamente una $r$-forma simétrica.
\end{lemma}

\begin{proof}
Sea \( \tau \) trasposición de \( i \) por \( j \). Tenemos
\begin{align*}
Sim(\beta)(v_1,\ldots, v_i,\ldots, v_j,\ldots, v_r)&=\frac{1}{r!}\sum_{\sigma\in \mathfrak{G}_r} \ ^{\sigma\tau}\beta (v_1,\ldots, v_j,\ldots, v_i,\ldots, v_r)\\
&=\frac{1}{r!}\sum_{\sigma'\in \mathfrak{G}_r} \ ^{\sigma'}\beta(v_1,\ldots, v_j,\ldots, v_i,\ldots, v_r)\\
&=Sim(\beta)(v_1,\ldots, v_j,\ldots, v_i,\ldots, v_r).
\end{align*}
Donde hemos usado que todo \( \sigma'\in \mathfrak{G}_r \) se puede reescribir como \( \sigma \tau \) siendo \( \sigma \) único si ya hemos fijado \( \tau \).
\end{proof}

\begin{lemma}
Si \( \beta \) ya es simétrica entonces \( Sim(\beta)=\beta \)
\end{lemma}

\begin{proof}
De acuerdo con la Observación \ref{rem:inv-trasp}(3) tenemos
\begin{align*}
Sim(\beta)(v_1\ldots v_r)&=\frac{1}{r!}\sum_{\sigma\in \mathfrak{G}_r} \  ^\sigma\beta(v_1,\ldots, v_r)\\
&=\frac{1}{r!}(r!)\beta(v_1,\ldots, v_r)\\
&=\beta(v_1,\ldots, v_r).
\end{align*}
\end{proof}


De manera análoga se define una función de alternancia para \(r\)-formas como sigue.
\begin{definition}
Sea \( \beta\in Multi_r(V) \), se define su \emph{forma alternada asociada} por la fórmula
\[
Alt(\beta)=\frac{1}{r!}\sum_{\sigma\in \mathfrak{G}_r} s(\sigma) ^\sigma\beta
\]
\end{definition}


\begin{lemma}
\( Alt(\beta) \) es una \(r\)-forma alternada.
\end{lemma}

\begin{proof}
	Sea \( \tau \) trasposición de \( i \) por \( j \). Tenemos
\[\begin{array}{l}
Alt(\beta)(v_1,\dots, v_i,\dots, v_j,\dots, v_r)=\\ \\=\frac{1}{r!}\sum_{\sigma\in
    \mathfrak{G}_r} s(\sigma) ^\sigma\beta(v_1,\ldots, v_i,\ldots, v_j,\ldots, v_r)\\ \\
=\frac{1}{r!}\sum_{\sigma\in \mathfrak{G}_r} s(\sigma) \beta(v_{\sigma(1)},\ldots, v_{\sigma(i)},\ldots, v_{\sigma(j)},\ldots, v_{\sigma(r)})
\\ \\=\frac{1}{r!}\sum_{\sigma\in \mathfrak{G}_r} s(\sigma)
    \beta(v_{\tau\sigma(1)},\ldots, v_{\tau\sigma(j)},\ldots,
    v_{\tau\sigma(i)},\ldots, v_{\tau\sigma(r)})\\ \\
=\frac{1}{r!}\sum_{\sigma\in \mathfrak{G}_r}-s(\tau\sigma) ^{\tau\sigma}\beta(v_1,\ldots, v_j,\ldots, v_i,\ldots, v_r)
\\ \\=-\frac{1}{r!}\sum_{\sigma'\in \mathfrak{G}_r}s(\sigma') ^{\sigma'}\beta(v_1,\ldots,
    v_i,\ldots, v_j,\ldots, v_r)\\ \\
=-Alt(\beta)(v_1,\ldots, v_j,\ldots, v_i,\ldots, v_r).
\end{array}\]
\end{proof}

\begin{lemma}
Si \( \beta\in Alt_r(V) \) entonces \( Alt(\beta)=\beta \).
\end{lemma}

\begin{proof}
Sabemos que \( ^\sigma\beta =s(\sigma)\beta \) de acuerdo con la Observación\ref{rem:inv-trasp}(2)  y por tanto
\begin{align*}
Alt(\beta)&=\frac{1}{r!}\sum_{\sigma\in \mathfrak{G}_r}s(\sigma) ^\sigma\beta\\
&=\frac{1}{r!}(r!)\beta\\
&=\beta
\end{align*}
\end{proof}

Además tenemos un diagrama que conecta todos los conceptos vistos hasta ahora, donde \( \rho \) es la biyección que lleva la \(1\)-forma \( \gamma\colon V_1\otimes\ldots \otimes V_r\to \RealSet \) en la \(r\)-forma \( \rho(\gamma)\colon V_1\times\ldots\times V_r\to \RealSet \) definida como
\[
\rho(\gamma)(v_1\ldots v_r)=\gamma(v_1\otimes \ldots \otimes v_r).
\]

\begin{figure}[h]
	\centering
	\begin{tikzcd}
		Alt_r(V) & Multi_r(V) \arrow[l, "Alt"'] \arrow[r ,"Sim"]& Sim_r(V)\\
		(\Lambda^r V)^* \arrow[r, "\pi_a^*"] \arrow[u, "\rho_a"]& (\otimes_r V)^* \arrow[u, "\rho", "\homeo"'] & (S^rV)^* \arrow[l, "\pi_s^*"'] \arrow[u ,"\rho_s"]
	\end{tikzcd}
	%\captionof{figure}{}
	\label{fig:iso-ext}
\end{figure}
Entonces \( \rho_s=Sim\circ \rho \circ \pi_s^* \) y \( \rho_a=Alt\circ \rho \circ \pi_a^* \) son también isomorfismos.

\begin{proposition}\label{ROA}
\( \rho_a \) lleva la \(1\)-forma \( \gamma\colon \Lambda^rV\to \RealSet \) en la \(r\)-forma alternada
\[
\rho_a(\gamma)(v_1,\ldots,v_r)=\gamma(v_1\wedge\ldots\wedge v_r).
\]
\end{proposition}

\begin{proof}
Teniendo en cuenta la conmutatividad del diagrama anterior
\[\begin{array}{l}
\rho_a(\gamma)(v_1\ldots v_r)=Alt\circ \rho_a\circ \pi_a^*(\gamma)
=Alt(\rho(\gamma\circ \pi_a))(v_1\ldots v_r)=
\\ \\ \frac{1}{r!}\sum_{\sigma\in \mathfrak{G}_r} s(\sigma) ^\sigma\rho(\gamma\circ \pi_a)(v_1\ldots v_r)=
\\ \\ \frac{1}{r!}\sum_{\sigma\in \mathfrak{G}_r} s(\sigma) \rho(\gamma\circ \pi_a)(v_{\sigma(1)}\ldots v_{\sigma(r)})=
\\ \\ \frac{1}{r!}\sum_{\sigma\in \mathfrak{G}_r} s(\sigma) \gamma\circ
    \pi_a(v_{\sigma(1)}\otimes\ldots\otimes v_{\sigma(r)}) = \\ \\
\frac{1}{r!}\sum_{\sigma\in \mathfrak{G}_r} s(\sigma) s(\sigma) \gamma\circ \pi_a(v_{1}\otimes\ldots\otimes v_{r})
\\ \\ \frac{1}{r!}(r!\gamma(v_1\wedge\ldots \wedge v_r))
=\gamma(v_1\wedge\ldots \wedge v_r).
\end{array}\]
Aquí hemos usado que \( v_{\sigma(1)}\otimes \ldots v_{\sigma(r)} \) y \( s(\sigma)(v_1\otimes\ldots \otimes v_r) \) representan la misma clase en \( \Lambda^rV \) como cociente de \( \otimes_rV \).
\end{proof}

De aquí se sigue inmediatamente que \(\rho_a \) es inyectiva. \\
Para la sobreyectividad, si \( \xi\colon V_1\times\ldots\times V_r\to \RealSet\) es una \(r\)-forma alternada, entonces \( \rho^{-1}(\xi)\colon V_1\otimes\ldots\otimes V_r\to \RealSet \) cumple
\begin{align*}
\rho^{-1}(\xi)(v_1\otimes\ldots\otimes v_i\otimes\ldots\otimes v_j\otimes\ldots\otimes v_r)&=\xi(v_1,\ldots,v_i,\ldots,v_j,\ldots,v_r)\\
&=-\xi (v_1,\ldots,v_j,\ldots,v_i, \ldots,v_r)=-\rho^{-1}(\xi)(v_1\otimes\ldots\otimes v_j\otimes\ldots\otimes v_i\otimes\ldots\otimes v_r),
\end{align*}
por lo que es \(\rho^{-1}\) es compatible con el cociente que define \( \Lambda^rV \) e induce una \(1\)-forma \( \tilde{\rho}^{-1}(\xi)\colon \Lambda^rV\to \RealSet \) con
\[
\pi_a^*(\tilde{\rho}^{-1}(\xi))=\tilde{\rho}^{-1}(\xi)\circ \pi =\rho^{-1}(\xi).
\]
Así pues \( \rho_a(\tilde{\rho}^{-1}(\xi))=\xi \).

\begin{remark}
Para \( \rho_s\colon (S^rV)^*\to Sim_r(V) \) tenemos análogamente que si \( \nu\in (S^rV)^* \) entonces
\[
\rho_s(\nu)(v_1,\ldots,v_r)=\nu(\underbracket{v_1v_2\cdots v_r}_{\text{prod simétrico}}).
\]
\end{remark}

Hemos probado así que existen  isomorfismos
\[
(\Lambda^rV)^*\homeo Alt_r(V) \mbox{ y } (S^rV)^*\homeo Sim_r(V).
\]

\section{Dimensiones}

En esta sección establecemos las dimensiones de los productos exteriores y simétricos sobre un espacio vectorial \(V\) de base 
\( \{e_i\}_{i=1}^n \).
\begin{proposition}
Los productos exteriores \( e_{i_1}\wedge \ldots\wedge e_{i_r} \) con \( i_1<\ldots< i_r \) forman base de \( \Lambda^rV \). En particular su dimensión es
\[
\binom{n}{r}=\frac{n!}{r!(n-r)!}.
\]
\end{proposition}

\begin{proof}
Podemos comprobar que toda r-forma en \( Alt_r(V) \) queda determinada por sus valores en \( (e_{i_1},\ldots,e_{i_r}) \) con \( i_1<\ldots <i_r \). Por tanto su dimensión es el número de posibles combinaciones de tales subíndices.
\end{proof}

También se pude determinar la dimensión de \( S^rV \), si bien la demostración es más complicada.

\begin{proposition}
Una base de \( S^rV \) la forman los productos simétricos \( e_{i_1}\cdots e_{i_r} \) con \( i_1\leq \ldots\leq i_r \) y su dimensión es
\[
dimS^r(V)=\binom{n+r-1}{r} =\frac{(n+r-1)!}{r!(n-1)!}
\]
\end{proposition}

\begin{proof}
La prueba se hará en clase de problemas.
\end{proof}

\begin{note}
Observese que para \( n=dim(V) \) tenemos \( \Lambda^nV\homeo \RealSet \), estando \( \Lambda^nV \) generado por \( e_1\wedge\ldots\wedge e_n \).
\end{note}

De acuerdo con los resultados anteriores y el diagrama \ref{fig:iso-ext}, las dimensiones de los espacios de las \(r\)-formas alternadas y simétricas sobre \(V\) son, respectivamente,  \(\dim Alt_r(V) = \binom{n}{r}\)  y \(\dim Sim_r(V) = \binom{n+r-1}{r}\).
 
\begin{proposition}
Toda \(n\)-forma alternada de \(Alt_r(V) \cong \Lambda^r(V) \cong \RealSet\) se puede interpretar como un determinante. Explícitamente, si $\mu \in Alt_n(V)\), entonces $\mu(v_1,\ldots,v_n) = \det A(v_1,\ldots,v_n) \mu(e_1,\ldots, e_n)\), donde \(A(v_1,\ldots,v_n)\) es la matriz formada por los vectores \(v_i\) como columnas.  
\end{proposition}

\begin{proof}
Si \( v_j=\sum_{i=1}^{n}\lambda_j^ie_i \) y vemos $\mu$ como una \(1\)-forma del producto tensorial \(\otimes_n V\), entonces tenemos por la multilinealidad del producto tensorial
\begin{align*}
\mu(v_1 \otimes \ldots \otimes v_n)&=\mu(\sum_{i=1}^n \lambda_1^ie_i\otimes\ldots \otimes\sum_{i=1}^n\lambda_n^i e_i)\\
&=\sum_{(i_1,\ldots,i_n)\in \{1,\ldots,n\}^n} \lambda_1^{i_1}\cdots\lambda_n^{i_n}\mu(e_{i_1}\otimes\ldots\otimes e_{i_n}).
\end{align*}
Ahora bien, por ser \( \mu \) alternada ésta se anula cuando aparecen índices repetidos. Así pues en la suma anterior desaparecen todos los sumandos  donde \( i_1\ldots i_n\) no sea una permutación de \( \{1,\ldots,n\} \) y por tanto 
\begin{align*}
\mu(v_1\otimes \ldots \otimes v_n)&=\sum_{\sigma\in \mathfrak{G}_n}\lambda_1^{\sigma(1)}\cdots\lambda_n^{\sigma(n)}\mu(e_{\sigma(1)}\otimes\ldots\otimes e_{\sigma(n)}) \\
&=\sum_{\sigma\in \mathfrak{G}_n} s(\sigma)\lambda_1^{\sigma(1)}\cdots\lambda_n^{\sigma(n)}\mu(e_1\otimes\ldots\otimes e_n) \\
&=\lambda\mu(e_1\otimes\ldots\otimes e_n),\\
\end{align*}
donde \( \lambda=\sum_{\sigma\in \mathfrak{G}_n}s(\sigma)\lambda_1^{\sigma(1)}\ldots\lambda_n^{\sigma(n)}=\det A(v_1,\ldots,v_n) \).
\end{proof}

\section{Producto simétrico y exterior entre formas}
Dadas una \(r\)-forma \( \alpha\colon\otimes_rV\to \RealSet \) y una \(s\)-forma \( \beta\colon\otimes_sV\to \RealSet \) habíamos definido la \((r+s)\)-forma \( \alpha\otimes\beta\colon\otimes_{r+s}V\to \RealSet \) por
\[
\alpha\otimes\beta(v_1\otimes\ldots\otimes v_r\otimes v_{r+1}\otimes\ldots\otimes v_{r+s})=\alpha(v_1\ldots v_r)\cdot\beta(v_{r+1}\ldots v_{r+s}).
\]

Si \(\alpha\) y \(\beta\) son simétricos entonces \( \alpha\otimes\beta \) no tiene que ser simétrico. Lo mismo ocurre si son alternadas. Es por ellos que buscamos definir nuevos productos que sean internos para las formas simétricas y alternadas.

\begin{definition}
El \emph{producto simétrico} de la \(r\)-forma simétrica \( \alpha \) con la \(s\)-forma simétrica \( \beta \) es la simetrización 
del producto anterior; esto es, 
\[
\alpha\cdot\beta=Sim(\alpha\otimes\beta).
\]
\end{definition}

De acuerdo con la definición anterior, tenemos explícitamente para  un elemento genérico de \(V\times \ldots \times V\) (\(r+s\) veces) 

\begin{align*}
  \alpha \cdot \beta(v_1,\dots,v_r,v_{r+1},\dots,v_{r+s})
  = \\
  \frac{1}{(r+s)!}
  \sum_{\sigma\in \mathfrak{G}_{r+s}}
  \alpha(v_{\sigma(1)}\dots v_{\sigma(r)})
  \beta(v_{\sigma(r+1)}\dots v_{\sigma(r+s)}).
\end{align*}

\begin{definition}
El \emph{producto exterior (o alternado)} de la \(r\)-forma alternada \( \alpha \) con la \(s\)-forma alternada \( \beta \) es
la forma alternada asociada a $\alpha \otimes \beta$; es decir,  
\[
\alpha\wedge\beta=Alt(\alpha\otimes\beta); 
\]
explícitamente,
\begin{align*}
&\alpha\wedge\beta(v_1,\ldots,v_r,v_{r+1},\ldots,v_{r+s})=\\
  &\frac{1}{(r+s)!}
    \sum_{\sigma\in \mathfrak{G}_{r+s}}
    s(\sigma)
    \alpha(v_{\sigma(1)} \ldots v_{\sigma(r)})
    \beta(v_{\sigma(r+1)}\ldots v_{\sigma(r+s)}).\\
\end{align*}
\end{definition}

Veamos ahora una serie de propiedades acerca de estos nuevos productos.

\begin{proposition}\label{prop:ext-proper}
El producto exterior de formas alternadas cumple las siguientes propiedades:
\begin{enumerate}
\item Bilinealidad: \( (a\alpha+b\beta)\wedge \eta=a(\alpha\wedge\eta)+b(\beta\wedge\eta); a,b\in \RealSet. \)
\item Asociatividad: \( (\alpha\wedge\beta)\wedge \gamma=\alpha\wedge(\beta\wedge\gamma) \).
\item Anticonmutatividad: \( \omega\wedge\eta=(-1)^{rs}\eta\wedge\omega \) si \( \omega\in \Lambda^r(V) \) y \( \eta\in \Lambda^s(V) \). En particular, \( \omega\wedge\omega=0 \) si \(r\) es impar.
\end{enumerate}
\end{proposition}

\begin{proof}
\begin{enumerate}
\item Esta propiedad es inmediata a partir de la definición.
\item Es consecuencia del Lema \ref{lem:transi} que sigue.
\item Observamos que cada término de
\begin{align*}
&Alt(\omega\otimes\eta)(v_1,\ldots,v_r,v_{r+1},\ldots,v_{r+s})=\\
&\frac{1}{(r+s)!}\sum_{\sigma\in \mathfrak{G}_{r+s}}s(\sigma)\omega(v_{\sigma(1)},\ldots,\omega_{\sigma(r)})\eta(v_{\sigma(r+1)},\ldots,v_{\sigma(r+s)})
\end{align*}
aparece en
\begin{align*}
&Alt(\eta\otimes\omega)(v_1,\ldots,v_r,v_{r+1},\ldots,v_{r+s})=\\
&\frac{1}{(r+s)!}\sum_{\sigma'\in \mathfrak{G}_{r+s}}s(\sigma')\eta(v_{\sigma'(1)},\ldots,\omega_{\sigma'(r)})\omega(v_{\sigma'(r+1)},\ldots,v_{\sigma'(r+s)})
\end{align*}
En efecto, dada \( \sigma \), sea \( \sigma' \) definida por
\(
\sigma'(1)=\sigma(r+1),\ldots,\sigma'(s)=\sigma(r+s),\sigma'(s+1)=\sigma(1),\ldots,\sigma'(s+r)=\sigma(r)
\)
Falta por ver la relación entre los signos. Es fácil ver que para pasar de \( \sigma'(1) \) a \( \sigma(1) \) necesitamos trasponer r posiciones y lo mismo ocurre para cualquier otro índice. Así pues \( s(\sigma)= s(\sigma')(-1)^{rs} \).
\end{enumerate}
\end{proof}

\begin{lemma}\label{lem:transi}
	\begin{enumerate}
\item [(1)] Supongamos que la \(r\)-forma \( \alpha \) cumple \( Alt(\alpha)=0 \). Entonces para toda \(s\)-forma \( \beta \) se tiene \( Alt(\alpha\otimes\beta)=Alt(\beta\otimes\alpha)=0 \).
\item [(2)] \( Alt(Alt(\omega\otimes\eta)\otimes\theta)=Alt(\omega\otimes\eta\otimes\theta)=Alt(\omega\otimes Alt(\eta\otimes \theta)) \).
\end{enumerate}
\end{lemma}

\begin{proof}
\begin{enumerate}
\item [(1)] Desarrollamos uno de los dos dado que son análogos.
\begin{align*}
&Alt(\alpha\otimes\beta)(v_1\ldots v_{r+s})=\\
&=\frac{1}{(r+s)!}\sum_{\sigma\in \mathfrak{G}_{r+s}} s(\sigma)\alpha(v_{\sigma(1)},\ldots,v_{\sigma(r)})\beta(v_{\sigma(r+1)},\ldots,v_{\sigma(r+s)}).
\end{align*}
Para probar que esta suma es nula se descompondrá \( \mathfrak{G}_{r+s} \) en subconjuntos disjuntos tales que sumas extendidas a esos subconjuntos serán todas nulas.\\
Empezamos tomando \( \mathfrak{G}\subseteq \mathfrak{G}_{r+s} \) formado por las permutaciones que dejen fijo \( r+1,\ldots,r+s \). Entonces si \( \sigma' \) es la restricción de \( \sigma \in \mathfrak{G}\) a \( \{1,\ldots,r\} \), tenemos que \( \sigma\mapsto \sigma' \) da un isomorfismo \( \mathfrak{G}\homeo \mathfrak{G}_r \) y además \( s(\sigma)=s(\sigma') \). Entonces
\begin{align*}
&\sum_{\sigma\in \mathfrak{G}} s(\sigma)\alpha(v_{\sigma(1)},\ldots, v_{\sigma(r)})\beta(v_{\sigma(r+1)},\ldots,v_{\sigma(r+s)})=\\
&=\sum_{\sigma\in \mathfrak{G}} s(\sigma)\alpha(v_{\sigma(1)},\ldots,v_{\sigma(r)})\beta(v_{r+1},\ldots,v_{r+s})=\\
&=(\sum_{\sigma'\in \mathfrak{G}_{r}}s(\sigma')\alpha(v_{\sigma'(1)},\ldots,v_{\sigma'(r)}))\beta(v_{r+1},\ldots,v_{r+s})=\\
&=r!Alt(\alpha)\beta(v_{r+1}\ldots v_{r+s})=0.
\end{align*}
Sea \( \sigma_1\notin \mathfrak{G} \). Entonces \( \mathfrak{G}\sigma_1=\{\sigma\sigma_1\colon\sigma\in \mathfrak{G}\} \) es disjunto con \( \mathfrak{G} \) pues si \( \mu\in \mathfrak{G}\sigma_1\cap \mathfrak{G}\) entonces \( \mu=\sigma\sigma_1 \) con \( \mu,\sigma\in \mathfrak{G} \) y se cumpliría \( \sigma_1=\sigma^{-1}\mu\in \mathfrak{G} \) que es una contradicción.\\
El sumando generado por \( \mathfrak{G}\sigma_1 \) es
\begin{align*}
&\lambda_{\sigma_1}=\sum_{\mu\in \mathfrak{G}\sigma_1}s(\mu)\alpha(v_{\mu(1)}\ldots v_{\mu(r)})\beta(v_{\mu(r+1)}\ldots v_{\mu(r+s)})=\\
&=\sum_{\sigma\in \mathfrak{G}} s(\sigma\sigma_1)\alpha(v_{\sigma\sigma_1(1)}\ldots v_{\sigma\sigma_1(r)})\beta(v_{\sigma\sigma_1(r+1)}\ldots v_{\sigma\sigma_1(r+s)})\quad (*)
\end{align*}
Tomando \( (w_1\ldots w_{r+s})=(v_{\sigma_1(1)}\ldots v_{\sigma_1(r+s)}) \), tenemos que \( w_{\sigma(i)}=v_{\sigma\sigma_1(i)} \) y por tanto
\begin{align*}
(*)&= s(\sigma_1)\sum_{\sigma\in \mathfrak{G}}s(\sigma)\alpha(w_{\sigma(1)}\ldots w_{\sigma(r)})\beta(w_{\sigma(r+1)}\ldots w_{\sigma(r+s)})\\
&= s(\sigma_1)(r!Alt(\alpha)(w_1\ldots w_r))\beta(w_{r+1}\ldots w_{r+s}) = 0.\\
\end{align*}
Si tomamos \( \sigma_2\notin \mathfrak{G}\cup \mathfrak{G}\sigma_1 \), tenemos \( \mathfrak{G}\cap \mathfrak{G}\sigma_2=\emptyset \) y \( \mathfrak{G}\sigma_1\cap \mathfrak{G}\sigma_2=\emptyset \).\\
Repitiendo el proceso anterior llegamos a \( \lambda_{\sigma_2}=0 \).

Por la finitud de \( \mathfrak{G}_{r+s} \) se llegará a cubrir \( \mathfrak{G}_{r+s} \) con conjuntos disjuntos de la forma \( \mathfrak{G}, \mathfrak{G}_{\sigma_1},\ldots,\mathfrak{G}_{\sigma_k} \) y cada uno de ellos no aporta nada a la suma total, que será por tanto nula.
\item [(2)] Observamos que por ser \( Alt(\alpha) \) una forma alternada \( Alt(Alt(\alpha))=Alt(\alpha) \). Por tanto
\[
\alpha = Alt(Alt(\eta\otimes\theta)-\eta\otimes\theta) = Alt(\eta\otimes\theta)-Alt(\eta\otimes\theta) =0.
\]
Entonces por (1), la linealidad y la transitividad del producto tensorial
\[
0=Alt(\omega\otimes\alpha)=Alt(\omega\otimes Alt(\eta\otimes\theta))-Alt(\omega\otimes\eta\otimes\theta).
\]
Por tanto \( Alt(\omega\otimes Alt(\eta\otimes\theta))=Alt(\omega\otimes\eta\otimes\theta) \).

De manera análoga se prueba 
\[
Alt(Alt(\omega\otimes Alt(\eta\otimes\theta)))=Alt(\omega\otimes\eta\otimes\theta).
 \]
\end{enumerate}
\end{proof}

\begin{note}
Usando la segunda parte del lema \ref{lem:transi} tenemos
\[
(\alpha\wedge\beta)\wedge\gamma=Alt(Alt(\alpha\otimes\beta)\otimes\gamma)=
Alt(\alpha\otimes Alt(\beta\otimes\gamma))=\alpha\wedge(\beta\wedge\gamma)
\]
lo que probaría (2) en La Proposición \ref{prop:ext-proper}.
\end{note}

La siguiente proposición nos dice que las formas alternadas quedan determinadas por las \(1\)-formas y el producto exterior. Sea $V$ un espacio vectorial de base \( \{e_i\}_{i=1}^n \).

\begin{proposition}\label{prop:base-alt}
El isomorfismo \( \rho_a\colon(\Lambda^rV)^*\to Alt_r(V)  \) de la Proposición \ref{ROA}
lleva el dual del elemento básico 
\[ e_{i_1}\wedge\ldots \wedge e_{i_r}\in \Lambda^r(V)
\]
en la \(r\)-forma alternada definida como el producto exterior de $1$-formas 
\[r!(e_{i_1}^*\wedge\ldots\wedge e_{i_r}^*) \ (1\leq i_1< \ldots<i_r\leq n).
\] 
En particular al recorrer  \( 1\leq i_1< \ldots<i_r\leq n  \) todas las
ordenaciones, el conjunto \( \{e_{i_1}^*\wedge\ldots\wedge e_{i_r}^*\}\) es
una base de \( Alt_r(V) \).
\end{proposition}

\begin{proof}
Lo haremos por inducción.\\
Para \(r=1\) es obvio. Supongamos el resultado cierto para \(r-1\) y veamos el caso \(r\). Sea $e_{j_1} \wedge \ldots \wedge e_{j_r}$ un elemento básico de \(\Lambda^r V\). Por definición

\begin{equation}
 \label{proof:prop-alt}
\begin{aligned}
&\rho_a^{-1}(e_{i_1}^*\wedge\ldots e_{i_r}^*)(e_{j_1}\wedge\ldots\wedge e_{j_r}) =(e_{i_1}^*\wedge\ldots e_{i_r}^*)(e_{j_1},\ldots, e_{j_r})=\\
&=\frac{1}{r!}\sum_{\sigma\in \mathfrak{G}_r} s(\sigma)e_{i_1}^*(e_{j_{\sigma(1)}})(e_{i_2}^*\wedge\ldots e_{i_r}^*)(e_{j_{\sigma(2)}},\ldots, e_{j_{\sigma(r)}}) = (1).
\end{aligned}
\end{equation}

Aquí usamos la transitividad del producto exterior de formas. Ahora \( e_{i_1}^*(e_{j_{\sigma(1)}})=1 \) si y sólo si  \( j_{\sigma(1)}=i_1 \).
Por tanto si (1) no es nulo debe existir un \( 1\leq t\leq r \) con \( j_t=i_1 \). Además, por hipótesis de inducción, \( (r-1)!((e_{i_2}^*\wedge\ldots e_{i_r}^*) \) corresponde al dual de \( e_{i_2}\wedge\ldots e_{i_r} \), por lo que necesariamente \( \{j_1,\ldots,j_{t-1},j_{t+1},\ldots,j_r\}=\{i_2,\ldots,i_r\} \) si (1) no es nulo. Por tanto,  \(\{i_1,\ldots, i_r\}\) coincide con  \(\{j_1,\ldots, j_r\}\) y teniendo en cuenta que están ordenados \(i_k = j_k\) para todo \(k\leq r\). Entonces,   

\begin{equation}
  \label{proof:prop-alt-2}
  \begin{aligned}
  (1) &= \frac{1}{r!} \sum_{\sigma\in \mathfrak{G}_r, \ \sigma(1)=1}
    s(\sigma)(e_{i_2}^{*} \wedge \dots e_{i_r}^{*})
    (e_{i_{\sigma(2)}}, \dots, e_{i_{\sigma(r)}})\\
    &=\frac{1}{r!} \sum_{\sigma'\in \mathfrak{G}_{r-1}} s(\sigma')(e_{i_2}^{*} \wedge \dots
    e_{i_r}^{*}) (e_{j_{\sigma'(2)}}, \dots, e_{j_{\sigma'(r)}}) = (2)
  \end{aligned}
\end{equation}
 Aquí \(\sigma'\) es la restricción de \(\sigma \in \mathfrak{G}_r\) con \(\sigma(1) = 1\). Obsérvese que \( s(\sigma')= s(\sigma) \), además  
\[
(r-1)!((e_{i_2}^*\wedge\ldots e_{i_r}^*)(e_{i_{\sigma'(2)}},\ldots, e_{i_{\sigma'(r)}})= s(\sigma')
\]
por lo que
\begin{align*}
(2) &=\frac{1}{r!}\frac{1}{(r-1)!}\sum_{\sigma'\in \mathfrak{G}_{r-1}}s(\sigma')^2\\
&=\frac{1}{r!}\frac{1}{(r-1)!}(r-1)! =\frac{1}{r!}.
\end{align*}
De donde se sigue el resultado.
\end{proof}

\begin{remark}\label{rem:reescalado}
  El factor \(r!\) de la proposición anterior es habitualmente eliminado reescalando previamente el producto exterior por medio de la fórmula
  \(\omega\wedge\eta=\frac{(r+s)!}{r!s!}\text{Alt}(\omega\otimes\eta)\) para la
  \(r\)-forma \(\omega\) y la \(s\)-forma \(\eta\). Con esta nueva definición el
  coeficiente \(r!\) desaparece de la proposición anterior.

  Con la nueva fórmula de \(Alt\) se tiene que (aplicada la transitividad y si
  \(\gamma\) es una $t$-forma):
  \[\begin{array}{c}
      \omega\wedge\eta\wedge\gamma=(\omega\wedge\eta)\wedge\gamma=
      \frac{(r+s+t)!}{(r+s)!t!}\frac{(r+s)!}{r!s!}\text{Alt}
      (\text{Alt}(\omega\otimes\eta)\otimes\gamma)= \\
      =\frac{(r+s+t)!}{r!s!t!}\text{Alt}(\text{Alt}(\omega\otimes\eta)\otimes\gamma).
    \end{array}\]
\end{remark}

\begin{remark}
  Otra ventaja de la definición reescalada de producto exterior es que la
  notación \(\omega_1\wedge \ldots \wedge \omega_r\) usada para describir la clase de
  \(\omega_{1}\otimes\dots\otimes\omega_{r}\in\bigotimes_{r}V^{*}\) en
  \(\Lambda^{r}V^{*}\), que era \(\omega_{1}\wedge\dots\wedge\omega_{r}\), es perfectamente 
  coherente con \(\omega_{1}\wedge\dots\wedge\omega_{r}\) como producto exterior
  de 1-formas ya que \(e_{i_{1}}^{*}\wedge\dots\wedge e_{i_{r}}^{*}\) con
  \(i_{1}<\dots<i_{r}\) son ahora los elementos básicos con ambas notaciones. Se
  tiene así la identificación \(\Lambda^{r}V^{*}\cong\text{Alt}_{r}(V)\).
\end{remark}

Terminamos con la siguiente proposición que nos da el producto exterior de
1-formas como un determinante.

\begin{proposition}\label{prop:prod-ext-det}
  Dadas las 1-formas sobre \(V\) \(\omega_{1},\dots,\omega_{r}\) y la secuencia
  de vectores \(v_{1},\dots,v_{r}\in V\) se tiene la igualdad:
  \[
    (\omega_{1}\wedge\dots\omega_{r})(v_{1},\dots,v_{r})=
    \frac{1}{r!}\det{A_{\omega_{1}\dots\omega_{r}}},
  \]
  donde \(A_{\omega_{1}\dots\omega_{r}}=(a_{ij})\) con \(a_{ij}=\omega_{j}(v_{1})\).
\end{proposition}

\begin{proof}
  Las propiedades del determinante (si multiplicamos una fila o una columna por
  \(\lambda\) el determinante queda multiplicado por \(\lambda\) y si una fila o
  columna se descompone en una suma entonces el determinante es la suma de los
  respectivos determinantes) se traducen en que la aplicación
  \(\gamma_{\omega_{1}\dots\omega_{r}}\colon V\times\dots\times V\to\RealSet\)
  tal que \((v_{1},\dots,v_{r})\mapsto \det{A_{\omega_{1}\dots\omega_{r}}}\) es
  una r-forma alternada.

  Entonces \(\Phi\colon V^{*}\times\dots\times
  V^{*}\to\Lambda^{r}(V)\) dada por \((\omega_{1},\dots,\omega_{r})\mapsto
  \gamma_{\omega_{1},\dots,\omega_{r}}\) se convierte en una aplicación
  multilineal. Análogamente lo es la aplicación \(\Psi\colon V^{*}\times\dots\times
  V^{*}\to\Lambda^{r}(V)\) dada por \((\omega_{1},\dots,\omega_{r})\mapsto
  \omega_{1}\wedge\dots\wedge\omega_{r}\).

  Entonces la proposición anterior quedará demostrada si probamos \(\Phi=\Psi\).
  Para ello, dada una base \(\{e_{i}\}_{i=1}^{n}\) de \(V\), bastará que
  probemos la igualdad para cualquier secuencia de elementos básicos
  \((e_{i_{1}}^{*},\dots,e_{i_{r}}^{*})\).

  Además, si en esa secuencia \(i_{t}=i_{s}\) para algún \(s\) y algún \(t\),
  entonces \(\Psi(e_{i_{1}}^{*},\dots,e_{i_{r}}^{*})=0\) por anticonmutatividad
  y \(\Phi(e_{i_{1}}^{*},\dots,e_{i_{r}}^{*})=0\) por tener la matriz
  correspondiente dos columnas iguales. Entonces podemos suponer que todas las \(i_{1},\dots,i_{r}\)
  serán distintas y, más aún, podemos reducirnos al caso \(i_{1}<\dots<i_{r}\) por
  anticonmutatividad. 
  
  En tal caso, para
  \(\Phi(e_{i_{1}}^{*},\dots,e_{i_{r}}^{*})=\gamma_{e_{i_{1}}^{*},\dots,e_{i_{r}}^{*}}\)
  con \(i_{1}<\dots<i_{r}\) se tiene que
  \(\gamma_{e_{i_{1}}^{*},\dots,e_{i_{r}}^{*}}(v_{1},\dots,v_{r})\) es el
  determinante de la matriz \(A=(a_{ij})\) con
  \(v_{i}=\sum_{j=1}^{n}a_{ij}e_{j}\), esto es, la matriz cuyos vectores
  columnas son los \(v_{i}\).

  Para \(\Psi(e_{i_{1}}^{*},\dots,e_{i_{r}}^{*})=e_{i_{1}}^{*}\wedge\dots\wedge
  e_{i_{r}}^{*}\) se tiene
  \[
    \begin{array}{l}
       (e_{i_{1}}^{*}\wedge\dots\wedge e_{i_{r}}^{*})(v_{1},\dots,v_{r})=(e_{i_{1}}^{*}
    \wedge\dots\wedge e_{i_{r}}^{*})(\sum_{j=1}^{n}a_{1j}e_{j},
    \dots,\sum_{j=1}^{n}a_{rj}e_{j})= \\
      \overset{(1)}{=}
      \sum_{(j_{1},\dots,j_{r})\in\{1,\dots,n\}^{r}}a_{1j_{1}}\cdots a_{rj_{r}}
    (e_{i_{1}}^{*}\wedge\dots\wedge e_{i_{r}}^{*})(e_{j_{1}},
    \dots,e_{j_{r}})= \\
    \overset{(2)}{=}\frac{1}{r!}\sum_{\sigma\in \mathfrak{G}_{r}}s(\sigma) a_{1\sigma(1)}
    \cdots a_{r\sigma(r)}=\det{A}.
    \end{array}
  \]

  En la igualdad \((1)\) se usa la multilinealidad y en la \((2)\) la
  Proposición \ref{prop:base-alt} y así \(r!(e_{i_{1}}^{*}\wedge\dots\wedge
  e_{i_{r}}^{*})\) se corresponde con el dual de \(e_{i_{1}}\wedge\dots\wedge
  e_{i_{r}}\in\Lambda^{r}(V)\) y por tanto los sumandos del término de la
  izquierda de \((2)\) no nulos son aquellos con
  \(\{j_{1},\dots,j_{r}\}=\{i_{1},\dots,i_{r}\}\).

  Entonces \(r!(e_{i_{1}}^{*}\wedge\dots\wedge
  e_{i_{r}}^{*})(e_{j_{1}},\dots,e_{j_{r}})= s(\sigma)\) donde \(\sigma\) es
  la permutación que lleva \((j_{1}\dots j_{r})\) a \((i_{1}\dots i_{r})\).
\end{proof}

\begin{remark}
  Si usamos el reescalado del producto exterior dado por la fórmula en la
  Observación \ref{rem:reescalado} entonces el factor \(\frac{1}{r!}\) desaparece de la
  igualdad de la Proposición \ref{prop:prod-ext-det}.

  Por eso a la definición del producto exterior con el reescalado se le llama el
  \emph{convenio del determinante}.
\end{remark}

\section{Formas alternadas y simétricas sobre una variedad}

Como casos particulares de formas diferenciables sobre una \(m\)-variedad \(M\)
se pueden considerar las formas diferenciables simétricas y alrernadas sobre
\(M\) así como el correspondiente producto exterior.

Si se identifica \(\Lambda^{r}(\RealSet^{m})\) con
\(\RealSet^{\binom{m}{r}}\) entonces
\(\Lambda^{r}(\Tangente{M})=\bigcup_{p\in M}\Lambda^{r}(\Tangente[p]{M})\) puede
ser dotada de una estructura de  \((m+\binom{m}{r})\)-variedad:

En efecto, dada una carta \((U,\varphi)\) de \(M\), la carta asociada para
\(\Lambda^{r}(\Tangente{M})\) es
\((\Lambda^{r}(\Tangente{U}),\Lambda^{r}(\varphi))\), donde
\(\Lambda^{r}(\Tangente{U})=\bigcup_{p\in U}\Lambda^{r}(\Tangente[p]{M})\) y
\[
\Lambda^{r}(\varphi)\colon\Lambda^{r}(\Tangente{U})\to\varphi(U)\times
\Lambda^{r}(\RealSet^{m})=\varphi(U)\times\RealSet^{\binom{m}{r}}
\] 
lleva 
\(\omega\in\Lambda^{r}(\Tangente[p]{M})\) en el par 
\(\Lambda^{r}(\varphi)(\omega)=(\varphi(p),\Lambda^{r}(\varphi)_{2}(\omega)\)), donde  
 \(\Lambda^{r}(\varphi)_{2}(\omega)\in\Lambda^{r}(\RealSet^{m})\) es la forma 
 definida por \(\Lambda^{r}(\varphi)_{2}(\omega)(v)=\omega(\widetilde{\varphi}^{-1}(p,v))\),
siendo \((\Tangente{U},\widetilde{\varphi})\) la carta de \(\Tangente{M}\)
asociada a \((U,\varphi)\).

Los cambios de carta de \(\Lambda^{r}(\Tangente{M})\) vienen dados por las
aplicaciones
\[\begin{array}{c}
    \varphi(U\cap V)\times\Lambda^{r}(\RealSet^{m})\to\psi(U\cap
    V)\times\Lambda^{r}(\RealSet^{m}) \\
    (x,\omega)\mapsto(\psi\varphi^{-1}(x),\Lambda^{r}(\J[x]{\psi\varphi^{-1}})(\omega))
  \end{array}\]
donde, si \(f\colon\RealSet^{m}\to\RealSet^{m},\
\Lambda^{r}(f)\colon\Lambda^{r}(\RealSet^{m})\to\Lambda^{r}(\RealSet^{m})\) está
dada por \(\Lambda^{r}(f)(\omega)\) con
\(\Lambda^{r}(f)(\omega)(v_{1},\dots,v_{r})=\omega(f(v_{1}),\dots,f(v_{r}))\).

Para esta estructura de variedad sobre
\(\Lambda^{r}(\Tangente{M})\) la proyección
\(\pi\colon\Lambda^{r}(\Tangente{M})\to M\), dada por \(\pi(\omega)=p,\
\omega\in\Lambda^{r}(\Tangente[p]{M})\), es una aplicación diferenciable y 
se definen las \emph{r-formas diferenciables alternadas} sobre
\(M\) como las secciones \(s\colon M\to\Lambda^{r}(\Tangente{M})\) de \(\pi\).

Equivalentemente se podría definir una \(r\)-forma alternada sobre \(M\) como una
aplicación diferenciable \(f\colon\bigotimes_{r}\Tangente{M}\to\RealSet\) tal
que para cada \(p\in M\) la restricción
\(f\colon\Tangente[p]{M}\otimes\dots \otimes\Tangente[p]{M}\to\RealSet\) define
una \(r\)-forma alternada.
\par
\medskip
De manera similar se podría definir las \emph{r-formas diferenciables simétricas}
sobre \(M\).
\par
\medskip
Dada una carta \((U,\varphi)\) de $M$ con coordenadas locales 
\(\varphi(p)=(x_{1}(p),\dots,x_{m}(p))\), a partir de las \(1\)-formas \(\dif{x_{1}},\dots,\dif{x_{m}}\) se obtiene 
una base de \(\Lambda^{r}(\Tangente{M})\) por medio de los productos exteriores
\[
  \dif{x_{i_{1}}}\wedge\dots\wedge\dif{x_{i_{r}}}, 
\]
para cada secuencia ordenada
\(1\leq i_{1}<\dots<i_{r}\leq m\). Así, una \(r\)-forma alternada \(\omega\) sobre \(M\)
tiene como expresión local en \((U,\varphi)\) 
\[
  \omega=\sum_{1\leq
    i_{1}<\dots<i_{r}\leq m}f_{i_{1}\dots
    i_{r}}\dif{x_{i_{1}}}\wedge\dots\wedge\dif{x_{i_{r}}},
\]
con \(f_{i_{1}\dots i_{r}}\colon\varphi(U)\to\RealSet\) diferenciable.

Habitualmente se denota por \(\Omega^{r}(M)\) el espacio vectorial de las
\(r\)-formas diferenciables alternadas sobre \(M\) para \(r\geq 0\).

Para \(r=0\) se toma \(\Omega^{0}(M)=\mathcal{F}(M)\) las funciones diferenciables
sobre \(M\). Para \(r=1,\ \Omega^{1}(M)\) son las \(1\)-formas sobre \(M\).

Toda aplicación diferenciable \(f\colon M^{m}\to N^{n}\) induce una aplicación
lineal \(f^{*}\colon\Omega^{r}(N)\to\Omega^{r}(M)\) que lleva la \(r\)-forma
alternada \(\omega\colon\bigotimes_{r}\Tangente{N}\to\RealSet\) en la \(r\)-forma
alternada \(f^{*}\omega\colon\bigotimes_{r}\Tangente{M}\to\RealSet\) dada por
la composición
\[
\bigotimes_{r}\Tangente{M}\overset{\otimes_{r}\dif{f}}{\to}\bigotimes_{r}\Tangente{N}\overset{\omega}{\to}\RealSet
\]
donde \(\otimes_{r}\dif{f}\) es la aplicación que para cada \(p\in M\) y
\(v_{1},\dotsm,v_{r}\in\Tangente[p]{M}\) se tiene
\[
  (\otimes_{r}\dif{f})_{p}(v_{1}\otimes\dots\otimes
  v_{r})=\dif{f_{p}} v_{1}\otimes\dots\otimes\dif{f_{p}} v_{r},
\]
siendo \(\dif{f}\colon\Tangente{M}\to\Tangente{N}\) la diferencial de \(f\).

Si tomamos cartas \((U,\varphi)\) de \(M\) y \((V,\psi)\) de \(N\) con
\(f(U)\subseteq V\), entonces, si la expresión local de \(\omega\) en
\((V,\psi)\) es
\[
  \omega=\sum_{1\leq
    i_{1}<\dots<i_{r}\leq n}g_{i_{1}\dots
    i_{r}}\dif{y_{i_{1}}}\wedge\dots\wedge\dif{y_{i_{r}}},
\]
la expresión local de \(f^{*}\omega\) en \((U,\varphi)\) es
\[
  f^{*}\omega=\sum_{1\leq
    i_{1}<\dots<i_{r}\leq n}g_{i_{1}\dots
    i_{r}}(f(x_{1},\dots,x_{m}))
  (\sum_{k=1}^{m}\pderiv{f_{i_{1}}}{x_{k}}\dif{x_{k}})
  \wedge\dots\wedge
  (\sum_{k=1}^{m}\pderiv{f_{i_{r}}}{x_{k}}\dif{x_{k}}),
\]
que habría que desarrollar teniendo en cuenta las propiedades del producto
exterior. En particular, \(\dif{x_{i}}\wedge\dif{x_{i}}=0\) y
\(\dif{x_{i}}\wedge\dif{x_{j}}=-\dif{x_{j}}\wedge\dif{x_{i}}\) por
anticonmutatividad.

Es importante señalar, que para el caso \(r=1\), es decir, si
\(\omega\colon\Tangente{N}\to\RealSet\) es una 1-forma, se tiene que la
expresión de \(f^{*}\omega\) es la siguiente:

\[
  f^{*}\omega=\sum_{i=1}^{n}g_{i}(f(x_{1},\dots,x_{m}))
  (\sum_{k=1}^{m}\pderiv{f_{i}}{x_{k}}\dif{x_{k}})=
  \sum_{i=1}^{n}g_{i}(f(x_{1},\dots,x_{m}))\dif{f}
\]

Por último, una propiedad interesante de la anterior aplicación es que
\[
  f^{*}(\omega\wedge\theta)=f^{*}\omega\wedge f^{*}\theta.
\]
\section{Diferencial Exterior}

Una importante propiedad de las formas alternadas sobre una \(m\)-variedad \(M\)
es que la forma gradiente de una función (Ejemplo \ref{ex:forma-dif})
\(\dif\colon\mathcal{F}(M)=\Omega^{0}(M)\to\Omega^{1}(M)\) es el primer eslabón
de toda una cadena de operadores
\[
  \Omega^{0}(M) \overset{\dif[1]{}}{\longrightarrow}
  \Omega^{1}(M) \overset{\dif[2]{}}{\longrightarrow}
  \Omega^{2}(M) \overset{\dif[3]{}}{\longrightarrow} \dots
\]
Para \(r\geq 1\) el operador
\(\dif=\dif[r+1]{}\colon\Omega^{r}(M)\to\Omega^{r+1}(M)\) se llama
\emph{diferencial exterior}. Damos una definición en los abiertos de
\(\RealSet^{m}\) y después daremos una idea de cómo extender la definición a
toda la variedad \(M\).

\begin{definition}
En un abierto \(V\subseteq\RealSet^{m}\) una \(r\)-forma diferenciable alternada se
escribe como \(\omega=\sum_{I}f_{I}\dif[I]{}\), variando \(I\) en  el
conjunto de secuencias ordenadas \(I=\{1\leq i_{1}<\dots<i_{r}\leq m\}\),  donde \(\dif[I]=\dif{x_{i_{1}}}\wedge\dots\wedge\dif{x_{i_{r}}}\) y \(f_{I}\colon V\to\RealSet\) es diferenciable.

Entonces si \(\dif{f_{I}}\) es la \(1\)-forma gradiente de \(f_I\), se define
\[
  \dif{\omega}=\sum_{I}\dif{f_{I}}\wedge\dif[I]=
  \sum_{I}(\sum_{i=1}^{m}\pderiv{f_{I}}{x_{i}}\dif{x_{i}})\wedge\dif[I].
\]
Desarrolándose esta fórmula de acuerdo con las propiedades del producto exterior. 
\end{definition}

Ahora dada una \(r\)-forma diferenciable alternada \(\omega\) sobre la variedad
\(M\), tomamos sus restricciones \(i_{U}^{*}\omega=\omega_{U}\) a las cartas
\((U,\varphi)\) de un atlas de \(M\). Entonces \(\varphi^{-1}\colon\varphi(U)\to
U\) permite definir la forma \((\varphi^{-1})^{*}\omega_{U}\) sobre el abierto
\(\varphi(U)\subseteq\RealSet^{m}\), donde está definida la diferencial exterior
\(\dif{((\varphi^{-1})^{*}\omega_{U})}\). Entonces pasamos esta diferencial a
\(U\) por medio de \(\varphi\colon U\to\varphi(U)\) como
\(\varphi^{*}(\dif{((\varphi^{-1})^{*}\omega_{U})})\).

Queda ver que todas estas formas están bien definidas en las intersecciones y definir entonces la diferencial de
\(\omega\) como la \((r+1)\)-forma $d\omega: \bigotimes_{r+1} TM \to \RealSet$ dada por 
\[
  (5)\quad (\dif{\omega})_{p}=\varphi^{*}(\dif{((\varphi^{-1})^{*}\omega_{p})}) \mbox{ para todo } p\in M.
\]

Esto se deja sin demostración. Se puede consultar en \emph{Introduction
  to Smooth Manifolds, John M. Lee}, donde también se pueden encontrar las
demostraciones de las siguientes propiedades de la diferencial exterior:

\begin{proposition}
  La diferencial exterior cuenta con las
  siguiente propiedades:
  \begin{enumerate}
  \item \(\dif{}\) es lineal.
  \item \(\dif{} \dif{}=0\).
  \item
    \(\dif{(\omega\wedge\eta)}=\dif\omega\wedge\eta+(-1)^{r}\omega\wedge\dif{\eta}\)
    si \(\omega\in\Omega^{r}(M)\).
  \item Si \(f\colon M\to N\) es una aplicación diferenciable entre variedades y
    \(\omega\in\Omega^{r}(N)\) entonces \(\dif{f^{*}\omega}=f^{*}\dif{\omega}\).
  \end{enumerate}
\end{proposition}

A partir de la secuencia 
\[
  \mathcal{F}(M)=\Omega^{0}(M) \overset{\dif[1]{}}{\longrightarrow}
  \Omega^{1}(M) \overset{\dif[2]{}}{\longrightarrow}
  \dots
  \Omega^{r}(M) \overset{\dif[r+1]{}}{\longrightarrow}
  \dots
\]
 y teniendo en cuenta que \(\dif[r+1]{}\circ\dif[r]{}=0\) se puede considerar, para todo \(r\geq 0\),  el
 espacio vectorial cociente \(\frac{\ker{\dif[r+1]{}}}{\text{Im}\dif[r]{}}\). Estos espacios forman 
 la llamada \emph{cohomología de De Rham} de la variedad \(M\). Las formas en \(\ker\dif[r+1]{}\), es decir aquellas
 con \(\dif{\omega}=0\), se llaman \emph{formas cerradas} y las formas en
 \(\text{Im}\dif[r]{}\), es decir aquellas \(r\)-formas \(\omega\) para las que
 existe una \((r-1)\)-forma \(\eta\) con \(\dif{\eta}=\omega\) se llaman
 \emph{formas exactas}. Así pues, la cohomología de De Rham se ocupa del álgebra que determinan aquellas formas cerradas
 sobre \(M\) que no son exactas.

 El curso
 \textbf{Cálculo en Variedades} trata sobre todo ello. 
 \end{document}

%%% Local Variables:
%%% TeX-master: "../VD_ebook"
%%% End:
