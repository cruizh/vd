\documentclass[../VD.tex]{subfiles}

\externaldocument{../VD}

\begin{document}

\setcounter{chapter}{9}
\chapter{Espacios de formas y tensores sobre una variedad}\label{chap:dual} 

\section{Introducción}

En \textbf{Álgebra Lineal} conocemos la extensión natural del espacio dual de
las formas.

\begin{definition}
  Si una 1-forma del espacio vectorial \(V\) es una aplicación lineal
  \(V\to\RealSet\), entonces dados \(n\) espacios vectoriales
  \(V_{1},\dots,V_{n}\) y otro espacio vectorial \(W\), una \emph{aplicación
    multilineal} \(f\colon V_{1}\times\dots\times V_{n}\to W\), esto es, \(\forall
  \lambda,\mu\in\RealSet,\ \forall v_{i},v'_{i}\in V_{i},\ 1\leq i\leq n\):
  
  \[
    f(v_{1},\dots,\lambda v_{i}+\mu v'_{i},\dots,v_{n})=\lambda
    f(\dots,v_{i},\dots)+\mu f(\dots,v'_{i},\dots).
  \]

  Cuando \(V_{1}=\dots=V_{n},\ W=\RealSet\) decimos que \(f\) es una n-forma
  sobre \(V\).
\end{definition}

Ahora se plantea cómo extender las n-formas al espacio tangente de una variedad
\(M\). Ya vimos en el caso de las 1-formas que esa construcción se puede abordar
de varias maneras:

\begin{enumerate}
\item Crear el espacio cotangente haciendo la construcción punto a punto y
  después definiendo una estructura coherente en todo \(M\): para cada \(p\in
  M\) sea \(\DualT[p]{M}\) y se define \(\DualT{M}=\bigcup_{p\in
    M}\DualT[p]{M}\), dotándolo de una estructura de variedad diferenciable
  compatible con la de \(M\) mediante la proyección
  \(\pi^{*}\colon\DualT{M}\to M\ \omega\in\DualT[p]{M}\mapsto p\) y las 1-formas
  se definen como secciones diferenciables de esa proyección.

\item Definir el espacio cotangente globalmente sobre todo \(\Tangente{M}\) por
  medio de funciones diferenciables \(f\colon\Tangente{M}\to\RealSet\) que punto
  a punto son 1-formas, es decir, \(f\colon\Tangente[p]{M}\to\RealSet\) es
  lineal para todo \(p\in M\).

\item Usar \(\mathbb{X}(M)\) el \(\mathcal{F}(M)\)-módulo de los campos sobre
  \(M\) y reescribir \(\DualT{M}\) como el dual de \(\mathbb{X}(M)\) como
  \(\mathcal{F}(M)\)-módulo. Esto es, usar la identificación
  \(\DualT{M}\cong\text{Hom}(\mathbb{X}(M),\mathcal{F}(M))\), que lleva la forma
  \(\omega\colon M\to\Tangente{M}\) a la función \(\omega(X)(p)=X(\omega(p)),\
  \forall p\in M\).
\end{enumerate}

Veamos cómo extender estas construccione equivalentes del espacio cotangente de
\(M\) para obtener la noción de n-forma sobre la variedad \(M\).

Para ello recordemos la noción de producto tensorial que permite reinterpretar
las aplicaciones multilineales como 1-formas, pasando la complicación de las
aplicaciones multilineales al espacio vectorial.

\begin{definition}
  Recordemos que el \emph{producto tensorial} de los espacios vectoriales
  \(V_{1},\dots,V_{n}\), \(V_{1}\bigotimes\dots\bigotimes V_{n}\), es el espacio
  vectorial obtenido por combinaciones lineales de los símbolos
  \(v_{1}\bigotimes\dots\bigotimes\ v_{n}\) sujetos a las relaciones:

  \begin{itemize}
  \item \(\lambda(v_{1}\bigotimes\dots\bigotimes
    v_{n})=v_{1}\bigotimes\dots\bigotimes \lambda v_{i}\bigotimes\dots\bigotimes
    v_{n},\ \forall i\)

  \item \(v_{1}\bigotimes\dots\bigotimes(v_{i}+v'_{i})\bigotimes\dots\bigotimes
    v_{n}=v_{1}\bigotimes\dots\bigotimes \lambda v_{i}\bigotimes\dots\bigotimes
    v_{n}+v_{1}\bigotimes\dots\bigotimes \lambda v'_{i}\bigotimes\dots\bigotimes
    v_{n},\ \forall i\)
  \end{itemize}

  Entonces \((V_{1}\bigotimes\dots\bigotimes
  V_{n})^{*}\cong\text{Multi}_{n}(V_{1},\dots,V_{n}),\ f\mapsto
  g_{f}(v_{1},\dots,v_{n})=f(v_{1}\bigotimes\dots\bigotimes\v_{n})\).

  
\end{definition}

\end{document}

%%% Local Variables:
%%% TeX-master: "../VD"
%%% End: