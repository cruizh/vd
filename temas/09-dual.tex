\documentclass[../VD.tex]{subfiles}

\externaldocument{../VD}

\begin{document}

\setcounter{chapter}{8}
\chapter{Tangente dual}\label{chap:dual}

\section{Clase}

Si \((U,\varphi)\) es una carta sobre \(M\), consideramos \(T U=\bigcup_{p\in U}T_{p}M\subseteq T M\). Entonces \(T U^*=\bigcap_{p\in U}T_{p}^*{M}\subseteq T^*(M)\).

Además, si tomamos una carta \((TU,\tilde{\varphi})\), donde \(\tilde{\varphi}\colon TU\to \varphi(U)\times \RealSet^m \), podemos ver su equivalente dual:

Sea \((T^*U,\tilde{\varphi})^*\), donde \(\tilde{\varphi}^*\colon T^*U\to \varphi(U)\times (\RealSet^m)^* \) dada por
\[
w\in T^*_p(M)\mapsto \tilde{\varphi}^*(w)=(\varphi(p),\tilde{\varphi}_2(w)\quad \text{con p en U}
\]

\begin{example}[1-forma]
Asociada a cada \(f\in \Fdif(M)\) hay asociada una 1-forma
\[
p\in M\mapsto ((\dif{f})_p\colon T_p(M)\to \RealSet)
\]
cuyas componentes locales (referidas a carta \((U,\varphi)\)) de \(M\)) son \label{def:grad_1}
\[
(\dfrac{\partial(f\circ \varphi^{-1})}{\partial x_1}|_{\varphi(p)},\ldots,\dfrac{\partial(f\circ \varphi^{-1})}{\partial x_m}|_{\varphi(p)})
\]
Localmente, \(e_i^p\) representa los elementos básicos de \(T_p(M)\) y \((e_i^p)^*\) representa los duales de los \(e_i^p\) y forman una base de \(T_p^*(M)\).
Esto se traduce en 
\[
(\dif{f})_p=\dfrac{\partial(f\circ \varphi^{-1})}{\partial x_1}|_{\varphi(p)}(e_1^p)^*+\ldots+\dfrac{\partial(f\circ \varphi^{-1})}{\partial x_m}|_{\varphi(p)}(e_m^p)^*
\]
En general, dada una 1-forma \(w\) con componentes licales
\begin{enumerate}
\item \(g^U\colon\varphi(U)\to (\RealSet^m)^*\)
\item \(g^V\colon \psi(V)\to (\RealSet^m)^*\)
\end{enumerate}
ha de cumplirse:
\[
g^V(\psi\circ \varphi^{-1}(x))=(J_{\psi\circ \varphi^{-1}(p)}\varphi\circ \psi^{-1})^t\cdot g^U(x)
\]

Sean \((U,\varphi)\) y \((V,\psi)\) dos cartas de \(M\) con \(U\cap V\neq \emptyset\). Entonces, para \(y\in \psi(U\cap V)\) y \(x=\varphi\circ \psi^{-1}(y)\) se tiene:
\begin{align*}
(\bigtriangledown f)^V_i|_y&=\dfrac{\partial(f)\circ \psi^{-1}}{\partial y_i}|_y\\
&=\dfrac{\partial(f\circ \varphi^{-1}\circ \varphi \circ \psi^{-1})}{\partial y_i}|_y\\
&=\sum_{k=1}^m \dfrac{\partial(f\circ \varphi^{-1})}{\partial x_k}|_x\cdot \dfrac{\partial(\varphi\circ \psi^{-1})}{\partial y_i}|_y\\
&=\text{producto escalar de la i-ésima columna de la matriz jacobiana} \\
&J_y \varphi\circ \psi^{-1} \text{con el vector } (\bigtriangledown f)^U(x)
\end{align*}
siendo el gradiente el visto en \ref{def:grad_1.}.

Resumiendo obtenemos
\[
(\bigtriangledown f)^V(\psi\circ \varphi^{-1}(x))=(\bigtriangledown f)^V(y)=(J_ {\psi\circ\varphi^{-1}(x)}(\varphi \circ \psi^{-1})^t\cdot (\bigtriangledown f)^U(x)
\]

Estro probaría que cumple la propiedad de compatibilidad vista anteriormente %need ref

Veamos un caso concreto. Si tenemos una carta \((U,\varphi)\) de \(M\) podemos ver la siguiente aplicación localmente:
\[
\varphi_i\colon U\to \varphi(U)\to^{\pi_i}\RealSet
\]
Cuyas 1-formas vienen dadas por
\begin{align*}
(\dif{\varphi_i})_p&=(\dfrac{\partial(\varphi_i\circ \varphi^{-1})}{\partial x_1}|_{\varphi(p)},\ldots,\dfrac{\partial(\varphi_i\circ \varphi^{-1})}{\partial x_m}|_{\varphi(p)})\\
&= \sum_{k=1}^{m} \underbracket{\dfrac{\partial(\varphi_i\circ \varphi^{-1})}{\partial x_k}|_{\varphi(p)}}_{\delta_{ik}}\cdot (e_k^p)^*\colon T_p M\to \RealSet
\end{align*}
\end{example}

En \(M=\RealSet^m\) (Con una única carta \((\RealSet^m,id)\)), toda 1-forma \(w\) se expresa con la notación habitual
\[
w=g_1\dif{x_1}+\ldots+g_m\dif{x_m}
\]
y si ahora \(f\in\Fdif{\RealSet^m}\), entonces:
\[
df=\dfrac{\partial(f)}{\partial x_1}dx_1+\ldots+\dfrac{\partial(f)}{\partial x_m}dx_m
\]
que es la notación usual de análisis

\begin{example}
Sea \(M^m\) una variedad diferenciable y \(T^*M\) espacio cotangente.
\[
T^*(T^*M)=\text{espacio cotangente a } T^*M
\]
variedad diferenciable de dimension \(4m\).

Asociado a \(M\), hay una 1-forma canónica \(\theta_M\) sobre \(T^*M\) (llamada de Liouville) definida de la siguiente manera:
\[
\theta_M\colon T^*M\to T^*(T^*M)
\]
con \(\theta_M(w)\in T^*_w(T^*M)\). Si \(w\in T^*M\) definimos \(\theta_M(w)(v)=w(d\pi^*(v))\)
\end{example}

\section{Una forma alternativa de ver las 1-formas}

Supongamos que tenemos una 1-forma \(o\in M \mapsto \omega_p\in T_p^*(M)\). Podemos defnir una aplicación duferenciable 
\[
f_\omega\colon TM\to \RealSet,\quad (f_\omega\in \Fdif(TM))
\]
dada por \(f_\omega(v)=\omega_p(v)\)

Veamos la diferenciabilidad de \(f_\omega\).

Dada una carta \((U,\varphi)\) de \(M\), se tiene:
 \begin{center}
 	\begin{tikzcd}
 	\bigcup_{p\in U}T_p M=
	TU
	\ar[r, "f_\omega"]
	\ar[d, "\tilde{\varphi}"] &
	\RealSet\\
	\varphi(U)\times\RealSet^m
	\ar[ur, "f_\omega\circ \tilde{\varphi}" ']
	\end{tikzcd}
\end{center}

Tendríamos que probar que \(f_\omega\circ \tilde{\varphi}(x,y)=\omega_{\varphi^{-1}(x)}(\tilde{\varphi}^{-1}(x,y))\) es diferenciable.\\
Además tenemos el siguiente diagrama:
 \begin{center}
	\begin{tikzcd}
	U
	\ar[r, "\omega"]
	\ar[d, "\varphi"] &
	T^*(U)
	\ar[d, "\tilde{\varphi}^*"]\\
	\varphi(U)
	\ar[r, "\tilde{\varphi}^*\circ \omega\circ \varphi^{-1}"] &
	\varphi(U)\times(\RealSet^m)^*
	\end{tikzcd}
\end{center}

Según las definiciones, 
\begin{align*}
\tilde{\varphi}^*(\omega_ {\varphi^{-1}(x)})=(x,\eta)
\end{align*}
con \(\eta=(\lambda_1(x),\ldots,\lambda_m(x))\in (\RealSet^m)^*\) las coordenadas respecto a los elemento de la base de \((\RealSet^m)^*\).

Dado \(y=(y_1,\ldots, y_m)\), si aplicamos \(\eta\) a \(y\) obtendremos:
\[
\eta(y)=\sum_{i=1}^{m}\lambda_i(x)y_i=(\tilde{\varphi}^*(\omega_ {\varphi^{-1}(x)}))_2\quad\text{diferenciable}
\]

Así pues, podeoms asociar a una 1-forma la función:
\[
\omega\mapsto f_{\omega}\in\Fdif(TM)
\]
Recíprocamente, dada una \(f\in\Fdif(TM)\) tal que para todo \(p\in M\)
\[
f\colon T_p(M)\to \RealSet
\]
es lineal, podemos definir una 1-forma que notaremos por \(\omega_f\) de la siguiente manera:
\[
p\in M\mapsto ((\omega_f)_p\colon T_pM\to \RealSet)
\]
con \((\omega_f)_p=f|_{T_p(M)}\)
\end{document}