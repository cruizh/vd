\documentclass[../VD.tex]{subfiles}

\externaldocument{../VD}

\begin{document}

\setcounter{chapter}{8}
\chapter{Espacio cotangente de una variedad}\label{chap:dual}

\section{Introducción}

El espacio cotangente de una variedad se basa en el \emph{estudio de los
  momentos} de la física.

\begin{definition}
  Sea \(W\) un espacio vectorial sobre un cuerpo \(K\). El \emph{espacio dual} de \(W\) será el conjunto de aplicaciones
  lineales \(f\colon W\to K\) con las operaciones obvias de suma y multiplicación por un escalar. 
\end{definition}

\begin{definition}[name=espacio cotangente a \(M\) en el punto \(p\)]
Sea \(M\) variedad diferenciable, se llama \emph{espacio cotangente de \(M\) en \(p\in M\)} al espacio dual de \(T_pM\), esto es, \(\DualT[p]{M}=\{\omega\colon\Tangente[p]{M}\to\RealSet
\text{ lineales}\}\). 
\end{definition}

\begin{proposition}
  Si \(M\) es una variedad de dimensión \(m\) entonces
  \(\DualT{M} = \cup_{p\in M} \DualT{M}\) admite una estructura de variedad diferenciable de dimensión \(2m\).
Además, la aplicación \(\pi^{*}\colon\DualT{M}\to M\), tal que
  \(\pi^{*}(\omega)=p\) si \(\omega\in\DualT[p]{M}\), es submersión. 
\end{proposition}

\begin{proof}
  Sea  \((U,\varphi)\) una carta de \(M\), sea \(\DualT{U}=\bigcup_{p\in
    U}\DualT[p]{M}\). Si \((\RealSet^{m})^{*}\) es el espacio dual de
  \(\RealSet^{m}\), sea   
  \(\widetilde{\varphi}^{*}\colon\DualT{U}\to\varphi(U)\times(\RealSet^{m})^{*}\)
  la aplicación que lleva 
   \(\omega\in\DualT[p]{M}\) en el par 
  \(\widetilde{\varphi}^{*}(\omega)=(\varphi(p),\varphi_{2}^{*}(\omega))\),
  donde \(\varphi_{2}^{*}(\omega)\colon\RealSet^{m}\to\RealSet\) es la \(1\)-forma
  que a cada \(y\in\RealSet^{m}\) le asocia el número real
  \(\varphi_{2}^{*}(\omega)(y)=\omega(\widetilde{\varphi}^{-1}(\varphi(p),y))\),
  siendo
  \(\widetilde{\varphi}\colon\Tangente{U}\to\varphi(U)\times\RealSet^{m}\) tal
  que \(\widetilde{\varphi}(v)=(\varphi(p),\varphi_{2}(v))\) con
  \(v\in\Tangente[p]{M}\).

  \vline
  
  Si ahora fijamos el isomorfismo
  \(\phi\colon\RealSet^{m}\times(\RealSet^{m})^{*}\cong\RealSet^{m}\times\RealSet^{m}\)
  que lleva las formas duales de elementos de la
  base canónica de \((\RealSet^{m})^{*}\), \(e_{i}^{*}\), en los correspondientes  vectores básicos \(e_{i}\), entonces la composición
  \(\phi\circ\varphi^{*}\colon\DualT{U}\to\varphi(U)\times\RealSet^{m}\) nos define una carta de \(\DualT{M}\).

  \vline

  Obsérvese que si \((e_{i}^{p})^{*}\) es el dual del vector básico
  \(e_{i}^{p}\in\Tangente[p]{M}\), entonces
  \(\widetilde{\varphi}^{*}((e_{i}^{p})^{*})=(\varphi(p),e_{i}^{*})\), pues si
  \(y=\sum\limits_{i=1}^{m}\lambda_{i}e_{i}\), entonces:
  
  \[\begin{array}{l}
      \widetilde{\varphi}^{-1}(\varphi(p),y)=\sum\limits_{i=1}^{m}\lambda_{i}e_{i}^{p}=
      v\in\Tangente[p]{M},
    \end{array}\]
  
  y
  
  \[\begin{array}{l}
      \varphi_{2}^{*}((e_{i}^{p})^{*})(y)=(e_{i}^{p})^{*}(v)=\lambda_{i}.
    \end{array}\]
  
  Por tanto, si \(\omega=\sum\limits_{i=1}^{m}\lambda_{i}(e_{i}^{p})^{*}\), entonces
  
  \[
    \phi\circ\widetilde{\varphi}^{*}(\omega)=(x,(\lambda_{1},\dots,\lambda_{m}))
    \in\RealSet^{m}\times\RealSet^{m}
  \]

  y la inversa de \(\widetilde{\varphi}^{*}\) lleva
  \((x,\lambda_{1},\dots,\lambda_{m})\) en la forma
  \(\sum\limits_{i=1}^{m}\lambda_{i}(e_{i}^{\varphi^{-1}(x)})^{*}\).

 
  \vline
  
  Queda ver que si \(\mathcal{A}=\{(U,\varphi)\}\) es un atlas de \(M\) entonces
  \(\{(\DualT{U},\phi\circ\widetilde{\varphi}^{*})\}\) es un atlas para \(\DualT{M}\).

  Si consideramos \(\phi\) como una identificación de coordenadas, bastará hacer
  las comprobaciones para las \(\widetilde{\varphi}^{*}\). Es inmediato que
  \(\widetilde{\varphi}^{*}(\Tangente{U})\) es un abierto de
  \(\RealSet^{m}\times(\RealSet^{m})^{*}\).

  Además, si \((U,\varphi),(V,\psi)\) son cartas de \(\mathcal{A}\) con \(U\cap
  V\neq\emptyset\), entonces \(\DualT{U}\cap\DualT{V}=\DualT{(U\cap V)}\) y
  \(\widetilde{\varphi}^{*}(\DualT{U}\cap\DualT{V})=\varphi(U\cap
  V)\times(\RealSet^{m})^{*}\) y
  \(\widetilde{\psi}^{*}(\DualT{U}\cap\DualT{V})=\psi(U\cap 
  V)\times(\RealSet^{m})^{*}\) son abiertos.

  \vline

  Queda probar que la aplicación siguiente es un difeomorfismo:
  \[\widetilde{\psi}^{*}\circ(\widetilde{\varphi}^{*})^{-1}\colon\varphi(U\cap
    V)\times(\RealSet^{m})^{*}\to\psi(U\cap V)\times(\RealSet^{m})^{*}.\]

  Dado \((x,\eta)\in\varphi(U\cap V)\times(\RealSet^{m})^{*}\), con
  \(\eta=(\mu_{1},\dots,\mu_{m})\), se tiene por
  definición
  \[
    \widetilde{\psi}^{*}\circ(\widetilde{\varphi}^{*})^{-1}(x,\eta)=
    (\psi\circ\varphi^{-1}(x),\psi_{2}^{*}((\widetilde{\varphi}^{*})^{-1}(x,\eta))),
    \quad (1)
  \]

  donde para un vector \(y=(\lambda_{1},\dots,\lambda_{m})\in\RealSet^{m}\) se
  tiene, siendo
  \(z=\psi\circ\varphi^{-1}(x)\),
  \[\begin{array}{l}
      \psi_{2}^{*}((\widetilde{\varphi}^{*})^{-1}(x,\eta))(y)=
      ((\widetilde{\varphi}^{*})^{-1}(x,\eta))(\widetilde{\psi}^{-1}(z,y))=\\
      ((\widetilde{\varphi}^{*})^{-1}(x,\eta))(\sum\limits_{i=1}^{m}
      \lambda_{i}e_{i}^{\psi^{-1}(z)})=\\
      (\sum\limits_{j=1}^{m}\mu_{j}(e_{j}^{\varphi^{-1}(x)})^{*})(\sum\limits_{i=1}^{m}
      \lambda_{i}e_{i}^{\varphi^{-1}(x)})=\\
      \eta(\varphi_{2}(\widetilde{\psi}^{-1}(z,y)))=\eta(
      \J[z]{\varphi\circ\psi^{-1}(y)})=\\
      (\J[z]{\varphi\circ\psi^{-1}})^{t}(\eta)(y).
    \end{array}\]

  Luego,
  \[
    \psi_{2}^{*}((\widetilde{\varphi}^{*})^{-1}(x,\eta))=
    (\J[z]{\varphi\circ\psi^{-1}})^{t}(\eta)=
    ((\J[x]{\psi\circ\varphi^{-1}})^{-1})^{t}. \quad(*)
  \]

  Aquí recordamos que si \(A\) es una matriz representando una aplicación lineal
  \(f\colon V\to W\), entonces \(A^{t}\) es la matriz que
  representa la aplicación lineal dual \(f^{*}\colon(W)^{*}\to(V)^{*},
  f^{*}(\omega)=\omega\circ f\).

  
  Por tanto, el cambio de la carta \((\DualT{U},\widetilde{\varphi}^{*})\) a la
  carta \((\DualT{V},\widetilde{\psi}^{*})\) en \(\DualT{M}\) está representado
  por la traspuesta de la inversa de la matriz jacobiana que representa el
  cambio de la carta \((\Tangente{U},\widetilde{\varphi})\) a la carta
  \((\Tangente{V},\widetilde{\psi})\) en \(TM\), así que los cambios de cartas en
  \(\DualT{M}\) son difeomorfismos.

  \vline

  La proyección \(\pi^{*}\colon\DualT{M}\to M\) es submersión pues para una
  carta \((U,\varphi)\) de \(M\) se tiene el diagrama:

  \begin{center}
    \centering
    \begin{tikzcd}[column sep = large]
      \DualT{U}
      \ar[d, "\widetilde{\varphi}^*"']
      \ar[r, "\pi^*"]
      &
      U
      \ar[d, "\varphi"] \\ 
      \varphi(U)\times(\RealSet^m)^*
      \ar[r, ""']
      &
      \varphi(U)
    \end{tikzcd}
  \end{center}
donde \(\varphi\circ\pi^{*}\circ(\widetilde{\varphi}^{*})^{-1}(x,\eta)=x\) es la proyección natural.
\end{proof}

\begin{definition}[name=1-forma]\label{def:1-forma}
  Se llama \emph{\(1\)-forma} sobre la variedad  \(M\) a cualquier sección \(\omega\) (diferenciable) de
  \(\pi^{*}\).
\end{definition}

\begin{remark}
  Igual que pasa en el caso de los campos tangentes, las 1-formas están
  determinadas por sus componentes locales. Así, si \(\omega\colon
  M\to\DualT{M}\) es una 1-forma y \((U,\varphi)\) es una carta de \(M\),
  tenemos
  
  \begin{center}
    \centering
    \begin{tikzcd}[column sep = large]
      U
      \ar[d, "\varphi"']
      \ar[r, "\omega"]
      &
      \DualT{U}
      \ar[d, "\widetilde{\varphi}^*"] \\ 
      \varphi(U)
      \ar[r, ""']
      &
      \varphi(U)\times(\RealSet^m)^*
    \end{tikzcd}
  \end{center}

  donde \(\widetilde{\varphi}^{*}\circ\omega\circ\varphi^{-1}(x)=(x,g^{U}(x))\),
  con \(g^{U}\colon\varphi(U)\to(\RealSet^{m})^{*}\).

  A las funciones \(g^{U}\) se las llama \textbf{componentes locales de la
    \(1\)-forma \(\omega\)}.

  Ahora la relación entre las componentes locales de dos cartas \((U,\varphi)\)
  y \((V,\psi)\) viene dada por la ecuación
  \begin{equation}\label{1forM}
    g^{V}(\psi\circ\varphi^{-1}(x))=(\J[\psi\circ\varphi^{-1}(x)]
    {\varphi\circ\psi^{-1}})^{t}g^{U}(x) 
  \end{equation}
que sigue del cambio de cartas visto anteriormente.

  Por tanto, dar una \(1\)-forma es dar una colección de componentes locales para un
  atlas de \(M\) sujetas a la relación anterior.
\end{remark}

\section{Ejemplos de \(1\)-formas sobre \(M\)}

\begin{example}\label{ex:forma-dif}
  El ejemplo clásico es la \(1\)-forma conocida como \emph{diferencial o forma
    gradiente}: cada función diferenciable \(f\in\mathcal{F}(M)\) tiene asociada una \(1\)-forma
  \[
    p\in M\mapsto (\dif{f})_p\colon\Tangente[p]{M}\to \RealSet
  \]
  cuyas componentes locales (referidas a carta \((U,\varphi)\) de \(M\)) son
  
  \begin{equation}\label{def:grad-1}
   (\nabla f)^{U}_{\varphi(p)}=(\Restrict{\dfrac{\partial(f\circ \varphi^{-1})}{\partial x_1}}{\varphi(p)},\ldots,\Restrict{\dfrac{\partial(f\circ \varphi^{-1})}{\partial x_m}}{\varphi(p)})
\end{equation}

Localmente, \(e_i^p\) representa los elementos básicos de \(\Tangente[p]{M}\) y
\((e_i^p)^*\) representa los duales de los \(e_i^p\) y forman una base de
\(\DualT[p]{M}\). Esto se traduce en 
\[
(\dif{f})_p=\Restrict{\dfrac{\partial(f\circ \varphi^{-1})}{\partial x_1}}{\varphi(p)}(e_1^p)^*+\ldots+\Restrict{\dfrac{\partial(f\circ \varphi^{-1})}{\partial x_m}}{\varphi(p)}(e_m^p)^*
\]
Para comprobar que las relaciones locales anteriores definen una \(1\)-forma sobre toda la variedad \(M\) debemos comprobar que se cumple la relación en 
 \ref{def:grad-1}  si tenemos dos cartas \((U,\varphi)\) y \((V,\psi)\) con \(U\cap V\neq\emptyset\).  En efecto, si \(y\in \psi(U\cap V)\), se tiene para 
\(x=\varphi\circ \psi^{-1}(y)\) :
\begin{align*}
\Restrict{(\nabla f)^V_i}{y}&=\Restrict{\dfrac{\partial(f\circ \psi^{-1})}{\partial y_i}}{y}\\
&=\Restrict{\dfrac{\partial(f\circ \varphi^{-1}\circ \varphi \circ \psi^{-1})}{\partial y_i}}{y}\\
&=\sum_{k=1}^m \Restrict{\dfrac{\partial(f\circ \varphi^{-1})}{\partial x_k}}{x}\cdot\Restrict{\dfrac{\partial(\varphi\circ \psi^{-1})}{\partial y_i}}{y}\\
&=\text{producto escalar de la i-ésima columna de la matriz jacobiana} \\
                            &\J[y]{\varphi\circ\psi^{-1}} \text{ con el vector } (\nabla f)^U(x)\\
  &=\Restrict{(\J[y]{\varphi\circ\psi^{-1}}^{t} \cdot (\nabla f)^{U})_i}{x}.
\end{align*}

Luego \((\nabla f)^{V}(y)=(\J[y]{\varphi\circ\psi^{-1}})^{t}(\nabla f)^{U}(x)\). Esto prueba que cumple la propiedad de compatibilidad y tenemos que 
\(df\) es una \(1\)-forma sobre todo \(M\). 
\end{example}


\begin{remark}[name=notación tradicional]
En el caso concreto de las componentes de la aplicación de coordenadas de una carta \((U, \varphi)\),
  \[
  \varphi_i\colon U\overset{\varphi}{\longrightarrow}
  \varphi(U)\overset{\pi_i}{\longrightarrow}\RealSet,\
  \varphi_{i}\in\mathcal{F}(U).
\]
la  \(1\)-forma gradiente de \(\varphi_i\) es 
\begin{align*}
(\dif{\varphi_i})_p&=(\Restrict{\dfrac{\partial(\varphi_i\circ \varphi^{-1})}{\partial x_1}}{\varphi(p)},\ldots,\Restrict{\dfrac{\partial(\varphi_i\circ \varphi^{-1})}{\partial x_m}}{\varphi(p)})\\
                   &= \sum_{k=1}^{m} \Restrict{\dfrac{\partial(\overbracket{\varphi_i\circ \varphi^{-1}}^{\pi_{i}})}{\partial x_k}}{\varphi(p)}\cdot (e_k^p)^*\\
                   & = \sum_{k=1}^{m} \underbracket{\dfrac{\partial x_{i}}{\partial x_k}}_{\delta_{ik}}\cdot (e_k^p)^*=(e_{i}^{p})^{*} \colon T_p M\to \RealSet
\end{align*}

Recodemos que a los elementos básicos \(\{e_{i}^{p}\}_{i=1}^{m}\) de
  \(\Tangente[p]{M}\)  se les suele denotar \((\pderiv{}{x_{i}})_{\varphi}\) o simplemente \((\pderiv{}{x_{i}})\) si se sobreentiende la aplicación de coordenadas \(\varphi\).

  De la misma manera, a los duales de esos elementos básicos,
  \(\{(e_{i}^{p})^{*}\}_{i=1}^{m}\) se les suele denotar
  \(\Restrict{\dif{x_{i}}}{\varphi}\) o \(\dif{x_{i}}\).
\end{remark}

\begin{example}

  Si \(M=\RealSet^m\), con la carta identidad, toda \(1\)-forma \(\omega\)
  se expresa con la notación habitual
  \[
    \omega=g_1\dif{x_1}+\ldots+g_m\dif{x_m}.
  \]
  Sea \(f\in\mathcal{F}(\RealSet^m)\), entonces:
  \[
    df=\dfrac{\partial f}{\partial x_1}dx_1+\ldots+\dfrac{\partial f}{\partial
      x_m}dx_m
  \]
que es la notación usual en cálculo de varias variables.
\end{example}

\begin{example}
Sea \(M^m\) una variedad diferenciable y \(T^*M\) espacio cotangente.
\[
T^*(T^*M)=\text{espacio cotangente a } T^*M
\]
variedad diferenciable de dimension \(4m\).

Asociado a \(M\), hay una \(1\)-forma canónica \(\theta_M\) sobre \(T^*M\) (llamada de Liouville) 
\[
\theta_M\colon T^*M\to T^*(T^*M)
\]
definida de la siguiente manera:  Si \(\omega\in T^*M\) y \(v\in T_{\omega}T^*M$, escribimos 
\(\theta_M(\omega)(v)=\omega(\dif{\pi^*}(v))\), con
\(\dif{\pi^{*}}(v)\in\Tangente[p]{M}\). Nótese que para \(\pi^*: T^*M \to M\) tenemos 
\[ 
\dif{\pi^{*}}\colon\Tangente{\DualT{M}}\to\Tangente{M}, \ \ v\longmapsto\dif{\pi^{*}}(v)\in\Tangente[p]{M}.
\]
Queda como ejercicio comprobar que \(\theta_{M}\) es sección diferenciable.
\end{example}


\section{Descripciones alternativas de las \(1\)-formas}

Supongamos una \(1\)-forma \(\omega\) tal que \(p\in M
\overset{\omega}{\mapsto}\omega_p\in\DualT[p]{M}\) (a cada punto corresponde un
covector asociado). Entonces podemos definir una aplicación diferenciable \(f_\omega\in\mathcal{F}(\Tangente{M})\), 
\[
f_\omega\colon\Tangente{M}\to\RealSet,
\]
dada por \(f_\omega(v)=\omega_p(v)\in\RealSet,\quad v\in\Tangente[p]{M}\).

Veamos la diferenciabilidad de \(f_\omega\). Dada una carta \((U,\varphi)\) de \(M\), se tiene:
 \begin{center}
 	\begin{tikzcd}
	TU
	\ar[r, "f_\omega"]
	\ar[d, "\widetilde{\varphi}"] &
	\RealSet\\
	\varphi(U)\times\RealSet^m
	\ar[ur, "f_\omega\circ\widetilde{\varphi}^{-1}" ']
	\end{tikzcd}
\end{center}
hay que probar que \(f_\omega\circ \widetilde{\varphi}^{-1}(x,y)=\omega_{\varphi^{-1}(x)}(\widetilde{\varphi}^{-1}(x,y))\)
es diferenciable. Para ello consideramos el siguiente diagrama:
 \begin{center}
	\begin{tikzcd}
	U
	\ar[r, "\omega"]
	\ar[d, "\varphi"] &
	\DualT{U}
	\ar[d, "\widetilde{\varphi}^*"]\\
	\varphi(U)
	\ar[r, "\widetilde{\omega}"] &
	\varphi(U)\times(\RealSet^m)^*
	\end{tikzcd}
\end{center}
donde \(\widetilde{\omega}=\widetilde{\varphi}^{*}\circ\omega\circ\varphi^{-1}\)
es diferenciable por ser \(\omega\) una \(1\)-forma sobre \(M\).

Según las definiciones, 
\begin{align*}
\widetilde{\varphi}^*(\omega_ {\varphi^{-1}(x)})=(x,\eta)
\end{align*}
con \(\eta=(\lambda_1(x),\ldots,\lambda_m(x))\in(\RealSet^m)^*\) las coordenadas
respecto a los elemento de la base de \((\RealSet^m)^*\).

Dado \(y=(y_1,\ldots, y_m)\in\RealSet^{m}\), si aplicamos \(\eta\) a \(y\)
obtendremos:
\[
  \eta(y)=\underbracket{\sum_{i=1}^{m}\lambda_i(x)y_i}_{(*)}\overset{\text{def}}{=}
  \varphi^{*}_{2}(\omega_{\varphi^{-1}(x)})=
  \omega_{\varphi^{-1}(x)}(\widetilde{\varphi}^{-1}(x,y))
\]
que es diferenciable por serlo \((*)\), ya que lo son las \(\lambda_{i}(x)\).
Luego \(f_{\omega}\circ\widetilde{\varphi}^{-1}\) es diferenciable.

\vline

Así pues, podemos asociar a cada \(1\)-forma \(\omega\) sobre \(M\) una función
\(f_{\omega}\in\mathcal{F}(\Tangente{M})\):
\[
  \omega\mapsto f_{\omega}\in\mathcal{F}(\Tangente{M}).
\]
Recíprocamente, dada una \(f\in\mathcal{F}(\Tangente{M})\) tal que para todo
\(p\in M\) 
\[
  \Restrict{f}{\Tangente[p]{M}}\colon\Tangente[p]{M}\to\RealSet
\]
es lineal, podemos definir una \(1\)-forma que notaremos por \(\omega_f\) de la
siguiente manera:
\[
  p\in M\overset{\omega_{f}}{\longmapsto}((\omega_f)_p\colon T_pM\to
  \RealSet)\in\DualT{M}
\]
con \((\omega_f)_p=f|_{T_p(M)}\). Si tomamos ahora \(f_{\omega_{f}}\) se tiene
\(f_{\omega_{f}}=f\).
\par
\medskip
De acuerdo con la notación habitual, se denota por \(\Omega^{1}(M)\) el conjunto de las \(1\)-formas sobre \(M\). 
Las observaciones anteriores  llevan al siguiente resultado.


\begin{proposition}
El conjunto  \(\Omega^{1}(M)\) es un espacio vectorial, de hecho es subespacio vectorial
  de \(\mathcal{F}(\Tangente{M})\) si identificamos la \(1\)-forma \(\omega\) con la función \(f_{\omega}\).
\end{proposition}
  
A partir dela descripción de \(\Omega^1(M)\) anterior, toda aplicación diferenciable 
  \(h\colon M\to N\) induce una aplicación lineal
  \(h^{*}\colon\Omega^{1}(N)\to\Omega^{1}(M)\) dada por la composición 
  \(h^{*}(\omega)=\omega\circ\dif{h}\colon\Tangente{M}\to\Tangente{N\to\RealSet}\).
  Alternativamente, podemos también definirla como la sección de
  \(\pi^{*}\colon\DualT{M}\to M\),
  \[h^{*}(\omega)(p)(v)=\omega(h(p))(\dif{h}(v)),\ \mbox{ para todo }
  v\in\Tangente[p]{M}.\]

La aplicación \(h^*\) se denota también por \(d^*h\) y se le llama la \emph{codiferencial} de \(h\). 

Pasemos ahora a una tercera descripción más sutil de las \(1\)-formas sobre \(M\). Sea \(X\colon M\to\Tangente{M}\) un campo tangente sobre \(M\) y sea
\(\omega\colon M\to\DualT{M}\) una \(1\)-forma. Se tiene definida una función diferenciable
  \[\langle\omega,X\rangle\colon M\to\RealSet,\text{ dada por }
  \langle\omega,X\rangle(p)=\omega_{p}(X(p))\in\RealSet.\]

  La diferenciabilidad sigue de que para la carta \((U,\varphi)\),
  \(\langle\omega,X\rangle\circ\varphi^{-1}\colon\varphi(U)\to\RealSet\) se
  expresa por medio de las componentes \(g^{U},h^{U}\) de \(X\) y \(\omega\),
  respectivamente, como:
  \begin{align*}
    \omega_{\varphi^{-1}(x)}X_{\varphi^{-1}(x)}=
    &(\sum_{i=1}^{m}h_{i}^{U}(x)(e_{i}^{\varphi^{-1}(x)})^{*})
      (\sum_{j=1}^{m}g_{j}^{U}e_{j}^{\varphi^{-1}(x)})\overset{\delta_{ij}}{=}\\
    &\sum_{i=1}^{m}h_{i}^{U}(x)g_{i}^{U}(x),
  \end{align*}
que obviamente es diferenciable. Así pues, tenemos definida una aplicación
  \(\phi \colon\Omega^{1}(M)\to\text{Hom}(\mathbb{X}(M),\mathcal{F}(M))\) que lleva
 la \(1\)-forma \(\omega\) en \(\phi_{\omega}(X)=\langle\omega,X\rangle\), que además
  de ser lineal es homomorfismo de \(\mathcal{F}(M)\)-módulos, pues
  \(\phi_{\omega}(f\cdot X)=f\cdot\phi_{\omega}(X)\), para todo
  \(f\in\mathcal{F}(M)\).


\begin{theorem}\label{teorprincformas}
  La aplicación 
  \(\phi\colon\Omega^{1}(M)\to\text{Hom}_{\mathcal{F}(M)}
  (\mathbb{X}(M),\mathcal{F}(M))\), es un isomorfismo.
\end{theorem}

  De esta manera, tenemos una descripción funcional de las \(1\)-formas sobre una
  variedad si además cambiamos \(\mathbb{X}(M)\) por \(\mathcal{D}(M)\) el
  espacio de las derivaciones de \(M\) (recordemos que podemos ver los campos
  como derivaciones).


Para demostrar el teorema hacen falta dos resultados previos.

\begin{lemma}[name=de extensión de campos]\label{lem:ext-campos}
  Si \(X\colon U\to\Tangente{U}\) es un campo definido en un abierto
  \(U\subseteq M\) con \(p\in U\), entonces existen un abierto \(V\subseteq U\) con \(p\in V\)
  y un campo \(\widetilde{X}\colon M\to\Tangente{M}\) tal que
  \(\Restrict{\widetilde{X}}{V}=X\). 
\end{lemma}

\begin{proof}
  Por el lema de extensión de funciones, podemos encontrar abiertos \(V\subseteq V'\subseteq \overline{V}' \subseteq U\) con \(p\in V\) 
  y una función diferenciable \(f\colon M\to\RealSet\) tal que 
  abierto con \(\Restrict{f}{V}=1,\Restrict{f}{M- \overline{V}'}=0\). Como \(U\) y \(\overline{V}'\) son abiertos, es inmediato que 
   \[
    \widetilde{X}\colon M\to\Tangente{M}\colon q\in M\mapsto\widetilde{X}(q)=
    \begin{dcases}
      f(q)X(q) & q\in U, \\ 0 & q\in M-\overline{V}',
    \end{dcases}
  \]
define un campo sobre todo \(M\) que extiende a \(X\) sobre \(V\).  
\end{proof}

\begin{corollary}\label{cor:ext-campos}
  Sea \(p\in M,v\in\Tangente[p]{M}\), entonces existe un campo \(X\colon
  M\to\Tangente{M}\) tal que \(X(p)=v\).
\end{corollary}

\begin{proof}
  Sea \((U,\varphi)\) una carta con \(p\in U\), 
  \(v=\sum_{i=1}^{m}\lambda_{i}e_{i}^{p}\), entonces definimos \(Y: U \to TU\) por 
  \( Y(q) =\sum_{i=1}^{m}\lambda_{i}e_{i}^{q}, \ q\in U\). Por el Lema \ref{lem:ext-campos}
   existe un campo \(X: M \to TM \) con \(X(p) = Y(p) = v\).
\end{proof}

\begin{lemma}\label{lem:xi-iguales} Sea \(\xi\colon \mathbb{X}(M) \to \mathcal{F}(M)\) un homomorfismo de \(\mathcal{F}(M)\)-módulos. Si los campos \(X,Y\) coinciden en un abierto \(U\) con \(p\in U\), entonces las funciones \(\xi(X)\) y $\xi(Y)\) representan el mismo germen en \(p\).
\end{lemma}

\begin{proof}
 Sean \(V\subseteq V'\subseteq U\) abiertos, \(V\) con \(p \in V\) y \(\overline{V}' \subseteq U\). Sea \(f \in \mathcal{F}(M)\) con  
 \(\Restrict{f}{V} = 1\) y \(\Restrict{f}{M-\overline{V}'} = 0\). Se define el campo 
 
\[
    Z\colon M\to\Tangente{M}\colon p\in M\mapsto Z(q)=
    \begin{dcases}
      f(q)(X-Y)(q) & q\in U, \\ 0 & q\in M-\overline{V}'.
    \end{dcases}
  \] 
 Nótese que \(Z\) es un campo sobre \(M\) ya que \(U\) y \(M- \overline{V}'$ son abiertos que cubren a \(M\). De hecho \(Z\) es el campo nulo pues si \(q\in U\), \((X-Y)(q)=0\) y si
  \(q\not\in V'\), \(f(q)=0\). Por tanto \(\xi(Z)\) es la función nula y para todo   \(q\in V\) tenemos 
  \(0=\xi(Z)(q)=\xi(f\cdot(X-Y))(q)=f(q)\xi(X-Y)(q)=1\cdot(\xi(X)-\xi(Y))(q)\)
  y por tanto \(\xi(X)(q) = \xi(Y)(q)\).
\end{proof}

\begin{proof}[Prueba del Teorema \ref{teorprincformas}]
  

  Probaremos primero la inyectividad. Supongamos
  \(\phi_{\omega}=\phi_{\omega'}\), sean \(p\in M\) y \(v\in\Tangente[p]{M}\). Por el
  Corolario \ref{cor:ext-campos}, existe \(X\in\mathbb{X}(M)\) con \(X(p)=v\).
  Entonces
  \(\omega_{p}(v)=\omega_{p}(X(p))=\langle\omega,X\rangle(p)=
  \phi_{\omega}(X)(p)=\phi_{\omega'}(X)(p)=\omega'_{p}(X(p))=\omega'_{p}(v)\).
  Luego \(\omega=\omega'\) y \(\phi\) es inyectiva.
  
  Para ver que \(\phi\) es sobreyectiva, sea \(\xi: \mathbb{X}(M) \to \mathcal{F}(M)\) un homomorfismo de \(\mathcal{F}(M)\)-módulos. Para una carta cualquiera de un atlas $\mathcal{A}$ de $M$, \((U,\varphi)\), definimos el campo sobre \(U\) $X^U_i = \widetilde{\varphi}\circ Y^U_i \circ \varphi\) definido por medio del diagrama 
  
 \begin{center}
	\begin{tikzcd}
	U
	\ar[r, "X^U_i"]
	\ar[d, "\varphi"] &
	\DualT{U}
	\ar[d, "\widetilde{\varphi}"]\\
	\varphi(U)
	\ar[r, "Y^U_i"] &
	\varphi(U)\times \RealSet^m
	\end{tikzcd}
\end{center}    
donde \(Y^U_i(x) = (x,e_i)\) para \(x\in \varphi(U)\) y \(\{e_i\}_{i=1}^m\) la base canónica de \(\mathbb{R}^m\). Esto es, $X^U_i(p)$ está representado por la curva $\varepsilon^p_i(t) = \varphi^{-1}(\varphi(p)+te_i)$ para todo $p \in U$. Además, para todo $v\in T_pM$ se tiene $v = \sum_{i=1}^m \lambda^U_i(p) X^U_i(p)$, donde $\lambda^U_i: U \to \mathbb{R}$ es diferenciable. 

Por el lema de extensión de funciones, para cada $p\in U$ podemos encontrar una función diferenciable $\widehat{\lambda}^U_i: M \to \mathbb{R}$ con $\lambda^U_i = \widehat{\lambda}^U_i$ sobre un abierto $U_{p,i}\subseteq U$ con $p\in U_{p,i}$. Análogamente, el Lema  \ref{lem:ext-campos} permite encontrar un campo $\widehat{X}^U_i : M \to TM$ que coincide con $X^U_i$ sobre un abierto $U'_{p,i}\subseteq U$ con $p\in U'_{p,i}$. Sin pérdida de generalidad, podemos suponer $U_p = U_{p,i} = U'_{p,i}$ para todo $i$.     

Si aplicamos la propiedad de Lindeloff al recubrimiento de $M$ por los abiertos $U_p$, obtenemos  un nuevo atlas $\mathcal{A}'$ tal que para cada carta en él (mantenemos la notación anaterior), $(U,\varphi)$,  el campo $X^U_i$ admite una extensión a un campo  $\widetilde{X}^U_i: M \to TM$ y las funciones 
$\lambda^U_i$ se extienden  a funciones diferenciables $\widehat{\lambda}^U_i: M \to \mathbb{R}$. 

A continuación definiremos una $1$-forma como aplicación diferenciable $\omega: TM \to \mathbb{R}$ cuyas restricción a $T_p M$ es lineal para todo $p\in M$.

Dados $p\in M$ y $v\in T_pM$, sea  $(U,\varphi)$ una carta de $\mathcal{A}'$ con $p\in U$. Entonces $v = \sum_{i=1}^m \lambda^U_i(p) X^U_i(p)$ y definimos $\omega_U: M \to TM$ por   
  \[
  \omega_U(v) = \sum_{i=1}^m \widehat{\lambda}^U_i(q) \xi(\widehat{X}^U_i) (q) = \sum_{i=1}^m \xi(\widehat{\lambda}^U_i \widehat{X}^U_i)(q) = \xi(\sum_{i=1}^m \widehat{\lambda}^U_i \widehat{X}^U_i)(q) .   
  \]
Obviamente $\omega_U$ es diferenciable y lineal por definición. Queda comprobar que $\omega_U = \omega_V$ sobre los puntos de cada intersección $U\cap V$ para tener una $1$-forma bien definida. 

Sea pues $U\cap V\neq \emptyset$, para las cartas $(U,\varphi)$ y $(V,\psi)$ de $\mathcal{A}'$. Para todo $q\in U\cap V$ tenemos la igualdad $X^V_i(q) = \sum_{j=1}^m \alpha_{ij}(q) X^U_j(q)$, donde $J_{\varphi(q)}(\psi\circ \varphi^{-1}) = (\alpha_{ij}(q))$ es la matriz jacobiana  . Esto sigue delas definiciones de los campos $X^U_i$ y $X^V_i$. 

En particular, para las funciones coordenadas de un vector $v\in T_q M = T_q (U\cap V)$ se tiene 
$\lambda^U_j(q) = \sum_{i=1}^m \alpha_{ij}(q) \lambda^V_i(q)$. Fijado $q$,  podemos aplicar de nuevo el lema de extensión y encontrar aplicaciones diferenciables $\widehat{\alpha}^q_{ij}: M  \to  \mathbb{R}$ 
que coinciden con $\alpha_{ij}$ en un abierto $\Omega_q \subseteq U\cap V$.   

Entones los campos $\sum_{i= 1}^m \widehat{\lambda}^V_i \widehat{X}^V_i$ y $\sum_{i= 1}^m \widehat{\lambda}^V_i(\sum_{j=1}^m \widehat {\alpha}^q_{ij}\widehat{X}^U_j)$ coinciden en $\Omega_q$ y por el Lema \ref{lem:xi-iguales} las funciones $\xi(\sum_{i=1}^m\widehat{\lambda}^V_i\widehat{X}^V_i)$ y $\xi(\sum_{i= 1}^m \widehat{\lambda}^V_i(\sum_{j=1}^m \widehat {\alpha}^q_{ij}\widehat{X}^U_j))$ coinciden en un abierto $W\subseteq \Omega_q$ con $q\in W$. Entonces para cualquier $v\in T_q M = T_qW$, se cumplen las igualdades
\[
\omega_V(v) = \xi(\sum_{i=1}^m\widehat{\lambda}^V_i\widehat{X}^V_i)(q) = \xi(\sum_{i= 1}^m \widehat{\lambda}^V_i(\sum_{j=1}^m \widehat {\alpha}^q_{ij}\widehat{X}^U_j))(q) =
\]
\[ = \sum_{i= 1}^m \widehat{\lambda}^V_i(q)(\sum_{j=1}^m \widehat{\alpha}^q_{ij}(q)\xi(\widehat{X}^U_j)(q))
= \sum_{i= 1}^m \lambda^V_i(q)(\sum_{j=1}^m \alpha_{ij}(q) \xi(\widehat{X}^U_j)(q)) =
\]
\[= \sum_{i=1}^m \lambda^U_i(q) \xi(\widehat{X}^U_j)(q) = \sum_{i=1}^m \widehat{\lambda}^U_i(q) \xi(\widehat{X}^U_j)(q) = \xi(\sum_{i=1}^m \widehat{\lambda}^U_i(q) \widehat{X}^U_j)(q) = \omega_U(v).
\]
\end{proof}

\begin{remark}
  Por tanto, hemos construido tres  maneras de ver el espacio cotangente:
  \begin{enumerate}
  \item Como unión de los duales de los espacios tangentes en los puntos de $M$, \(\DualT{M}=\bigcup_{p\in M}\DualT[p]{M}\)
  \item Como funciones diferenciables \(f\colon\Tangente{M}\to\RealSet\) cuyas restricciones  $f: T_pM \to \mathbb{R}$ son lineales. 
  \item Como homomorfismos de $\mathcal{F}(M)$-módulos entre campos y funciones sobre $M$, 
  \[\Omega^{1}(M)\cong\text{Hom}_{\mathcal{F}(M)}(\mathbb{X}(M),\mathcal{F}(M)).
  \]
Esto es, $\Omega^1(M)$ es el $\mathcal{F}(M)$-módulo dual de $\mathbb{X}(M)$.   
 
  \end{enumerate}
\end{remark}
\end{document}





%%% Local Variables:
%%% TeX-master: "../VD_ebook"
%%% End: