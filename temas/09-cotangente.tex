\documentclass[../VD.tex]{subfiles}

\externaldocument{../VD}

\begin{document}

\setcounter{chapter}{8}
\chapter{Espacio cotangente de una variedad}\label{chap:dual}

\section{Introducción}

El espacio cotangente de una variedad se basa en el \emph{estudio de los
  momentos} de la física.

\begin{definition}
  Sea \(W\) espacio vectorial, el dual de \(W\) será el conjunto de aplicaciones
  lineales \(f\colon W\to K\) con \(K\) cuerpo del espacio vectorial. 
\end{definition}

\begin{definition}[name=espacio cotangente a \(M\) en el punto \(p\)]
Sea \(M\) variedad diferenciable, \(p\in M,\
\Tangente[p]{M}\), entonces
\(\DualT[p]{M}=\{\omega\colon\Tangente[p]{M}\to\RealSet
\text{lineales}\}\)
\end{definition}

\begin{proposition}
  Si \(M\) es una variedad diferenciable de dimensión \(m\) entonces
  \(\DualT{M}\) es una variedad diferenciable de dimensión \(2m\).

  Además, la aplicación \(\pi^{*}\colon\DualT{M}\to M\), tal que
  \(\pi^{*}(\omega)=p\) (\(\omega\in\DualT[p]{M}\)), es submersión. 
\end{proposition}

\begin{proof}
  Dada \((U,\varphi)\) carta de \(M\), sea \(\DualT{U}=\bigcup_{p\in
    U}\DualT[p]{M}\). Si \((\RealSet^{m})^{*}\) es el espacio dual de
  \(\RealSet^{m}\), entonces definimos la aplicación dada por  
  \(\widetilde{\varphi}^{*}\colon\DualT{U}\to\varphi(U)\times(\RealSet^{m})^{*}\)
  de la siguiente manera:

  \vline
  
  Dada la \textbf{aplicación lineal} \(\omega\in\DualT[p]{M}\),
  \(\widetilde{\varphi}^{*}(\omega)=(\varphi(p),\varphi_{2}^{*}(\omega))\),
  donde \(\varphi_{2}^{*}(\omega)\colon\RealSet^{m}\to\RealSet\) es la forma
  lineal que a cada \(y\in\RealSet^{m}\) asocia
  \(\varphi_{2}^{*}(\omega)(y)=\omega(\widetilde{\varphi}^{-1}(\varphi(p),y))\)
  siendo
  \(\widetilde{\varphi}\colon\Tangente{U}\to\varphi(U)\times\RealSet^{m}\) tal
  que \(\widetilde{\varphi}(v)=(\varphi(p),\varphi_{2}(v))\) con
  \(v\in\Tangente[p]{M}\).

  \vline
  
  Si ahora fijamos el isomorfismo
  \(\phi\colon\RealSet^{m}\times(\RealSet^{m})^{*}\cong\RealSet^{m}\times\RealSet^{m}\)
  que lleva los covectores de la
  base canónica de \((\RealSet^{m})^{*}\), \(e_{i}^{*}\), en los vectores de la base
  canónica de \(\RealSet^{m}\), \(e_{i}\), entonces la composición
  \(\phi\circ\varphi^{*}\colon\DualT{U}\to\varphi(U)\times\RealSet^{m}\) (ya no
  va a \(\varphi(U)\times(\RealSet^{m})^{*}\)).

  \vline

  Obsérvese que si \((e_{i}^{p})^{*}\) es el dual del vector básico
  \(e_{i}^{p}\in\Tangente[p]{M}\), entonces
  \(\widetilde{\varphi}^{*}((e_{i}^{p})^{*})=(\varphi(p),e_{i}^{*})\), pues si
  \(y=\sum\limits_{i=1}^{m}\lambda_{i}e_{i}\), entonces:
  
  \[\begin{array}{l}
      \widetilde{\varphi}^{-1}(\varphi(p),y)=\sum\limits_{i=1}^{m}\lambda_{i}e_{i}^{p}=
      v\in\Tangente[p]{M},
    \end{array}\]
  
  y
  
  \[\begin{array}{l}
      \varphi_{2}^{*}((e_{i}^{p})^{*})(y)=(e_{i}^{p})^{*}(v)=\lambda_{i}.
    \end{array}\]
  
  Por tanto, si \(\omega=\sum\limits_{i=1}^{m}\lambda_{i}(e_{i}^{p})^{*}\), entonces
  
  \[
    \phi\circ\widetilde{\varphi}^{*}(\omega)=(x,(\lambda_{1},\dots,\lambda_{m}))
    \in\RealSet^{m}\times\RealSet^{m}
  \]

  y la inversa de \(\widetilde{\varphi}^{*}\) lleva
  \((x,\lambda_{1},\dots,\lambda_{m})\) en la forma
  \(\sum\limits_{i=1}^{m}\lambda_{i}(e_{i}^{\varphi^{-1}(x)})^{*}\).

  Observamos que \(\widetilde{\varphi}^{*}\) es un homeomorfismo.
  
  \vline
  
  Queda ver que si \(\mathcal{A}=\{(U,\varphi)\}\) es un atlas de \(M\) entonces
  \(\{(\DualT{U},\phi\circ\widetilde{\varphi}^{*})\}\) es un atlas para \(\DualT{M}\).

  En primer lugar, la topología de \(\DualT{M}\) es la unión de las topologías
  en los \(p\in M\), de la misma forma que hacíamos en \(\Tangente{M}\), que son
  isomorfas a las de \(\RealSet^{m}\). En particular, \(\DualT{U}\) es abierto.

  Si consideramos \(\phi\) como una identificación de coordenadas, bastará hacer
  las comprobaciones para las \(\widetilde{\varphi}^{*}\). Es inmediato que
  \(\widetilde{\varphi}^{*}(\Tangente{U})\) es un abierto de
  \(\RealSet^{m}\times(\RealSet^{m})^{*}\cong_{\phi}\RealSet^{m}\times
  \RealSet^{m}=\RealSet^{2m}\).

  Además, si \((U,\varphi),(V,\psi)\) son cartas de \(\mathcal{A}\) con \(U\cap
  V\neq\emptyset\), entonces \(\DualT{U}\cap\DualT{V}=\DualT{(U\cap V)}\) y
  \(\widetilde{\varphi}^{*}(\DualT{U}\cap\DualT{V})=\varphi(U\cap
  V)\times(\RealSet^{m})^{*}\) y
  \(\widetilde{\psi}^{*}(\DualT{U}\cap\DualT{V})=\psi(U\cap 
  V)\times(\RealSet^{m})^{*}\) son abiertos.

  \vline

  Queda probar que la aplicación siguiente es un difeomorfismo:
  \[\widetilde{\psi}^{*}\circ(\widetilde{\varphi}^{*})^{-1}\colon\varphi(U\cap
    V)\times(\RealSet^{m})^{*}\to\psi(U\cap V)\times(\RealSet^{m})^{*}.\]

  Dado \((x,\eta)\in\varphi(U\cap V)\times(\RealSet^{m})^{*}\), con
  \(\eta=(\mu_{1},\dots,\mu_{m})\), se tiene por
  definición
  \[
    \widetilde{\psi}^{*}\circ(\widetilde{\varphi}^{*})^{-1}(x,\eta)=
    (\psi\circ\varphi^{-1}(x),\psi_{2}^{*}((\widetilde{\varphi}^{*})^{-1}(x,\eta)))
    \quad (1)
  \]

  donde para un vector \(y=(\lambda_{1},\dots,\lambda_{m})\in\RealSet^{m}\) se
  tiene, siendo
  \(z=\psi\circ\varphi^{-1}(x)\),
  \[\begin{array}{l}
      \psi_{2}^{*}((\widetilde{\varphi}^{*})^{-1}(x,\eta))(y)=
      ((\widetilde{\varphi}^{*})^{-1}(x,\eta))(\widetilde{\psi}^{-1}(z,y))=\\
      ((\widetilde{\varphi}^{*})^{-1}(x,\eta))(\sum\limits_{i=1}^{m}
      \lambda_{i}e_{i}^{\psi^{-1}(z)})=\\
      (\sum\limits_{j=1}^{m}\mu_{j}(e_{j}^{\varphi^{-1}(x)})^{*})(\sum\limits_{i=1}^{m}
      \lambda_{i}e_{i}^{\varphi^{-1}(x)})=\\
      \eta(\varphi_{2}(\widetilde{\psi}^{-1}(z,y)))=\eta(
      \J[z]{\varphi\circ\psi^{-1}(y)})=\\
      (\J[z]{\varphi\circ\psi^{-1}})^{t}(\eta)(y)
    \end{array}\]

  Luego,
  \[
    \psi_{2}^{*}((\widetilde{\varphi}^{*})^{-1}(x,\eta))=
    (\J[z]{\varphi\circ\psi^{-1}})^{t}(\eta)=
    ((\J[x]{\psi\circ\varphi^{-1}})^{-1})^{t}\quad(*)
  \]

  Aquí recordamos que si \(A\) es una matriz representando una aplicación lineal
  \(f\colon V\to W\), entonces \(A^{t}\) es la matriz que
  representa la aplicación lineal dual \(f^{*}\colon(W)^{*}\to(V)^{*},
  f^{*}(\omega)=\omega\circ f\).

  
  Por tanto, el cambio de la carta \((\DualT{U},\widetilde{\varphi}^{*})\) a la
  carta \((\DualT{V},\widetilde{\psi}^{*})\) en \(\DualT{M}\) está representado
  por la traspuesta de la inversa de la matriz jacobiana que representa el
  cambio de la carta \((\Tangente{U},\widetilde{\varphi})\) a la carta
  \((\Tangente{V},\widetilde{\psi})\), así que los cambios de cartas en
  \(\DualT{M}\) son diferenciables.

  \vline

  La proyección \(\pi^{*}\colon\DualT{M}\to M\) es submersión pues para una
  carta \((U,\varphi)\) de \(M\) se tiene el diagrama:

  \begin{center}
    \centering
    \begin{tikzcd}[column sep = large]
      \DualT{U}
      \ar[d, "\widetilde{\varphi}^*"']
      \ar[r, "\pi^*"]
      &
      U
      \ar[d, "\varphi"] \\ 
      \varphi(U)\times(\RealSet^m)^*
      \ar[r, ""']
      &
      \varphi(U)
    \end{tikzcd}
  \end{center}

  donde \(\varphi\circ\pi^{*}\circ(\widetilde{\varphi}^{*})^{-1}(x,\eta)=x\) es la
  proyección natural.
\end{proof}

\begin{definition}[name=1-forma]\label{def:1-forma}
  Se llama \emph{1-forma} de \(M\) a cualquier sección \(\omega\) (diferenciable) de
  \(\pi^{*}\).
\end{definition}

\begin{remark}
  Igual que pasa en el caso de los campos tangentes, las 1-formas están
  determinadas por sus componentes locales. Así, si \(\omega\colon
  M\to\DualT{M}\) es una 1-forma y \((U,\varphi)\) es una carta de \(M\),
  tenemos
  
  \begin{center}
    \centering
    \begin{tikzcd}[column sep = large]
      U
      \ar[d, "\varphi"']
      \ar[r, "\omega"]
      &
      \DualT{U}
      \ar[d, "\widetilde{\varphi}^*"] \\ 
      \varphi(U)
      \ar[r, ""']
      &
      \varphi(U)\times(\RealSet^m)^*
    \end{tikzcd}
  \end{center}

  donde \(\widetilde{\varphi}^{*}\circ\omega\circ\varphi^{-1}(x)=(x,g^{U}(x))\),
  con \(g^{U}\colon\varphi(U)\to(\RealSet^{m})^{*}\).

  A las funciones \(g^{U}\) se las llama \textbf{componentes locales de la
    1-forma \(\omega\)}.

  Ahora la relación entre las componentes locales de dos cartas \((U,\varphi)\)
  y \((V,\psi)\) viene dada por la ecuación
  \[
    g^{V}(\psi\circ\varphi^{-1}(x))=(\J[\psi\circ\varphi^{-1}(x)]
    {\varphi\circ\psi^{-1}})^{t}g^{U}(x)
  \]

  que sigue del cambio de cartas visto anteriormente.

  Por tanto, dar una 1-forma es dar una colección de componentes locales para un
  atlas de \(M\) sujetas a la relación anterior.
\end{remark}

\section{Ejemplos de 1-formas sobre \(M\)}

\begin{example}
  El ejemplo clásico es la 1-forma conocida como \emph{diferencial o gradiente}:
  asociada a cada \(f\in\mathcal{F}(M)\) hay asociada una 1-forma
  \[
    p\in M\mapsto((\dif{f})_p\colon\Tangente[p]{M}\to \RealSet)
  \]
  cuyas componentes locales (referidas a carta \((U,\varphi)\) de \(M\)) son
  
  \[\label{def:grad-1}
   (\nabla f)^{U}_{\varphi(p)}=(\Restrict{\dfrac{\partial(f\circ \varphi^{-1})}{\partial x_1}}{\varphi(p)},\ldots,\Restrict{\dfrac{\partial(f\circ \varphi^{-1})}{\partial x_m}}{\varphi(p)})
\]
Localmente, \(e_i^p\) representa los elementos básicos de \(\Tangente[p]{M}\) y
\((e_i^p)^*\) representa los duales de los \(e_i^p\) y forman una base de
\(\DualT[p]{M}\). Esto se traduce en 
\[
(\dif{f})_p=\Restrict{\dfrac{\partial(f\circ \varphi^{-1})}{\partial x_1}}{\varphi(p)}(e_1^p)^*+\ldots+\Restrict{\dfrac{\partial(f\circ \varphi^{-1})}{\partial x_m}}{\varphi(p)}(e_m^p)^*
\]
 Si \(U\cap V\neq\emptyset\), dada una 1-forma \(\omega\) con componentes
locales 
\begin{enumerate}
\item \(g^U\colon\varphi(U)\to (\RealSet^m)^*,\text{ para la carta }(U,\varphi)\)
\item \(g^V\colon \psi(V)\to (\RealSet^m)^*,\text{ para la carta }(V,\psi{})\)
\end{enumerate}
ha de cumplirse:
\[
g^V(\psi\circ \varphi^{-1}(x))=(\J[\psi\circ \varphi^{-1}(x)]{\varphi\circ
  \psi^{-1}})^tg^U(x),\quad x\in\varphi(U\cap V)
\]

Sean \(y\in \psi(U\cap V),\ x=\varphi\circ \psi^{-1}(y)\) se tiene:
\begin{align*}
\Restrict{(\nabla f)^V_i}{y}&=\Restrict{\dfrac{\partial(f\circ \psi^{-1})}{\partial y_i}}{y}\\
&=\Restrict{\dfrac{\partial(f\circ \varphi^{-1}\circ \varphi \circ \psi^{-1})}{\partial y_i}}{y}\\
&=\sum_{k=1}^m \Restrict{\dfrac{\partial(f\circ \varphi^{-1})}{\partial x_k}}{x}\cdot\Restrict{\dfrac{\partial(\varphi\circ \psi^{-1})}{\partial y_i}}{y}\\
&=\text{producto escalar de la i-ésima columna de la matriz jacobiana} \\
                            &\J[y]{\varphi\circ\psi^{-1}} \text{ con el vector } (\nabla f)^U(x)\\
  &=(\J[y]{\varphi\circ\psi^{-1}})_{\cdot i}\cdot\Restrict{(\nabla f)^{U}}{x}
\end{align*}

Luego \((\nabla f)^{V}(y)=(\J[y]{\varphi\circ\psi^{-1}})^{t}(\nabla f)^{U}(x)\),
siendo el gradiente el visto en este \cref{def:grad-1}.

Esto probaría que cumple la propiedad de compatibilidad que relaciona las
componentes locales.

Veamos un caso concreto. Si tenemos una carta \((U,\varphi)\) de \(M\) podemos
ver la siguiente aplicación localmente:
\[
  \varphi_i\colon U\overset{\varphi}{\longrightarrow}
  \varphi(U)\overset{\pi_i}{\longrightarrow}\RealSet,\
  \varphi_{i}\in\mathcal{F}(U)
\]
Cuyas 1-formas vienen dadas por
\begin{align*}
(\dif{\varphi_i})_p&=(\Restrict{\dfrac{\partial(\varphi_i\circ \varphi^{-1})}{\partial x_1}}{\varphi(p)},\ldots,\Restrict{\dfrac{\partial(\varphi_i\circ \varphi^{-1})}{\partial x_m}}{\varphi(p)})\\
                   &= \sum_{k=1}^{m} \Restrict{\dfrac{\partial(\overbracket{\varphi_i\circ \varphi^{-1}}^{\pi_{i}})}{\partial x_k}}{\varphi(p)}\cdot (e_k^p)^*\\
                   & = \sum_{k=1}^{m} \underbracket{\dfrac{\partial x_{i}}{\partial x_k}}_{\delta_{ik}}\cdot (e_k^p)^*=(e_{i}^{p})^{*}\colon T_p M\to \RealSet
\end{align*}
\end{example}

\begin{remark}[name=notación tradicional]
  Localmente, a los elementos básicos \(\{e_{i}^{p}\}_{i=1}^{m}\) de
  \(\Tangente[p]{M}\)  se les suele denotar \((\pderiv{}{x_{i}})_{\varphi}\).

  De la misma manera, a los duales de esos elementos básicos,
  \(\{(e_{i}^{p})^{*}\}_{i=1}^{m}\) se les suele denotar
  \(\Restrict{\dif{x_{i}}}{\varphi}\).
\end{remark}

\begin{example}
  Si \(M=\RealSet^m\) (\(\mathcal{A}=(\RealSet^m,id)\)), toda 1-forma \(\omega\)
  se expresa con la notación habitual
  \[
    \omega=g_1\dif{x_1}+\ldots+g_m\dif{x_m}.
  \]
  Sea \(f\in\mathcal{F}(\RealSet^m)\), entonces:
  \[
    df=\dfrac{\partial f}{\partial x_1}dx_1+\ldots+\dfrac{\partial f}{\partial
      x_m}dx_m
  \]
  
  que es la notación usual en cálculo de varias variables.
\end{example}

\begin{example}
Sea \(M^m\) una variedad diferenciable y \(T^*M\) espacio cotangente.
\[
T^*(T^*M)=\text{espacio cotangente a } T^*M
\]
variedad diferenciable de dimension \(4m\).

Asociado a \(M\), hay una 1-forma canónica \(\theta_M\) sobre \(T^*M\) (llamada de Liouville) definida de la siguiente manera:
\[
\theta_M\colon T^*M\to T^*(T^*M)
\]
con \(\theta_M(\omega)\in T^*_\omega(T^*M)\). Si \(\omega\in T^*M\) definimos
\(\theta_M(\omega)(v)=\omega(\dif{\pi^*}(v))\), con
\(\dif{\pi^{*}}(v)\in\Tangente[p]{M}\) pues:
\[
  v\in\Tangente[\omega]{\DualT{M}},\ \pi^{*}\colon\DualT{M}\to M\leadsto
  \begin{array}{l}\\ \dif{\pi^{*}}\colon\Tangente{\DualT{M}}\to\Tangente{M}\\
  \quad\quad\quad\quad v\longmapsto\dif{\pi^{*}}(v)\in\Tangente[p]{M}\end{array}
\]

se prueba que \(\theta_{M}\) es sección diferenciable.
\end{example}

\section{Una forma alternativa de ver las 1-formas}

Supongamos una 1-forma \(\omega\) tal que \(p\in M
\overset{\omega}{\mapsto}\omega_p\in\DualT[p]{M}\) (a cada punto corresponde un
covector asociado). Entonces podemos definir una aplicación diferenciable 
\[
f_\omega\colon\Tangente{M}\to\RealSet,\quad (f_\omega\in\mathcal{F}(\Tangente{M}))
\]
dada por \(f_\omega(v)=\omega_p(v)\in\RealSet,\quad v\in\Tangente[p]{M}\).

Veamos la diferenciabilidad de \(f_\omega\).

Dada una carta \((U,\varphi)\) de \(M\), se tiene:
 \begin{center}
 	\begin{tikzcd}
	TU
	\ar[r, "f_\omega"]
	\ar[d, "\widetilde{\varphi}"] &
	\RealSet\\
	\varphi(U)\times\RealSet^m
	\ar[ur, "f_\omega\circ\widetilde{\varphi}^{-1}" ']
	\end{tikzcd}
\end{center}

Tendríamos que probar que \(f_\omega\circ
\widetilde{\varphi}^{-1}(x,y)=\omega_{\varphi^{-1}(x)}(\widetilde{\varphi}^{-1}(x,y))\)
es diferenciable.\\
Además tenemos el siguiente diagrama:
 \begin{center}
	\begin{tikzcd}
	U
	\ar[r, "\omega"]
	\ar[d, "\varphi"] &
	T^*(U)
	\ar[d, "\tilde{\varphi}^*"]\\
	\varphi(U)
	\ar[r, "\widetilde{\omega}"] &
	\varphi(U)\times(\RealSet^m)^*
	\end{tikzcd}
\end{center}

donde \(\widetilde{\omega}=\widetilde{\varphi}^{*}\circ\omega\circ\varphi^{-1}\)
es diferenciable por ser \(\omega\) 1-forma.

Según las definiciones, 
\begin{align*}
\widetilde{\varphi}^*(\omega_ {\varphi^{-1}(x)})=(x,\eta)
\end{align*}
con \(\eta=(\lambda_1(x),\ldots,\lambda_m(x))\in(\RealSet^m)^*\) las coordenadas
respecto a los elemento de la base de \((\RealSet^m)^*\).

Dado \(y=(y_1,\ldots, y_m)\in\RealSet^{m}\), si aplicamos \(\eta\) a \(y\)
obtendremos:
\[
  \eta(y)=\underbracket{\sum_{i=1}^{m}\lambda_i(x)y_i}_{(*)}\overset{\text{def}}{=}
  \varphi^{*}_{2}(\omega_{\varphi^{-1}(x)})=
  \omega_{\varphi^{-1}(x)}(\widetilde{\varphi}^{-1}(x,y))
\]

que es diferenciable por serlo \((*)\), ya que lo son las \(\lambda_{i}(x)\).
Luego \(f_{\omega}\circ\widetilde{\varphi}^{-1}\) es diferenciable.

\vline

Así pues, podemos asociar a cada 1-forma \(\omega\) sobre \(M\) una función
\(f_{\omega}\in\mathcal{F}(\Tangente{M})\):
\[
  \omega\mapsto f_{\omega}\in\mathcal{F}(\Tangente{M})
\]
Recíprocamente, dada una \(f\in\mathcal{F}(\Tangente{M})\) tal que para todo
\(p\in M\) 
\[
  \Restrict{f}{\Tangente[p]{M}}\colon\Tangente[p]{M}\to\RealSet
\]
es lineal, podemos definir una 1-forma que notaremos por \(\omega_f\) de la
siguiente manera:
\[
  p\in M\overset{\omega_{f}}{\longmapsto}((\omega_f)_p\colon T_pM\to
  \RealSet)\in\DualT{M}
\]

con \((\omega_f)_p=f|_{T_p(M)}\). Si tomamos ahora \(f_{\omega_{f}}\) se tiene
\(f_{\omega_{f}}=f\).

\begin{remark}
  \(\Omega^{1}(M)=\{\text{1-formas sobre }M\}\).
\end{remark}

\begin{proposition}
  \(\Omega^{1}(M)\) es un espacio vectorial, de hecho es subespacio vectorial
  \(\begin{array}{c}\\ \Omega^{1}(M)\subseteq\mathcal{F}(\Tangente{M})\\
      \omega\mapsto f_{\omega}\end{array}\).
\end{proposition}
      
\end{document}

%%% Local Variables:
%%% TeX-master: "../VD"
%%% End: