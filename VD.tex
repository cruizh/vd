\documentclass[ebook,oneside]{memoir}

\usepackage{polyglossia}
\setdefaultlanguage{spanish}

\usepackage{amsmath,amssymb,mathtools}
\usepackage{amsthm,thmtools}
\usepackage{faktor} % Espacios cociente
\usepackage{exercise}
\usepackage{cool}
\usepackage[inline]{enumitem}
\usepackage{subcaption}
\usepackage{mathrsfs}

\usepackage{tikz, tikz-cd, pgfplots}
\usetikzlibrary{decorations.markings,arrows, shapes.geometric}

\usepackage[unicode]{hyperref}
\usepackage{nameref,cleveref}

\usepackage{xcolor}
\hypersetup{
    colorlinks,
    linkcolor={red!50!black},
    citecolor={blue!50!black},
    urlcolor={blue!80!black}
}

% rhcool
\newcommand{\NaturalSet}{\ensuremath{\mathbb{N}}}   % Naturales
\newcommand{\IntegerSet}{\ensuremath{\mathbb{Z}}}   % Enteros
\newcommand{\RationalSet}{\ensuremath{\mathbb{Q}}}  % Racionales
\newcommand{\RealSet}{\ensuremath{\mathbb{R}}}      % Reales
\newcommand{\ComplexSet}{\ensuremath{\mathbb{C}}} % Complejos
\newcommand{\Parts}[1]{\ensuremath{\mathcal{P}(#1)}} % Partes de un conjunto
\newcommand{\Field}{\ensuremath{\mathbb{K}}}        % Cuerpo arbitrario
\newcommand{\ProjectiveSpace}{\ensuremath{\mathbb{P}}} % Espacio proyectivo
\DeclarePairedDelimiter{\Norm}{\lVert}{\rVert}
\makeatletter
\newcommand{\Set}[2][]{%
\listval{#1}{0}% Record length of list
\setcounter{COOL@listlen}{\value{COOL@listpointer}}
\ifthenelse{\value{COOL@listlen} = 0}{%
  \{#2\}%
}{% else
  \ifthenelse{\value{COOL@listlen} = 1}{%
    \{ #2 \}_{#1}%
  }{% else
    \ifthenelse{\value{COOL@listlen} = 3}{%
      \{ #2 \}_{\listval{#1}{1} = \listval{#1}{2}}^{\listval{#1}{3}}%
    }{% else
      \PackageError{cool}{Invalid list length for ‘Set’}%
      {can only have none, one or three elements for second argument}%
    }%
  }%
}%
}
\makeatother


% rhtopology
\makeatletter
\newcommand{\COOL@notation@InteriorParen}{p}
\newcommand{\Interior}[1]{\operatorname{int}\COOL@decide@paren{Interior}{#1}}
\newcommand{\Exterior}[1]{\operatorname{ext}\COOL@decide@paren{Interior}{#1}}
\newcommand{\Clausura}[1]{\overline{#1}}
\newcommand{\Frontera}[1]{\partial{}#1}
\newcommand{\Dist}[2]{\operatorname{d}\left( #1,\, #2 \right)}
\newcommand{\Bola}[2]{\operatorname{B}\left( #1,\, #2 \right)}
\newcommand{\Id}[1]{\mathbf{1}_{#1}}
\newcommand{\Restrict}[2]{{% we make the whole thing an ordinary symbol
  \left.\kern-\nulldelimiterspace{}% automatically resize the bar with \right
  #1 % the function
  \vphantom{\big|} % pretend it's a little taller at normal size
  \right|_{#2} % this is the delimiter
}} % Source: https://tex.stackexchange.com/questions/22252
\newcommand{\T}[1]{T_{#1}} % Clases de espacio topológico
\newcommand{\sumacon}{\mathop{\#}} % Suma conexa
\newcommand{\homeo}{\cong} % Homeomorfo
\newcommand{\homotop}{\simeq} % Homotópico
\newcommand{\COOL@notation@GFParen}{p}
\newcommand{\GF}[1]{\operatorname{\pi_{1}}\COOL@decide@paren{GF}{#1}} % Grupo
% fundamental
\newcommand{\Free}[1]{\operatorname{F}\left(#1\right)}
\newcommand{\normal}{\lhd} % Subgrupo normal
\DeclarePairedDelimiter{\SG}{\langle}{\rangle} % Subgrupo generado
\DeclarePairedDelimiter{\SGN}{\lhd}{\rhd} % Subgrupo normal generado
\DeclareMathOperator{\Ker}{Ker}
\makeatother

% thmrh
\declaretheorem[
  name=Teorema,
  refname={Teorema,Teoremas}%,
  % numberwithin=section]{theorem}
  ]{theorem}
  
\declaretheorem[
  name=Proposición,
  refname={Proposición,Proposiciones}]{proposition}

\declaretheorem[
  name=Propiedad,
  refname={Propiedad,Propiedades}]{property}
  
\declaretheorem[
  name=Ejemplo,
  refname={Ejemplo,Ejemplos}]{example}

\declaretheorem[
  name=Corolario,
  refname={Corolario,Corolarios}]{corollary}

\declaretheorem[
  name=Criterio,
  refname={Criterio,Criterios}]{criterion}

\declaretheorem[
  name=Lema,
  refname={Lema,Lemas}]{lemma}

\declaretheorem[
  name=Definición,
  style=definition,
  refname={Definición,Definiciones}]{definition}

\declaretheorem[
  name=Observación,
  style=remark,
  refname={Observación,Observaciones}]{remark}
  
\declaretheorem[
  name=Nota,
  style=remark,
  refname={Nota,Notas},
  sibling=remark]{note}

\numberwithin{equation}{chapter}

\newenvironment{subproof}[1][\proofname]{%
  \renewcommand{\qedsymbol}{$\blacksquare$}%
  \begin{proof}[#1]%
}{%
  \end{proof}%
}

\title{Variedades Diferenciables}
\author{Carlos José Ruiz-Henestrosa Ruiz}
\date{2018}

\begin{document}

\maketitle

\chapter{Cartas, atlas y compatibilidad}
\label{chap:cartas}


\begin{definition}[carta]
  Dado un conjunto \(X\), una \emph{carta} de \(X\) es un par \((\mathcal{U},
  \varphi)\) con \(\mathcal{U} \subseteq X\) y \(\varphi \colon \mathcal{U} \to
  \RealSet^{n}\) inyectiva y su imagen \(\varphi(\mathcal{U})\) es un abierto de
  \(\RealSet^{n}\).
\end{definition}

\begin{definition}[cartas compatibles]
  Dos cartas de \(X\) \((\mathcal{U}, \varphi)\), \((\mathcal{W}, \psi)\) se
  dicen \emph{compatibles} si \(\mathcal{U} \cap \mathcal{W} = \emptyset\) o, en
  caso contrario, \(\varphi(\mathcal{U} \cap \mathcal{W})\) y \(\psi(\mathcal{U}
  \cap \mathcal{W})\) son abiertos y además \(\psi \circ \varphi^{-1} \colon
  \varphi(\mathcal{U} \cap \mathcal{W}) \to \psi(\mathcal{U} \cap \mathcal{W})\)
  y su inversa \(\varphi \circ \psi^{-1}\) son diferenciables (son
  difeomorfismos).

  Una carta se dice compatible con un atlas si es compatible con todas las
  cartas del atlas.
\end{definition}

\begin{definition}[atlas]
  Un \emph{atlas} de \(X\) es una colección de cartas compatibles de la misma
  dimensión \(\Set{(U_{i},\varphi_{i})}_{i \in \Lambda}\) con \(X = \bigcup_{i
    \in \Lambda} U_{i}\).
\end{definition}

\begin{lemma}[compatibilidad de la carta restricción]
  \label{lem:compat-restriccion}
  Si \((U,\varphi)\) una carta de \(X\) y \(\Omega \subseteq \varphi(U)\) es
  abierto de \(\RealSet^{n}\). Entonces si \(V = \varphi^{-1}(\Omega)\) y
  \(\varphi' = \Restrict{\varphi}{V}\) tenemos que \((V,\varphi')\) es carta de
  \(X\) compatible con \((U,\varphi)\).
\end{lemma}

\begin{proof}
  Nótese que \(U \cap V = V\). \(\varphi(U \cap V) = \varphi(V) = \Omega\) es
  abierto de \(\RealSet^{n}\) y \(\varphi'(U \cap V) = \varphi(U \cap V)\), por
  lo que las cartas son compatibles.
\end{proof}

\begin{lemma}[compatibilidad heredada de subcartas]
  \label{lem:compat-subcartas}
  Sean \((W,\psi), (U,\varphi)\) cartas compatibles de \(X\) con \(W \subseteq
  U\). Entonces \((W,\psi)\) es compatible con toda carta \((V,\rho)\)
  compatible con \((U,\varphi)\).
\end{lemma}

\begin{proof}
  Supongamos \(V \cap U \neq \emptyset\), porque de lo contrario no habría nada
  que demostrar.

  Entonces conocemos que
  \begin{itemize}
  \item \(\psi(W \cap U) = \psi(W)\) y \(\varphi(W \cap U) = \varphi(W)\) son
    abiertos.
  \item \(\psi \circ \varphi^{-1} \colon \varphi(W \cap U) \to \psi(W \cap U)\)
    y \(\varphi \circ \psi^{-1} \colon \psi(W \cap U) \to \varphi(W \cap U)\)
    son diferenciables.
  \item \(\rho(U \cap V)\) y \(\varphi(U \cap V)\) son abiertos.
  \item \(\rho \circ \varphi^{-1} \colon \varphi(U \cap V) \to \rho(U \cap V)\)
    y \(\varphi \circ \rho^{-1} \colon \rho(U \cap V) \to \varphi(U \cap V)\)
    son diferenciables.
  \end{itemize}

  Hay que probar lo siguiente:
  \begin{enumerate}
  \item \(\rho \circ \psi^{-1}\) es diferenciable.
  \item \(\psi \circ \rho^{-1}\) es diferenciable.
  \item \(\psi(V \cap W)\) es abierto.
  \item \(\rho(V \cap W)\) es abierto.
  \end{enumerate}

  Prueba:
  \begin{enumerate}
  \item \(\rho \circ \psi^{-1} = (\rho \circ \varphi^{-1}) \circ (\varphi \circ
    \psi^{-1})\) que es diferenciable por ser composición de diferenciables
    (teniendo en cuenta las restricciones necesarias).
  \item \(\psi \circ \rho^{-1} = (\psi \circ \varphi^{-1}) \circ (\varphi \circ
    \rho^{-1})\) que es diferenciable por ser composición de diferenciables.
  \item \(\rho(V \cap W) = \rho(V \cap W \cap U) = \rho(V) \cap \rho(U \cap W)\)
    que es intersección de abiertos, luego es abierto (el último paso puede
    hacerse porque \(\rho\) es inyectiva). El último término es abierto porque
    es imagen inversa de un abierto.
  \item \(\psi(V \cap W) = \psi(V) \cap \psi(W)\) que es intersección de
    abiertos, luego es abierto. \(\psi(V)\) es abierto porque \(\psi(V) =
    \psi{(\varphi^{-1}{(\varphi{(V)})})} = (\psi \circ \varphi^{-1}){(\varphi{(V)})} =
    {(\varphi \circ \psi^{-1})}^{-1}{(\varphi{(V)})}\), i.e.\ es imagen inversa de un abierto.
  \end{enumerate}
\end{proof}

\begin{lemma}[compatibilidad transitiva a través de atlas]
  \label{lem:compat-trans-atlas}
  Si \((W, \psi), (W', \psi')\) son compatibles con un atlas \(A\) de \(X\),
  entonces son compatibles entre sí.
\end{lemma}

\begin{proof}
  Hay que probar que
  \begin{enumerate}
  \item \(\psi(W \cap W')\) es abierto.
  \item \(\psi'(W \cap W')\) es abierto.
  \item \(\psi \circ \psi'^{-1}\) es diferenciable.
  \item \(\psi' \circ \psi^{-1}\) es diferenciable.
  \end{enumerate}
  % TODO
\end{proof}

\begin{definition}[atlas compatibles]
  Dos atlas \(\mathcal{A}\) y \(\mathcal{B}\) sobre \(X\) se dicen
  \emph{compatibles} si cada carta de \(\mathcal{A}\) es compatible con cada
  carta de \(\mathcal{B}\) y recíprocamente.
\end{definition}

\begin{definition}[estructura diferenciable]
  Dado un atlas \(\mathcal{A}\) sobre \(X\), se llama \emph{estructura
    diferenciable} generada por \(\mathcal{A}\) a la familia de todas las cartas
  compatibles con \(\mathcal{A}\).

  También se denomina \emph{atlas maximal} que contiene a \(\mathcal{A}\).
\end{definition}

\begin{lemma}
  Las estructuras diferenciables generadas por \(\mathcal{A}\) y \(\mathcal{B}\)
  coinciden si y solo si \(\mathcal{A}\) y \(\mathcal{B}\) son compatibles.
\end{lemma}

\begin{proof}
  Utilizaremos la notación \(\langle  \mathcal{A} \rangle\) para denotar la estructura
  diferenciable generada por el atlas \(\mathcal{A}\).

  \begin{subproof}[\(\implies\)]
    Supongamos que \(\langle \mathcal{A} \rangle = \langle \mathcal{B} \rangle = M\). Entonces, como
    \(\mathcal{A} \subseteq M\) y \(\mathcal{B} \subseteq M\), \(\mathcal{A}\) y
    \(\mathcal{B}\) son compatibles porque \(M\) es un atlas (por el
    \cref{lem:compat-subcartas} de \nameref{lem:compat-subcartas} aplicado a
    cada carta de los atlas).
  \end{subproof}

  \begin{subproof}[\(\impliedby\)]
    Supongamos que \(\mathcal{A}\) y \(\mathcal{B}\) son compatibles. Hay que
    probar que \(\langle \mathcal{A} \rangle = \langle \mathcal{B} \rangle\). Lo
    haremos por doble contención. Haremos la prueba para \(\langle \mathcal{A}
    \rangle \subseteq \langle \mathcal{B} \rangle\), porque la contención
    contraria es completamente análoga.

    Sea \((U, \varphi) \in
    \langle \mathcal{A} \rangle\). Hay que probar que es compatible con
    cualquier carta \((V,\psi) \in \mathcal{B}\).

      Si \(U \cap V = \emptyset\), ya está probado. En caso contrario,
      \((U,\varphi)\) es compatible con \(\mathcal{A}\). Además, como suponemos
      que \(\mathcal{A}\) y \(\mathcal{B}\) son compatibles, \((V,\psi)\)
      también es compatible con \(\mathcal{A}\). Por el
      \cref{lem:compat-trans-atlas}, \((U,\varphi)\) y \((V,\psi)\) son
      compatibles.
  \end{subproof}
\end{proof}

\begin{definition}[variedad diferenciable]
  \label{def:vd}
  Una \emph{variedad diferenciable} de dimensión \(n\) es un conjunto \(X\) con
  un atlas \(\mathcal{A}\) numerable cumpliendo la \nameref{def:Hausdorff}.

\end{definition}

\begin{definition}[propiedad de Hausdorff]
  \label{def:Hausdorff}
  Se dice que \(\mathcal{A}\) cumple la propiedad de Hausdorff si, dados \(x,y
  \in X\), existen cartas compatibles con \(\mathcal{A}\) \((U, \varphi)\) y
  \((V, \psi)\) con \(x \in U, y \in V, U \cap V = \emptyset\).
\end{definition}

\begin{example}
  \(\RealSet^{n}\) con \(\Id \colon \RealSet^{n} \to \RealSet^{n}\):
  \((\RealSet^{n}, \Id)\) es una carta que lo hace \(n\)-variedad.

  Cualquier \(U \subseteq \RealSet^{n}\) abierto proporciona una carta \((U,
  i)\) compatible con \(\mathcal{A}\), con \(i \colon U \to \RealSet^{n}\) la
  inclusión.
\end{example}

\begin{example}
  En \(\RealSet\), la carta anterior no es compatible con la carta \((\RealSet,
  f)\), \(f(x) = x^{3}\), porque \(\Id \circ f^{-1}\) no es diferenciable.
\end{example}

\begin{example}
  \(\ComplexSet^{n}\) es una \(2n\)-variedad:
  Dados \(a_{j} + \I b_{j}, j = 1,\dots,n \mapsto
  (a_{1},\dots,a_{n},b_{1},\dots,b_{n}) \in \RealSet^{2n}\), luego \(\mathcal{A}
  = \Set{(\ComplexSet^{n}, \varphi)}\) es un atlas de una \(2n\)-variedad.
\end{example}

\begin{example}
  El espacio de matrices reales \(m \times n\) es una \(mn\)-variedad con la
  aplicación \(\varphi(A) =
  (a_{11},\dots,a_{1n},a_{21},\dots,a_{2n},\dots,a_{mn})\).

  Análogamente, las matrices complejas \(m \times n\) forman una \(2mn\)-variedad.
\end{example}

\begin{definition}[topología natural]
  \label{def:topnat}
  Sea \(\mathcal{A}\) un atlas de \(X\). Se define la famila \(\mathcal{T}_{\mathcal{A}}\)
  como la de aquellos conjuntos \(G \subseteq X\) tales que
  \[
    \varphi(G \cap U) \text{ es abierto de } \varphi(U) \forall (U,\varphi) \in \mathcal{A}
  \]

  A \(\mathcal{T}_{\mathcal{A}}\) la llamamos \emph{topología natural} del atlas \(\mathcal{A}\)
\end{definition}

\begin{lemma}
  \(\mathcal{T}_{\mathcal{A}}\) es topología sobre \(X\).
\end{lemma}

\begin{proof}\item
  \begin{subproof}[\(\emptyset, X \in \mathcal{A}\)]
    Trivial.
  \end{subproof}

  \begin{subproof}[\(\Set{G_{\alpha}}_{\alpha \in \Gamma} \subseteq
    \mathcal{T}_{\mathcal{A}} \implies \bigcup_{\alpha \in \Gamma} G_{\alpha}
    \in \mathcal{T}_{\mathcal{A}}\)]
    Sea \((U,\varphi)\) una carta de \(\mathcal{A}\).
    \begin{align*}
      &\varphi(U \cap G_{\alpha}) \text{ es abierto de } \varphi(U)\\
      \implies & \bigcup_{\alpha \in \Gamma} \varphi(U \cap G_{\alpha}) \text{ es abierto de } \varphi(U)\\
      \implies & \varphi(U \cap G) \text{ es abierto de } \varphi(U)
    \end{align*}
  \end{subproof}

  \begin{subproof}[\(G_{1}, \dots, G_{n} \in \mathcal{T}_{\mathcal{A}} \implies
    \bigcap_{i = 1}^{n} G_{i} \in \mathcal{T}_{\mathcal{A}}\)]
    La prueba es análoga a la del apartado anterior.
  \end{subproof}
\end{proof}

\begin{lemma}
  Para toda carta \((W, \psi)\) compatible con \(\mathcal{A}\), \(W \in
  \mathcal{T}_{\mathcal{A}}\).
\end{lemma}

\begin{proof}
  Por compatibilidad de \((W,\psi)\) con toda \((U,\varphi) \in \mathcal{A}\),
  se tiene que \(\varphi(U \cap W) \subseteq \varphi(U) \subseteq \RealSet^{n}\) es abierto de
  \(\RealSet^{n}\), y por tanto es abierto de \(\varphi(U)\), por lo que \(W \in
  \mathcal{T}_{\mathcal{A}}\)
\end{proof}

\begin{lemma}
  \label{lem:topnat-caract}
  Son equivalentes:
  \begin{gather}
    \label{eq:topnat-caract-1}
    G \in \mathcal{T}_{\mathcal{A}}\\
    \label{eq:topnat-caract-2}
    (W, \psi) \in \langle \mathcal{A} \rangle \implies \psi(W \cap G) \text{ abierto
      de } \psi(W)\\
    \label{eq:topnat-caract-3}
    \forall p \in G \  \exists (V,\rho) \text{ compatible con } \mathcal{A}
    \text{ con } p \in V \subseteq G
  \end{gather}
\end{lemma}

\begin{proof}\item
  
  \begin{subproof}[\(\ref{eq:topnat-caract-1} \implies \ref{eq:topnat-caract-2}\)]
    % TODO
  \end{subproof}

  \begin{subproof}[\(\ref{eq:topnat-caract-2} \implies \ref{eq:topnat-caract-3}\)]
    Sea \(p \in G\). Entonces \(\exists (U,\varphi) \in \mathcal{A}\) con \(p
    \in U\). Por \ref{eq:topnat-caract-2}, \(\varphi(U \cap G)\) es abierto de
    \(\varphi(U)\).

    Sea \(V = U \cap G\). \((V, \rho) = (U \cap G), \Restrict{\varphi}{U \cap
      G}\) es compatible con \(\mathcal{A}\).

    Por los \cref{lem:compat-restriccion,lem:compat-subcartas} y \(p \in U \cap
    G = V \subseteq G\).
  \end{subproof}

  \begin{subproof}[\(\ref{eq:topnat-caract-3} \implies
    \ref{eq:topnat-caract-1}\)]
    Sea \((U,\varphi) \in \mathcal{A}\). Sea \(x \in \varphi(G \cap U)\).
    Entonces \(\exists p \in G \cap U : \varphi(p) = x\). Por
    \eqref{eq:topnat-caract-3}, \(\exists (V,\rho)\) compatible con
    \(\mathcal{A}\), luego \(\varphi(V \cap U)\) es abierto de \(\RealSet^{n}\)
    (o de \(\varphi(U)\)).
    \(p \in V \subseteq G \implies p \in V \cap U \subseteq G \cap U \implies x
    \in \varphi(V \cap U) \subseteq \varphi(G \cap U) \implies x\) es punto
    interior de \(\varphi(G \cap U)\), por lo que \(\varphi(G \cap U)\) es
    abierto (todos sus puntos son interiores).
  \end{subproof}
\end{proof}

\begin{lemma}
  Para toda carta \((W,\psi) \in \mathcal{A}\) (de hecho basta que sea
  compatible con \(\mathcal{A}\)), \(W \in
  \mathcal{T}_{\mathcal{A}}\) y \(\psi \colon W \to \psi(W)\) es homeomorfismo.

  Recíprocamente, si \(\mathcal{T}\) es una topología sobre \(X\) tal que
  \(\forall (U,\varphi) \in \mathcal{A} \quad U \in \mathcal{T}\) y \(\varphi
  \colon U \to \varphi(U)\) es homeomorfismo, entonces \(\mathcal{T} =
  \mathcal{T}_{\mathcal{A}}\)
\end{lemma}

\begin{proof}
  % TODO
\end{proof}

\section{Propiedades de la topología natural de una variedad diferenciable}

\begin{definition}[base]
  Sea \((X,\mathcal{T})\) un espacio topológico. Una subfamilia de abiertos
  \(\mathcal{B} \subseteq \mathcal{T}\) se dice \emph{base} de la topología si
  \(\forall G \in \mathcal{T}, x \in G, \ \exists B \in \mathcal{B} : x \in B
  \subseteq G\)
\end{definition}

\begin{definition}[1ºN]
  \label{def:1n}
  Se dice que \((X,\mathcal{T})\) \emph{cumple el 1º axioma de numerabilidad} o
  que es \emph{1-numerable} si existe una familia numerable \(\mathcal{B}_{x}\)
  de abiertos (depende de \(x\)) que contienen a \(x\) y tal que para todo \(G \in \mathcal{T}\)
  con \(x \in G\) existe \(B \in \mathcal{B}_{x}\) con \(x \in B \subseteq G\).
\end{definition}

\begin{definition}[2ºN]
  \label{def:2n}
  Se dice que \((X,\mathcal{T})\) \emph{cumple el 2º axioma de numerabilidad} o
  que es \emph{2-numerable} si existe una base numerable de la topología.
\end{definition}

\begin{remark}
  Si \((X,\mathcal{T})\) es \nameref{def:2n}, entonces es \nameref{def:1n}.
\end{remark}

\begin{proof}
  Basta tomar \(\mathcal{B}_{x} = \Set{B \in \mathcal{B} : x \in B}\)
\end{proof}

\begin{proposition}
  \(\RealSet^{n}\) cumple \nameref{def:2n}.
\end{proposition}

\begin{proof}
  Tomamos \(\mathcal{B}\) como sigue (es numerable):
  \[
    \mathcal{B} = \Set{B(q,\frac{1}{n}),\ n \geq 1,\ q \in \RationalSet^{n}}
  \]

  Sea \(G\) abierto de \(\RealSet^{n}\) y \(x \in G\). Por definición, existe
  \(\varepsilon > 0\) tal que \(\Bola{x}{\varepsilon} \subseteq G\). Sea
  \(n_{0}\) con \(\frac{1}{n_{0}} < \frac{\varepsilon}{2}\). Sea \(q \in
  \RationalSet^{n}\) con \(\Dist{q}{x} < \frac{1}{n_{0}}\). Entonces \(x \in
  \Bola{q_{0}}{\frac{1}{n_{0}}}\). Afirmamos que \(\Bola{q_{0}}{\frac{1}{n_{0}}}
  \subseteq \Bola{x,\varepsilon} \subseteq G\). Probémoslo.

  Sea \(y \in \Bola{q_{0}}{\frac{1}{n_{0}}}\). Entonces \(\Dist{x}{y} \leq
  \Dist{x}{q_{0}} + \Dist{q_{0}}{y} < \frac{1}{n_{0}} + \frac{1}{n_{0}} < \varepsilon\).
\end{proof}

\begin{lemma}
  Si \((X,\mathcal{T})\) cumple \nameref{def:2n}, entonces \(\forall A \subseteq
  X\), \((A,\mathcal{T}_{\mathcal{A}})\) lo cumple también.
\end{lemma}

\begin{proof}
  Sea \(\mathcal{B}\) base numerable de \((X,\mathcal{T})\). Entonces
  \(\mathcal{B}_{A} = \Set{B \cap A : B \in \mathcal{B}}\) es base numerable de
  \((A,\mathcal{T}_{A})\).
\end{proof}

\begin{lemma}
  La \nameref{def:topnat} de una variedad diferenciable \(M\) cumple \nameref{def:2n}.
\end{lemma}

\begin{proof}
  Sea \(\mathcal{A} = \Set{(U_{j}, \varphi_{j})}_{j \in J}\) un atlas numerable
  de \(M\). Cada \(\varphi_{j}(U_{j})\) abierto de \(\RealSet^{n}\) cumple
  \nameref{def:2n}. Existe \(\mathcal{B}_{j} = \Set{B_{k}^{j}}_{k \geq 1}\) base
  numerable de abiertos de \(\varphi_{j}(U_{j})\).
  \(\varphi_{j}^{-1}(B_{k}^{j})\) abierto de \(U_{j}\), luego es abierto de
  \(\mathcal{T}_{\mathcal{A}}\). Por tanto \(\varphi_{j} \colon U_{j} \to
  \varphi(U_{j})\) es homeomorfismo.

  Sea \(\mathcal{B}_{M} = \Set{\varphi^{-1}(B_{k}^{j})}_{j \in J, k \geq 1}\).
  \(\mathcal{B}_{M}\) es numerable y veamos que es base de
  \(\mathcal{T}_{\mathcal{A}}\):

  Sea \(G \in \mathcal{T}_{\mathcal{A}}\) y \(p \in G\). Entonces,
  \(\varphi_{j}(p) \in \varphi_{j}(G \cap U_{j})\), que es abierto de
  \(\varphi_{j}(U_{j})\). Por tanto, \(\exists k_{0}\) con \(\varphi_{j}(p) \in
  B_{k_{0}} \subseteq \varphi_{j}(G \cap U_{j})\), lo que implica que \(p \in
  \varphi_{j}^{-1}(B_{k_{0}}^{j}) \subseteq G \cap U_{j} \subseteq G\).
\end{proof}

\begin{proposition}
  La \nameref{def:topnat} de \(M\) cumple la \nameref{def:Hausdorff}.
\end{proposition}

\begin{proof}
  % TODO
\end{proof}

\begin{proposition}
  En todo espacio topológico \((X,\mathcal{T})\) que cumpla \nameref{def:2n},
  cualquier \(A \subseteq X\) cumple la \nameref{def:Lindelöff}.
\end{proposition}

\begin{definition}[propiedad de Lindelöff]
  \label{def:Lindelöff}
  Si \(A \subseteq \bigcup_{\alpha \in \Lambda} G_{\alpha}\) una familia de
  abiertos, entonces \(\exists \Set{\alpha_{i}}_{i \geq 1}\) numerable con \(A
  \subseteq \bigcup_{i=1}^{\Infinity} G_{\alpha_{i}}\).
\end{definition}

\begin{corollary}
  Todo \(A \subseteq M\) con \(M\) \nameref{def:vd} cumple la
  \nameref{def:Lindelöff} para la \nameref{def:topnat}.
\end{corollary}

\begin{proof}
  Sea \(\Set{B_{i}}_{i \geq 1}\) una base numerable de \(\mathcal{T}\).
  Para cada \(n \geq 1\) sea \(\Lambda_{n} = \Set{\alpha : B_{n} \subseteq
    G_{\alpha}}\).

  Si \(\Lambda_{n} \neq \emptyset\) elegimos un \(\alpha(n) \in \Lambda_{n}\).
  Afirmamos que \(A \subseteq \bigcup_{n=1}^{\Infinity} G_{\alpha(n)}\).
  Probémoslo:

  Sea \(a \in A\). Entonces \(\exists \alpha_{a}\) con \(a \in G_{\alpha_{a}}\).
  Como \(\mathcal{B}\) es base, \(\exists n_{a}\) con \(a \in B_{n_{a}}
  \subseteq G_{\alpha_{a}}\). Por tanto, \(\alpha_{a} \in \Lambda_{n_{a}}\), por
  lo que \(\Lambda_{n_{a}} \neq \emptyset\) y hemos elegido \(\alpha(n_{a}) =
  \alpha_{a}\). Por tanto, \(a \in B_{n_{a}} \subseteq G_{\alpha(n_{a})}\).
\end{proof}

\begin{proposition}
  Todo abierto (de la \nameref{def:topnat}) de una \nameref{def:vd} \(M\) de
  dimensión \(n\) es también una \nameref{def:vd} de dimensión \(n\).
\end{proposition}

\begin{proof}
  % TODO
\end{proof}

\section{Ejercicios}
\label{sec:r1}

\begin{Answer}[number=1]
  \Question Llamamos \(V_{n} = \Set{(x_{1},\dots,x_{n+1}) \in S^{n} \colon
    x_{n+1} \neq 1}\) y \(V_{s} = \Set{(x_{1},\dots,x_{n+1}) \in S^{n} \colon
    x_{n+1} \neq -1}\).

  Considerando los hemisferios \(U_{i}^{+}\) y \(U_{i}^{-}\), para cada \(i\) se
  consideran las proyecciones \(\varphi_{i} \colon U_{i}^{+} \to B^{n}\) y
  \(\psi_{i} \colon U_{i}^{-1} \to B^{n}\) consistentes en eliminar la
  coordenada \(i\) del vector (proyectan el hemisferio sobre la bola unidad).
  \(\varphi_{i}^{-1}(y_{1},\dots,y_{n}) =
  (y_{1},\dots,y_{i-1},\sqrt{1-\Norm{y}^{2}},y_{i},\dots,y_{n})\) y análogamente
  tenemos \(\psi_{i}^{-1}\).

  Este apartado será consecuencia del siguiente por el
  \cref{lem:compat-trans-atlas}.

  \Question
  Hay que probar lo siguiente:
  \begin{enumerate}
  \item \(\varphi_{i} \circ \rho_{n}^{-1} \colon \rho_{n}(V_{n} \cap U_{i}^{+})
    \to \varphi_{i}(V_{n} \cap U_{i}^{+})\) es diferenciable.
  \item \(\rho_{n}(V_{n} \cap U_{i}^{+})\) es abierto de \(\RealSet^{n}\).
  \item \(\varphi_{i}(V_{n} \cap U_{i}^{+})\) es abierto de \(\RealSet^{n}\).
  \end{enumerate}

  \begin{proof}
    \begin{enumerate}
    \item
      \begin{align*}
        \rho_{n}^{-1}(y_{1},\dots,y_{n})
        &= \left( \frac{2y_{1}}{1+\Norm{y}^{2}},\dots, \frac{2y_{n}}{1+\Norm{y}^{2}},
          \frac{\Norm{y}^{2}}{1+\Norm{y}^{2}} \right)\\
        (\varphi_{i} \circ \rho_{n}^{-1})(y_{1},\dots,y_{n})
        &= \left( \frac{2y_{1}}{1+\Norm{y}^{2}},\dots, \underbrace{\frac{2y_{i}}{1+\Norm{y}^{2}}}_{\text{eliminado}},\dots,
          \frac{\Norm{y}^{2}}{1+\Norm{y}^{2}} \right)
      \end{align*}
      Esta función es claramente diferenciable.

    \item \(\rho_{n}(V_{n} \cap U_{n+1}^{+}) = \rho_{n}(U_{n+1}^{+} \setminus
      \Set{N}) = \RealSet^{n} \setminus D^{n}\), que es abierto.

      Si \(i \neq 1, n+1\), la imagen es un semi-\(\RealSet^{n}\) abierto.

      Si \(i = 1\), la imagen es \(B^{n}\), que es abierta.

      Análogamente con \(\rho_{s}\) cambiando \(1\) por \(n+1\).
    \item \(\varphi_{n+1}(V_{n} \cap U_{i}^{+}) = \varphi_{n+1}(U_{i}^{+} \setminus
      \Set{N}) = B^{n} \setminus \Set{\vec{0}}\), que es abierto.

      \(U_{i}^{+} \cap V_{norte} = U_{i}^{+} \  \forall i \neq n+1\).
      \(\varphi_{i}(U_{i}^{+}) = B^{n}\) que es abierto.
    \end{enumerate}
  \end{proof}
\end{Answer}

\begin{Answer}[number=2]
  Consideramos la siguiente aplicación para la carta de \(U_{i}\):
  \begin{align*}
    \varphi_{i} \colon U_{i} &\to \RealSet^{n}\\
    [x_{1}, \dots, x_{n+1}] &\mapsto \left(
                              \frac{x_{1}}{x_{i}},
                              \dots,
                              \frac{x_{i-1}}{x_{i}},
                              \frac{x_{i+1}}{x_{i}},
                              \dots,
                              \frac{x_{n}}{x_{i}}
                              \right)
  \end{align*}

  Para comprobar que efectivamente \(\mathcal{A} \coloneqq \Set{(U_{i},\varphi_{i})}\) es un atlas
  debemos comprobar que las cartas son compatibles dos a dos.
  Comprobemos que \(\varphi_{i}(U_{i} \cap U_{j})\) es abierto de
  \(\RealSet^{n}\): la imagen es \(\RealSet^{n} \setminus H_{j}\),
  con
  \[
    H_{j} =
    \begin{cases}
      \Set{x_{j} = 0} & \text{si } j < i\\
      \Set{x_{j-1} = 0} & \text{si } j > i
    \end{cases}
  \]

  luego es abierto.

  Hay que ver que \(\varphi_{j} \circ \varphi_{i}^{-1} \colon \varphi_{i}(U_{i}
  \cap U_{j}) \to \varphi_{j}(U_{i} \cap U_{j})\) es diferenciable:

  \begin{align*}
    (y_{1},\dots,y_{n})
    &\mapsto [y_{1},\dots,y_{i-1},1,y_{i},\dots,y_{n}]\\
    &\mapsto \left(
      \frac{y_{1}}{y_{j}},
      \dots,
      \frac{y_{j-1}}{y_{j}},
      \frac{y_{j+1}}{y_{j}},
      \dots,
      \frac{y_{i-1}}{y_{j}},
      \frac{1}{y_{j}},
      \frac{y_{i+1}}{y_{j}},
      \dots,
      \frac{y_{n}}{y_{j}}
      \right)
  \end{align*}
  que es diferenciable.

  Probemos que la \nameref{def:topnat} del atlas es la cociente:
  sea \(G \in \mathcal{T}_{\mathcal{A}}\). Entonces \(\varphi_{i}(G \cap
  U_{i})\) es abierto de \(\RealSet^{n}\). Veamos que \(\pi^{-1}(G)\) es abierto
  de \(\RealSet^{n+1}\setminus\Set{0}\).
  \[
    \pi^{-1}(U_{i}) = \RealSet^{n+1} \setminus \Set{x_{i} = 0}
  \]
\end{Answer}

\end{document}

%%% Local Variables:
%%% mode: latex
%%% mode: flyspell
%%% mode: outline
%%% ispell-local-dictionary: "castellano"
%%% End: